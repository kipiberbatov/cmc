\begin{definition}
  Let
    $R$ be a ring,
    $V$ be a finite-dimensional $R$-module,
    $\omega \in \Lambda^2 V^*$.
  We say that $\omega$ is \textbf{non-degenerate} or \textbf{symplectic}
  if the associated map
  \begin{equation}
    \tilde{\omega} \colon V \to V^*,\
    X \in V \mapsto \tilde{\omega}(X) := i_X \omega \in V^*,
  \end{equation}
  is an isomorphism.

  The pair $(V, \omega)$ is called a \textbf{symplectic module}
  (or \textbf{symplectic vector space} if $R$ is a field).
\end{definition}
\begin{proposition}
  Let
    $R$ be a ring without,
    $(V, \omega)$ be a finite-dimensional symplectic module over $R$.
  Assume that for any $x \in R,\ x + x = 0 \Rightarrow x = 0$.
  Then $\dim V$ is an even number.
\end{proposition}
\begin{proof}
  Let $n = \dim V$.
  In a basis of $V$ $\omega$ is represented by an antisymmetric matrix $A$.
  But then
  \begin{equation}
    \det A = \det(A^T) = \det(-A) = (-1)^n \det A.
  \end{equation}
  If $n$ is odd, then $\det A + \det A = 0$.
  By assumption this means that $\det A = 0$
  which contradicts the nondegeneracy of $\omega$.
  Hence, $n$ is even.
\end{proof}
\begin{definition}
  Let $M$ be a smooth manifold, $\omega \in \Omega^\bullet M$.
  We say that:
  \begin{enumerate}
    \item
      $\omega$ is \textbf{closed} if $d \omega = 0$
    \item
      $\omega$ is \textbf{exact} if there exists $\eta \in \Omega^\bullet M$
      such that $d \eta = \omega$.
  \end{enumerate}
\end{definition}
\begin{proposition}
  Let $M$ be a smooth manifold, $\omega \in \Omega^\bullet M$.
  If $\omega$ is exact, then it is closed.
\end{proposition}
\begin{proof}
  Let $\eta \in \Omega^\bullet M$ be such that $d \eta = \omega$.
  Then $d \omega = d (d \eta) = 0$, i.e., $\omega$ is closed.
\end{proof}
\begin{definition}
  Let $M$ be a smooth manifold, $\omega \in \Omega^2 M$.
  We say that $\omega$ is a \textbf{symplectic form}
  if it is non-degenerate
  (with base module $\mathfrak{X} M$ over $\mathcal{F} M$) and closed.

  The pair $(M, \omega)$ is called a \textbf{symplectic manifold}.
\end{definition}
\begin{proposition}
  Let $(M, \omega)$ be a symplectic manifold.
  Then $M$ is even-dimensional.
\end{proposition}
\begin{definition}
  Let $Q$ be a smooth manifold.
  Consider the cotangent bundle $T^* Q$ with bundle projection
  $\pi \colon T^* Q \to Q$
  with differential $d \pi \colon T(T^* Q) \to T Q$.
  Define the \textbf{tautological one-form}
  $\theta \colon T^* Q \to T^* (T^* Q)$ as follows:
  for any $(q, p) \in T^*Q$ (i.e,. $q \in Q$, $p \in \Hom(T_q Q, \R)$),
  \begin{equation}
    \restrict{\theta}{(q, p)}
    := p \circ \restrict{d \pi}{(q, p)} \in T^*_{(q, p)}(T^* Q).
  \end{equation}
  In other words, if we denote $M := T^* Q$, then $\theta$ is a section of its
  cotangent bundle $T^* M$, i.e., an $1$-form on $M$. 
\end{definition}
\begin{discussion}
  Let $Q$ be a smooth manifold, $\pi \colon T^* Q \to Q$ be the projection.
  Then a $1$-form on $Q$ is a section of $\colon T^* Q$, i.e., a smooth map
  $\mu \colon Q \to \colon T^* Q$ such that $\pi \circ \mu = \id_Q$.
  As such it has a pullback
  $\mu^* \colon \Omega^\bullet(T^* Q) \to \Omega^\bullet Q$.
\end{discussion}
\begin{proposition}
  Let
    $Q$ be a smooth manifold,
    $\theta$ be the tautological one-form on $T^* Q$,
    $\mu \in \Omega^1 Q$.
  Then
  \begin{equation}
    \mu^* \theta = \mu.
  \end{equation}
\end{proposition}
\begin{proof}
  Let $q \in Q$.
  Then
  \begin{equation}
    \restrict{\mu^* \theta}{q}
    = \restrict{\theta}{\mu q} \circ \restrict{d \mu}{q}
    = \restrict{\mu}{q} \circ \restrict{(d \pi)}{\mu q}
      \circ \restrict{d \mu}{q}
    = \restrict{\mu}{q} \circ \restrict{d(\pi \circ \mu)}{q}
    = \restrict{\mu}{q}.
  \end{equation}
  Since $q$ is arbitrary, $\mu^* \theta = \mu$.
\end{proof}
\begin{definition}
  Let
    $Q$ be a smooth manifold,
    $\theta \in \Omega^1(T^* Q)$ be the tautological one-form.
  Define $\omega := - d \theta$.
  The pair $(T^* Q, \omega)$ is called the \textbf{phase space} of $Q$.
  (In this setting $Q$ is usually called the \textbf{configuration space}.)
\end{definition}
\begin{remark}
  Let $Q$ be a smooth manifold.
  The elements of $T^* Q$ are of the form $(q, p)$ where $q \in Q$ and
  $p \in T^*_q Q = \Hom(T_q Q, \R)$.
  $q$ is called \textbf{generalised position}, while $p$ is called
  \textbf{generalised momentum}.
\end{remark}
\begin{proposition}
  Let
    $Q$ be a smooth manifold,
    $(T^* Q, \omega)$ be its phase space.
  Then $(T^* Q, \omega)$ is a symplectic manifold.
\end{proposition}
\begin{definition}
  Let $Q$ be a smooth manifold of dimension $n$.
  Consider a point $q_0 \in Q$ and let $(U, \hat{\varphi})$ be a chart around
  $q_0$, i.e., $U$ is a neighbourhood of $q_0$ and
  $\hat{\varphi} \colon U \to \R^n$ is a diffeomorphism.
  Let $\{\hat{q}^i \colon U \to \R\}_{i = 1}^n$ be the corresponding local
  coordinates, i.e., if $\{\pi^i \colon \R^n \to \R\}_{i = 1}^n$ are the
  projection maps, then $\{\hat{q}^i = \pi^i \circ \hat{\varphi}\}_{i = 1}^n$.
  Let $i \in \{1, ..., n\}$.
  Define \textbf{position coordinate} $q^i \colon T^* U \to \R$ by
  \begin{equation}
    q^i := \hat{q}^i \circ \restrict{\pi}{U}.
  \end{equation}
  Also, define \textbf{momentum coordinate} $p_i \colon T^* U \to \R$
  as follows: for any $(q, p) \in T^* U$,
  \begin{equation}
    p_i(q, p)
    := p\left(\restrict{\frac{\partial}{\partial \hat{q}^i}}{q}\right).
  \end{equation}
\end{definition}
\begin{proposition}
  Let
    $Q$ be a smooth manifold of dimension $n$,
    $q_0 \in Q$,
    $(U, \hat{\varphi})$ be a chart around $q_0$,
    $\{\hat{q}^i \colon U \to \R\}_{i = 1}^n$ be the corresponding local
      coordinates,
    $\{q^i \colon T^* U \to \R\}_{i = 1}^n$ be the corresponding position
      coordinates,
    $\{p_i \colon T^* U \to \R\}_{i = 1}^n$ be the corresponding momentum
      coordinates.
  Then the map $\varphi \colon T^* U \to \R^{2 n}$ defined by
  \begin{equation}
    \varphi(q, p) = (q^1(q, p), ..., q^n(q, p), p_1(q, p), ..., p_n(q, p))
  \end{equation}
  is a diffeomorphism, i.e., $(T^* U, \varphi)$ is a chart around $(q_0, 0)$.
  (The covector in $T^*_{q_0}$ does not matter, so we make the trivial choice by
  taking zero.)

  These local coordinates are called \textbf{generalised coordinates}.
\end{proposition}
\begin{remark}
  From now on, given a manifold $Q$ and a chart $(U, \hat{\varphi})$, unless
  stated otherwise, we will fix the notation and use the objects defined above:
  the projection map $\pi \colon T^* Q \to Q$, the tautological one-form
  $\theta$ and the canonical symplectic form $\omega = - d \theta$;
  for $i = 1, ..., n$ the coordinate maps $\hat{q}^i$, $q^i$, and $p_i$;
  the chart $(T^* U, \varphi)$.
\end{remark}
\begin{proposition}
  Let
    $Q$ be a smooth manifold,
    $\xi \in \Omega^1(T^* Q)$ has the following property:
    for any $1$-form $\mu$ on $Q$, $\mu^* \xi= 0$.
  Then $\xi = 0$.
\end{proposition}
\begin{proof}
  Let
    $n := \dim Q$, $(U, \hat{\varphi})$ be a chart on $Q$ and
    $\{f_i, g^i \in \mathcal{F}(T^* U)\}_{i = 1}^n$ be such that
  \begin{equation}
    \restrict{\xi}{U} = \sum_{i = 1}^n f_i\, d q^i + \sum_{i = 1}^n g^i\, d p_i.
  \end{equation}
  Take arbitrary $\{h_j \in \mathcal{F} U\}_{j = 1}^n$ so that
  \begin{equation}
    \restrict{\mu}{U} = \sum_{j = 1}^n h_j\, d \hat{q}^j.
  \end{equation}
  Note that
  $q^i \circ \restrict{\mu}{U} = \hat{q}^i$ and
  $p_i \circ \restrict{\mu}{U} = h_i$.
  Hence,
  \begin{equation}
    0 
    = \restrict{(\mu^* \xi)}{U}
    = \sum_{i = 1}^n (f_i \circ \restrict{\mu}{U})\,
      d(q^i \circ \restrict{\mu}{U})
    + \sum_{i = 1}^n (g^i \circ \restrict{\mu}{U})\,
      d(p_i \circ \restrict{\mu}{U})
    = \sum_{i = 1}^n (f_i \circ \restrict{\mu}{U})\, d \hat{q}^i
    + \sum_{i = 1}^n (g^i \circ \restrict{\mu}{U})\, d h_i.
  \end{equation}
  Fix $q_0 \in U$, $p_0 \in T^*_{q_0} Q$ so that $(p_0, q_0) \in T^* U$.
  Denote
  \begin{equation}
    c_i
    :=
    p_0\left(\restrict{\frac{\partial}{\partial \hat{q}^i}}{q_0}\right),
    i = 1, ..., n,
  \end{equation}
  so that
  \begin{equation}
    p_0 = \sum_{i = 1}^n c_i \restrict{d \hat{q}^i}{q_0}.
  \end{equation}
  \begin{enumerate}
    \item
      We will first prove that
      for any $i \in \{1, ..., n\}$, $f_i(q_0, p_0) = 0$.
      Define the constant functions
      \begin{equation}
        h_i(q) := c_i,\ i \in \{1, ..., n\},\ q \in U.
      \end{equation}
      Then for any $i \in \{1, ..., n\}$, $d h_i = 0$.
      Hence,
      \begin{equation}
        \begin{split}
          0
          & = \restrict{(\mu^* \xi)}{q_0} \\
          & = \sum_{i = 1}^n
              f_i(q_0, \sum_{j = 1}^n h_j(q_0) \restrict{d \hat{q}^j}{q_0})\,
              \restrict{d \hat{q}^i}{q_0} \\
          & = \sum_{i = 1}^n
              f_i(q_0, \sum_{j = 1}^n c_j \restrict{d \hat{q}^j}{q_0})\,
              \restrict{d \hat{q}^i}{q_0} \\
          & = \sum_{i = 1}^n f_i(q_0, p_0)\, \restrict{d \hat{q}^i}{q_0}.
        \end{split}
      \end{equation}
      Therefore, for any $i \in \{1, ..., n\}$, $f_i(q_0, p_0) = 0$.
    \item
      We will now prove that
      for any $i \in \{1, ..., n\}$, $g^i(q_0, p_0) = 0$.
      Define the linear functions
      \begin{equation}
        h_i(q) := c_i + \hat{q}^i(q) - \hat{q}^i(q_0).
      \end{equation}
      Then for any $i \in \{1, ..., n\}$,
      $d h_i = d \hat{q}^i$ and $h_i(q) = c_i$.
      Hence,
      \begin{equation}
        \begin{split}
          0
          & = \restrict{(\mu^* \xi)}{q_0} \\
          & = \sum_{i = 1}^n
              g^i(q_0, \sum_{j = 1}^n h_j(q_0) \restrict{d \hat{q}^j}{q_0})\,
              \restrict{d h_i}{q_0} \\
          & = \sum_{i = 1}^n
              g^i(q_0, \sum_{j = 1}^n c_j \restrict{d \hat{q}^j}{q_0})\,
              \restrict{d \hat{q}^i}{q_0} \\
          & = \sum_{i = 1}^n g^i(q_0, p_0)\, \restrict{d \hat{q}^i}{q_0}.
        \end{split}
      \end{equation}
      Therefore, for any $i \in \{1, ..., n\}$, $g^i(q_0, p_0) = 0$.
  \end{enumerate}
  Since $(q_0, p_0) \in T^* U$ was arbitrary, we conclude that
  for any $i \in \{1, ..., n\}$, $f_i = g^i = 0$.
  Hence, $\restrict{\xi}{U} = 0$.
  Taking an atlas $\{(U_\alpha, \hat{\varphi}_\alpha)\}_{\alpha \in A}$ of $Q$
  (for some index set $A$), we conclude that $\xi = 0$.
\end{proof}
\begin{corollary}
  Let
    $Q$ be a smooth manifold,
    $\theta$ be the tautological one-form on $T^* Q$,
    $\eta \in \Omega^1(T^* Q)$ has the following property:
    for any $1$-form $\mu$ on $Q$, $\mu^* \eta = \mu$.
  Then $\eta = \theta$.
\end{corollary}
\begin{proof}
  Write $\eta = \theta + \xi$, i.e., $\xi := \eta - \theta$.
  Then, for any $\mu \in \Omega^1 Q$,
  \begin{equation}
    \mu
    = \mu^* \eta
    = \mu^* \theta + \mu^* \xi
    = \mu + \mu^* \xi
    \Rightarrow \mu^* \xi = 0.
  \end{equation}
  But from the previous proposition it follows that $\xi = 0$,
  and hence $\eta = \theta$.
\end{proof}
\begin{proposition}
  Let
    $Q$ be a smooth manifold of dimension $n$,
    $(U, \hat{\varphi})$ be a chart,
    $(q, p) \in T^* U$,
    $ i \in \{1, ..., n\}$.
  Then
  \begin{equation}
    \restrict{d \pi}{(q, p)}
    \left(\restrict{\frac{\partial}{\partial q^i}}{(q, p)}\right)
    = \restrict{\frac{\partial}{\partial \hat{q}^i}}{q}
  \end{equation}
  and
  \begin{equation}
    \restrict{d \pi}{(q, p)}
    \left(\restrict{\frac{\partial}{\partial p^i}}{(q, p)}\right)
    = 0.
  \end{equation}
\end{proposition}
\begin{proof}
  Let $f \colon Q \to \R$ be smooth.
  Define the functions
  $\hat{g} := f \circ \hat{\varphi}^{-1} \colon \R^n \to \R$ and
  $g := f \circ \pi \circ \varphi^{-1} \colon \R^{2 n} \to \R$.
  Let $(X^1, ..., X^n, Y^1, ..., Y^n) := \varphi(p, q) \in \R^n$.
  This means that $(X^1, ..., X^n) = \hat{\varphi}(q)$.
  Then
  \begin{equation}
    g(X^1, ..., X^n, Y^1, ..., Y^n)
    = f(\pi(q, p))
    = f(q)
    = \hat{g}(X^1, ..., X^n).
  \end{equation}
  Hence,
  \begin{equation}
    \begin{split}
      \frac{\partial g}{\partial x^i}(X^1, ..., X^n, Y^1, ..., Y^n)
      & = \lim_{h \to 0}
        \frac
        {g(X^1, ..., X^i + h, ..., X^n, Y^1, ..., Y^n)
         - g(X^1, ..., X^n, Y^1, ..., Y^n)}
        {h} \\
      & = \lim_{h \to 0}
        \frac{\hat{g}(X^1, ..., X^i + h, ..., X^n) - \hat{g}(X^1, ..., X^n)}{h}
        \\
      & = \frac{\partial \hat{g}}{\partial \hat{x}^i}(X^1, ..., X^n).
    \end{split}
  \end{equation}
  Similarly, since $g$ is constant with respect to the last $n$ coordinates,
  \begin{equation}
    \frac{\partial g}{\partial x^{n + i}}(X^1, ..., X^n, Y^1, ..., Y^n) = 0
  \end{equation}
  Take the standard coordinate systems (given by projections)
  $\{\hat{x}^k\}_{k = 1}^n$ on $\R^n$ and
  $\{x^k\}_{k = 1}^{2 n}$ on $\R^{2 n}$.
  Then, by the definitions of differential and partial derivative on manifold,
  \begin{equation}
    (\restrict{d \pi}{(q, p)}
      \left(\restrict{\frac{\partial}{\partial q^i}}{(q, p)}\right)) f
    = \restrict{\frac{\partial}{\partial q^i}}{(q, p)}(f \circ \pi)
    = \frac{\partial(f \circ \pi \circ \varphi^{-1})}{x^i}(\varphi(q, p))
    = \frac{\partial(f \circ \hat{\varphi}^{-1})}{\hat{x}^i}(\hat{\varphi}(q))
    = \restrict{\frac{\partial}{\partial \hat{q}^i}}{q} f,
  \end{equation}
  from which it follows that the first equality holds.
  Similarly,
  \begin{equation}
    (\restrict{d \pi}{(q, p)}
      \left(\restrict{\frac{\partial}{\partial p^i}}{(q, p)}\right)) f
    = \restrict{\frac{\partial}{\partial p^i}}{(q, p)}(f \circ \pi)
    = \frac{\partial(f \circ \pi \circ \varphi^{-1})}{x^{i + n}}(\varphi(q, p))
    = 0,
  \end{equation}
  from which it follows that the second equality holds.
\end{proof}
\begin{proposition}[Tautological one-form in generalised coordinates]
  Let
    $Q$ be a smooth manifold of dimension $n$,
    $(U, \hat{\varphi})$ be a chart.
  Then
  \begin{equation}
    \restrict{\theta}{U} = \sum_{i = 1}^n p_i\, d q^i.
  \end{equation}
\end{proposition}
\begin{proof}
  Let $(q, p) \in T^* U$.
  Recall that $\restrict{\theta}{(q, p)} = p \circ \restrict{d \pi}{(q, p)}$.
  Hence,
  \begin{equation}
    \restrict{\theta}{(q, p)}
    \left(\restrict{\frac{\partial}{\partial q^i}}{(q, p)}\right)
    = p\left(\restrict{\frac{\partial}{\partial \hat{q}^i}}{q}\right)
    = p_i(q, p),
  \end{equation}
  and
  \begin{equation}
    \restrict{\theta}{(q, p)}
    \left(\restrict{\frac{\partial}{\partial p^i}}{(q, p)}\right)
    = p(0)
    = 0.
  \end{equation}
  Therefore,
  \begin{equation}
    \restrict{\theta}{(q, p)}
    = \sum_{i = 1}^n
      \restrict{\theta}{(q, p)}
      \left(\restrict{\frac{\partial}{\partial q^i}}{(q, p)}\right)\,
      \restrict{d q^i}{(q, p)}
    + \sum_{i = 1}^n
      \restrict{\theta}{(q, p)}
      \left(\restrict{\frac{\partial}{\partial p^i}}{(q, p)}\right)\,
      \restrict{d p^i}{(q, p)}
    = \sum_{i = 1}^n p_i(q, p)\, \restrict{d q^i}{(q, p)},
  \end{equation}
  from which the proposition follows.
\end{proof}
\begin{corollary}[Canonical symplectic in generalised coordinates]
  Let
    $Q$ be a smooth manifold of dimension $n$,
    $(U, \hat{\varphi})$ be a chart.
  Then
  \begin{equation}
    \restrict{\omega}{U} = \sum_{i = 1}^n d q^i \wedge d p_i.
  \end{equation}
\end{corollary}
\begin{proof}
  Let $i \in \{1, ..., n\}$.
  Then
  \begin{equation}
    - d(p_i\, d q^i) = - d p_i \wedge d q^i = d q^i \wedge d p_i.
  \end{equation}
  Summing up for all $i$, we get the desired result.
\end{proof}
\begin{definition}
  Let $(M, \omega)$ be a symplectic manifold, $f \in \mathcal{F} M$.
  We say that $X \in \mathfrak{X} M$ is a \textbf{Hamiltonian vector field} for
  $f$ if
  \begin{equation}
    i_X \omega + d_0 f = 0.
  \end{equation}
\end{definition}
\begin{proposition}
  Let $(M, \omega)$ be a symplectic manifold, $f \in \mathcal{F} M$.
  Then there exists a unique Hamiltonian vector field for $f$.
\end{proposition}
\begin{proof}
  The non-degeneracy of $\omega$ means that we can interpret the symplectic form
  as the isomorphism $\tilde{\omega} \colon \mathfrak{X} M \to \Omega^1 M$,
  given by
  \begin{equation}
    (\tilde{\omega} X) := i_X \omega,\ X \in \mathfrak{X} M.
  \end{equation}
  Hence, the problem at hand has a unique solution
  $X = \tilde{\omega}^{-1}(- d_0 f)$.
\end{proof}
\begin{definition}
  Let $(M, \omega)$ be a symplectic manifold.
  Define the map $\hamiltonian \colon \mathcal{F} M \to \mathfrak{X} M$ by
  \begin{equation}
    \hamiltonian = - \tilde{\omega}^{-1} \circ d_0.
  \end{equation}
  It maps a function to its corresponding Hamiltonian vector field.
  We will write $\hamiltonian_f$ instead of $\hamiltonian(f)$
  for $f \in \mathcal{F} M$.
\end{definition}
\begin{proposition}
  Let
    $Q$ be a smooth manifold of dimension $n$,
    $(U, \hat{\varphi})$ be a chart on $Q$,
    $f \in \mathcal{F}(T^* Q)$.
  Then
  \begin{equation}
    \hamiltonian_f
    = \sum_{i = 1}^n
    \left(
      - \frac{\partial f}{\partial p^i} \frac{\partial}{\partial q^i}
      + \frac{\partial f}{\partial q^i} \frac{\partial}{\partial p^i}
    \right).
  \end{equation}
\end{proposition}
\begin{proof}
  First, note that
  $i_{\frac{\partial}{\partial q^i}} \omega = d p^i$ and
  $i_{\frac{\partial}{\partial p^i}} \omega = - d q^i$.
  Denote
  \begin{equation}
    X
    := \sum_{i = 1}^n
    \left(
      - \frac{\partial f}{\partial p^i} \frac{\partial}{\partial q^i}
      + \frac{\partial f}{\partial q^i} \frac{\partial}{\partial p^i}
    \right).
  \end{equation}
  Then
  \begin{equation}
    i_X \omega
    = \sum_{i}^n
    \left(
      - \frac{\partial f}{\partial p^i} i_{\frac{\partial}{\partial q^i}} \omega
      + \frac{\partial f}{\partial q^i} i_{\frac{\partial}{\partial p^i}} \omega
    \right)
    = \sum_{i}^n
    \left(
      - \frac{\partial f}{\partial p^i} d p^i
      - \frac{\partial f}{\partial q^i} d q^i
    \right)
    = - d f.
  \end{equation}
  Hence, $\hamiltonian_f = X$.
\end{proof}
\begin{proposition}
  Let $(M, \omega)$ be a symplectic manifold, $f, g \in \mathcal{F} M$.
  Then
  \begin{equation}
    \hamiltonian_{f g} = f \hamiltonian_g + g \hamiltonian_f.
  \end{equation}
\end{proposition}
\begin{proof}
  Follows directly from the Leibniz rule for $d_0$.
\end{proof}
\begin{definition}
  Let $(M, \omega)$ be a symplectic manifold, $X \in \mathfrak{X} M$.
  We say that $X$ is a \textbf{symplectic vector field} if $L_X \omega = 0$.
\end{definition}
\begin{remark}
  Since $L_{\lie{X}{Y}} = \lie{L_X}{L_Y} = L_X \circ L_Y - L_Y \circ L_X$,
  the symplectic vector fields form a Lie subalgebra of the Lie algebra of
  vector fields.
\end{remark}
\begin{proposition}
  Let $(M, \omega)$ be a symplectic manifold, $f \in \mathcal{F} M$.
  Then $\hamiltonian_f$ is a symplectic vector field.
\end{proposition}
\begin{proof}
  $
    L_{\hamiltonian_f} \omega
    = i_{\hamiltonian_f}(d \omega) + d(i_{\hamiltonian_f} \omega)
    = i_{\hamiltonian_f} 0 - d(d f)
    = 0.
  $
\end{proof}
\begin{proposition}
  Let
    $(M, \omega)$ be a symplectic manifold,
    $X \in \mathfrak{X} M$ be a symplectic vector fields.
  Then
  \begin{equation}
    d(i_X \omega) = 0.
  \end{equation}
\end{proposition}
\begin{proof}
  $
    d(i_X \omega)
    = L_X \omega - i_X(d \omega)
    = 0 - 0
    = 0.
  $
\end{proof}
\begin{proposition}
  Let $M$ be a smooth manifold, $X, Y \in \mathfrak{X} M$.
  Then
  \begin{equation}
    L_X \circ i_Y = i_{\lie{X}{Y}} + i_Y \circ L_X.
  \end{equation}
\end{proposition}
\begin{proposition}
  Let
    $(M, \omega)$ be a symplectic manifold,
    $X, Y \in \mathfrak{X} M$ be symplectic vector fields.
  Then
  \begin{equation}
    \lie{X}{Y} = \hamiltonian_{i_Y(i_X \omega)}.
  \end{equation}
\end{proposition}
\begin{proof}
  \begin{equation}
    i_{\lie{X}{Y}} \omega
    = (L_X \circ i_Y - i_Y \circ L_X) \omega
    = (L_X \circ i_Y) \omega
    = ((d \circ i_X + i_X \circ d) \circ i_Y) \omega
    = d(i_X(i_Y \omega))
    = - d(i_Y(i_X \omega)).
  \end{equation}
  We get the desired result from the definition of $\hamiltonian$.
\end{proof}
\begin{definition}
  Let $(M, \omega)$ be a symplectic manifold.
  Define the \textbf{Poisson bracket}
  $\poisson{\cdot}{\cdot} \colon \mathcal{F} M \to \mathcal{F} M$ by
  \begin{equation}
    \poisson{f}{g}
    := i_{\hamiltonian_g}(i_{\hamiltonian_f} \omega),\
    f, g \in \mathcal{F} M.
  \end{equation}
\end{definition}
\begin{corollary}
  Let $(M, \omega)$ be a symplectic manifold, $f, g \in \mathcal{F} M$.
  Then
  \begin{equation}
    \lie{\hamiltonian_f}{\hamiltonian_g}
    = \hamiltonian_{i_{\hamiltonian_g}(i_{\hamiltonian_f} \omega)}
    = \poisson{f}{g}.
  \end{equation}
\end{corollary}
\begin{proposition}[Leibniz rule holds for the Poisson bracket]
  Let $(M, \omega)$ be a symplectic manifold, $f, g, h \in \mathcal{F} M$.
  Then
  \begin{equation}
    \poisson{f}{g h} = \poisson{f}{g} h + g \poisson{f}{h}.
  \end{equation}
\end{proposition}
\begin{proof}
  $
    \poisson{f}{g h}
    = i_{\hamiltonian_{g h}}(i_{\hamiltonian_f} \omega)
    = i_{h \hamiltonian_g + g \hamiltonian_{h}}(i_{\hamiltonian_f} \omega)
    = (i_{\hamiltonian_g}(i_{\hamiltonian_f} \omega))\, h
      + g\, (i_{\hamiltonian_h}(i_{\hamiltonian_f} \omega)) 
    = \poisson{f}{g} h + g \poisson{f}{h}.
  $
\end{proof}
\begin{proposition}
  Let $(M, \omega)$ be a symplectic manifold, $f, g, h \in \mathcal{F} M$.
  Then
  \begin{equation}
    \poisson{f}{g} = L_{\hamiltonian_f} g.
  \end{equation}
\end{proposition}
\begin{proof}
  $
    \poisson{f}{g}
    = i_{\hamiltonian_g}(i_{\hamiltonian_f} \omega)
    = - i_{\hamiltonian_f} \circ i_{\hamiltonian_g} \omega
    = i_{\hamiltonian_f}(d g)
    = L_{\hamiltonian_f} g.
  $
\end{proof}
\begin{definition}
  Let $(M, \omega)$ be a symplectic manifold.
  Define
  ${\rm ad} \colon \mathcal{F} M \to (\mathcal{F} M \to \mathcal{F} M)$ by
  \begin{equation}
    {\rm ad}_f g := \poisson{f}{g},\ f, g \in \mathcal{F} M,
  \end{equation}
\end{definition}
\begin{proposition}
  Let $(M, \omega)$ be a symplectic manifold.
  Then
  \begin{equation}
    \lie{{\rm ad}_f}{{\rm ad}_g} = {\rm ad}_{\poisson{f}{g}}.
  \end{equation}
  (Here the bracket $\lie{\cdot}{\cdot}$ is the commutator of operators.)
\end{proposition}
\begin{proof}
  From the previous proposition it follows that
  \begin{equation}
    {\rm ad}_f = L_{X_f},\ f \in \mathcal{F} M.
  \end{equation}
  Hence,
  \begin{equation}
    \lie{{\rm ad}_f}{{\rm ad}_g}
    = \lie{L_{\hamiltonian_f}}{L_{\hamiltonian_g}}
    = L_{\lie{\hamiltonian_f}{\hamiltonian_g}}
    = L_{\hamiltonian_{\poisson{f}{g}}}
    = {\rm ad}_{\poisson{f}{g}}.
  \end{equation}
\end{proof}
\begin{corollary}
  Let $(M, \omega)$ be a symplectic manifold.
  Then $(\mathcal{F} M, \poisson{\cdot}{\cdot})$ is a Lie algebra over $\R$.
\end{corollary}
\begin{proof}
  Bilinearity and antisymmetry are trivial to check.
  The Jacobi identity is equivalent to the adjoint map being a Lie algebra
  homomorphism, which was the previous proposition.
\end{proof}
\begin{definition}
  Let
    $R$ be a commutative ring with unity ring,
    $(A, +, \cdot)$ be an $R$-module
    with additional structures of
    an associative algebra $(A, *)$ and
    a Lie algebra $(A, \poisson{\cdot}{\cdot})$.
  We say that $A$ is a Poisson algebra if the Lie bracket acts as a derivation,
  i.e., for all $f, g, h \in A$,
  \begin{equation}
    \poisson{f}{g * h} = \poisson{f}{g} * h + g * \poisson{f}{h}.
  \end{equation}
\end{definition}
\begin{corollary}
  Let $(M, \omega)$ be a symplectic manifold.
  Then $\mathcal{F} M$ is a Poisson algebra over $\R$.
  Here, addition, scalar multiplication, and multiplication are given by the
  corresponding pointwise operations, while the Lie bracket is given by the
  Poisson bracket.
\end{corollary}
