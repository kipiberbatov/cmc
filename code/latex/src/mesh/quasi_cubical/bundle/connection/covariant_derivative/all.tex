\begin{definition}
  Let
    $\mathcal{K}$ be a quasi-cubical mesh,
    $E$ be a vector bundle on $\mathcal{K}$.
  A \textbf{connection derivative} on $E$ is a map
  \begin{equation}
    \widetilde{\mathcal{D}} \colon \Gamma E \to \Omega^1(M, E)
  \end{equation}
  that satisfies the product rule:
  for any $f \in C^0 \mathcal{K}$ and $\sigma \in \Gamma E$,
  \begin{equation}
    \widetilde{\mathcal{D}}(\sigma \otimes f)
    = (\widetilde{\mathcal{D}} \sigma) \usmile f + \sigma \otimes \delta_0 f.
  \end{equation}
\end{definition}
\begin{remark}
  There is one-to one correspondence between connection derivatives and
  connections.
  Indeed, to a connection derivative $\widetilde{\mathcal{D}}$ corresponds a
  connection $\nabla$ (and vice versa) as follows:
  for any $X \in \mathfrak{X} \mathcal{K}$ and $\sigma \in \Gamma E$,
  \begin{equation}
    (\widetilde{\mathcal{D}} \sigma)(X) = \nabla_X \sigma.
  \end{equation}
\end{remark}
\begin{definition}
  Let
    $\mathcal{K}$ be a quasi-cubical mesh,
    $D = \dim \mathcal{K}$,
    $E$ be a vector bundle on $\mathcal{K}$,
    $\widetilde{\mathcal{D}}$ be a connection on $E$, 
    $p \in \{0, ..., D\}$.
  Define a graded map $\widetilde{\mathcal{D}}$,
  \begin{equation}
    \mathcal{D}_p
    \colon \Omega^p(\mathcal{K}, E) \to \Omega^{p + 1}(\mathcal{K}, E)\
    (p = 0, ..., D - 1),
  \end{equation}
  as follows:
  for any $\sigma \in \Gamma E$ and $\omega \in \Omega^p(\mathcal{K})$,
  \begin{equation}
    \mathcal{D}_p(\sigma \otimes \omega)
    := (\widetilde{\mathcal{D}} \sigma) \usmile \omega
    + \sigma \otimes (\delta_p \omega),
  \end{equation}
  and extended by bilinearity for all tensors.
  $\mathcal{D}$ is called the \textbf{(discrete) covariant exterior derivative}
  induced by $\widetilde{\mathcal{D}}$.
  (Moreover, if $\widetilde{\mathcal{D}}$ is induced by some connection
  $\nabla$, we may say that $\mathcal{D}$ is induced by $\nabla$.)
\end{definition}
\begin{proposition}
  Let
    $\mathcal{K}$ be a quasi-cubical mesh,
    $E$ be a vector bundle on $\mathcal{K}$,
    $\widetilde{\mathcal{D}}$ be a connection derivative on $E$,
    $\mathcal{D}$ be its corresponding covariant exterior derivative,
    $p, q \in \N$,
    $\tau \in \Omega^p(\mathcal{K}, E)$,
    $\omega \in \Omega^q \mathcal{K}$.
  Then $\mathcal{D}$ satisfies the following version of the graded Leibniz rule:
  \begin{equation}
    \mathcal{D}_{p + q}(\tau \usmile \omega)
    = \mathcal{D}_p \tau \usmile \omega + (-1)^p \tau \usmile \delta_q \omega.
  \end{equation}
\end{proposition}
\begin{proof}
  It is enough to prove the proposition when $\tau$ is a simple tensor.
  So, let $\tau = \sigma \otimes \eta$
  for some $\sigma \in \Gamma E$ and $\eta \in \Omega^p \mathcal{K}$.
  Then
  \begin{equation}
    \begin{split}
      \mathcal{D}_{p + q}(\tau \usmile \omega)
      & = \mathcal{D}_{p + q}((\sigma \otimes \eta) \usmile \omega) \\
      & = \mathcal{D}_{p + q}(\sigma \otimes (\eta \smile \omega)) \\
      & = \widetilde{\mathcal{D}} \sigma \usmile (\eta \smile \omega)
          + \sigma \otimes \delta_{p + q}(\eta \smile \omega) \\
      & = (\widetilde{\mathcal{D}} \sigma \usmile \eta) \usmile \omega
          + \sigma \otimes
          (\delta_p \eta \smile \omega + (-1)^p \eta \smile \delta_q \omega) \\
      & = ( \widetilde{\mathcal{D}} \sigma \usmile \eta
            + \sigma \otimes \delta_p \eta)
          \usmile \omega
          + (-1)^p (\sigma \otimes \eta) \usmile \delta_q \omega \\
      & = \mathcal{D}_p(\sigma \otimes \eta) \usmile \omega
          + (-1)^p (\sigma \otimes \eta) \usmile \delta_q \omega \\
      & = \mathcal{D}_p \tau \usmile \omega
          + (-1)^p \tau \usmile \delta_q \omega. \qedhere
    \end{split}
  \end{equation}
\end{proof}
