\begin{proposition}
  Let $X$ and $Y$ be sets, $f \colon X \to Y$.
  Then
  \begin{equation}
    X \simeq \coprod_{y \in Y} f^{-1}[\{x\}].
  \end{equation}
\end{proposition}
\begin{definition}
  Let $\mathcal{K}$ be a quasi-cubical mesh.
  A \textbf{discrete vector bundle} on $\mathcal{K}$ is a triple
  ${\bf E} = (E, \pi, \mathcal{K})$,
  where $E$ is a set and $\pi \colon E \to \mathcal{K}_0$ is a projection
  operator having the following property:
  for any node $\mathcal{N} \in \mathcal{K}_0$,
  $\pi^{-1}[\{\mathcal{N}\}] \subseteq E$ is a vector space, called the
  \textbf{fibre} of $E$ at $\mathcal{N}$,
  and denoted by $\restrict{E}{\mathcal{N}}$.

  Often $\pi$ and $\mathcal{K}$ are implicitly known.
  In that case we will simply say that $E$ is a discrete vector bundle.
\end{definition}
\begin{example}
  Let $\mathcal{K}$ be a quasi-cubical mesh.
  For any $\mathcal{N} \in \mathcal{K}_0$, define by $T_\mathcal{N} \mathcal{K}$
  the free vector space generated by the adjacent edges to $\mathcal{N}$.
  Define the \textbf{discrete tangent bundle}
  \begin{equation}
    T \mathcal{K}
    := \coprod_{\mathcal{N} \in \mathcal{K}_0} T_\mathcal{N} \mathcal{K}.
  \end{equation}
  (The projection $\pi$ is the disjoint union projection.)
\end{example}
\begin{definition}
  Let $\mathcal{K}$ be a quasi-cubical mesh,
  ${\bf E} = (E, \pi, \mathcal{K})$ be a discrete vector bundle.
  Define the triple ${\bf E}^* := (E', \pi', \mathcal{K})$ as follows.
  \begin{equation}
    E' := \coprod_{\mathcal{N} \in \mathcal{K}_0} \restrict{E'}{\mathcal{N}}
  \end{equation}
  and $\pi'$ is the disjoint union projection.
  We call ${\bf E}^*$  (or simply $E^*$) the \textbf{dual bundle}.
\end{definition}
\begin{definition}
  Let $\mathcal{K}$ be a quasi-cubical mesh,
  ${\bf E} = (E, \pi_E, \mathcal{K})$ and ${\bf F} = (F, \pi_F, \mathcal{K})$
  be discrete vector bundles.
  Define the triple ${\bf E} \oplus {\bf F}:= (G, \pi_G, \mathcal{K})$
  as follows.
  \begin{equation}
    G :=
    \coprod_{\mathcal{N} \in \mathcal{K}_0}
      \restrict{E}{\mathcal{N}} \oplus \restrict{F}{\mathcal{N}}
  \end{equation}
  and $\pi_G$ is the disjoint union projection.
  We call ${\bf E} \oplus {\bf F}$ (or simply $E \oplus F$) the
  \textbf{direct sum bundle}.

  In a similar way, we define the \textbf{tensor product bundle}
  ${\bf E} \otimes {\bf F}$ (or simply $E \otimes F$)
  as the bundle whose fibres are
  $\restrict{E}{\mathcal{N}} \otimes \restrict{F}{\mathcal{N}}$
  for any node $\mathcal{N} \in \mathcal{K}_0$.

  Finally, we define the \textbf{Hom bundle}
  $\Hom({\bf E}, {\bf F})$ (or simply $\Hom(E, F)$)
  as the bundle whose fibres are
  $\Hom\left(\restrict{E}{\mathcal{N}}, \restrict{F}{\mathcal{N}}\right)$
  for any node $\mathcal{N} \in \mathcal{K}_0$.
\end{definition}
\begin{proposition}
  Let $\mathcal{K}$ be a quasi-cubical mesh, $E$, $F$, and $G$
  be discrete vector bundles over $\mathcal{K}$.
  The following canonical isomorphisms are straightforward generalisations of
  their vector space counterparts:
  \begin{subequations}
    \begin{alignat}{2}
      & (E \oplus F)^*          && \simeq E^* \oplus F^*, \\
      & (E \otimes F)^*         && \simeq E^* \otimes F^*, \\
      & E \oplus (F \oplus G)   && \simeq (E \oplus F) \oplus G, \\
      & E \oplus F              && \simeq F \oplus E, \\
      & E \otimes (F \otimes G) && \simeq (E \otimes F) \otimes G, \\
      & E \otimes F             && \simeq F \otimes E, \\
      & E \otimes (F \oplus G)  && \simeq E \otimes F \oplus E \otimes G, \\
      & (E \oplus F) \otimes G  && \simeq E \otimes G \oplus F \otimes G.
    \end{alignat}
  \end{subequations}
\end{proposition}
