\begin{proposition}
  Let
    $\mathcal{K}$ be a quasi-cubical mesh,
    $(E, \pi, \mathcal{K})$ be a discrete vector bundle over $\mathcal{K}$.
  A \textbf{section} on $E$ is a function
  $\mathcal{X} \colon \mathcal{K}_0 \to E$
  such that
  $\pi \circ \mathcal{X} = \id_{\mathcal{K}_{0}}$.

  The space of all sections on $E$ is denoted by $\Gamma E$.
\end{proposition}
\begin{remark}
  Let
    $\mathcal{K}$ be a quasi-cubical mesh,
    $E$ be a vector bundle over $\mathcal{K}$.
  We can think of a bundle as assignment of a vector space
  $\restrict{E}{\mathcal{N}}$ at each node $\mathcal{N}$ of $\mathcal{K}$.
  A section on $E$ is then an assignment of a single vector
  $\restrict{e}{\mathcal{N}} \in\restrict{E}{\mathcal{N}}$
  at each node $\mathcal{N}$ of $\mathcal{K}$.
\end{remark}
\begin{example}
  Let $\mathcal{K}$ be a quasi-cubical mesh.
  The \textbf{space of discrete vector fields} on $\mathcal{K}$ is
  $\Gamma(T \mathcal{K})$ -- the space of sections on the discrete tangent
  bundle of $\mathcal{K}$.
\end{example}
\begin{definition}
  Let
    $\mathcal{K}$ be a quasi-cubical mesh,
    $E$ be a vector bundle over $\mathcal{K}$.
  Define addition on $\Gamma E$ in a component-wise manner.
  Precisely, if $X, Y \in \Gamma E$, then
  \begin{equation}
    X + Y \in \Gamma E,\ \forall \mathcal{N} \in \mathcal{K}_0,
    \restrict{(X + Y)}{\mathcal{N}}
    := \restrict{X}{\mathcal{N}} + \restrict{Y}{\mathcal{N}}.
  \end{equation}
\end{definition}
\begin{proposition}
  Let
    $\mathcal{K}$ be a quasi-cubical mesh,
    $E$ be a vector bundle over $\mathcal{K}$.
  Then $\Gamma E$ is an abelian group under the addition operation defined
  above.
  The zero element is the section giving zero at any point, while negation is
  defined component-wise in the obvious way.
\end{proposition}
\begin{definition}
  Let
    $\mathcal{K}$ be a quasi-cubical mesh,
    $E$ be a vector bundle over $\mathcal{K}$,
    $X \in \Gamma E$,
    $f \in C^0 \mathcal{K}$.
  We define $f\, X \in \Gamma E$ as follows:
  for any $\mathcal{N} \in \mathcal{K}_0$,
  \begin{equation}
    \restrict{(f\, X)}{\mathcal{N}}
    := f(\mathcal{N}_\bullet)\, \restrict{X}{\mathcal{N}}.
  \end{equation}
\end{definition}
\begin{proposition}
  Let
    $\mathcal{K}$ be a quasi-cubical mesh,
    $E$ be a vector bundle over $\mathcal{K}$.
  The space $\Gamma E$ is an $(C^0 \mathcal{K})$-module under the multiplication
  operation defined above.
\end{proposition}
\begin{proposition}
  Let
   $\mathcal{K}$ be a quasi-cubical mesh,
    $E$ and $F$ be vector bundles over $\mathcal{K}$.
  The following equalities show functoriality of the $\Gamma$ operator,
  sending a bundle over $\mathcal{K}$ to a module over $C^0 \mathcal{K}$.
  (Note that operations on the left are on discrete vector bundles,
  while those on the right are on $(C^0 \mathcal{K})$-modules.)
  \begin{subequations}
    \begin{alignat}{2}
      & \Gamma(E^*)
      && \simeq (\Gamma E)^*, \\
      %
      & \Gamma(E \oplus F)
      && \simeq \Gamma E \oplus \Gamma F, \\
      %
      & \Gamma(E \otimes F)
      && \simeq \Gamma E \otimes \Gamma F, \\
      %
      & \Gamma(\Hom(E, F))
      && \simeq \Hom(\Gamma E, \Gamma F).
    \end{alignat}
  \end{subequations}
\end{proposition}
