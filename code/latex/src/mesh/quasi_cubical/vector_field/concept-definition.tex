\begin{definition}
  Let $\mathcal{K}$ be a quasi-cubical mesh.
  We say that $X \in \Hom(C_0 \mathcal{K}, C_1 \mathcal{K})$
  is a \textbf{combinatorial vector field} if for any
  $\mathcal{N} \in \mathcal{K}_0$ and $\mathcal{E} \in \mathcal{K}_1$,
  \begin{equation}
    \mathcal{N} \notin \partial \mathcal{E} \Rightarrow
    \mathcal{E}^\bullet(X \mathcal{N}_\bullet) = 0
  \end{equation}
  (here $a_\bullet$ and $a^\bullet$ denote the corresponding basis chains and
  cochains to a cell $a$).
  In other words, a combinatorial vector field assigns a basis $0$-chain
  $\mathcal{N}_\bullet$ a $1$-chain whose coefficients are zero on the edges
  that do not contain $\mathcal{N}$.
  We will write its coefficients by
  \begin{equation}
    X^{\mathcal{E}}_{\mathcal{N}} := \mathcal{E}^\bullet(X \mathcal{N}_\bullet).
  \end{equation}

  The space of all combinatorial vector fields will be denoted by
  $\mathfrak{X} K$.
\end{definition}
