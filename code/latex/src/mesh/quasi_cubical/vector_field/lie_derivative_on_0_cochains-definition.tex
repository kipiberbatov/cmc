\begin{definition}
  Let $\mathcal{K}$ be a quasi-cubical mesh.
  The \textbf{Lie derivative on $0$-cochains} is defined as
  \begin{equation}
    L \colon \mathfrak{X} K \to \Hom(C^0 \mathcal{K}, C^0 \mathcal{K}),
    L_{\mathcal{X}} := i_{\mathcal{X}} \circ \delta,\
    \mathcal{X} \in \mathfrak{X} K.
  \end{equation}
  The above expression applied to a function $f \in C^0 \mathcal{K}$ equals to
  \begin{equation}
    L_{\mathcal{X}} f
    = i_{\mathcal{X}} (\delta f)
    = (\delta_0 f) \circ \mathcal{X}
    = f \circ \partial_1 \circ X.
  \end{equation}
\end{definition}
\begin{remark}
  No we are going to analyse to what extent the Leibniz rule is reproduced in
  the discrete setting.
  Namely, for
  $\mathcal{X} \in \mathfrak{X} K$, $f, g \in \mathcal{F} K$
  we are estimating
  \begin{equation}
    \Delta_\mathcal{X}(f, g) :=
    (L_{\mathcal{X}} f) \smile g
    + f \smile (L_{\mathcal{X}} g)
    - L_{\mathcal{X}} (f \smile g).
  \end{equation}
  Let $\mathcal{N} \in \mathcal{K}_0$.
  Then
  \begin{equation}
    \begin{split}
        \Delta_\mathcal{X}(f, g) \mathcal{N}_\bullet
      & =
        ((L_{\mathcal{X}} f) \smile g + f \smile (L_{\mathcal{X}} g)
        - L_{\mathcal{X}} (f \smile g)) \mathcal{N}_\bullet \\
      & = f (\partial(X \mathcal{N}_\bullet))\, g \mathcal{N}_\bullet
        + f \mathcal{N}_\bullet\, g (\partial(X \mathcal{N}_\bullet))
        - (f \smile g)(\partial(X \mathcal{N}_\bullet)) \\
      & = \sum_{\mathcal{E} \succ \mathcal{N}}
        \mathcal{X}_\mathcal{N}^\mathcal{E}
        ( (\sum_{\mathcal{N}' \prec \mathcal{E}}
          \varepsilon(\mathcal{E}, \mathcal{N}') f_{\mathcal{N}'})
          g_{\mathcal{N}}
          + f_{\mathcal{N}} (\sum_{\mathcal{N}' \prec \mathcal{E}}
          \varepsilon(\mathcal{E}, \mathcal{N}') g_{\mathcal{N}'})
          - \sum_{\mathcal{N}' \prec \mathcal{E}}
          \varepsilon(\mathcal{E}, \mathcal{N}')
          f_{\mathcal{N}'} g_{\mathcal{N}'}
        ) \\
      & = \sum_{\mathcal{E} \succ \mathcal{N}}
        \mathcal{X}_\mathcal{N}^\mathcal{E}\,
        \varepsilon(\mathcal{E}, \mathcal{N})\,
        \, (\delta_0 f)(\mathcal{E})\,
        (\delta_0 g)(\mathcal{E}).
    \end{split}
  \end{equation}
\end{remark}
