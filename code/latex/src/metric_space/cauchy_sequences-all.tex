\begin{discussion}
  We already proved that any converging sequence is a Cauchy sequence.
  The converse, i.e., that any Cauchy sequence has a limit,
  is not true in general.
  (This motivated the introduction of complete metric spaces.)
  For instance, there are Cauchy sequences of rational numbers that
  do not converge to a rational number.
  For example, consider the sequence $a \colon \N \to \Q$,
  \begin{subequations}
    \begin{alignat}{1}
      & a_0 := 1, \\
      & a_{n + 1} := \frac{1}{a_n} + \frac{a_n}{2},\ n \in \N.
    \end{alignat}
  \end{subequations}
  Let $d \colon \Q \times \Q \to \Q$ be the standard metric given by
  \begin{equation}
    d(x, y) := \abs{x - y},\ x, y \in \Q.
  \end{equation}
  (We define the codomain to be $\Q$ since we assume $\R$ is not constructed
  yet.)
  It is not hard to check that $a \in \Cauchy(\Q, d)$.
  However, assume it has a limit $A \in \Q$.
  Then, it will satisfy the relation
  \begin{equation}
    A = \frac{1}{A} + \frac{A}{2}
    \Rightarrow A = \frac{2 + A^2}{2 A}
    \Rightarrow 2 A^2 = 2 + A^2
    \Rightarrow A^2 = 2.
  \end{equation}
  But there are is no rational whose square is $2$.
  To bee able to find the square root of $2$, we need to \emph{complete} the
  space of rationals.
\end{discussion}
\begin{proposition}
  Let $(X, d)$ be a metric space $Y \subseteq X$.
  Then $(Y, \restrict{d}{Y})$ is also a metric space.
\end{proposition}
\begin{proof}
  All the requirements for a function to be a metric have only universal
  quantifiers.
  Hence, they translate to subsets as well.
\end{proof}
\begin{definition}
  Let $(X, d)$ and $(Y, \delta)$ be metric spaces.
  A function $f \colon X \to Y$ is called an
  \textbf{isomorphism of metric spaces}
  if it is bijective and for any $x, y \in X$,
  \begin{equation}
    \delta(f x, f y) = d(x, y).
  \end{equation}
\end{definition}
\begin{definition}
  Let $(X, d)$ be a metric space, $Y \subseteq X$.
  We say that $Y$ is a \textbf{dense subset of $X$} if
  for any $x \in X$ there exists a Cauchy sequence in $(Y, \restrict{d}{Y})$
  that converges (as a sequence in $X$) to $x$.
\end{definition}
\begin{definition}
  Let $(X, d)$ be a metric space, $a$ and $b$ be sequences in $X$.
  Define the sequence $\hat{d}(a, b) \colon \N \to \R$ by
  \begin{equation}
    \hat{d}(a, b) := \{d(a_n, b_n)\}_{n = 0}^\infty.
  \end{equation}
\end{definition}
\begin{proposition}
  Let $(X, d)$ be a metric space, $x, y, z \in X$.
  Then
  \begin{equation}
    \abs{\hat{d}(x, y) - \hat{d}(y, z)} \leq \hat{d}(x, z).
  \end{equation}
\end{proposition}
\begin{proof}
  From triangle inequality and symmetry it follows that
  $\hat{d}(x, y) - \hat{d}(y, z) \leq \hat{d}(x, z)$ and
  $\hat{d}(y, z) - \hat{d}(x, y) \leq \hat{d}(x, z)$.
  This gives the desired inequality.
\end{proof}
\begin{notation}
  We will denote $\Cauchy(\R) := \Cauchy(\R, d_\R)$, where
  $d_\R$ is the standard metric on $\R$,
  $d_\R(x, y) := \abs{x - y}$ for $x, y \in \R$.
\end{notation}
\begin{proposition}
  Let $(X, d)$ be a metric space, $a, b \in \Cauchy(X, d)$.
  Then $\hat{d}(a, b) \in \Cauchy(\R)$.
  (And hence, converging since $\R$ is complete.)
\end{proposition}
\begin{proof}
  Denote $x := \hat{d}(a, b) \colon \N \to \R$.
  Let $m, n \in \N$.
  Then
  \begin{equation}
    \begin{split}
      \abs{x_m - x_n}
      & = \abs{d(a_m, b_m) - d(a_n, b_n)}
        = \abs{d(a_m, b_m) - d(a_n, b_m) + d(a_n, b_m) - d(a_n, b_n)} \\
      & \leq \abs{d(a_m, b_m) - d(a_n, b_m)} + \abs{d(a_n, b_m) - d(a_n, b_n)}
        \leq d(a_m, a_n) + d(b_m, b_n).
    \end{split}
  \end{equation}
  Now let $\varepsilon > 0$.
  Choose $N_a \in \N$ such that
  for all $m, n > N_a$, $d(a_m, a_n) < \varepsilon / 2$,
  and $N_b \in \N$ such that
  for all $m, n > N_b$, $d(b_m, b_n) < \varepsilon / 2$.
  Define $N := \max(N_A, N_B)$.
  Then for any $m, n > N$,
  \begin{equation}
    \abs{x_m - x_n}
    \leq d(a_m, a_n) + d(b_m, b_n)
    \leq \varepsilon / 2 + \varepsilon / 2
    = \varepsilon,
  \end{equation}
  which means that $x \in \Cauchy(\R)$.
\end{proof}
\begin{corollary}
  Let $(X, d)$ be a metric space, $a$ and $b$ be convergent sequences.
  Then
  \begin{equation}
    \lim_{n \to \infty} d(a_n, b_n)
    = d\left(\lim_{n \to \infty} a_n, \lim_{n \to \infty} b_n\right).
  \end{equation}
\end{corollary}
\begin{remark}
  The above equality can be stated as
  \begin{equation}
    \lim_\R(\hat{d}(a, b)) = d(\lim_{(X, d)} a, \lim_{(X, d)} b)
  \end{equation}
  which can be stated as the commutative diagram relation
  \begin{equation}
    \lim_\R \circ \hat{d} = d \circ (\lim_{(X, d)} \times \lim_{(X, d)}).
  \end{equation}
\end{remark}
\begin{definition}
  Let $(X, d)$ be a metric space.
  Define $\delta \colon \Cauchy(X, d) \times \Cauchy(X, d) \to \R$ as follows:
  for any $a, b \in \Cauchy(X, d)$,
  \begin{equation}
    \delta(a, b) := \lim_{n \to \infty} d(a_n, b_n).
  \end{equation}
\end{definition}
\begin{proposition}
  Let
    $(X, d)$ be a metric space,
    $Y := \Cauchy(X, d)$,
    $\delta \colon Y \times Y \to \R$ be defined as above.
  Then $(Y, \delta)$ is a \textbf{pseudometric space}.
  (A metric space without non-degeneracy, but with requiring zero distance for
  equal points.)
\end{proposition}
\begin{proof}
  Let $a, b, c \in \Cauchy(X, d)$.
  \begin{enumerate}
    \item
      \textbf{Equal points have zero distance.}
      Since for any $n \in \N$, $d(a_n, a_n) = 0$, $\hat{d}(a, b)$ is the zero
      sequence, whose limit is zero, i.e.,
      $\delta(a, a) = \lim \hat{d}(a, a) = 0$.
    \item
      \textbf{Non-negativity.}
      Since for any $n \in \N$, $d(a_n, b_n) \geq 0$,
      then the non-negativity is preserved in the limit case, i.e.,
      $\delta(a, b) = \lim \hat{d}(a, b) \geq 0$.
    \item
      \textbf{Symmetry.}
      Since for any $n \in \N$, $d(a_n, b_n) = d(b_n, a_n)$,
      then the symmetry is preserved in the limit case, i.e.,
      $\delta(a, b) = \lim \hat{d}(a, b) = \lim \hat{d}(b, a) = \delta(b, a)$.
    \item
      \textbf{Triangle inequality.}
      Since for any $n \in \N$, $d(a_n, c_n) \leq d(a_n, b_n) + d(b_n, c_n)$,
      then the triangle inequality is preserved in the limit case, i.e.,
      $\delta(a, c) \leq \delta(a, b) + \delta(b, c)$.
  \end{enumerate}
\end{proof}
\begin{remark}
  Note that positive definiteness of the metric is not preserved for Cauchy
  sequences.
  Indeed, two sequences which coincide after some point but have different
  initial values.
  Then they agree in the limiting case and so their distance is zero,
  but they are unequal.

  Another interesting case occurs when the two sequences differ entirely but
  converge to one another in the limiting case.
  For intance, the constant zero sequence and the sequence
  $\{1 / (n + 1)\}_{n = 0}^\infty$.
\end{remark}
\begin{proposition}
  Let
    $(X, d)$ be a metric space,
    $\delta$ be the pseudometric on $\Cauchy(X, d)$.
  Define the relation $\sim$ on $\Cauchy(X, d)$ by
  \begin{equation}
    a \sim b \Leftrightarrow \delta(a, b) = 0,\
    a, b \in \Cauchy(X, d).
  \end{equation}
  Then $\sim$ is an equivalence relation on $\Cauchy(X, d)$.
\end{proposition}
\begin{proof}
  Reflexivity and symmetry are obvious, so let us consider transitivity.
  Consider $a, b, c \in \Cauchy(X, d)$ and let
  $a \sim b$ and $b \sim c$.
  Then for any $n \in \N$,
  \begin{equation}
    d(a_n, c_n) \leq d(a_n, b_n) + d(b_n, c_n).
  \end{equation}
  Let $\varepsilon > 0$.
  Choose
  $N_A \in \N$ such that for all $n > N_A$, $d(a_n, b_n) < \varepsilon / 2$, and
  $N_B \in \N$ such that for all $n > N_B$, $d(b_n, c_n) < \varepsilon / 2$.
  Let $N := \max(N_a, N_B)$.
  Then for any $n > N$,
  \begin{equation}
    0
    \leq d(a_n, c_n)
    \leq d(a_n, b_n) + d(b_n, c_n)
    < \varepsilon / 2 + \varepsilon / 2
    = \varepsilon.
  \end{equation}
  Hence, $\lim_{n \to \infty} d(a_n, c_n) = 0$, i.e. $a \sim c$.
  Therefore, $\sim$ is an equivalence relation.
\end{proof}
\begin{proposition}
  Let $(X, d)$ be a metric space, $a, a', b, b' \in \Cauchy(X, d)$.
  Assume that $a \sim a'$ and $b \sim b'$.
  Then $\delta(a, b) = \delta(a', b')$.
\end{proposition}
\begin{definition}
  Let $(X, d)$ be a metric space.
  Define the $\rho$ on $\Cauchy(X, d) / \sim$ by
  \begin{equation}
    \rho([a]_\sim, [b]_\sim) := \delta(a, b).
  \end{equation}
  (Correctness of the definition is follows from the previous proposition.)
\end{definition}
\begin{proposition}
  Let $(X, d)$ be a metric space.
  Then $(\Cauchy(X, d) / \sim, \rho)$ is a metric space.
\end{proposition}
\begin{proposition}
  Let $(X, d)$ be a metric space.
  Then the metric $\rho$ on $\Cauchy(X, d) / \sim$ is complete.
\end{proposition}
\begin{proposition}
  Let $(X, d)$ be a metric space.
  Consider the map
  $f \colon X \to \Cauchy(X, d) / \sim$ defined by embedding constants
  as constant sequences:
  \begin{equation}
    f(x) = [(x, x, ..., x, ...)]_\sim.
  \end{equation}
  Then $f$ is an embedding and its image is a dense subset of
  $\Cauchy(X, d) / \sim$.
  Moreover, if $X$ is complete, then $f$ is an isomorphism.
\end{proposition}
\begin{remark}
  All of what we did in this section has the following interpretation.
  The space $(Y, \rho)$ of equivalence calsees of Cauchy sequences on $(X, d)$
  is a complete metric space that has $(X, d)$
  isomorphic to a dense subset of $X$ with the induced metric.
  For this reason $Y$ is called the \textbf{completion of $X$}.
  Moreover, if $(X, d)$ is already complete, then $(Y, \rho)$ is isomorphic to
  $(X, d)$.
  In other words, the completion of a complete metric space is essentially the
  same space, as one could expect.
\end{remark}
\begin{remark}
  In constructive setting one often works with the so called
  \textbf{rapidly converging sequences} and \textbf{rapid Cauchy sequences},
  thus removing the dependence on a real argument in the respective definitions
  of a converging sequence and a Cauchy sequence.
\end{remark}
\begin{definition}
  Let $(X, d)$ be a metric space, $a \colon \N \to X$.
  We say that $a$ is a \textbf{rapidly converging sequence} if
  there exists $A \in X$ such that for any $k \in \N$ there exists $N \in N$
  such that for all $m, n > N$,
  \begin{equation}
    \exists A \in X,\
      \forall k \in \N,\
        \exists N \in \N,\
          \forall n \in \N,\
             n > N \Rightarrow d(a_n, A) < 2^{-k}.
  \end{equation}
\end{definition}
\begin{definition}
  Let $(X, d)$ be a metric space, $a \colon \N \to X$.
  We say that $a$ is a \textbf{rapid Cauchy sequence} if
  for any $k \in \N$ there exists $N \in N$ such that for all $m, n > N$,
  \begin{equation}
    \forall k \in \N,\
      \exists N \in \N,\
        \forall m, n \in \N,\
           m, n > N \Rightarrow d(a_m, a_n) < 2^{-k}.
  \end{equation}
\end{definition}
