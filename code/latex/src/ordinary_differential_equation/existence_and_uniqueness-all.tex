\begin{definition}
  Let $x \in \R$, $I \subseteq \R$.
  We say that $I$ is \textbf{an interval around $x$} if $I$ is a connected set
  containing an open neighbourhood of $x$.
\end{definition}
\begin{definition}
  Let
    $t_0 \in \R$,
    $n \in \N^+$,
    $x_0 \in \R^n$,
    $D \subseteq \R \times \R^n$ be an open neighbourhood of $(t_0, x_0)$,
    $F \colon D \to \R^n$ be continuous.
  The \textbf{initial value-problem for the system of ordinary differential
  equations for right-hand side $F$} is the following problem: find
    an interval $I$ around $t_0$,
    a neighbourhood $C \subseteq \R^n$ of $x_0$,
    and a curve $x \colon I \to C$,
  such that for any $t \in I$, $(t, x(t)) \in D$ and
  \begin{subequations}
    \label{equation:ordinary_differential_equation/initial_value_problem}
    \begin{alignat}{1}
      & \dot{x}(t) = F(t, x(t)),\ \forall t \in I, \\
      & x(t_0) = x_0.
    \end{alignat}
  \end{subequations}
\end{definition}
\begin{discussion}
  Under the assumptions of the above definition, assume that $x$ is a solution
  and let $t \in I$.
  Integrating over the segment $[t_0, t]$ the differential equation, we get
  \begin{equation}
    \int_{t_0}^t F(s, x(s))\, d s
    = \int_{t_0}^s \dot{x}(s)\, d s
    = x(t) - x(t_0)
    = x(t) - x_0,
  \end{equation}
  which can be restated as the system of integral equations
  \begin{equation}
    \label{equation:ordinary_differential_equation/initial_value_problem/integral}
    x(t) = x_0 + \int_{t_0}^t F(s, x(s)).
  \end{equation}
  Conversely, if $x$ solves
  \Cref{equation:ordinary_differential_equation/initial_value_problem/integral},
  then $x(t_0) = x_0$.
  Moreover, if $x$ is assumed to be differentiable, then differentiating it we
  get $\dot{x}(t) = F(t, x(t))$.

  In general the integral system
  \Cref{equation:ordinary_differential_equation/initial_value_problem/integral}
  is more general than the differential system
  \Cref{equation:ordinary_differential_equation/initial_value_problem}
  and may include non-smooth solutions.
  However they are equivalent under smoothness assumptions which will mostly
  interest us.
  Nevertheless, the analysis is still easier to perform on the integral one to
  which we will stick.
  We will show that under mild assumptions the integral formulation gives rise
  locally to a contraction operator to which the Banach fixed point theorem
  applies.
\end{discussion}
\begin{definition}
  Let
    $(X, d_X)$ and $(Y, d_Y)$ be metric spaces,
    $f \colon X \to Y$,
    $K \in \R$.
  We say that $f$ is \textbf{Lipschitz continuous with Lipschitz constant $L$}
  if for any $x_1, x_2 \in X$,
  \begin{equation}
    d_Y(f x_1, x_2) \leq L d_X(x_1, x_2).
  \end{equation}
  If the constant $L$ merely exists (but is not being fixed) we say that
  $f$ is Lipschitz continuous.
\end{definition}
\begin{definition}
  Let $n \in \N^+$, $D \subseteq \R \times \R^n$ be open, $S \subset D$.
  We say that $S$ is a \textbf{spatial surface} if there exists $t \in \R$ such
  that for any $x \in S$, the temporal (first) component of $S$ is $t$.
\end{definition}
\begin{definition}
  Let
    $n \in \N^+$,
    $D \subseteq \R \times \R^n$ be an open set,
    $F \colon D \to \R^n$ be continuous,
    $L \in \R^+$.
  We say that $F$ is \textbf{Lipschitz continuous with respect to the spatial
  arguments with Lipschitz constant $L$} if for any spatial surface
  $S \subset D$, $\restrict{F}{S}$ is Lipschitz continuous with Lipschitz
  constant $L$.

  If the constant $L$ merely exists (but is not being fixed) we say that
  $F$ is Lipschitz continuous with respect to the spatial arguments.
\end{definition}
\begin{notation}
  Let $n \in \N^+$, $x_0 \in \R^n$, $\varepsilon \in \R^+$.
  By $\mathcal{C}(x_0, \varepsilon)$ we will denote the closed ball centered at
  $x_0$ with
  radius $\varepsilon$, i.e.,
  \begin{equation}
    \mathcal{C}(x_0, \varepsilon)
    := \set{x \in \R^n}{\norm{x - x_0} \leq \varepsilon}.
  \end{equation}
\end{notation}
\begin{proposition}
  Let
    $t_0 \in \R$,
    $n \in \N^+$,
    $x_0 \in \R^n$,
    $D \subseteq \R \times \R^n$ be an open neighbourhood of $(t_0, x_0)$,
    $F \colon D \to \R^n$ be continuous and Lipschitz continuous with respect to
      the spatial arguments.
  Then there exist $\tau, h \in \R^+$ such that:
  \begin{itemize}
    \item
      for any
      $x \in C^0(\mathcal{C}(t_0, \tau), \mathcal{C}(x_0, h))$ and
      $t \in \mathcal{C}(t_0, \tau)$,
      \begin{equation}
        x_0 + \int_{t_0}^t F(s, x(s))\, d s \in \mathcal{C}(x_0, h),
      \end{equation}
      or equivalently,
      \begin{equation}
        \norm{\int_{t_0}^t F(s, x(s))\, d s} \leq h;
      \end{equation}
    \item
      the resulting operator
      \begin{equation}
        \label{equation:ordinary_differential_equation/initial_value_problem/operator}
        \begin{split}
          & T
            \colon C^0(\mathcal{C}(t_0, \tau), \mathcal{C}(x_0, h))
            \to C^0(\mathcal{C}(t_0, \tau), \mathcal{C}(x_0, h)), \\
          & (T x)(t) := x_0 + \int_{t_0}^t F(s, x(s))\, d s,
        \end{split}
      \end{equation}
      is a contraction.
  \end{itemize}
\end{proposition}
\begin{proof}
  For a compact (bounded and closed) subset $Y$ of $D$ denote
  \begin{equation}
    M_Y := \norm{F}_{Y, \infty} = \max_{y \in Y}(\abs{f(y)})
  \end{equation}
  (existing since $F$ is continuous, hence attaing maxima on compact subsets).
  Clearly if $Z \subseteq Y$ is a compact, then $M_Z \leq M_Y$.
  Let $L \in \R^+$ be a (spatial) Lipschitz constant for $f$.

  Let $q \in [0, 1)$ be arbitrary.
  ($q$ will be our final Lipschitz constant that will control how small we will
  choose our closed balls around $t_0$ and $x_0$.)
  Take any $\tilde{\tau} = q / L$ and choose $h \in \R^+$ sufficiently small so
  that
  \begin{equation}
    \mathcal{C}(t_0, \tau) \times \mathcal{C}(x_0, h) \subset D.
  \end{equation}
  Define $\tau := \min(\tilde{\tau}, q h / M_{\mathcal{C}(x_0, h)})$.
  We claim that $(\tau, h)$ satisfy the requirements of the proposition.
  Let
    $x \in C^0(\mathcal{C}(t_0, \tau), \mathcal{C}(x_0, h))$ and
    $t \in \mathcal{C}(t_0, \tau)$.
  \begin{itemize}
    \item
      \textbf{The integral operator $T$ is well defined.}
      \begin{equation}
        \begin{split}
          \norm{\int_{t_0}^t F(s, x(s))\, d s}
          \leq \int_{t_0}^{t} \norm{F(s, x(s))}\, d s
          \leq M_{\mathcal{C}(x_0, h)}) \abs{t - t_0}
          \leq M_{\mathcal{C}(x_0, h)}) \tau
          \leq q h
          < h.
        \end{split}
      \end{equation}
    \item
      \textbf{$T$ is a contraction with Lipschitz constant $q$.}
      Let $x, y \in C^0(\mathcal{C}(t_0, \tau), \mathcal{C}(x_0, h))$.
      Choose $t \in \mathcal{C}(t_0, \tau)$ such that
      \begin{equation}
        \norm{T x - T y}_{C^0(\mathcal{C}(t_0, \tau), \mathcal{C}(x_0, h))}
        = \norm{(T x - T y)(t)}_{\R^n}.
      \end{equation}
      Then
      \begin{equation}
        \begin{split}
          \norm{T x - T y}_\infty
          & = \norm{(T x - T y)(t)}_U \\
          & = \norm{\int_{t_0}^t (F(s, x(s)) - F(s, y(s)))\, d s}_U \\
          & \leq \int_{t_0}^t \norm{(F(s, x(s)) - F(s, y(s)))}_U\, d s \\
          & \leq L \int_{t_0}^t \norm{x(s) - y(s)}_U\, d s \\
          & \leq L \int_{t_0}^t \norm{x - y}_\infty\, d s \\
          & = L \abs{t - t_0} \norm{x - y}_\infty \\
          & \leq L \tau \norm{x - y}_\infty \\
          & \leq q \norm{x - y}_\infty.
        \end{split}
      \end{equation}
  \end{itemize}
  Hence, $T$ is a contraction with Lipschitz constant $q$.
\end{proof}
\begin{definition}
  Let
    $I$ be an open interval,
    $n \in \N^+$,
    $U$ be an open subset of $\R \times \R^n$.
    $V$ be a vector space,
    $k \in \N^+$,
    $m \geq k - 1$,
    $u \in C^m(U \to \R^n)$ (usually a flow velocity),
    $\partial_t \colon C^k(U, V) \to C^{k - 1}(U, V)$
      be the temporal partial derivative,
    $d_x \colon C^k(U, V) \to C^{k - 1}(U, V \times (\R^n)^*)$
      be the spatial differential (the differential when time is fixed),
    $i_u \colon C^{k - 1}(U, V \times (\R^n)^*) \to C^{k - 1}(U, V)$
      be the contraction with the flow field.
  The material derivative on $V$-valued functions is defined by
  \begin{equation}
    \frac{D}{D t} \colon C^k(U, V) \to C^{k - 1}(U, V),\
    \frac{D}{D t} := \partial_t + i_u \circ d_x.
  \end{equation}
\end{definition}
\begin{proposition}
  Let
    $I$ be an open interval,
    $n \in \N^+$,
    $A$ be an open subset of $\R^n$
    $c \colon I \to A$,
    $V$ be a vector space,
    $B$ be an open subset of $V$,
    $f \colon A \to B$.
  Then
  \begin{equation}
    \frac{d(f \circ c)}{d t} = i_{\dot{c}}(d f \circ c) \colon I \to \R.
  \end{equation}
\end{proposition}
\begin{proposition}
  Let
    $I$ be an interval,
    $n \in \N^+$,
    $U$ be an open subset of $\R \times \R^n$.
    $V$ be a vector space,
    $k \in \N^+$,
    $u \in C^{k - 1}(U, \R^n)$ (usually a flow velocity),
    $x \colon I \to \R^n$ be an integral curve of $u$,
    $\tilde{x} := \id_I \odot x \colon I \to I \times \R^n$,
    $g \in C^k(I, \R^n)$.
  Then
  \begin{equation}
    \frac{d (g \circ \tilde{x})}{d t} = \frac{D g}{D t} \circ \tilde{x}.
  \end{equation}
\end{proposition}
\begin{proof}
  The flow velocity and motion $x$ are related by
  \begin{equation}
    \dot{x} = u \circ x.
  \end{equation}
  Hence,
  \begin{equation}
    \begin{split}
      \frac{d (g \circ \tilde{x})}{d t}
      & = i_{\dot{\tilde{x}}}(d g \circ \tilde{x}) \\
      & = i_{1 \oplus \dot{x}}((\partial_t g \oplus d_x g) \circ \tilde{x}) \\
      & = \partial_t g \circ \tilde{x}
        + i_{u \circ \tilde{x}} ((d_x g) \circ \tilde{x}) \\
      & = \partial_t g \circ \tilde{x} + (i_u(d_x g)) \circ \tilde{x}) \\
      & = ((\partial_t + i_u \circ d_x) g) \circ \tilde{x} \\
      & = \frac{D g}{D t} \circ \tilde{x}.
    \end{split}
  \end{equation}
\end{proof}
\begin{corollary}
  Let
    $I$ be an interval,
    $n \in \N^+$,
    $U$ be an open subset of $\R \times \R^n$.
    $V$ be a vector space,
    $k \in \N^+$,
    $u \in C^{k - 1}(U, \R^n)$ (usually a flow velocity),
    $x \colon I \to \R^n$ be an integral curve of $u$,
    $\tilde{x} := \id_I \odot x \colon I \to I \times \R^n$,
    $g \in C^k(I, \R^n)$.
  \begin{equation}
    \frac{d^k (g \circ \tilde{x})}{d t^k} = \frac{D^k g}{D t^k} \circ \tilde{x}.
  \end{equation}
\end{corollary}
\begin{proof}
  We proceed by the induction.
  In the base case $k = 0$ noth sides equal to $g \circ \tilde{x}$
  For the induction step assume the proposition is true for the integer $k$.
  Then for $k + 1$ we get
  \begin{equation}
    \begin{split}
      \frac{d^{k + 1} (g \circ \tilde{x})}{d t^{k + 1}}
      & = \frac{d}{d t}\left(\frac{d^k (g \circ \tilde{x})}{d t^k}\right)
      & \qquad (\text{Definition of operator power}) \\
      & = \frac{d}{d t}\left(\frac{D^k g}{D^k t} \circ \tilde{x}\right)
      & \qquad (\text{Induction hypothesis}) \\
      & = \frac{D}{D t}\left(\frac{D^k g}{D^k t}\right) \circ \tilde{x}
      & \qquad (\text{Pevious proposition}) \\
      & = \frac{D^{k + 1} g}{D t^{k + 1}} \circ \tilde{x}
      & \qquad (\text{Definition of operator power}).
    \end{split}
  \end{equation}
\end{proof}
\begin{theorem}[Picard-Lindel\"{o}f]
  Let
    $t_0 \in \R$,
    $n \in \N^+$,
    $x_0 \in \R^n$,
    $D \subseteq \R \times \R^n$ be an open neighbourhood of $(t_0, x_0)$,
    $k \in \N$,
    $F \in C^k(D, \R^n)$ be
      Lipschitz continuous with respect to the spatial arguments.
  Then there exist $\tau, h \in \R^+$ such that the problem
  \Cref{equation:ordinary_differential_equation/initial_value_problem}
  has a unique solution
  $x \in C^{k + 1}(\mathcal{C}(t_0, \tau), \mathcal{C}(x_0, h))$.

  In particular if $k = \infty$ so that $F \in C^\infty(D, \R^n)$, then
  $x \in C^\infty(\mathcal{C}(t_0, \tau), \mathcal{C}(x_0, h))$.
\end{theorem}
\begin{proof}
  Denote
    $I := \mathcal{C}(t_0, \tau)$,
    $V := \mathcal{C}(x_0, h)$,
  \begin{equation}
     u := \restrict{F}{I \times V} \in C^k(I \times V, \R^n).
   \end{equation} 
  Let $q \in [0, 1)$ be arbitrary.
  Take (from the proof of the previous proposition) $\tau, h \in \R^+$ such that
  the operator $T$ defined on
  $C^0(I, V)$ in
  \Cref{equation:ordinary_differential_equation/initial_value_problem/operator}
  is a contraction with Lipschitz constant $q$.
  According to the Banach fixed point theorem, since $C^0$ spaces between closed
  subsets of Euclidean spaces are complete, there exists a unique
  $x \in C^0(I, V)$ that is a fixed point
  of $T$.
  We need to show that the solution $x$ is in fact of class $C^{k + 1}$, not
  merely continuous.
  First we use the fact that
  \begin{equation}
    \frac{d}{d t} \int_{t_0}^t u(s, x(s))\, d s = u(t, x(t))
  \end{equation}
  to get $x'(t) = u(t, x(t))$.
  Define
  \begin{subequations}
    \begin{alignat}{1}
      & \tilde{x} \colon I \to I \times V, \\
      & \tilde{x}(t) := (t, x(t)),\ \forall t \in I,
    \end{alignat}
  \end{subequations}
  so that $\dot{x} = u \circ \tilde{x}$.
  From the previous propoition it follows that $u$ is $k$ time differentiable
  and
  \begin{equation}
    \frac{d^{k + 1} x}{d t^{k + 1}}
    = \frac{d^k (u \circ \tilde{x})}{d t^k}
    = \frac{D^k u}{D t^k} \circ \tilde{x}.
  \end{equation}
  Hence, $x \in C^{k + 1}(I, V)$, as wanted.
\end{proof}
