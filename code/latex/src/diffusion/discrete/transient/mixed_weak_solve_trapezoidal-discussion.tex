\begin{discussion}
  \label{cmc/diffusion/discrete/transient/mixed_weak_solve_trapezoidal-discussion}
  We are going to derive a solution to
  \Cref{cmc/diffusion/discrete/transient/mixed_weak-formulation}
  using the trapezoidal rule for time integration.
  We will assume that the heat capacity $\tilde{\pi}$ is time-independent which will
  allow us to rearrange the time derivative:
  \begin{equation}
    C(\tilde{w}, \frac{\partial \tilde{u}} {\partial t})
    = \frac{d}{d t} C(\tilde{w}, \tilde{u}).
  \end{equation}
  For further simplicity we will also assume that all the rest input data (heat
  source, thermal conductivity, boundary conditions) are also time-independent.
  We can then integrate the equation (the conservation law)
  \begin{equation}
    - B(\tilde{w}, q) - \frac{d}{d t} C(\tilde{w}, \tilde{u}) = - F(\tilde{w})
  \end{equation}
  with respect to $t$ in the interval $[\alpha, \beta] \subset I$ to get
  \begin{equation}
    - B(\tilde{w}, \int_\alpha^\beta q\, d t)
    - (C(\tilde{w}, \tilde{u}(\beta)) - C(\tilde{w}, \tilde{u}(\alpha)))
    = - (\beta - \alpha) F(\tilde{w}).
  \end{equation}
  If we partition $I$ into segements with size $\tau$ with moments of
  time $\{t_s := t_0 + \tau s\}_{s \geq 0}$, and if we denote
  \begin{subequations}
    \begin{alignat}{3}
      & q^s
      && := q(t_s), \enspace
      && s \geq 0, \\
      %
      & \tilde{u}^s
      && := \tilde{u}(t_s), \enspace
      && s \geq 0,
    \end{alignat}
  \end{subequations}
  we get
  \begin{equation}
    - \frac{\tau}{2} (B(\tilde{w}, q^s) + B(\tilde{w}, q^{s + 1}))
    - (C(\tilde{w}, \tilde{u}^{s + 1}) - C(\tilde{w}, \tilde{u}^s))
    = - \tau F(\tilde{w}).
  \end{equation}
  By multiplying the above equation with $2 / \tau$ and rearranging we get:
  \begin{equation}
    - B(\tilde{w}, q^{s + 1}) - \frac{2}{\tau} C(\tilde{w}, \tilde{u}^{s + 1})
    = - 2 F(\tilde{w}) + B(\tilde{w}, q^s)
    - \frac{2}{\tau} C(\tilde{w}, \tilde{u}^s).
  \end{equation}
  At step $0$ we calculate initial data as
  \begin{subequations}
    \begin{alignat}{2}
      & q^0
      && = (- \kappa \star_1 \delta_0)(u_0), \\
      %
      & \tilde{u}^0
      && = \star_0 u_0.
    \end{alignat}
  \end{subequations}
  At any step $s > 0$ we get the following system for
  $(q^s, \tilde{u}^s) \in C^{D - 1} K \times C^D K$:
  \begin{subequations}
    \begin{alignat}{3}
      & \forall r [E T^{-1}] \in \Ker \tr_{\Gamma_N, D - 1}, \,
      && A(r, q^s) - B^T(r, \tilde{u}^s)
      && = - G(r), \\
      & \forall \tilde{w} [\Theta L^D] \in C^D K, \,
      && - B(\tilde{w}, q^s) - \frac{2}{\tau} C(\tilde{w}, \tilde{u}^s)
      && = - 2 F(\tilde{w}) + B(\tilde{w}, q^{s - 1})
        - \frac{2}{\tau} C(\tilde{w}, \tilde{u}^{s - 1}), \\
      %
      &
      && \tr_{\Gamma_N, d - 1} q^s
      && = g_N.
    \end{alignat}
  \end{subequations}
  Let
  \begin{subequations}
    \begin{alignat}{2}
      & J
      && := \set{i \in \{0, ..., n_{D - 1}\}}
        {c_{d - 1, i} \in (\Gamma_N)_{D - 1}}, \\
      %
      & \overline{J}
      && := \{0, ..., n_{D - 1}\} \setminus J.
    \end{alignat}
  \end{subequations}
  Initial conditions give us $Q^0$ and $U^0$.
  Denote the matrices ${\bf A}$, ${\bf B}$, ${\bf C}$ and vectors
  ${\bf Q}^s$, ${\bf U}^s$, ${\bf F}$, ${\bf G}$
  of the corresponding operators in standard bases.
  Let $s > 0$.
  We get the following system for ${\bf Q}^s$ and ${\bf U}^s$:
  \begin{subequations}
    \begin{alignat}{3}
      & \sum_{j = 0}^{n_{D - 1} - 1} {\bf A}_{i, j} {\bf Q}^s_j
        - \sum_{k = 0}^{n_d - 1} ({\bf B}^T)_{i, k} {\bf U}^s_k
      && = - {\bf G}_i, \:
      && i \in \overline{J}, \\
      %
      & - \sum_{i = 0}^{n_{D - 1} - 1} {\bf B}_{k, i} {\bf Q}^s
        - \frac{2}{\tau} ({\bf C} {\bf U}^s)_k
      && = - 2 {\bf F}_k
        + {\bf B} {\bf Q}^{s - 1}
        - \frac{2}{\tau} {\bf C} {\bf U}^{s - 1}_k, \:
      && k \in \{0, ..., n_d - 1\}, \\
      %
      & {\bf Q}^s_i
      && = g_N(c_{d - 1, i}), \:
      && i \in J.
    \end{alignat}
  \end{subequations}
  The system can be rewritten as:
  \begin{subequations}
    \begin{alignat}{3}
      & \sum_{j \in \overline{J}} {\bf A}_{i, j} {\bf Q}^s_j
        - \sum_{k = 0}^{n_d - 1} ({\bf B}^T)_{i, k} {\bf U}^s_k
      && = - {\bf G}_i -
        \sum_{j \in J} {\bf A}_{i, j} g_N(c_{d - 1, j}), \:
      && i \in \overline{J}, \\
      %
      & - \sum_{i \in J} {\bf B}_{k, i} {\bf Q}^s_i
        - \sum_{l = 0}^{n_d - 1}
          \frac{2}{\tau}{\bf C}_{k, l} {\bf U}^s_l
      && = - 2 {\bf F}_k
        + \sum_{i \in J} {\bf B}_{k, i} g_N(c_{d - 1, i})
        + {\bf B} {\bf Q}^{s - 1}
        - \frac{2}{\tau} {\bf C} {\bf U}^{s - 1}_k, \:
      && i \in \overline{J}.
    \end{alignat}
  \end{subequations}
  Let $\overline{\bf A}$ be the restriction of ${\bf A}$ to the rows and colums
  in $\overline{J}$,
  $\overline{\bf B}$ be the restristion of ${\bf B}$ to the colums in
  $\overline{J}$,
  $\overline{\bf Q}^s$ be the restriction of ${\bf Q}^{s}$ to the indices
  in $\overline{J}$,
  $\widetilde{\bf F} \in \R^{n_d}$ be defined as,
  \begin{equation}
    \widetilde{\bf F}_k
    := 2 {\bf F}_k - \sum_{i \in J} {\bf B}_{k, i} g_N(c_{d - 1, i}),\
    k = 0, ..., n_d - 1,
  \end{equation}
  $\overline{\bf G}$ be the restriction of ${\bf G}$ to the indices in
  $\overline{J}$ (since $A$ is diagonal, ${\bf G}$ is not modified before
  restriction).
  Hence, we get the restricted system
  \begin{subequations}
    \begin{alignat}{4}
      &
      && \overline{\bf A}\, \overline{\bf Q}^s
      && - \overline{\bf B}^T {\bf U}^s
      && = - \overline{\bf G}, \\
      %
      & - 
      && \overline{\bf B}\, \overline{\bf Q}^s
      && - \frac{2}{\tau} {\bf C} {\bf U}^s
      && = - \widetilde{\bf F}
        + {\bf B} {\bf Q}^{s - 1}
        - \frac{2}{\tau} {\bf C} {\bf U}^{s - 1}.
    \end{alignat}
  \end{subequations}
  We can solve for $\overline{\bf Q}^s$ as follows:
  \begin{equation}
    \overline{\bf Q}^s
    = \overline{\bf A}^{-1}
      (- \overline{\bf G} + \overline{\bf B}^T {\bf U}^s)
    = - \overline{\bf P} + \overline{\bf R} {\bf U}^s,
  \end{equation}
  where we have denoted
  \begin{align}
    \overline{\bf P} & := \overline{\bf A}^{-1} \overline{\bf G}, \\
    \overline{\bf R} & := \overline{\bf A}^{-1} \overline{\bf B}^T.
  \end{align}
  % Hence,
  % \begin{equation}
  %   {\bf B} {\bf Q}^s + \frac{2}{\tau} {\bf C} {\bf U}^s + \widetilde{\bf F}
  %   = - \overline{\bf B}\, \overline{\bf Q}^{s + 1}
  %     + \frac{2}{\tau} {\bf C} {\bf U}^{s + 1}
  %   = - \overline{\bf B}\, \overline{\bf A}^{-1}
  %     (- \overline{\bf G} + \overline{\bf B}^T {\bf U}^{s + 1})
  %     + \frac{2}{\tau} {\bf C} {\bf U}^{s + 1}.
  % \end{equation}
  This means that
  \begin{equation}
    - \overline{\bf B}\, \overline{\bf Q}^s - \frac{2}{\tau} {\bf C} {\bf U}^s
    = - \overline{\bf B}
      \overline{\bf A}^{-1} (- \overline{\bf G} + \overline{\bf B}^T {\bf U}^s)
    - \frac{2}{\tau} {\bf C} {\bf U}^s
    = - (\overline{\bf B} \overline{\bf A}^{-1} \overline{\bf B}^T
      + \frac{2}{\tau} {\bf C}) {\bf U}^s
      + \overline{\bf B} \overline{\bf A}^{-1} \overline{\bf G}.
  \end{equation}
  Define the left-hand side matrix ${\bf N}_\tau \in M_{n_d \times n_d}(\R)$,
  \begin{equation}
    {\bf N}_\tau
    := \overline{\bf B}\, \overline{\bf A}^{-1} \overline{\bf B}^T
      + \frac{2}{\tau} {\bf C}.
  \end{equation}
  Hence, the conversation law becomes
  \begin{equation}
    {\bf N}_\tau {\bf U}^s
    = \widetilde{\bf F}
    + \overline{\bf B} \overline{\bf A}^{-1} \overline{\bf G} 
    - {\bf B} {\bf Q}^{s - 1}
    + \frac{2}{\tau} {\bf C} {\bf U}^{s - 1}.
  \end{equation}
  % This translates to
  % \begin{equation}
  %   (\overline{\bf B}\, \overline{\bf A}^{-1} \overline{\bf B}^T
  %   - \frac{2}{\tau} {\bf C}) {\bf U}^{s + 1}
  %   = \overline{\bf B} \overline{\bf A}^{-1} \overline{\bf G}
  %   - \widetilde{\bf F} - {\bf B} {\bf Q}^s - \frac{2}{\tau} {\bf C} {\bf U}^s.
  % \end{equation}
  Define the constant right-hand side vector ${\bf Z} \in \R^{n_d}$,
  \begin{equation}
    {\bf Z}
    := \overline{\bf B} \overline{\bf A}^{-1} \overline{\bf G}
      + \widetilde{\bf F}
    = \overline{\bf B} \overline{\bf P} + \widetilde{\bf F}.
  \end{equation}
  This leads to the following linear $n_d \times n_d$ system:
  \begin{equation}
    {\bf N}_\tau {\bf U}^s
    = {\bf Z}
    - {\bf B} {\bf Q}^{s - 1}
    + \frac{2}{\tau} {\bf C} {\bf U}^{s - 1}.
  \end{equation}
  Define 
  \begin{align}
    {\bf V}_\tau & := {\bf N}_\tau^{-1} {\bf Z}, \\
    {\bf Y}_\tau^s
    & := - {\bf B} {\bf Q}^{s - 1} + \frac{2}{\tau} {\bf C} {\bf U}^{s - 1}, \\
    {\bf W}_\tau^s & := {\bf N}_\tau^{-1} {\bf Y}_\tau^s.
  \end{align}
  To find ${\bf V}_\tau$ and ${\bf W}_\tau^s$ we first find the Cholesky
  decomposition of ${\bf N}_\tau$:
  \begin{equation}
    {\bf N}_\tau = {\bf L}_\tau {\bf L}_\tau^T.
  \end{equation}
  Hence,
  \begin{equation}
    {\bf U}^s
    = {\bf N}_\tau^{-1}
      ({\bf Z} - {\bf B} {\bf Q}^s + \frac{2}{\tau} {\bf C} {\bf U}^s)
    = {\bf V}_\tau + {\bf W}_\tau^s.
  \end{equation}
  Summarasing, we get the following algorithmic procedure.
\end{discussion}
