\begin{discussion}
  Let
    $(M, g)$ be a Riemannian manifold,
    $\mathcal{B}$ be a manifold diffeomorphic to an open submanifold of $M$,
    $I$ be an interval.
  A map $\kappa \colon I \times \mathcal{B} \to M$
  is called an \textbf{deformation} if it is smooth and for any $t \in I$,
  $\kappa_t := \kappa(t, \cdot) \colon \mathcal{B} \to M$ is an embedding.
\end{discussion}
\begin{discussion}
  Let $t \in I$.
  We denote the image of $\kappa_t$ as $\mathcal{B}_t$.
  Then $\mathcal{B}_t$ is diffeomorphic to $\mathcal{B}$.
\end{discussion}
\begin{discussion}
  Now fix $t_0, t_1 \in I$ with $t_0 < t_1$.
  Define the diffeomorphism
  \begin{equation}
    \varphi
    := \kappa_1 \circ \kappa_0^{-1}
    \colon \mathcal{B}_{t_0} \to \mathcal{B}_{t_1}
  \end{equation}
  which is also called a deformation.
  Here, $X := \mathcal{B}_{t_0}$ is called the \textbf{reference configuration},
  while $Y := \mathcal{B}_{t_1}$ is called the \textbf{deformed configuration}.
\end{discussion}
\begin{discussion}
  Consider the differential of $\varphi$,
  \begin{equation}
    F := d \varphi \colon T X \to T Y.
  \end{equation}
  Traditionally is called the \textbf{deformation gradient},
  but in our formalism it is more logical to call it
  \textbf{deformation differential}, which we will do.
  Since $\varphi$ is a diffeomorphism, $F$ is also a diffeomorphism, and also
  for any $x \in X$, a linear isomorphism
  $F_x \colon T_x X \to T_{\varphi(x)} Y$.
  Since $\varphi$ is an instance of a deformation, $F_x$ has a positive
  determinant in any coordinate system in $M$ (inherited by both $X$ and $Y$).
\end{discussion}
\begin{discussion}
  Define the bundle map $F^T \colon T Y \to T X$ as follows:
  for any $x \in X$, $F^T_{\varphi(x)}$ is the linear transpose to $F_x$
  with respect to the inner products $g_x$ and $g_{\varphi(x)}$.
  In other words, for any $x \in X$, $v \in T_x X$, $w \in T_{\varphi(x)} Y$,
  \begin{equation}
    g_x(v, F^T_{\varphi(x)} w) = g_{\varphi(x)}(F_x v, w).
  \end{equation}
\end{discussion}
\begin{discussion}
  Denote $C := F^T \circ F \colon T X \to T X$.
  Since for any $y \in Y$, $F^T_y \colon T_y Y \to T_{\varphi^{-1} y} X$,
  we conclude that for any $x \in X$,
  \begin{equation}
    C_x = (F^T \circ F)_x = F^T_{\varphi(x)} \circ F_x \colon T_x X \to T_x X.
  \end{equation}
  Hence, $C$ can equivalently be considered as a section of the bundle
  $\End(T M)$,
  i.e., by the tensor field
  $\tilde{C} \colon M \to \End(T M)$
  defined for any $x \in M$ and $v \in T_x M$ by
  \begin{equation}
    \tilde{C}(x) v = C(x, v) \in T_x M.
  \end{equation}
  For this reason we will identify $C$ with the tensor field $\tilde{C}$.
  $C$ is a $(1, 1)$ tensor field on the initial configuration.
  $C$ is called the \textbf{Green's deformation tensor}, the
  \textbf{right Cauchy-Green deformation tensor}, or the
  \textbf{Cauchy strain tensor}.
  We will make use also of the tensor field $U = \sqrt{C}$.
\end{discussion}
\begin{remark}
  Note that although $C$ is a tensor field on the initial configuration $X$,
  it does not encode an intrinsic property on it but a relation with another
  configuration.
  However, it may be possible to interpret $C$ as an intrinsic property if we
  stop treating the metric tensor as fixed.
  In that case, we start with some initial metric tensor $g$ on $X$,
  but the new metric tensor $h$ on $X$
  \begin{equation}
    h = \restrict{g}{X}  \circ C,
  \end{equation}
  where we use the representation
  $\restrict{g}{X} \colon \mathcal{X} X \to \mathcal{X}^* X$.
  This new metric tensor in fact implicitly measures the distances in $Y$.
  Indeed, take $x \in X$ and $v, w \in T_x X$.
  Then
  \begin{equation}
    h_x(v, w)
    = g_x(C_x v, w)
    = g_x(F^T_x F_x v, w)
    = g_{\varphi(x)}(F_x v, F_x w)
    = g_{\varphi(x)}((d \varphi)_x v, (d \varphi)_x w),
  \end{equation}
  which is the inner product on $T_{\varphi(x)}$.
\end{remark}
\begin{remark}
  In a similar fashion we can consider the map
  \begin{equation}
    B := F \circ F^T \colon T Y \to T Y
  \end{equation}
  identical to a section of $\End(T Y)$.
  Hence, $B$ is a $(1, 1)$ tensor field on the deformed configuration.
  $B$ is called the \textbf{Finger's deformation tensor}, the
  \textbf{left Cauchy-Green deformation tensor} or the
  \textbf{Green strain tensor}.
\end{remark}
\begin{discussion}
  What we called the deformation map $\varphi$ in fact represents a general
  motion of the continuum body $\mathcal{B}$.
  The actual meaning of deformation is the change of distances.
  On a Riemannian manifold there are two distance measurements.
  One is the local one, that measures how lengths of vectors in the tangent
  spaces change.
  The global one works on the manifold itself and measures how lengths of
  geodesics change.
  We will primarily work with the local one but we will see how it gives us
  global insights as well.
\end{discussion}
\begin{discussion}
  Let $V$ and $W$ be finite dimensional isomorphic vector spaces with inner
  products $g \colon V \times V \to \R$ and $h \colon W \times W \to \R$.
  Consider a linear isomorphism $F \colon V \to W$.
  We want to isolate the deformation information encoded in $F$.
  In other words, we want to ignore rotations.
  More precisely, let $X$ be some set (of invariants).
  We want to characterise the functions
  $f \colon {\rm Iso}(V, W) \coprod {\rm Aut}(V) \to X$
  satisfying the following property: for any $S \in O((W, h), (V, g))$,
  \begin{equation}
    f(S \circ F) = f(F).
  \end{equation}
  Consider the polar decomposition of $F$:
  the unique orthogonal map $R \colon V \to W$ and a
  symmetric positive definite map $U \colon W \to W$ such that
  $F = R \circ U$.
  Concretely,
  \begin{equation}
    F^T F = U^T R^T R U = U^T U = U^2,
  \end{equation}
  and this equation has unique positive-definite solution for $U$, denoted by
  $U := \sqrt{U^2} = \sqrt{F^T F} = \sqrt{C}$.
  This also gives the unique $R$, $R = F U^{-1} = F \sqrt{C}^{-1}$.
  Going back to our characterisation of $f$, if we plug
  $S = R^{T} \colon W \to V$,
  we get
  \begin{equation}
    f(F) = f(R^T F) = f(R^T R U) = f(U).
  \end{equation}
  Hence, the deformation information of $F$ can be encoded in the symmetric
  positive definite map $U$.
  The eigenvalues of $U$ (the \textbf{singular values} of $F$)
  are the stretches of the deformation.
\end{discussion}
\begin{discussion}
  We will now show that if $\psi \colon X \to Z$ is another deformation with
  the same Green deformation tensor $C_\psi = C$, then $\psi$ and $\varphi$ are
  related by a local isometry.
  (Recall that $\varphi \colon X \to Y$.)
  Indeed, let $\tau = \psi \circ \varphi^{-1} \colon Y \to Z$.
  Since,
  \begin{equation}
    d \tau = d \psi \circ d \varphi^{-1} \circ T Y \to T Z,
  \end{equation}
  then
  \begin{equation}
    \begin{split}
        (d \tau)^T \circ (d \tau)
      & = (d \psi \circ d \varphi^{-1})^T \circ (d \psi) \circ (d \varphi^{-1})
        = (d \varphi)^{-T} \circ (d \psi)^T \circ (d \psi) \circ
          (d \varphi)^{-1}
        = F^{-T} C_\psi F^{-1} \\
      & = F^{-T} C F^{-1}
        = F^{-T} F^T F F^{-1}
        = I,
    \end{split}
  \end{equation}
  which is precisely the condition for $\tau$ being a local isometry.
\end{discussion}
\begin{discussion}
  It can be proven that if the ambient manifold $M$ is the Euclidean space
  $\R^n$, then a local isometry $\tau \colon Y \to Z$ is a global one,
  i.e., it can be represented as a rigid body motion:
  \begin{equation}
    \tau(y) = Q y + a,\ Q \in {\rm SO}(n),\ a \in \R^n.
  \end{equation}
  For a general Riemannian manifold, local distances are measured by the inner
  products on the tangent spaces which make them Euclidean spaces.
  Globally, however, distances are measured by lengths of geodesics and global
  isometries preserve those distances.
  In the online discussion
  \url{https://math.stackexchange.com/questions/116195/how-to-go-from-local-to-global-isometry}
  it is shown that under certain conditions for $M$ we can still make the
  conclusion that local isometry implies global isometry.
  Therefore, the Green deformation tensor $C$ (or its square root $U$) uniquely
  reconstructs the deformation up to a local (and hence a global) isometry.
  This means that $U$ (or invertible expressions of it considered below)
  uniquely determine the deformation, disregarding the rigid body motions still
  contained in the deformation differential $F$.
  The eigenvalues of $U$ are called \textbf{stretches}.
\end{discussion}
\begin{discussion}
  If we do not have a deformation but only a rigid body motion, then
  $U$ is the identity $I$ and stretches will be all equal to $1$.
  \textbf{Strain measures} are reversible expressions
  $E \colon {\rm Symm}_+(T X) \to {\rm Symm}(T X)$
  defined as analytic functions measuring the extent to which $U$
  differs from $I$.
  This is imposed by the constraint $E(I) = 0$.
  We also impose another condition that makes all strain measures equal to
  second order as well, namely $E'(I) = I$.
  Possible strain measures include:
  \begin{enumerate}
    \item
      \textbf{logarithmic strain}, \textbf{natural strain},
      \textbf{true strain}, or \textbf{Hencky strain}:
      \begin{equation}
        E_{0}(U) := \ln U = \frac{1}{2} \ln{C}.
      \end{equation}
    \item
      the \textbf{Seth-Hill strain measures} or \textbf{Doyle-Ericksen tensors}:
      for any $m \in \R \setminus \{0\}$,
      \begin{equation}
        E_{(m)}(U) := \frac{1}{2 m}(U^{2 m} - I) = \frac{1}{2 m}(C^m - I).
      \end{equation}
      In particular, we have the \textbf{Green-Lagrangian strain tensor}
      \begin{equation}
        E_{(1)}(U) := \frac{1}{2}(U^2 - I) = \frac{1}{2}(C- I),
      \end{equation}
      and the \textbf{Biot strain tensor}
      \begin{equation}
        E_{(1 / 2)}(U) := U - I = \sqrt{C} - I.
      \end{equation}
    \item
      for any $m \in \R \setminus \{0\}$,
      \begin{equation}
        E^{(m)}(U) := \frac{1}{2 m}(U^m - U^{-m}).
      \end{equation}
  \end{enumerate}
\end{discussion}
\begin{discussion}
  Another way to describe motion is via \textbf{displacements}.
  In other words we are looking for an object $u(x)$ that measures the increment
  of the motion of a point $x \in X$ to $y = \varphi(x) \in Y$.
  On a Euclidean space this is done by the displacement vector
  $u(x) = \varphi(x) - x$.
  However, in the generic setting of curved manifolds, it is not possible to
  take point differences.
  Despite that, on Riemannian manifolds something similar is possible.
  Indeed, let $x \in X$ and $v \in T_x X = T_x M$ is sufficiently ``small''.
  Then there exists unique geodesic $\gamma$ starting at $x$, in the direction
  of $v$, i.e., $\gamma \in [0, 1] \to M$ with $\gamma(0) = x$ and
  $\gamma'(0) = v$.
  Define the exponential map of $v$ at $x$ by
  \begin{equation}
    \exp_x(v) := \gamma(1).
  \end{equation}
  In other words, it gives the point that we traverse through a geodesic
  starting at $x$ in the direction of $v$ and moving for a unit amount of time.
  If the exponent map is defined everywhere, it is a map
  \begin{equation}
    \exp \colon T M \to M.
  \end{equation}
  In particular, on a Euclidean space $\R^n$, to a point $x$ in direction $v$
  corresponds the geodesic $\gamma(t) = x + t v$.
  Hence,
  \begin{equation}
    \exp_x(v) = \gamma(1) = x + v,
  \end{equation}
  which coincides with the intuition of starting at $x$ and moving along the
  vector $v$.
  Now consider the reverse problem: given two points $x$ and $y$ in $M$,
  find a vector $v$ such that $\exp_x(v) = y$.
  This problem may have multiple solutions if more than one geodesic connects
  $x$ and $y$ but we are interested in the small geodesics that connect $x$ and
  $y$.
  For the sake of simplicity we will assume that there is unique such small
  geodesic $\gamma \colon [0, 1] \to M$ with $\gamma(0) = x$ and
  $\gamma(1) = y$.
  In this case, if $v = \gamma'(0)$, then by definition
  \begin{equation}
    \exp_x(v) := \gamma(1) = y.
  \end{equation}
  Denote this vector by $v := y - x \in T_x X$.
  We will call it the \textbf{displacement} of $x$.
  This gives us the \textbf{displacement vector field}
  \begin{equation}
    u \colon X \to T X, u(x) = \phi(x) - x \in T_x X.
  \end{equation}
  Hence,
  \begin{equation}
    \varphi = \restrict{\exp}{T X} \circ u \colon X \to Y.
  \end{equation}
  By the chain rule,
  \begin{equation}
    F = d \varphi = d(\restrict{\exp}{T X}) \circ d u.
  \end{equation}
\end{discussion}
