\documentclass{article}
\usepackage[margin = 0.6in, paper = a4paper]{geometry}
\usepackage{xcolor}
\usepackage[fleqn]{amsmath}
\usepackage{amssymb}
\usepackage{amsthm}
\usepackage{makecell}
\usepackage{mathtools}
\usepackage{bookmark}
\addtocontents{toc}{\protect{\pdfbookmark[0]{\contentsname}{toc}}}
\addtocontents{lof}{\protect{\pdfbookmark[0]{\listfigurename}{lof}}}
\addtocontents{lot}{\protect{\pdfbookmark[0]{\listtablename}{lot}}}
\usepackage{hyperref}
\hypersetup{
  pdfauthor = {Kiprian Berbatov},
  pdftitle = {Combinatorial Mesh Calculus},
  pdfsubject = {Combinatorial Mesh Calculus},
  pdfkeywords = {Combinatorial, Mesh, Calculus, Geometry, Microstructure},
  pdfcreator = {pdflatex},
  pdfproducer = {Latex2e with hyperref},
  colorlinks = true,
  linkcolor = blue,
  citecolor = green,
  urlcolor = cyan,
  bookmarksnumbered = true,
  bookmarksopen = false
}
\usepackage[nameinlink]{cleveref}
\usepackage{graphicx}
\graphicspath{{build/release/pdf}}
\usepackage{subcaption}
\usepackage{cprotect}
\usepackage[most]{tcolorbox}
\newcommand\coloredcomponent[2]
{
  \tcolorboxenvironment{#1}
  {
    breakable,
    enhanced,
    colback = white,
    colframe = #2,
    boxrule = 1pt,
    left = 2pt,
    right = 2pt,
    top = 2pt,
    bottom = 2pt,
    sharp corners,
    before skip = \topsep,
    after skip = \topsep,
  }
}

\counterwithin{equation}{section}
\theoremstyle{definition}

\newtheorem{theorem}{Theorem}[section]
\coloredcomponent{theorem}{blue}

\newtheorem{proposition}[theorem]{Proposition}
\coloredcomponent{proposition}{blue}

\newtheorem{lemma}[theorem]{Lemma}
\coloredcomponent{lemma}{blue}

\newtheorem{corollary}[theorem]{Corollary}
\coloredcomponent{corollary}{blue}

\newtheorem{hypothesis}[theorem]{Hypothesis}
\coloredcomponent{hypothesis}{red}

\newtheorem{notation}[theorem]{Notation}
\coloredcomponent{notation}{green}

\newtheorem{definition}[theorem]{Definition}
\coloredcomponent{definition}{green}

\newtheorem{formulation}[theorem]{Formulation}
\coloredcomponent{formulation}{brown}

\newtheorem{discussion}[theorem]{Discussion}
\coloredcomponent{discussion}{yellow}

\newtheorem{example}[theorem]{Example}
\coloredcomponent{example}{purple}

\newtheorem{remark}[theorem]{Remark}
\coloredcomponent{remark}{orange}

\newtheorem{algorithm}[theorem]{Algorithm}
\coloredcomponent{algorithm}{cyan}

\newtheorem{solution}[theorem]{Solution}
\coloredcomponent{solution}{cyan}

\coloredcomponent{proof}{cyan}

\renewcommand{\thefootnote}{\arabic{footnote}}

\newcommand{\N}{\mathbb{N}}
\newcommand{\Z}{\mathbb{Z}}
\newcommand{\Q}{\mathbb{Q}}
\newcommand{\R}{\mathbb{R}}
\newcommand{\C}{\mathbb{C}}
\renewcommand{\S}{\mathbb{S}}

\newcommand{\set}[2]{\left\{ #1 \mid #2 \right\}}
\newcommand{\restrict}[2]{\left. #1 \right|_{#2}}

\DeclarePairedDelimiter\ceil{\lceil}{\rceil}
\DeclarePairedDelimiter\floor{\lfloor}{\rfloor}

\newcommand{\norm}[1]{\left\lVert#1\right\rVert}
\newcommand{\abs}[1]{\left\lvert#1\right\rvert}
\newcommand{\inner}[2]{\langle#1,#2\rangle}
\newcommand{\lie}[2]{[#1,#2]}
\newcommand{\poisson}[2]{\{#1,#2\}}
\newcommand{\hamiltonian}{\nu}

\newcommand{\linearspan}{\mathop{\rm span}\nolimits}
\newcommand{\Ker}{\mathop{\rm Ker}\nolimits}
\renewcommand{\Im}{\mathop{\rm Im}\nolimits}
\newcommand{\Hom}{\mathop{\rm Hom}\nolimits}
\newcommand{\End}{\mathop{\rm End}\nolimits}
\newcommand{\Aut}{\mathop{\rm Aut}\nolimits}
\newcommand{\id}{\mathop{\rm id}\nolimits}
\newcommand{\tr}{\mathop{\rm tr}\nolimits}
\newcommand{\sym}{\mathop{\rm sym}\nolimits}

\newcommand{\grad}{\mathop{\rm grad}\nolimits}
\renewcommand{\div}{\mathop{\rm div}\nolimits}

\newcommand{\Aff}{\mathop{\rm Aff}\nolimits}
\newcommand{\Con}{\mathop{\rm Con}\nolimits}

\newcommand{\Cl}{\mathop{\rm Cl}\nolimits}
\newcommand{\Link}{\mathop{\rm Link}\nolimits}

\newcommand{\Cauchy}{\mathop{\rm Cauchy}\nolimits}
\newcommand{\Convergent}{\mathop{\rm Convergent}\nolimits}

\newcommand{\sgn}{\mathop{\rm sgn}\nolimits}
\newcommand{\OR}{\mathop{\rm or}\nolimits}
\newcommand{\vol}{\mathop{\rm vol}\nolimits}
\newcommand{\usmile}{\underline{\smile}}
\newcommand{\uwedge}{\underline{\wedge}}

\newcommand{\newterm}[1]{\textbf{#1}}

\newcommand{\amount}{\mathsf{X}}
\newcommand{\potential}{\mathsf{Y}}
\newcommand{\mass}{\mathsf{M}}
\newcommand{\length}{\mathsf{L}}
\renewcommand{\time}{\mathsf{T}}
\newcommand{\temperature}{\theta}
\newcommand{\charge}{\mathsf{C}}

\newcommand{\topStrut}{\rule{0pt}{2.6ex}}

\setcounter{tocdepth}{4}
\setcounter{secnumdepth}{4}

\title{Combinatorial Mesh Calculus (CMC) \\
       \url{https://github.com/kipiberbatov/cmc}}
\author{Kiprian Berbatov}
\date{7 October 2025}

\begin{document}

\pdfsuppresswarningpagegroup=1

\maketitle

\tableofcontents
\NewCommandCopy\oricontentsline\contentsline
\makeatletter
\RenewDocumentCommand\contentsline{mmmm}
{%
  \oricontentsline{#1}{#2}{#3}{#4}%
  {\let\numberline\@gobble
    \bookmark[
      rellevel=1,
      keeplevel,
      dest=#4,
    ]{#2}}%
}
\makeatother

\listoffigures

\listoftables

\bookmarksetup{startatroot}

\part{Algebra}

\section{Commutative rings with unity}
\label{section:commutative_rings_with_unity}
\begin{definition}
  Let
    $R$ be a commutative ring with unity,
    $V$ be an $R$-algebra with multiplication operation $[\cdot, \cdot]$.
  We say that $V$ is a \textbf{Lie algebra} if $[\cdot, \cdot]$ is alternating
  and satisfies the \textbf{Jacobi identity}: for any $x, y, z \in V$,
  \begin{equation}
    [x, [y, z]] + [y, [z, x]] + [z, [x, y]] = 0.
  \end{equation}
\end{definition}

\begin{example}
  Let $R$ be a commutative ring with unity.
  The following are examples of modules over $R$.
  \begin{enumerate}
    \item
      For any $n \in \N$, the space $R^n$ is a module over $R$ under
      component-wise addition and multiplication with a scalar.
    \item
      For any $m, n \in \N$, the space $M_{m \times n}(R)$ of $m \times n$
      matrices with elements in $R$ is a module over $R$ under under
      component-wise addition and multiplication with a scalar.
    \item
      For any set $X$, the ring $R^X$ can also be considered as a module over
      $R$ with pointwise addition and multiplication with a scalar.
      It generalises the previous two cases when $X = \{1, ..., n\}$ and
      $X = \{1, ..., m\} \times \{1, ..., n\}$ respectively.
  \end{enumerate}
\end{example}

\begin{definition}
  Let $R$ be a commutative ring with unity, $S$ be a subset of $R$.
  We say that $S$ is a \newterm{subring with unity} of $R$ if $S$ is a
  commutative ring with unity with operations on $S$ being the restrictions on
  $S$ of the operations in $R$.
\end{definition}

\begin{proposition}
  Let $R$ be a commutative ring with unity, $S$ be a subset of $R$.
  Then $S$ is a subring with unity if contains $1$ and is closed under
  negation, addition, and multiplication, i.e.,
  for any $a, b \in R$,
  \begin{equation}
    1 \in \R,\ -a \in \R,\ a + b \in \R,\ a b \in R.
  \end{equation}
\end{proposition}

\begin{example}
  The following are examples of subrings.
  \begin{enumerate}
    \item
      The number sets $\Z$, $\Q$, $\R$, $\C$ form a chain of subrings
      (any of them is a subring with unity of the next one).
    \item
      For any $n \in \N,\ n \geq 2$, the set $n \Z := \set{n x}{x \in \Z}$ is a
      subring of $\Z$ that has no unity.
  \end{enumerate}
\end{example}

\begin{definition}
  Let $R$ be a commutative ring with unity.
  We say that $R$ is a \textbf{field} if it has at least two elements and any
  nonzero element has a multiplicative inverse, i.e.,
  \begin{equation}
    \forall a \in R \setminus \{0\}\ \exists b \in R,\ a * b = 1.
  \end{equation}
\end{definition}

\begin{example}
  The following are examples and counterexamples of fields.
  \begin{enumerate}
    \item
      Among the number sets, $\Q$, $\R$, $\C$ are fields, $\Z$ is not a field.
    \item
      Let $n$ be an integer.
      Then $\Z_n$ is a field if and only if $n$ is a prime number.
    \item
      Let $X$ be a nonempty set, $R$ be a field.
      Then the set of functions $R^X := \set{f}{f \colon X \to R}$ is not a
      field except in the trivial case of $X$ having one element.
      Indeed, let $X$ has at least two elements, and take a function $f \in R^X$
      that is zero at a point $x_0$ but nonzero at some other point $x_1$
      (so it is not identically zero).
      Then $f$ is not invertible since for any $g \in R^X$,
      \begin{equation}
        (f * g)(x_0) = f(x_0) * g(x_0) = 0 * g(x_0) = 0 \neq 1.
      \end{equation}
  \end{enumerate}
\end{example}

\begin{remark}
  The reason for introducing commutative rings with unity instead of working
  with fields is that, as explained above, sets of functions $R^X$ are not
  fields.
  We will encounter them a lot when working with algebraic structures on smooth
  manifolds and meshes, predominantly modules over subrings of $R^X$.
\end{remark}


\section{Modules over commutative rings with unity}
\label{section:modules_over_commutative_rings_with_unity}
\begin{definition}
  Let
    $R$ be a commutative ring with unity,
    $V$ be an $R$-algebra with multiplication operation $[\cdot, \cdot]$.
  We say that $V$ is a \textbf{Lie algebra} if $[\cdot, \cdot]$ is alternating
  and satisfies the \textbf{Jacobi identity}: for any $x, y, z \in V$,
  \begin{equation}
    [x, [y, z]] + [y, [z, x]] + [z, [x, y]] = 0.
  \end{equation}
\end{definition}

\begin{remark}
  When the space of scalars is a field, a module becomes a vector space.
  Hence, modules generalise vector spaces.
\end{remark}

\begin{example}
  Let $R$ be a commutative ring with unity.
  The following are examples of modules over $R$.
  \begin{enumerate}
    \item
      For any $n \in \N$, the space $R^n$ is a module over $R$ under
      component-wise addition and multiplication with a scalar.
    \item
      For any $m, n \in \N$, the space $M_{m \times n}(R)$ of $m \times n$
      matrices with elements in $R$ is a module over $R$ under under
      component-wise addition and multiplication with a scalar.
    \item
      For any set $X$, the ring $R^X$ can also be considered as a module over
      $R$ with pointwise addition and multiplication with a scalar.
      It generalises the previous two cases when $X = \{1, ..., n\}$ and
      $X = \{1, ..., m\} \times \{1, ..., n\}$ respectively.
  \end{enumerate}
\end{example}

\begin{definition}
  Let
    $R$ be a commutative ring with unity,
    $V$ be a module over $R$.
    $S \subset V$.
  We say that the $S$ is \textbf{linearly independent} if
  for any $n \in \N$, $\lambda_1, ..., \lambda_n \in R$, $v_1, ..., v_n \in V$,
  \begin{equation}
    \lambda_1 v_1 + ... + \lambda_n v_n = 0
    \Rightarrow \lambda_1 = ... = \lambda_n = 0.
  \end{equation}
  We say that the set $S$ is \textbf{linearly dependent} if
  there exist $n \in \N$, $\lambda_1, ..., \lambda_n \in R \setminus \{0\}$,
  $v_1, ..., v_n \in V$ such that
  \begin{equation}
    \lambda_1 v_1 + ... + \lambda_n v_n = 0.
  \end{equation}
\end{definition}

\begin{definition}
  Let
    $R$ be a commutative ring with unity,
    $V$ be a module over $R$.
    $S \subset V$.
  We say that $S$ \textbf{spans} $V$ if
  for any $v \in V$
  there exist $n \in \N$, $\lambda_1, ..., \lambda_n \in R$,
  $v_1, ..., v_n \in V$ such that
  \begin{equation}
    v = \lambda_1 v_1 + ... + \lambda_n v_n.
  \end{equation}
\end{definition}

\begin{definition}
  Let
    $R$ be a commutative ring with unity,
    $V$ be a module over $R$.
    $S \subset V$.
  We say that $S$ is a \textbf{basis} of $V$ if it is linearly independent and
  spans $V$.
\end{definition}

\begin{definition}
  Let $R$ be a commutative ring with unity, $V$ be a module over $R$.
  We say that $V$ is \textbf{free} if it admits a basis.
\end{definition}

\begin{example}
  The module of tuples and matrices over a commutative ring with unity are free.
  A set of modules that do not have a basis consists of $\Z_n$ as modules over
  $\Z$ for any $n \geq 2$.
  Indeed, any vector $v \in \Z_n$ is linearly independent since $n v = 0$.
\end{example}

\begin{definition}
  Let
    $R$ be a commutative ring with unity,
    $V$ and $W$ be $R$-modules,
    $f \colon V \to W$.
  We say that $f$ is \textbf{linear} (or a \textbf{homomorphism}) if for any
  $\lambda, \mu \in R$, $u, v \in V$,
  \begin{equation}
    f(\lambda u + \mu v) = \lambda f(u) + \mu f(v).
  \end{equation}
  The space of all homomorphisms between $V$ and $W$ is denoted by $\Hom(V, W)$.
  An \textbf{endomorphism} is a homorphism from a module to itself.
  The space of all endomorphisms on $V$ is denoted by $\End V := \Hom(V, V)$.
\end{definition}

\begin{proposition}
  Let
    $R$ be a commutative ring with unity,
    $V$ and $W$ be $R$-modules,
  Then $\Hom(V, W)$ is an $R$-module under pointwise addition and scalar
  multiplication.
\end{proposition}

\begin{proof}
  Since $\Hom(V, W)$ is a subset of the space of functions from $V$ to $W$,
  which, as we already know, form a module, it is enough to show that
  $\Hom(V, W)$ is closed under addition and scalar multiplication.
  \begin{itemize}
    \item
      Let $f, g \in \Hom(V, W)$.
      We want to show that $f + g \in \Hom(V, W)$.
      Take arbitrary $u, v \in U$, $\lambda, \mu \in R$.
      Then
      \begin{equation}
        (f + g)(\lambda u + \mu v)
        = f(\lambda u + \mu v) + g(\lambda u + \mu v)
        = \lambda f(u) + \mu f(v) + \lambda g(u) + \mu g(v)
        = \lambda (f + g)(u) + \mu (f + g)(v).
      \end{equation}
      Hence, $f + g \in \Hom(V, W)$.
    \item
      Let $f \in \Hom(V, W)$, $r \in R$.
      We want to show that $r f \in \Hom(V, W)$.
      Take arbitrary $u, v \in U$, $\lambda, \mu \in R$.
      Then
      \begin{equation}
        (r f)(\lambda u + \mu v)
        = r f(\lambda u + \mu v)
        = r \lambda f(u) + r \mu f(v)
        = \lambda (r f)(u) + \mu (r f)(v).
      \end{equation}
      Hence, $r f\in \Hom(V, W)$.
  \end{itemize}
  Hence $\Hom(V, W)$ is a subspace of $V^W$.
\end{proof}


\section{Algebras over rings}
\label{section:algebras_over_rings}
\begin{definition}
  Let
    $R$ be a commutative ring with unity,
    $V$ be an $R$-algebra with multiplication operation $[\cdot, \cdot]$.
  We say that $V$ is a \textbf{Lie algebra} if $[\cdot, \cdot]$ is alternating
  and satisfies the \textbf{Jacobi identity}: for any $x, y, z \in V$,
  \begin{equation}
    [x, [y, z]] + [y, [z, x]] + [z, [x, y]] = 0.
  \end{equation}
\end{definition}

\begin{definition}
  Let $R$ be a commutative ring with unity, $(A, \mu)$ be an $R$-algebra.
  \begin{enumerate}
    \item
      We say that $(A, \mu)$ is \textbf{associative algebra} if
      for any $a, b, c \in A$, $\mu(a, \mu(b, c)) = \mu(\mu(a, b), c)$.
      In other words, $(A, \mu)$ is a ring.
    \item
      We say that $(A, \mu)$ is \textbf{commutative algebra} if
      for any $a, b, c \in A$, $\mu(a, b) = \mu(b, a)$.
    \item
      Let $1 \in R$.
      We say that $(A, \mu, 1)$ is \textbf{unital algebra} if
      for any $a, \in A$, $\mu(a, 1) = \mu(1, a) = a$.
      ($1$ is the \textbf{unit element} or \textbf{unity}.)
    \item
      We say that $(A, \mu)$ is \textbf{alternating algebra} if
      for any $a \in A$, $\mu(a, a) = 0$.
    \item
      We say that $(A, \mu)$ is \textbf{anti-commutative algebra} if
      for any $a, b \in A$, $\mu(a, b) = - \mu(b, a)$.
  \end{enumerate}
\end{definition}

\begin{proposition}
  Let $R$ be a commutative ring with unity,
  Let $(A, \mu)$ be an algebra.
  \begin{itemize}
    \item
      If $A$ is alternating, then $A$ is anti-commutative.
    \item
      If $A$ is anti-commutative and the ring $R$ has the property
      \begin{equation}
        \label{equation:ring/characteristics_is_not_divisible_by_2}
        \forall x \in R,\ x + x = 0 \Rightarrow x = 0,
      \end{equation}
      then $A$ is alternating.
  \end{itemize}
\end{proposition}

\begin{proof}
  First, let $A$ be alternating.
  Then for any $a, b \in A$,
  \begin{equation}
    0 = \mu(a + b, a + b)
    = \mu(a, a) + \mu(a, b) + \mu(b, a) + \mu(b, b)
    = \mu(a, b) + \mu(b, a)
    \Rightarrow \mu(b, a) = - \mu(a, b).
  \end{equation}
  Conversely, let $A$ be anti-commutative. Taking $b = a$ in the definition
  leads to
  \begin{equation}
    \mu(a, a) = - \mu(a, a) \Rightarrow \mu(a, a) + \mu(a, a) = 0.
  \end{equation}
  Under the assumption of
  \Cref{equation:ring/characteristics_is_not_divisible_by_2},
  we conclude that $\mu(a, a) = 0$.
\end{proof}


\section{Lie algebras}
\label{section:lie_algebras}
\begin{definition}
  Let
    $R$ be a commutative ring with unity,
    $V$ be an $R$-algebra with multiplication operation $[\cdot, \cdot]$.
  We say that $V$ is a \textbf{Lie algebra} if $[\cdot, \cdot]$ is alternating
  and satisfies the \textbf{Jacobi identity}: for any $x, y, z \in V$,
  \begin{equation}
    [x, [y, z]] + [y, [z, x]] + [z, [x, y]] = 0.
  \end{equation}
\end{definition}

\begin{proposition}
  Let
    $R$ be a commutative ring with unity,
    $V$ be an associative algebra over $R$.
  Define the \textbf{commutator}$ [\cdot, \cdot] \colon V \times V \to V$
  as follows: for any $x, y \in V$,
  \begin{equation}
    [x, y] := x y - y x.
  \end{equation}
  Then $(V, [\cdot, \cdot])$ is a Lie algebra.
  In particular, when $X$ is a module over $R$, $V := {\rm End}_R X$ is a Lie
  algebra with Lie bracket given by
  \begin{equation}
    [\varphi, \psi] := \varphi \circ \psi - \psi \circ \varphi,\
    \varphi, \psi \in {\rm End}_R X.
  \end{equation}
\end{proposition}

\begin{proof}
  Let $x, y, z \in A$.
  $[\cdot, \cdot]$ is alternating since $[x, x] = x x - x x = 0$.
  Next, we calculate
  \begin{equation}
    [x, [y, z]]
    = [x, y z - z y]
    = x (y z - z y) - (y z - z y) x
    = x y z - x z y - y z x + z y x.
  \end{equation}
  Similarly,
  $[y, [z, x]] = y z x - y x z - z x y + x z y$ and
  $[z, [x, y]] = z x y - z y x - x y z + y x z$.
  We can see that each of the six permutations of $(x, y, z)$ in products occurs
  twice with opposite signs, which means that their sum is zero.
  Hence, the Jacobi identity is also satisfied.
\end{proof}

\begin{definition}
  Let
    $R$ be a commutative ring with unity,
    $V$ and $W$ be $R$-modules,
    $f \colon V \to W$.
  We say that $f$ is \textbf{linear} (or a \textbf{homomorphism}) if for any
  $\lambda, \mu \in R$, $u, v \in V$,
  \begin{equation}
    f(\lambda u + \mu v) = \lambda f(u) + \mu f(v).
  \end{equation}
  The space of all homomorphisms between $V$ and $W$ is denoted by $\Hom(V, W)$.
  An \textbf{endomorphism} is a homorphism from a module to itself.
  The space of all endomorphisms on $V$ is denoted by $\End V := \Hom(V, V)$.
\end{definition}

\begin{definition}
  $R$ be a commutative ring with unity, $V$ be a Lie algebra over $R$.
  Define the \textbf{adjoint map}
  \begin{equation}
    {\rm adj} \colon V \to {\rm End}_R V,\
    {\rm adj}_x y := [x, y],\ x, y \in V.
  \end{equation}
\end{definition}

\begin{proposition}
  $R$ be a commutative ring with unity, $V$ be a Lie algebra over $R$.
  Then the adjoint map is a homorphism of Lie algebras, i.e.,
  for any $x, y \in V$,
  \begin{equation}
    {\rm adj}_{[x, y]}
    = [{\rm adj}_x, {\rm adj}_y]
    := {\rm adj}_x \circ {\rm adj}_y - {\rm adj}_y \circ {\rm adj}_x.
  \end{equation}
\end{proposition}

\begin{proof}
  Let $x, y, z \in V$.
  Then
  \begin{equation}
    \begin{split}
      [{\rm adj}_x, {\rm adj}_y] z
      & = ({\rm adj}_x \circ {\rm adj}_y - {\rm adj}_y \circ {\rm adj}_x) z
      = [x, [y, z]] - [y, [x, z]] \\
      & = [x, [y, z]] + [y, [z, x]]
      = - [z, [x, y]]
      = [[x, y], z]
      = {\rm adj}_{[x, y]} z.
      \qedhere
    \end{split}
  \end{equation}
\end{proof}


\section{Derivations on algebras}
\label{section:derivations_on_algebras}
\begin{definition}
  Let
    $R$ be a commutative ring with unity,
    $V$ be an $R$-algebra with multiplication operation $[\cdot, \cdot]$.
  We say that $V$ is a \textbf{Lie algebra} if $[\cdot, \cdot]$ is alternating
  and satisfies the \textbf{Jacobi identity}: for any $x, y, z \in V$,
  \begin{equation}
    [x, [y, z]] + [y, [z, x]] + [z, [x, y]] = 0.
  \end{equation}
\end{definition}

\begin{example}[Derivative of a single variable function is derivation]
  Let
    $R = \R$,
    $A = C^\infty(\R)$ be the space of infinitely differentiable functions,
    $D$ is the derivative operator, i.e., $D f = f'$ for $f \in A$.
  Then $D$ is derivation.
\end{example}

\begin{example}
  Let $R$ be a commutative ring with unity,
  $V$ be a Lie algebra over $R$.
  Then the adjoint homomorphism ${\rm adj} \colon V \to {\rm End}_R(V)$ induces
  for each $x \in V$ a derivation ${\rm adj}_x \colon V \to V$.
  Indeed, take $x, y, z \in A$.
  Then
  \begin{equation}
    {\rm adj}_x [y, z]
    = [x, [y, z]]
    = - [z, [x, y]] - [y, [z, x]]
    = [[x, y], z] + [y, [x, z]]
    = [{\rm adj}_x y, z] + [y, {\rm adj}_x z].
  \end{equation}
\end{example}

\begin{proposition}
  Let $R$ be a commutative ring with unity,
  $A$ be a unital, commutative, and associative $R$-algebra.
  Then ${\rm Der}_A$ is an $A$-module under pointwise addition and
  multiplication with a scalar.
\end{proposition}

\begin{proof}
  Let $D, D_1, D_2 \in {\rm Der}_R(A)$.
  Then $D_1 + D_2$ is a homomorphism from module theory.
  To prove it satisfies the Leibniz rule, let $a, b \in A$.
  Then
  \begin{equation}
    (D_1 + D_2)(a * b)
    = D_1(a * b) + D_2(a * b)
    = D_1 a * b + a * D_1 b + D_2 a * b + a * D_2 b
    = (D_1 + D_2) a * b + a * (D_1 + D_2) b.
  \end{equation}
  Further, let $a \in A$.
  Then $(a \cdot D)(b) := a * D b$ for any $b \in B$.
  It is trivial to check it it is a homomorphism.
  To check it is a derivation, let $b, c \in A$.
  Then
  \begin{equation}
    (a \cdot D)(b * c)
    = a * D(b * c)
    = a * (D b * c + b * D c)
    = (a * D b) * c + b * (a * D c)
    = (a \cdot D) b * c + b * (a \cdot D) c.
  \end{equation}
  Hence, ${\rm Der}_R(A)$ is an $A$-module.
\end{proof}

\begin{proposition}
  Let $R$ be a commutative ring with unity,
  $A$ be a unital, commutative, and associative $R$-algebra,
  $X, Y \in {\rm Der}_R(A)$.
  Then $X \circ Y - Y \circ X \in {\rm Der}_R(A)$.
\end{proposition}

\begin{proof}
  Linearity of $[X, Y]$ follows from module theory.
  Hence, we only need to show it satisfies the Leibniz rule.
  Let $f, g \in A$.
  Then
  \begin{equation}
    X(Y(f g))
    = X((Y f) * g + f * (Y g))
    = (X(Y f)) * g + (Y f) * (X g) + (X f) * (Y g) + f * (X (Y g)).
  \end{equation}
  Analogously,
  \begin{equation}
    Y(X(f g)) = (Y(X f)) * g + (X f) * (Y g) + (Y f) * (X g) + f * (Y (X g)).
  \end{equation}
  Hence,
  \begin{equation}
    [X, Y](f g)
    = {\color{red} (X(Y f)) * g}\,
      {\color{blue} + f * (X (Y g))}\,
      {\color{red} - (Y(X f)) * g}\,
      {\color{blue}- f * (Y (X g))}
    = {\color{red}([X, Y] f) * g} + {\color{blue} f * ([X, Y] g)}.
    \qedhere
  \end{equation}
\end{proof}

\begin{corollary}
  Let $R$ be a commutative ring with unity,
  $A$ be a unital, commutative, and associative $R$-algebra.
  Then ${\rm Der}_R(A)$ is a Lie algebra with Lie bracket given by
  \begin{equation}
    [X, Y] := X \circ Y - Y \circ X,\ X, Y \in {\rm Der}_R(A). 
  \end{equation}
\end{corollary}

\begin{proposition}
  Let $R$ be a commutative ring with unity, $n$ be  positive integer.
  Consider the set $R^n$ as a ring with pointwise addition and multiplication,
  and as an $R$-algebra with pointwise scalar multiplication.
  Then
  \begin{equation}
    {\rm Der}_R(R^n) = 0.
  \end{equation}
\end{proposition}

\begin{proof}
  Let $D \in {\rm Der}_R(R^n)$.
  Consider the standard basis $(e_1, ..., e_n)$ of $R^n$ and let the matrix of
  $D$ in the basis be $A$, i.e., for $1 \leq i, j \leq n$,
  \begin{equation}
    D e_i = \sum_{j = 1}^n A_{j, i} e_j.
  \end{equation}
  Let $1 \leq i \leq n$.
  Then, since mltiplication in $R^n$ is commutative,
  \begin{equation}
    D e_i = D(e_i * e_i) = D e_i * e_i + e_i * D e_i = 2 D e_i * e_i,
  \end{equation}
  which translates to
  \begin{equation}
    \sum_{j = 1}^n A_{j, i} e_j
    = 2 \sum_{j = 1}^n A_{j, i} e_j * e_i
    = 2 A_{i, i} e_i.
  \end{equation}
  For $j \neq i$ this leads to $A_{j, i} = 0$, while for $j = i$ it leads to
  $A_{i, i} = 2 A_{i, i} \Rightarrow A_{i, i} = 0$.
  Therefore, for all $i, j \in \{1, ..., n\}$, $A_{i, j} = 0$.
  Hence, $A = 0$, which means that $D = 0$.
\end{proof}

\begin{definition}
  Let
  {
    $R$ be a commutative ring with unity,
    $A$ and $B$ be $R$-algebras,
    $D_A$ and $D_B$ be derivations on $A$ and $B$ respectively.
  }
  Define the map $D \colon A \otimes B \to A \otimes B$ as follows:
  it is the unique linear map such that for simple tensors
  $a \otimes b \in A \otimes B$,
  \begin{equation}
    D(a \otimes b) := D_A a \otimes b + a \otimes D_B b.
  \end{equation}
  We call the map $D$ the \textbf{tensor product of derivations}
  and denote it by $D_A \otimes D_B$.
\end{definition}

\begin{proposition}
  Let
  {
    $R$ be a commutative ring with unity,
    $A$ and $B$ be $R$-algebras,
    $D_A$ and $D_B$ be derivations on $A$ and $B$ respectively.
  }
  Then the tensor product of derivations $D := D_A \otimes D_B$
  is indeed a derivation on $A \otimes B$.
\end{proposition}

\begin{proof}
  Consider the simple tensors
  $c_1 := a_1 \otimes b_1,\ c_2 := a_2 \otimes b_2 \in A \otimes B$.
  Then
  \begin{equation}
    \begin{split}
      D(c_1 \cdot c_2)
      & = D((a_1 \otimes b_1) \cdot (a_2 \otimes b_2)) \\
      & = D((a_1 \cdot a_2) \otimes (b_1 \cdot b_2)) \\
      & = D_A(a_1 \cdot a_2) \otimes (b_1 \cdot b_2)
        + (a_1 \cdot a_2) \otimes D_B(b_1 \cdot b_2) \\
      & = (D_A a_1 \cdot a_2 + a_1 \cdot D_A a_2) \otimes (b_1 \cdot b_2)
        + (a_1 \cdot a_2) \otimes (D_B b_1 \cdot b_2 + b_1 \cdot D_B b_2) \\
      & = (D_A a_1 \otimes b_1 + a_1 \otimes D_B b_1) \cdot (a_2 \otimes b_2)
        + (a_1 \otimes b_1) \cdot (D_A a_2 \otimes b_2 + a_2 \otimes D_B b_2) \\
      & = D(a_1 \otimes b_1) \cdot (a_2 \otimes b_2)
        + (a_1 \otimes b_1) \cdot D(a_2 \otimes b_2) \\
      & = D c_1 \cdot c_2 + c_1 \cdot D c_2.
    \end{split}
  \end{equation}
  By linearity, the Leibniz rule is satisfied for all tensors.
\end{proof}


\section{Exterior algebra}
\label{section:exterior_algebra}
\begin{definition}
  Let
    $R$ be a commutative ring with unity,
    $V$ be an $R$-algebra with multiplication operation $[\cdot, \cdot]$.
  We say that $V$ is a \textbf{Lie algebra} if $[\cdot, \cdot]$ is alternating
  and satisfies the \textbf{Jacobi identity}: for any $x, y, z \in V$,
  \begin{equation}
    [x, [y, z]] + [y, [z, x]] + [z, [x, y]] = 0.
  \end{equation}
\end{definition}

\input{exterior_algebra/wedge_on_vectors_is_antisymmetric-proposition.tex}
\input{exterior_algebra/wedge_on_vectors_is_antisymmetric-proof.tex}
\begin{proposition}
  Let
    $R$ be a commutative ring with unity,
    $D \in \N$,
    $V$ be a free module over $R$ of dimension $D$,
    $e_0, ..., e_{D - 1}$ be a basis of $V$.
  Then the $2^D$-element set $S(e)$,
  \begin{equation}
    S(e) :=
    \set{e_{i_0} \wedge ... \wedge e_{i_{p - 1}}}
    {p \in \{0, ..., D\},\ 0 \leq i_0 < ... < i_{p - 1} \leq D - 1}
  \end{equation}
  (for $p = 0$ the empty wedge product is defined to be $1$)
  forms a basis of $\Lambda^\bullet V$.
\end{proposition}

\input{exterior_algebra/ordered_lists_form_a_basis-remark.tex}
\begin{proposition}
  Let
    $R$ be a commutative ring with unity,
    $D \in \N$,
    $V$ be a free module over $R$ of dimension $D$,
    $v_0, ..., v_{D - 1} \in V$.
  Then $(v_0, ..., v_{D - 1})$ form a basis of $V$ if and only if
  \begin{equation}
    v_0 \wedge ... \wedge v_{D - 1}
  \end{equation}
  forms a basis of the $1$-dimensional module $\Lambda^D V$.
\end{proposition}

\begin{proposition}
  Let
    $R$ be a commutative ring with unity,
    $D \in \N$,
    $V$ be a free module over $R$ of dimension $D$,
    $e_0, ..., e_{D - 1}$ be a basis of $V$,
    $v_0, ..., v_{D - 1} \in V$ such that for $j = 0, ..., D - 1$,
  \begin{equation}
    v_j = \sum_{i = 0}^{D - 1} e_i A_{i, j}
  \end{equation}
  (in matrix form, $(v_0, ..., v_{D - 1}) = (e_0, ..., e_{D - 1}) A$).
  Then
  \begin{equation}
    v_0 \wedge ... \wedge v_{D - 1} = (\det A) e_0 \wedge ... \wedge e_{D - 1}.
  \end{equation}
  As a consequence, the matrix $A$ is a change of basis matrix if and only if
  $\det A$ is an invertible element of $R$
  (a nonzero element when $R$ is a field).
\end{proposition}

\input{vector_space/dual-notation.tex}
\input{exterior_algebra/p_th_power_commutes_with_dual_up_to_isomorphism-proposition.tex}
\input{exterior_algebra/commutes_with_dual_up_to_isomorphism-corollary.tex}
\input{exterior_algebra/division_of_top_dimensional_forms-notation.tex}
\begin{definition}
  Let
    $D \in \N$,
    $V$ be a real vector space of dimension $D$,
    $v_0, ..., v_{D - 1} \in V$.
  Consider the $1$-dimensional space $\Lambda^D V$.
  Define an equivalence relation on the nonzero elements of $\Lambda^D V$
  as follows: for any $v, w \in \Lambda^D V \setminus \{0\}$,
  \begin{equation}
    v \equiv w \Leftrightarrow v / w > 0.
  \end{equation}
  This equivalence relation partitions $\Lambda^D V \setminus \{0\}$ into two
  equivalence classes
  (corresponding to positive and negative elements with respect to a choice).
  An \textbf{orientation on $V$} is a choice of $1$ of the equivalence classes
  on $\Lambda^D V$.
  Equivalently, we will also specify orientation by choosing an element on
  $\Lambda^D (V^*) \equiv (\Lambda^D V)^*$.

  An \textbf{oriented vector space} is a vector space with a chosen orientation.
\end{definition}

\begin{notation}
  Let
    $D \in \N$,
    $p \in \{0, ..., D\}$.
  By $C^D_p$ we will denote the set of all ordered lists with $p$ elements
  without repetition whose elements are from the set $\{0, ..., D - 1\}$.
\end{notation}

\begin{notation}
  Let
    $D \in \N$,
    $R$ be a commutative ring with unity,
    $V$ be a $D$-dimensional $R$-moDule,
    $p \in \{0, ..., d\}$,
    $I_p := (i_0, ..., i_{p - 1}) \in C^D_p$,
    $v_0, ..., v_{D - 1} \in V$.
  By $v_{I_p}$ we will denote the $p$-vector
  \begin{equation}
    v_{I_p} := v_{i_0} \wedge ... \wedge v_{i_{p - 1}}.
  \end{equation}
\end{notation}

\begin{proposition}
  Let
    $D \in \N$,
    $R$ be a commutative ring with unity,
    $V$ be a $D$-dimensional $R$-module,
    $e_\bullet := (e_0, ..., e_{D - 1})$ be a basis of $V$,
    $v_\bullet := (v_0, ..., v_{D - 1})$ be a set of vectors,
    $a \in M_{D \times D}(\R)$ be the transformation matrix from
      $e_\bullet$ to $v_\bullet$,
    $p \in \{0, ..., D\}$,
    $I_p \in C^D_p$.
  Then
  \begin{equation}
    v_{I_p} =
    \sum_{J_p \in C^D_p} \det \left(\restrict{a}{I_p \times J_p}\right) e_{J_p}.
  \end{equation}
\end{proposition}


\section{Inner products and Hodge star}
\label{section:inner_products_and_hodge_star}
\begin{definition}
  Let
    $R$ be a commutative ring with unity,
    $V$ be an $R$-algebra with multiplication operation $[\cdot, \cdot]$.
  We say that $V$ is a \textbf{Lie algebra} if $[\cdot, \cdot]$ is alternating
  and satisfies the \textbf{Jacobi identity}: for any $x, y, z \in V$,
  \begin{equation}
    [x, [y, z]] + [y, [z, x]] + [z, [x, y]] = 0.
  \end{equation}
\end{definition}

\input{inner_product/induced_isomorphism_with_dual-proposition.tex}
\input{inner_product/orthogonal_and_orthonormal_basis-definition.tex}
\input{exterior_algebra/inner_product-definition.tex}
\input{exterior_algebra/volume_d_vector-definition.tex}
\input{inner_product/there_are_two_volume_vectors_with_norm_one-remark.tex}
\begin{definition}
  Let
    $R$ be a commutative ring with unity,
    $V$ be an $R$-algebra with multiplication operation $[\cdot, \cdot]$.
  We say that $V$ is a \textbf{Lie algebra} if $[\cdot, \cdot]$ is alternating
  and satisfies the \textbf{Jacobi identity}: for any $x, y, z \in V$,
  \begin{equation}
    [x, [y, z]] + [y, [z, x]] + [z, [x, y]] = 0.
  \end{equation}
\end{definition}

\input{hodge_star/standard_basis_in_2d-example.tex}
\begin{proposition}[Alternative formula for Hodge star]
  Let
    $d \in \N$,
    $(V, g)$ be an oriented real inner product space of dimension $d$,
    $\vol$ be the volume $d$-vector on $\Lambda^d V$,
    $p \in \N$.
  Define the operator $\tilde{\star}_p$ as the unique operator
  \begin{equation}
    \tilde{\star}_p \colon \Lambda^p V \to \Lambda^{d - p} V
  \end{equation}
  such that for any $\omega, \eta \in \Lambda^p V$,
  \begin{equation}
    \omega \wedge \tilde{\star}_p \eta = (\Lambda^p g)(\omega, \eta) \vol.
  \end{equation}
  Then $\tilde{\star}_p = \star_p$.
\end{proposition}

\begin{proposition}
  Let
    $D \in \N$
    $(V, g)$ be an oriented real inner product space of dimension $D$,
    $p \in \N$.
  Then
  \begin{equation}
    \star_{D - p} \circ \star_p = (-1)^{p (D - p)} \id_{\Lambda^p V}.
  \end{equation}
\end{proposition}


\section{Chain complexes}
\label{section:chain_complexes}
\begin{definition}
  Let
    $R$ be a commutative ring with unity,
    $V$ be an $R$-algebra with multiplication operation $[\cdot, \cdot]$.
  We say that $V$ is a \textbf{Lie algebra} if $[\cdot, \cdot]$ is alternating
  and satisfies the \textbf{Jacobi identity}: for any $x, y, z \in V$,
  \begin{equation}
    [x, [y, z]] + [y, [z, x]] + [z, [x, y]] = 0.
  \end{equation}
\end{definition}

\input{chain_complex/tensor_product-definition.tex}
\input{chain_complex/tensor_product_is_chain_complex-proposition.tex}

\section{Differential graded algebras}
\label{section:differential_graded_algebras}
\begin{definition}
  Let
    $R$ be a commutative ring with unity,
    $V$ be an $R$-algebra with multiplication operation $[\cdot, \cdot]$.
  We say that $V$ is a \textbf{Lie algebra} if $[\cdot, \cdot]$ is alternating
  and satisfies the \textbf{Jacobi identity}: for any $x, y, z \in V$,
  \begin{equation}
    [x, [y, z]] + [y, [z, x]] + [z, [x, y]] = 0.
  \end{equation}
\end{definition}

\input{differential_graded_algebra/tensor_product-definition.tex}
\input{differential_graded_algebra/tensor_product_is_differential_graded_algebra-proposition.tex}

\part{Category theory}

\section{Functors}
\label{section:functors}
\begin{definition}
  Let $\mathcal{C}$ and $\mathcal{D}$ be categories.
  A \textbf{functor} $F$ is a pair $(F_{\rm Obj}, F_{\rm Mor})$, where
  $F_{\rm Obj} \colon {\rm Obj}_{\mathcal{C}} \to {\rm Obj}_{\mathcal{D}}$ and
  $F_{\rm Mor} \colon {\rm Mor}_{\mathcal{C}} \to {\rm Mor}_{\mathcal{D}}$
  and the following conditions are satisfied.
  \begin{enumerate}
    \item
      \textbf{$F$ maps domains and codomains accordingly.}
      For any $f \in {\rm Mor}_{\mathcal{C}}$,
      \begin{subequations}
        \begin{alignat}{1}
          & {\rm Domain}_{\mathcal{D}}(F_{\rm Mor} f)
          = F_{\rm Obj}({\rm Domain}_{\mathcal{D}} f), \\
          & {\rm Codomain}_{\mathcal{D}}(F_{\rm Mor} f)
          = F_{\rm Obj}({\rm Codomain}_{\mathcal{D}} f).
        \end{alignat}
      \end{subequations}
    \item
      \textbf{$F$ respects identities.}
      For any $X \in {\rm Obj}_{\mathcal{C}}$,
      \begin{equation}
        F_{\rm Mor} \id_X = \id_{F_{\rm Obj} X}
      \end{equation}
    \item
      \textbf{$F$ respects composition.}
      for any $f, g \in {\rm Mor}_{\mathcal{C}}$, with
      ${\rm Domain} g = {\rm Codomain} f$,
      \begin{equation}
        F(g \circ f) = F g \circ F f.
      \end{equation}
  \end{enumerate}
  It is a common convention to abuse the notation and use the single symbol $F$
  for both $(F_{\rm Obj}$ and $F_{\rm Mor})$.
\end{definition}
\begin{example}[Forgetful functors]
  Let $\mathcal{C}$ be a category of algebraic structures
  (e.g., groups, rings, modules, etc.).
  We define the forgetful functor $F$ as follows: it maps an algebraic structure
  to its carrier set (e.g., the group $(\R, +, -, 0)$ to the set $\R$) and a
  mrphism to itself.
  All the conditions for a functor are trivially satisfied.
\end{example}
\begin{example}[Power-set functor]
  In the category of sets consider the endofunctor
  $F = (F_{\rm Obj}, F_{\rm Mor})$ defined as follows.
  For a set $X$ define $F_{\rm Obj}(X) = \mathcal{P} X$ and for a function
  $f \colon X \to Y$ define $F f \colon \mathcal{P} X \to \mathcal{P} Y$ by
  \begin{equation}
    A \in \mathcal{P} X
    \mapsto (F f) A := \set{f(x)}{x \in A} \in \mathcal{P} Y.
  \end{equation}
\end{example}


\section{Universal algebraic constructions}
\label{section:universal_algebraic_constructions}
\begin{definition}
  Let
    $R$ be a commutative ring with unity,
    $V$ be a module over $R$,
    $E$ be an associative algebra with unity,
    $\iota \colon V \to E$ be an $R$-linear map.
  We say that $E$ is a \textbf{tensor algebra of $V$} if for any associative
  algebra $A$ and any $R$-linear $f \colon V \to A$,
  there exists a unique homomorphism of algebras with unity
  $F \colon E \to A$ such that $f = F \circ \iota$.
\end{definition}
\begin{proposition}
  Any two tensor algebras over $V$ are isomorphic.
\end{proposition}
\begin{proof}
  This is a universal property.
\end{proof}
\begin{notation}
  Let
    $R$ be a commutative ring with unity,
    $V$ be a module over $R$.
  By $T^\bullet V$  we denote free associative algebra with unity over $V$.
\end{notation}
\begin{proposition}
  Let
    $R$ be a commutative ring with unity,
    $V$ be a module over $R$.
  Then $T^\bullet V$ is a tensor algebra of $V$.
  (And it will be used as a canonical representative, i.e., as
  \emph{the} tensor algebra of $V$.)
\end{proposition}
\begin{definition}
  Let
    $R$ be a commutative ring with unity,
    $V$ be a module over $R$,
    $E$ be an alternating and associative algebra,
    $\iota \colon V \to E$ be an $R$-linear map.
  We say that $E$ is an \textbf{exterior algebra of $V$} if for any alternating
  and associative algebra $A$ and any $R$-linear map $f \colon V \to A$,
  there exists a unique alternating algebra homomorphism
  $F \colon E \to A$ such that $f = F \circ \iota$.
\end{definition}
\begin{proposition}
  Any two exterior algebras over $V$ are isomorphic.
\end{proposition}
\begin{proof}
  This is a universal property.
\end{proof}
\begin{notation}
  Let
    $R$ be a commutative ring with unity,
    $V$ be a module over $R$,
    $I$ be the ideal over the tensor algebra $T^\bullet V$ on $M$ defined to be
    the smallest ideal of $T^\bullet V$ such that for any $v \in V$,
    $v \otimes v \in I$.
  Define $\Lambda^\bullet V := T^\bullet V / I$.
\end{notation}
\begin{proposition}
  Let
    $R$ be a commutative ring with unity,
    $V$ be a module over $R$.
  Then $\Lambda^\bullet V$ is an exterior algebra on $V$.
  (And it will be used as a canonical representative, i.e., as
  \emph{the} exterior algebra of $V$.)
\end{proposition}


\part{Real analysis}

\section{Metric spaces}
\label{section:metric_spaces}
\begin{definition}
  Let
    $R$ be a ring,
    $V$ be a finite-dimensional $R$-module,
    $\omega \in \Lambda^2 V^*$.
  We say that $\omega$ is \textbf{non-degenerate} or \textbf{symplectic}
  if the associated map
  \begin{equation}
    \tilde{\omega} \colon V \to V^*,\
    X \in V \mapsto \tilde{\omega}(X) := i_X \omega \in V^*,
  \end{equation}
  is an isomorphism.

  The pair $(V, \omega)$ is called a \textbf{symplectic module}
  (or \textbf{symplectic vector space} if $R$ is a field).
\end{definition}
\begin{proposition}
  Let
    $R$ be a ring without,
    $(V, \omega)$ be a finite-dimensional symplectic module over $R$.
  Assume that for any $x \in R,\ x + x = 0 \Rightarrow x = 0$.
  Then $\dim V$ is an even number.
\end{proposition}
\begin{proof}
  Let $n = \dim V$.
  In a basis of $V$ $\omega$ is represented by an antisymmetric matrix $A$.
  But then
  \begin{equation}
    \det A = \det(A^T) = \det(-A) = (-1)^n \det A.
  \end{equation}
  If $n$ is odd, then $\det A + \det A = 0$.
  By assumption this means that $\det A = 0$
  which contradicts the nondegeneracy of $\omega$.
  Hence, $n$ is even.
\end{proof}
\begin{definition}
  Let $M$ be a smooth manifold, $\omega \in \Omega^\bullet M$.
  We say that:
  \begin{enumerate}
    \item
      $\omega$ is \textbf{closed} if $d \omega = 0$
    \item
      $\omega$ is \textbf{exact} if there exists $\eta \in \Omega^\bullet M$
      such that $d \eta = \omega$.
  \end{enumerate}
\end{definition}
\begin{proposition}
  Let $M$ be a smooth manifold, $\omega \in \Omega^\bullet M$.
  If $\omega$ is exact, then it is closed.
\end{proposition}
\begin{proof}
  Let $\eta \in \Omega^\bullet M$ be such that $d \eta = \omega$.
  Then $d \omega = d (d \eta) = 0$, i.e., $\omega$ is closed.
\end{proof}
\begin{definition}
  Let $M$ be a smooth manifold, $\omega \in \Omega^2 M$.
  We say that $\omega$ is a \textbf{symplectic form}
  if it is non-degenerate
  (with base module $\mathfrak{X} M$ over $\mathcal{F} M$) and closed.

  The pair $(M, \omega)$ is called a \textbf{symplectic manifold}.
\end{definition}
\begin{proposition}
  Let $(M, \omega)$ be a symplectic manifold.
  Then $M$ is even-dimensional.
\end{proposition}
\begin{definition}
  Let $Q$ be a smooth manifold.
  Consider the cotangent bundle $T^* Q$ with bundle projection
  $\pi \colon T^* Q \to Q$
  with differential $d \pi \colon T(T^* Q) \to T Q$.
  Define the \textbf{tautological one-form}
  $\theta \colon T^* Q \to T^* (T^* Q)$ as follows:
  for any $(q, p) \in T^*Q$ (i.e,. $q \in Q$, $p \in \Hom(T_q Q, \R)$),
  \begin{equation}
    \restrict{\theta}{(q, p)}
    := p \circ \restrict{d \pi}{(q, p)} \in T^*_{(q, p)}(T^* Q).
  \end{equation}
  In other words, if we denote $M := T^* Q$, then $\theta$ is a section of its
  cotangent bundle $T^* M$, i.e., an $1$-form on $M$. 
\end{definition}
\begin{discussion}
  Let $Q$ be a smooth manifold, $\pi \colon T^* Q \to Q$ be the projection.
  Then a $1$-form on $Q$ is a section of $\colon T^* Q$, i.e., a smooth map
  $\mu \colon Q \to \colon T^* Q$ such that $\pi \circ \mu = \id_Q$.
  As such it has a pullback
  $\mu^* \colon \Omega^\bullet(T^* Q) \to \Omega^\bullet Q$.
\end{discussion}
\begin{proposition}
  Let
    $Q$ be a smooth manifold,
    $\theta$ be the tautological one-form on $T^* Q$,
    $\mu \in \Omega^1 Q$.
  Then
  \begin{equation}
    \mu^* \theta = \mu.
  \end{equation}
\end{proposition}
\begin{proof}
  Let $q \in Q$.
  Then
  \begin{equation}
    \restrict{\mu^* \theta}{q}
    = \restrict{\theta}{\mu q} \circ \restrict{d \mu}{q}
    = \restrict{\mu}{q} \circ \restrict{(d \pi)}{\mu q}
      \circ \restrict{d \mu}{q}
    = \restrict{\mu}{q} \circ \restrict{d(\pi \circ \mu)}{q}
    = \restrict{\mu}{q}.
  \end{equation}
  Since $q$ is arbitrary, $\mu^* \theta = \mu$.
\end{proof}
\begin{definition}
  Let
    $Q$ be a smooth manifold,
    $\theta \in \Omega^1(T^* Q)$ be the tautological one-form.
  Define $\omega := - d \theta$.
  The pair $(T^* Q, \omega)$ is called the \textbf{phase space} of $Q$.
  (In this setting $Q$ is usually called the \textbf{configuration space}.)
\end{definition}
\begin{remark}
  Let $Q$ be a smooth manifold.
  The elements of $T^* Q$ are of the form $(q, p)$ where $q \in Q$ and
  $p \in T^*_q Q = \Hom(T_q Q, \R)$.
  $q$ is called \textbf{generalised position}, while $p$ is called
  \textbf{generalised momentum}.
\end{remark}
\begin{proposition}
  Let
    $Q$ be a smooth manifold,
    $(T^* Q, \omega)$ be its phase space.
  Then $(T^* Q, \omega)$ is a symplectic manifold.
\end{proposition}
\begin{definition}
  Let $Q$ be a smooth manifold of dimension $n$.
  Consider a point $q_0 \in Q$ and let $(U, \hat{\varphi})$ be a chart around
  $q_0$, i.e., $U$ is a neighbourhood of $q_0$ and
  $\hat{\varphi} \colon U \to \R^n$ is a diffeomorphism.
  Let $\{\hat{q}^i \colon U \to \R\}_{i = 1}^n$ be the corresponding local
  coordinates, i.e., if $\{\pi^i \colon \R^n \to \R\}_{i = 1}^n$ are the
  projection maps, then $\{\hat{q}^i = \pi^i \circ \hat{\varphi}\}_{i = 1}^n$.
  Let $i \in \{1, ..., n\}$.
  Define \textbf{position coordinate} $q^i \colon T^* U \to \R$ by
  \begin{equation}
    q^i := \hat{q}^i \circ \restrict{\pi}{U}.
  \end{equation}
  Also, define \textbf{momentum coordinate} $p_i \colon T^* U \to \R$
  as follows: for any $(q, p) \in T^* U$,
  \begin{equation}
    p_i(q, p)
    := p\left(\restrict{\frac{\partial}{\partial \hat{q}^i}}{q}\right).
  \end{equation}
\end{definition}
\begin{proposition}
  Let
    $Q$ be a smooth manifold of dimension $n$,
    $q_0 \in Q$,
    $(U, \hat{\varphi})$ be a chart around $q_0$,
    $\{\hat{q}^i \colon U \to \R\}_{i = 1}^n$ be the corresponding local
      coordinates,
    $\{q^i \colon T^* U \to \R\}_{i = 1}^n$ be the corresponding position
      coordinates,
    $\{p_i \colon T^* U \to \R\}_{i = 1}^n$ be the corresponding momentum
      coordinates.
  Then the map $\varphi \colon T^* U \to \R^{2 n}$ defined by
  \begin{equation}
    \varphi(q, p) = (q^1(q, p), ..., q^n(q, p), p_1(q, p), ..., p_n(q, p))
  \end{equation}
  is a diffeomorphism, i.e., $(T^* U, \varphi)$ is a chart around $(q_0, 0)$.
  (The covector in $T^*_{q_0}$ does not matter, so we make the trivial choice by
  taking zero.)

  These local coordinates are called \textbf{generalised coordinates}.
\end{proposition}
\begin{remark}
  From now on, given a manifold $Q$ and a chart $(U, \hat{\varphi})$, unless
  stated otherwise, we will fix the notation and use the objects defined above:
  the projection map $\pi \colon T^* Q \to Q$, the tautological one-form
  $\theta$ and the canonical symplectic form $\omega = - d \theta$;
  for $i = 1, ..., n$ the coordinate maps $\hat{q}^i$, $q^i$, and $p_i$;
  the chart $(T^* U, \varphi)$.
\end{remark}
\begin{proposition}
  Let
    $Q$ be a smooth manifold,
    $\xi \in \Omega^1(T^* Q)$ has the following property:
    for any $1$-form $\mu$ on $Q$, $\mu^* \xi= 0$.
  Then $\xi = 0$.
\end{proposition}
\begin{proof}
  Let
    $n := \dim Q$, $(U, \hat{\varphi})$ be a chart on $Q$ and
    $\{f_i, g^i \in \mathcal{F}(T^* U)\}_{i = 1}^n$ be such that
  \begin{equation}
    \restrict{\xi}{U} = \sum_{i = 1}^n f_i\, d q^i + \sum_{i = 1}^n g^i\, d p_i.
  \end{equation}
  Take arbitrary $\{h_j \in \mathcal{F} U\}_{j = 1}^n$ so that
  \begin{equation}
    \restrict{\mu}{U} = \sum_{j = 1}^n h_j\, d \hat{q}^j.
  \end{equation}
  Note that
  $q^i \circ \restrict{\mu}{U} = \hat{q}^i$ and
  $p_i \circ \restrict{\mu}{U} = h_i$.
  Hence,
  \begin{equation}
    0 
    = \restrict{(\mu^* \xi)}{U}
    = \sum_{i = 1}^n (f_i \circ \restrict{\mu}{U})\,
      d(q^i \circ \restrict{\mu}{U})
    + \sum_{i = 1}^n (g^i \circ \restrict{\mu}{U})\,
      d(p_i \circ \restrict{\mu}{U})
    = \sum_{i = 1}^n (f_i \circ \restrict{\mu}{U})\, d \hat{q}^i
    + \sum_{i = 1}^n (g^i \circ \restrict{\mu}{U})\, d h_i.
  \end{equation}
  Fix $q_0 \in U$, $p_0 \in T^*_{q_0} Q$ so that $(p_0, q_0) \in T^* U$.
  Denote
  \begin{equation}
    c_i
    :=
    p_0\left(\restrict{\frac{\partial}{\partial \hat{q}^i}}{q_0}\right),
    i = 1, ..., n,
  \end{equation}
  so that
  \begin{equation}
    p_0 = \sum_{i = 1}^n c_i \restrict{d \hat{q}^i}{q_0}.
  \end{equation}
  \begin{enumerate}
    \item
      We will first prove that
      for any $i \in \{1, ..., n\}$, $f_i(q_0, p_0) = 0$.
      Define the constant functions
      \begin{equation}
        h_i(q) := c_i,\ i \in \{1, ..., n\},\ q \in U.
      \end{equation}
      Then for any $i \in \{1, ..., n\}$, $d h_i = 0$.
      Hence,
      \begin{equation}
        \begin{split}
          0
          & = \restrict{(\mu^* \xi)}{q_0} \\
          & = \sum_{i = 1}^n
              f_i(q_0, \sum_{j = 1}^n h_j(q_0) \restrict{d \hat{q}^j}{q_0})\,
              \restrict{d \hat{q}^i}{q_0} \\
          & = \sum_{i = 1}^n
              f_i(q_0, \sum_{j = 1}^n c_j \restrict{d \hat{q}^j}{q_0})\,
              \restrict{d \hat{q}^i}{q_0} \\
          & = \sum_{i = 1}^n f_i(q_0, p_0)\, \restrict{d \hat{q}^i}{q_0}.
        \end{split}
      \end{equation}
      Therefore, for any $i \in \{1, ..., n\}$, $f_i(q_0, p_0) = 0$.
    \item
      We will now prove that
      for any $i \in \{1, ..., n\}$, $g^i(q_0, p_0) = 0$.
      Define the linear functions
      \begin{equation}
        h_i(q) := c_i + \hat{q}^i(q) - \hat{q}^i(q_0).
      \end{equation}
      Then for any $i \in \{1, ..., n\}$,
      $d h_i = d \hat{q}^i$ and $h_i(q) = c_i$.
      Hence,
      \begin{equation}
        \begin{split}
          0
          & = \restrict{(\mu^* \xi)}{q_0} \\
          & = \sum_{i = 1}^n
              g^i(q_0, \sum_{j = 1}^n h_j(q_0) \restrict{d \hat{q}^j}{q_0})\,
              \restrict{d h_i}{q_0} \\
          & = \sum_{i = 1}^n
              g^i(q_0, \sum_{j = 1}^n c_j \restrict{d \hat{q}^j}{q_0})\,
              \restrict{d \hat{q}^i}{q_0} \\
          & = \sum_{i = 1}^n g^i(q_0, p_0)\, \restrict{d \hat{q}^i}{q_0}.
        \end{split}
      \end{equation}
      Therefore, for any $i \in \{1, ..., n\}$, $g^i(q_0, p_0) = 0$.
  \end{enumerate}
  Since $(q_0, p_0) \in T^* U$ was arbitrary, we conclude that
  for any $i \in \{1, ..., n\}$, $f_i = g^i = 0$.
  Hence, $\restrict{\xi}{U} = 0$.
  Taking an atlas $\{(U_\alpha, \hat{\varphi}_\alpha)\}_{\alpha \in A}$ of $Q$
  (for some index set $A$), we conclude that $\xi = 0$.
\end{proof}
\begin{corollary}
  Let
    $Q$ be a smooth manifold,
    $\theta$ be the tautological one-form on $T^* Q$,
    $\eta \in \Omega^1(T^* Q)$ has the following property:
    for any $1$-form $\mu$ on $Q$, $\mu^* \eta = \mu$.
  Then $\eta = \theta$.
\end{corollary}
\begin{proof}
  Write $\eta = \theta + \xi$, i.e., $\xi := \eta - \theta$.
  Then, for any $\mu \in \Omega^1 Q$,
  \begin{equation}
    \mu
    = \mu^* \eta
    = \mu^* \theta + \mu^* \xi
    = \mu + \mu^* \xi
    \Rightarrow \mu^* \xi = 0.
  \end{equation}
  But from the previous proposition it follows that $\xi = 0$,
  and hence $\eta = \theta$.
\end{proof}
\begin{proposition}
  Let
    $Q$ be a smooth manifold of dimension $n$,
    $(U, \hat{\varphi})$ be a chart,
    $(q, p) \in T^* U$,
    $ i \in \{1, ..., n\}$.
  Then
  \begin{equation}
    \restrict{d \pi}{(q, p)}
    \left(\restrict{\frac{\partial}{\partial q^i}}{(q, p)}\right)
    = \restrict{\frac{\partial}{\partial \hat{q}^i}}{q}
  \end{equation}
  and
  \begin{equation}
    \restrict{d \pi}{(q, p)}
    \left(\restrict{\frac{\partial}{\partial p^i}}{(q, p)}\right)
    = 0.
  \end{equation}
\end{proposition}
\begin{proof}
  Let $f \colon Q \to \R$ be smooth.
  Define the functions
  $\hat{g} := f \circ \hat{\varphi}^{-1} \colon \R^n \to \R$ and
  $g := f \circ \pi \circ \varphi^{-1} \colon \R^{2 n} \to \R$.
  Let $(X^1, ..., X^n, Y^1, ..., Y^n) := \varphi(p, q) \in \R^n$.
  This means that $(X^1, ..., X^n) = \hat{\varphi}(q)$.
  Then
  \begin{equation}
    g(X^1, ..., X^n, Y^1, ..., Y^n)
    = f(\pi(q, p))
    = f(q)
    = \hat{g}(X^1, ..., X^n).
  \end{equation}
  Hence,
  \begin{equation}
    \begin{split}
      \frac{\partial g}{\partial x^i}(X^1, ..., X^n, Y^1, ..., Y^n)
      & = \lim_{h \to 0}
        \frac
        {g(X^1, ..., X^i + h, ..., X^n, Y^1, ..., Y^n)
         - g(X^1, ..., X^n, Y^1, ..., Y^n)}
        {h} \\
      & = \lim_{h \to 0}
        \frac{\hat{g}(X^1, ..., X^i + h, ..., X^n) - \hat{g}(X^1, ..., X^n)}{h}
        \\
      & = \frac{\partial \hat{g}}{\partial \hat{x}^i}(X^1, ..., X^n).
    \end{split}
  \end{equation}
  Similarly, since $g$ is constant with respect to the last $n$ coordinates,
  \begin{equation}
    \frac{\partial g}{\partial x^{n + i}}(X^1, ..., X^n, Y^1, ..., Y^n) = 0
  \end{equation}
  Take the standard coordinate systems (given by projections)
  $\{\hat{x}^k\}_{k = 1}^n$ on $\R^n$ and
  $\{x^k\}_{k = 1}^{2 n}$ on $\R^{2 n}$.
  Then, by the definitions of differential and partial derivative on manifold,
  \begin{equation}
    (\restrict{d \pi}{(q, p)}
      \left(\restrict{\frac{\partial}{\partial q^i}}{(q, p)}\right)) f
    = \restrict{\frac{\partial}{\partial q^i}}{(q, p)}(f \circ \pi)
    = \frac{\partial(f \circ \pi \circ \varphi^{-1})}{x^i}(\varphi(q, p))
    = \frac{\partial(f \circ \hat{\varphi}^{-1})}{\hat{x}^i}(\hat{\varphi}(q))
    = \restrict{\frac{\partial}{\partial \hat{q}^i}}{q} f,
  \end{equation}
  from which it follows that the first equality holds.
  Similarly,
  \begin{equation}
    (\restrict{d \pi}{(q, p)}
      \left(\restrict{\frac{\partial}{\partial p^i}}{(q, p)}\right)) f
    = \restrict{\frac{\partial}{\partial p^i}}{(q, p)}(f \circ \pi)
    = \frac{\partial(f \circ \pi \circ \varphi^{-1})}{x^{i + n}}(\varphi(q, p))
    = 0,
  \end{equation}
  from which it follows that the second equality holds.
\end{proof}
\begin{proposition}[Tautological one-form in generalised coordinates]
  Let
    $Q$ be a smooth manifold of dimension $n$,
    $(U, \hat{\varphi})$ be a chart.
  Then
  \begin{equation}
    \restrict{\theta}{U} = \sum_{i = 1}^n p_i\, d q^i.
  \end{equation}
\end{proposition}
\begin{proof}
  Let $(q, p) \in T^* U$.
  Recall that $\restrict{\theta}{(q, p)} = p \circ \restrict{d \pi}{(q, p)}$.
  Hence,
  \begin{equation}
    \restrict{\theta}{(q, p)}
    \left(\restrict{\frac{\partial}{\partial q^i}}{(q, p)}\right)
    = p\left(\restrict{\frac{\partial}{\partial \hat{q}^i}}{q}\right)
    = p_i(q, p),
  \end{equation}
  and
  \begin{equation}
    \restrict{\theta}{(q, p)}
    \left(\restrict{\frac{\partial}{\partial p^i}}{(q, p)}\right)
    = p(0)
    = 0.
  \end{equation}
  Therefore,
  \begin{equation}
    \restrict{\theta}{(q, p)}
    = \sum_{i = 1}^n
      \restrict{\theta}{(q, p)}
      \left(\restrict{\frac{\partial}{\partial q^i}}{(q, p)}\right)\,
      \restrict{d q^i}{(q, p)}
    + \sum_{i = 1}^n
      \restrict{\theta}{(q, p)}
      \left(\restrict{\frac{\partial}{\partial p^i}}{(q, p)}\right)\,
      \restrict{d p^i}{(q, p)}
    = \sum_{i = 1}^n p_i(q, p)\, \restrict{d q^i}{(q, p)},
  \end{equation}
  from which the proposition follows.
\end{proof}
\begin{corollary}[Canonical symplectic in generalised coordinates]
  Let
    $Q$ be a smooth manifold of dimension $n$,
    $(U, \hat{\varphi})$ be a chart.
  Then
  \begin{equation}
    \restrict{\omega}{U} = \sum_{i = 1}^n d q^i \wedge d p_i.
  \end{equation}
\end{corollary}
\begin{proof}
  Let $i \in \{1, ..., n\}$.
  Then
  \begin{equation}
    - d(p_i\, d q^i) = - d p_i \wedge d q^i = d q^i \wedge d p_i.
  \end{equation}
  Summing up for all $i$, we get the desired result.
\end{proof}
\begin{definition}
  Let $(M, \omega)$ be a symplectic manifold, $f \in \mathcal{F} M$.
  We say that $X \in \mathfrak{X} M$ is a \textbf{Hamiltonian vector field} for
  $f$ if
  \begin{equation}
    i_X \omega + d_0 f = 0.
  \end{equation}
\end{definition}
\begin{proposition}
  Let $(M, \omega)$ be a symplectic manifold, $f \in \mathcal{F} M$.
  Then there exists a unique Hamiltonian vector field for $f$.
\end{proposition}
\begin{proof}
  The non-degeneracy of $\omega$ means that we can interpret the symplectic form
  as the isomorphism $\tilde{\omega} \colon \mathfrak{X} M \to \Omega^1 M$,
  given by
  \begin{equation}
    (\tilde{\omega} X) := i_X \omega,\ X \in \mathfrak{X} M.
  \end{equation}
  Hence, the problem at hand has a unique solution
  $X = \tilde{\omega}^{-1}(- d_0 f)$.
\end{proof}
\begin{definition}
  Let $(M, \omega)$ be a symplectic manifold.
  Define the map $\hamiltonian \colon \mathcal{F} M \to \mathfrak{X} M$ by
  \begin{equation}
    \hamiltonian = - \tilde{\omega}^{-1} \circ d_0.
  \end{equation}
  It maps a function to its corresponding Hamiltonian vector field.
  We will write $\hamiltonian_f$ instead of $\hamiltonian(f)$
  for $f \in \mathcal{F} M$.
\end{definition}
\begin{proposition}
  Let
    $Q$ be a smooth manifold of dimension $n$,
    $(U, \hat{\varphi})$ be a chart on $Q$,
    $f \in \mathcal{F}(T^* Q)$.
  Then
  \begin{equation}
    \hamiltonian_f
    = \sum_{i = 1}^n
    \left(
      - \frac{\partial f}{\partial p^i} \frac{\partial}{\partial q^i}
      + \frac{\partial f}{\partial q^i} \frac{\partial}{\partial p^i}
    \right).
  \end{equation}
\end{proposition}
\begin{proof}
  First, note that
  $i_{\frac{\partial}{\partial q^i}} \omega = d p^i$ and
  $i_{\frac{\partial}{\partial p^i}} \omega = - d q^i$.
  Denote
  \begin{equation}
    X
    := \sum_{i = 1}^n
    \left(
      - \frac{\partial f}{\partial p^i} \frac{\partial}{\partial q^i}
      + \frac{\partial f}{\partial q^i} \frac{\partial}{\partial p^i}
    \right).
  \end{equation}
  Then
  \begin{equation}
    i_X \omega
    = \sum_{i}^n
    \left(
      - \frac{\partial f}{\partial p^i} i_{\frac{\partial}{\partial q^i}} \omega
      + \frac{\partial f}{\partial q^i} i_{\frac{\partial}{\partial p^i}} \omega
    \right)
    = \sum_{i}^n
    \left(
      - \frac{\partial f}{\partial p^i} d p^i
      - \frac{\partial f}{\partial q^i} d q^i
    \right)
    = - d f.
  \end{equation}
  Hence, $\hamiltonian_f = X$.
\end{proof}
\begin{proposition}
  Let $(M, \omega)$ be a symplectic manifold, $f, g \in \mathcal{F} M$.
  Then
  \begin{equation}
    \hamiltonian_{f g} = f \hamiltonian_g + g \hamiltonian_f.
  \end{equation}
\end{proposition}
\begin{proof}
  Follows directly from the Leibniz rule for $d_0$.
\end{proof}
\begin{definition}
  Let $(M, \omega)$ be a symplectic manifold, $X \in \mathfrak{X} M$.
  We say that $X$ is a \textbf{symplectic vector field} if $L_X \omega = 0$.
\end{definition}
\begin{remark}
  Since $L_{\lie{X}{Y}} = \lie{L_X}{L_Y} = L_X \circ L_Y - L_Y \circ L_X$,
  the symplectic vector fields form a Lie subalgebra of the Lie algebra of
  vector fields.
\end{remark}
\begin{proposition}
  Let $(M, \omega)$ be a symplectic manifold, $f \in \mathcal{F} M$.
  Then $\hamiltonian_f$ is a symplectic vector field.
\end{proposition}
\begin{proof}
  $
    L_{\hamiltonian_f} \omega
    = i_{\hamiltonian_f}(d \omega) + d(i_{\hamiltonian_f} \omega)
    = i_{\hamiltonian_f} 0 - d(d f)
    = 0.
  $
\end{proof}
\begin{proposition}
  Let
    $(M, \omega)$ be a symplectic manifold,
    $X \in \mathfrak{X} M$ be a symplectic vector fields.
  Then
  \begin{equation}
    d(i_X \omega) = 0.
  \end{equation}
\end{proposition}
\begin{proof}
  $
    d(i_X \omega)
    = L_X \omega - i_X(d \omega)
    = 0 - 0
    = 0.
  $
\end{proof}
\begin{proposition}
  Let $M$ be a smooth manifold, $X, Y \in \mathfrak{X} M$.
  Then
  \begin{equation}
    L_X \circ i_Y = i_{\lie{X}{Y}} + i_Y \circ L_X.
  \end{equation}
\end{proposition}
\begin{proposition}
  Let
    $(M, \omega)$ be a symplectic manifold,
    $X, Y \in \mathfrak{X} M$ be symplectic vector fields.
  Then
  \begin{equation}
    \lie{X}{Y} = \hamiltonian_{i_Y(i_X \omega)}.
  \end{equation}
\end{proposition}
\begin{proof}
  \begin{equation}
    i_{\lie{X}{Y}} \omega
    = (L_X \circ i_Y - i_Y \circ L_X) \omega
    = (L_X \circ i_Y) \omega
    = ((d \circ i_X + i_X \circ d) \circ i_Y) \omega
    = d(i_X(i_Y \omega))
    = - d(i_Y(i_X \omega)).
  \end{equation}
  We get the desired result from the definition of $\hamiltonian$.
\end{proof}
\begin{definition}
  Let $(M, \omega)$ be a symplectic manifold.
  Define the \textbf{Poisson bracket}
  $\poisson{\cdot}{\cdot} \colon \mathcal{F} M \to \mathcal{F} M$ by
  \begin{equation}
    \poisson{f}{g}
    := i_{\hamiltonian_g}(i_{\hamiltonian_f} \omega),\
    f, g \in \mathcal{F} M.
  \end{equation}
\end{definition}
\begin{corollary}
  Let $(M, \omega)$ be a symplectic manifold, $f, g \in \mathcal{F} M$.
  Then
  \begin{equation}
    \lie{\hamiltonian_f}{\hamiltonian_g}
    = \hamiltonian_{i_{\hamiltonian_g}(i_{\hamiltonian_f} \omega)}
    = \poisson{f}{g}.
  \end{equation}
\end{corollary}
\begin{proposition}[Leibniz rule holds for the Poisson bracket]
  Let $(M, \omega)$ be a symplectic manifold, $f, g, h \in \mathcal{F} M$.
  Then
  \begin{equation}
    \poisson{f}{g h} = \poisson{f}{g} h + g \poisson{f}{h}.
  \end{equation}
\end{proposition}
\begin{proof}
  $
    \poisson{f}{g h}
    = i_{\hamiltonian_{g h}}(i_{\hamiltonian_f} \omega)
    = i_{h \hamiltonian_g + g \hamiltonian_{h}}(i_{\hamiltonian_f} \omega)
    = (i_{\hamiltonian_g}(i_{\hamiltonian_f} \omega))\, h
      + g\, (i_{\hamiltonian_h}(i_{\hamiltonian_f} \omega)) 
    = \poisson{f}{g} h + g \poisson{f}{h}.
  $
\end{proof}
\begin{proposition}
  Let $(M, \omega)$ be a symplectic manifold, $f, g, h \in \mathcal{F} M$.
  Then
  \begin{equation}
    \poisson{f}{g} = L_{\hamiltonian_f} g.
  \end{equation}
\end{proposition}
\begin{proof}
  $
    \poisson{f}{g}
    = i_{\hamiltonian_g}(i_{\hamiltonian_f} \omega)
    = - i_{\hamiltonian_f} \circ i_{\hamiltonian_g} \omega
    = i_{\hamiltonian_f}(d g)
    = L_{\hamiltonian_f} g.
  $
\end{proof}
\begin{definition}
  Let $(M, \omega)$ be a symplectic manifold.
  Define
  ${\rm ad} \colon \mathcal{F} M \to (\mathcal{F} M \to \mathcal{F} M)$ by
  \begin{equation}
    {\rm ad}_f g := \poisson{f}{g},\ f, g \in \mathcal{F} M,
  \end{equation}
\end{definition}
\begin{proposition}
  Let $(M, \omega)$ be a symplectic manifold.
  Then
  \begin{equation}
    \lie{{\rm ad}_f}{{\rm ad}_g} = {\rm ad}_{\poisson{f}{g}}.
  \end{equation}
  (Here the bracket $\lie{\cdot}{\cdot}$ is the commutator of operators.)
\end{proposition}
\begin{proof}
  From the previous proposition it follows that
  \begin{equation}
    {\rm ad}_f = L_{X_f},\ f \in \mathcal{F} M.
  \end{equation}
  Hence,
  \begin{equation}
    \lie{{\rm ad}_f}{{\rm ad}_g}
    = \lie{L_{\hamiltonian_f}}{L_{\hamiltonian_g}}
    = L_{\lie{\hamiltonian_f}{\hamiltonian_g}}
    = L_{\hamiltonian_{\poisson{f}{g}}}
    = {\rm ad}_{\poisson{f}{g}}.
  \end{equation}
\end{proof}
\begin{corollary}
  Let $(M, \omega)$ be a symplectic manifold.
  Then $(\mathcal{F} M, \poisson{\cdot}{\cdot})$ is a Lie algebra over $\R$.
\end{corollary}
\begin{proof}
  Bilinearity and antisymmetry are trivial to check.
  The Jacobi identity is equivalent to the adjoint map being a Lie algebra
  homomorphism, which was the previous proposition.
\end{proof}
\begin{definition}
  Let
    $R$ be a commutative ring with unity ring,
    $(A, +, \cdot)$ be an $R$-module
    with additional structures of
    an associative algebra $(A, *)$ and
    a Lie algebra $(A, \poisson{\cdot}{\cdot})$.
  We say that $A$ is a Poisson algebra if the Lie bracket acts as a derivation,
  i.e., for all $f, g, h \in A$,
  \begin{equation}
    \poisson{f}{g * h} = \poisson{f}{g} * h + g * \poisson{f}{h}.
  \end{equation}
\end{definition}
\begin{corollary}
  Let $(M, \omega)$ be a symplectic manifold.
  Then $\mathcal{F} M$ is a Poisson algebra over $\R$.
  Here, addition, scalar multiplication, and multiplication are given by the
  corresponding pointwise operations, while the Lie bracket is given by the
  Poisson bracket.
\end{corollary}


\section{More on Cauchy sequences and complete metric spaces}
\label{section:cauchy_sequences}
\begin{discussion}
  We already proved that any converging sequence is a Cauchy sequence.
  The converse, i.e., that any Cauchy sequence has a limit,
  is not true in general.
  (This motivated the introduction of complete metric spaces.)
  For instance, there are Cauchy sequences of rational numbers that
  do not converge to a rational number.
  For example, consider the sequence $a \colon \N \to \Q$,
  \begin{subequations}
    \begin{alignat}{1}
      & a_0 := 1, \\
      & a_{n + 1} := \frac{1}{a_n} + \frac{a_n}{2},\ n \in \N.
    \end{alignat}
  \end{subequations}
  Let $d \colon \Q \times \Q \to \Q$ be the standard metric given by
  \begin{equation}
    d(x, y) := \abs{x - y},\ x, y \in \Q.
  \end{equation}
  (We define the codomain to be $\Q$ since we assume $\R$ is not constructed
  yet.)
  It is not hard to check that $a \in \Cauchy(\Q, d)$.
  However, assume it has a limit $A \in \Q$.
  Then, it will satisfy the relation
  \begin{equation}
    A = \frac{1}{A} + \frac{A}{2}
    \Rightarrow A = \frac{2 + A^2}{2 A}
    \Rightarrow 2 A^2 = 2 + A^2
    \Rightarrow A^2 = 2.
  \end{equation}
  But there are is no rational whose square is $2$.
  To bee able to find the square root of $2$, we need to \emph{complete} the
  space of rationals.
\end{discussion}
\begin{proposition}
  Let $(X, d)$ be a metric space $Y \subseteq X$.
  Then $(Y, \restrict{d}{Y})$ is also a metric space.
\end{proposition}
\begin{proof}
  All the requirements for a function to be a metric have only universal
  quantifiers.
  Hence, they translate to subsets as well.
\end{proof}
\begin{definition}
  Let $(X, d)$ and $(Y, \delta)$ be metric spaces.
  A function $f \colon X \to Y$ is called an
  \textbf{isomorphism of metric spaces}
  if it is bijective and for any $x, y \in X$,
  \begin{equation}
    \delta(f x, f y) = d(x, y).
  \end{equation}
\end{definition}
\begin{definition}
  Let $(X, d)$ be a metric space, $Y \subseteq X$.
  We say that $Y$ is a \textbf{dense subset of $X$} if
  for any $x \in X$ there exists a Cauchy sequence in $(Y, \restrict{d}{Y})$
  that converges (as a sequence in $X$) to $x$.
\end{definition}
\begin{definition}
  Let $(X, d)$ be a metric space, $a$ and $b$ be sequences in $X$.
  Define the sequence $\hat{d}(a, b) \colon \N \to \R$ by
  \begin{equation}
    \hat{d}(a, b) := \{d(a_n, b_n)\}_{n = 0}^\infty.
  \end{equation}
\end{definition}
\begin{proposition}
  Let $(X, d)$ be a metric space, $x, y, z \in X$.
  Then
  \begin{equation}
    \abs{\hat{d}(x, y) - \hat{d}(y, z)} \leq \hat{d}(x, z).
  \end{equation}
\end{proposition}
\begin{proof}
  From triangle inequality and symmetry it follows that
  $\hat{d}(x, y) - \hat{d}(y, z) \leq \hat{d}(x, z)$ and
  $\hat{d}(y, z) - \hat{d}(x, y) \leq \hat{d}(x, z)$.
  This gives the desired inequality.
\end{proof}
\begin{notation}
  We will denote $\Cauchy(\R) := \Cauchy(\R, d_\R)$, where
  $d_\R$ is the standard metric on $\R$,
  $d_\R(x, y) := \abs{x - y}$ for $x, y \in \R$.
\end{notation}
\begin{proposition}
  Let $(X, d)$ be a metric space, $a, b \in \Cauchy(X, d)$.
  Then $\hat{d}(a, b) \in \Cauchy(\R)$.
  (And hence, converging since $\R$ is complete.)
\end{proposition}
\begin{proof}
  Denote $x := \hat{d}(a, b) \colon \N \to \R$.
  Let $m, n \in \N$.
  Then
  \begin{equation}
    \begin{split}
      \abs{x_m - x_n}
      & = \abs{d(a_m, b_m) - d(a_n, b_n)}
        = \abs{d(a_m, b_m) - d(a_n, b_m) + d(a_n, b_m) - d(a_n, b_n)} \\
      & \leq \abs{d(a_m, b_m) - d(a_n, b_m)} + \abs{d(a_n, b_m) - d(a_n, b_n)}
        \leq d(a_m, a_n) + d(b_m, b_n).
    \end{split}
  \end{equation}
  Now let $\varepsilon > 0$.
  Choose $N_a \in \N$ such that
  for all $m, n > N_a$, $d(a_m, a_n) < \varepsilon / 2$,
  and $N_b \in \N$ such that
  for all $m, n > N_b$, $d(b_m, b_n) < \varepsilon / 2$.
  Define $N := \max(N_A, N_B)$.
  Then for any $m, n > N$,
  \begin{equation}
    \abs{x_m - x_n}
    \leq d(a_m, a_n) + d(b_m, b_n)
    \leq \varepsilon / 2 + \varepsilon / 2
    = \varepsilon,
  \end{equation}
  which means that $x \in \Cauchy(\R)$.
\end{proof}
\begin{corollary}
  Let $(X, d)$ be a metric space, $a$ and $b$ be convergent sequences.
  Then
  \begin{equation}
    \lim_{n \to \infty} d(a_n, b_n)
    = d\left(\lim_{n \to \infty} a_n, \lim_{n \to \infty} b_n\right).
  \end{equation}
\end{corollary}
\begin{remark}
  The above equality can be stated as
  \begin{equation}
    \lim_\R(\hat{d}(a, b)) = d(\lim_{(X, d)} a, \lim_{(X, d)} b)
  \end{equation}
  which can be stated as the commutative diagram relation
  \begin{equation}
    \lim_\R \circ \hat{d} = d \circ (\lim_{(X, d)} \times \lim_{(X, d)}).
  \end{equation}
\end{remark}
\begin{definition}
  Let $(X, d)$ be a metric space.
  Define $\delta \colon \Cauchy(X, d) \times \Cauchy(X, d) \to \R$ as follows:
  for any $a, b \in \Cauchy(X, d)$,
  \begin{equation}
    \delta(a, b) := \lim_{n \to \infty} d(a_n, b_n).
  \end{equation}
\end{definition}
\begin{proposition}
  Let
    $(X, d)$ be a metric space,
    $Y := \Cauchy(X, d)$,
    $\delta \colon Y \times Y \to \R$ be defined as above.
  Then $(Y, \delta)$ is a \textbf{semi-metric space}.
  (A metric space without non-degeneracy.)
\end{proposition}
\begin{proof}
  Let $a, b, c \in \Cauchy(X, d)$.
  \begin{enumerate}
    \item
      \textbf{Non-negativity.}
      Since for any $n \in \N$, $d(a_n, b_n) \geq 0$,
      then the non-negativity is preserved in the limit case,
      i.e., $d(a, b) \geq 0$.
    \item
      \textbf{Symmetry.}
      Since for any $n \in \N$, $d(a_n, b_n) = d(b_n, a_n)$,
      then the symmetry is preserved in the limit case,
      i.e., $\delta(a, b) = \delta(b, a)$.
    \item
      \textbf{Triangle inequality.}
      Since for any $n \in \N$, $d(a_n, c_n) \leq d(a_n, b_n) + d(b_n, c_n)$,
      then the triangle inequality is preserved in the limit case,
      i.e., $\delta(a, c) \leq \delta(a, b) + \delta(b, c)$.
  \end{enumerate}
\end{proof}
\begin{remark}
  Note that positive definiteness of the metric is not preserved for Cauchy
  sequences.
  Indeed, two sequences which coincide after some point but have different
  initial values.
  Then they agree in the limiting case and so their distance is zero,
  but they are unequal.

  Another interesting case occurs when the two sequences differ entirely but
  converge to one another in the limiting case.
  For intance, the constant zero sequence and the sequence
  $\{1 / (n + 1)\}_{n = 0}^\infty$.
\end{remark}
\begin{proposition}
  Let
    $(X, d)$ be a metric space,
    $\delta$ be the semi-metric on $\Cauchy(X, d)$.
  Define the relation $\sim$ on $\Cauchy(X, d)$ by
  \begin{equation}
    a \sim b \Leftrightarrow \delta(a, b) = 0,\
    a, b \in \Cauchy(X, d).
  \end{equation}
  Then $\sim$ is an equivalence relation on $\Cauchy(X, d)$.
\end{proposition}
\begin{proof}
  Reflexivity and symmetry are obvious, so let us consider transitivity.
  Consider $a, b, c \in \Cauchy(X, d)$ and let
  $a \sim b$ and $b \sim c$.
  Then for any $n \in \N$,
  \begin{equation}
    d(a_n, c_n) \leq d(a_n, b_n) + d(b_n, c_n).
  \end{equation}
  Let $\varepsilon > 0$.
  Choose
  $N_A \in \N$ such that for all $n > N_A$, $d(a_n, b_n) < \varepsilon / 2$, and
  $N_B \in \N$ such that for all $n > N_B$, $d(b_n, c_n) < \varepsilon / 2$.
  Let $N := \max(N_a, N_B)$.
  Then for any $n > N$,
  \begin{equation}
    0
    \leq d(a_n, c_n)
    \leq d(a_n, b_n) + d(b_n, c_n)
    < \varepsilon / 2 + \varepsilon / 2
    = \varepsilon.
  \end{equation}
  Hence, $\lim_{n \to \infty} d(a_n, c_n) = 0$, i.e. $a \sim c$.
  Therefore, $\sim$ is an equivalence relation.
\end{proof}
\begin{proposition}
  Let $(X, d)$ be a metric space, $a, a', b, b' \in \Cauchy(X, d)$.
  Assume that $a \sim a'$ and $b \sim b'$.
  Then $\delta(a, b) = \delta(a', b')$.
\end{proposition}
\begin{definition}
  Let $(X, d)$ be a metric space.
  Define the $\rho$ on $\Cauchy(X, d) / \sim$ by
  \begin{equation}
    \rho([a]_\sim, [b]_\sim) := \delta(a, b).
  \end{equation}
  (Correctness of the definition is follows from the previous proposition.)
\end{definition}
\begin{proposition}
  Let $(X, d)$ be a metric space.
  Then $(\Cauchy(X, d) / \sim, \rho)$ is a metric space.
\end{proposition}
\begin{proposition}
  Let $(X, d)$ be a metric space.
  Then the metric $\rho$ on $\Cauchy(X, d) / \sim$ is complete.
\end{proposition}
\begin{proposition}
  Let $(X, d)$ be a metric space.
  Consider the map
  $f \colon X \to \Cauchy(X, d) / \sim$ defined by embedding constants
  as constant sequences:
  \begin{equation}
    f(x) = [(x, x, ..., x, ...)]_\sim.
  \end{equation}
  Then $f$ is an embedding and its image is a dense subset of
  $\Cauchy(X, d) / \sim$.
  Moreover, if $X$ is complete, then $f$ is an isomorphism.
\end{proposition}
\begin{remark}
  All of what we did in this section has the following interpretation.
  The space $(Y, \rho)$ of equivalence calsees of Cauchy sequences on $(X, d)$
  is a complete metric space that has $(X, d)$
  isomorphic to a dense subset of $X$ with the induced metric.
  For this reason $Y$ is called the \textbf{completion of $X$}.
  Moreover, if $(X, d)$ is already complete, then $(Y, \rho)$ is isomorphic to
  $(X, d)$.
  In other words, the completion of a complete metric space is essentially the
  same space, as one could expect.
\end{remark}
\begin{remark}
  In constructive setting one often works with the so called
  \textbf{rapidly converging sequences} and \textbf{rapid Cauchy sequences},
  thus removing the dependence on a real argument in the respective definitions
  of a converging sequence and a Cauchy sequence.
\end{remark}
\begin{definition}
  Let $(X, d)$ be a metric space, $a \colon \N \to X$.
  We say that $a$ is a \textbf{rapidly converging sequence} if
  there exists $A \in X$ such that for any $k \in \N$ there exists $N \in N$
  such that for all $m, n > N$,
  \begin{equation}
    \exists A \in X,\
      \forall k \in \N,\
        \exists N \in \N,\
          \forall n \in \N,\
             n > N \Rightarrow d(a_n, A) < 2^{-k}.
  \end{equation}
\end{definition}
\begin{definition}
  Let $(X, d)$ be a metric space, $a \colon \N \to X$.
  We say that $a$ is a \textbf{rapid Cauchy sequence} if
  for any $k \in \N$ there exists $N \in N$ such that for all $m, n > N$,
  \begin{equation}
    \forall k \in \N,\
      \exists N \in \N,\
        \forall m, n \in \N,\
           m, n > N \Rightarrow d(a_m, a_n) < 2^{-k}.
  \end{equation}
\end{definition}


\section{Differentiability on normed affine spaces}
\label{section:differentiability_on_normed_affine_spaces}
\begin{definition}
  Let
    $(M, U, \norm{\cdot}_U)$ and $(N, V, \norm{\cdot}_V)$
      be normed affine spaces,
    $X$ and $Y$ be open subsets of $M$ and $N$ respectively,
    $f \colon X \to Y$,
    $x_0 \in X$,
    $u \in U$.
  We say that \textbf{$f$ is differentiable at $x_0$ in the direction $u$}
  if the following limit exists:
  \begin{equation}
    \nabla_u f(x_0)
    := \lim_{\varepsilon \to 0}
      \frac{f(x_0 + \varepsilon u) - f(x_0)}{\varepsilon} \in V.
  \end{equation}
  We call $\nabla_u f(x_0)$ the
  \textbf{directional derivative of $f$ at $x_0$ in direction $u$}.
\end{definition}
\begin{remark}
  Denote $I := \set{t \in \R}{x_0 + t u \in Y}$. 
  Define $g \colon I \to Y$ by
  \begin{equation}
    g(t) = f(x_0 + t u).
  \end{equation}
  Then $\nabla_u f(x_0) = g'(0)$.
\end{remark}
\begin{proposition}[Directional derivative is a linear operator]
  Let
    $(M, U, \norm{\cdot}_U)$ and $(N, V, \norm{\cdot}_V)$
      be normed affine spaces,
    $X$ and $Y$ be open subsets of $M$ and $N$ respectively.
    $x_0 \in X$,
    $u \in U$,
    $f, g \colon X \to Y$,
    $\lambda, \mu \in \R$.
  If $f$ and $g$ are differentiable in the direction of $u$ at $x_0$, so is
  $\lambda f + \mu g$.
  Moreover,
  \begin{equation}
    \nabla_u (\lambda f + \mu g)(x_0)
    = \lambda \nabla f(x_0) + \mu \nabla_u g(x_0).
  \end{equation}
\end{proposition}
\begin{definition}
  Let
    $(M, U, \norm{\cdot}_U)$ and $(N, V, \norm{\cdot}_V)$
      be normed affine spaces,
    $X$ and $Y$ be open subsets of $M$ and $N$ respectively,
    $f \colon X \to Y$,
    $x_0 \in X$.
  We say that \textbf{$f$ is Gateux-differentiable at $x_0$} if all directional
  derivatives of $f$ at $x_0$ exist.
\end{definition}
\begin{definition}
  Let
    $(M, U, \norm{\cdot}_U)$ and $(N, V, \norm{\cdot}_V)$
      be normed affine spaces,
    $X$ and $Y$ be open subsets of $M$ and $N$ respectively,
    $f \colon X \to Y$.
  We say that $f$ is \textbf{Gateaux-differentiable}
  if it is Gateux-differentiable at any $x_0 \in X$.
\end{definition}
\begin{definition}
  Let
    $(M, U, \norm{\cdot}_U)$ and $(N, V, \norm{\cdot}_V)$
      be normed affine spaces,
    $X$ and $Y$ be open subsets of $M$ and $N$ respectively,
    $f \colon X \to Y$,
    $x_0 \in X$.
  We say that $f$ is \textbf{Fr\'{e}chet differentiable at $x_0$} if
  there exists $A \in \Hom(U, V)$ such that
  for any $\varepsilon > 0$
  there exists $\delta > 0$ such that
  $B(f(x_0), \delta) \subseteq Y$ and
  for any $h \in V$ with $\norm{h} < \delta$,
  \begin{equation}
    \norm{f(x_0 + h) - f(x_0) - A h}_V \leq \varepsilon \norm{h}_U.
  \end{equation}
  We say that $A$ is \textbf{a differential of $f$ at $x_0$}.
\end{definition}
\begin{proposition}
  Let
    $(M, U, \norm{\cdot}_U)$ and $(N, V, \norm{\cdot}_V)$
      be normed affine spaces,
    $X$ and $Y$ be open subsets of $M$ and $N$ respectively,
    $f \colon X \to Y$,
    $x_0 \in X$,
    $f$ be differentiable at $x_0$,
    $A, B \in \Hom(U, V)$ be differentials of $f$ at $x_0$.
  Then $A = B$.
  In other words, whenever the differential exists, it is unique.
\end{proposition}
\begin{proof}
  Let $\varepsilon > 0$.
  Choose $\delta_A > 0$ such that for all $\norm{h_A}_U < \delta_A$,
  \begin{equation}
    \norm{f(x_0 + h_A) - f(x_0) - B h_A}_V \leq \varepsilon \norm{h_A}_U / 2,
  \end{equation}
  and $\delta_B > 0$ such that for all $\norm{h_B}_U < \delta_B$,
  \begin{equation}
    \norm{f(x_0 + h_B) - f(x_0) - A h_B}_V \leq \varepsilon \norm{h_B}_U / 2.
  \end{equation}
  Let $\delta = \min(\delta_A, \delta_B)$.
  Then for any $h \in U$ with $\norm{h}_U < \delta$,
  \begin{equation}
    \begin{split}
      \norm{(B - A) h}
      & = \norm{B h - A h} \\
      & \leq
        \norm{(f(x_0 + h) - f(x_0) - A h) - (f(x_0 + h) - f(x_0) - B h)}_V \\
      & \leq \norm{(f(x_0 + h) - f(x_0) - A h)}_V
        + \norm{(f(x_0 + h) - f(x_0) - B h)}_V \\
      & \leq \varepsilon h.
    \end{split}
  \end{equation}
  This means that $\norm{B - A}_{\Hom(U, V)} \leq \varepsilon$.
  (More precisely, the latter is true for small $h$ but the estimate for the
  homomorphism norm comes from linearity of $B - A$.)
  Since $\varepsilon$ is arbitrary, this means that
  $\norm{B - A}_{\Hom(U, V)} = 0$.
  Hence $B - A = 0$, i.e., $A = B$.
\end{proof}
\begin{notation}
  Let
    $(M, U, \norm{\cdot}_U)$ and $(N, V, \norm{\cdot}_V)$
      be normed affine spaces,
    $X$ and $Y$ be open subsets of $M$ and $N$ respectively,
    $f \colon X \to Y$,
    $x_0 \in X$,
    $f$ be differentiable at $x_0$.
  We denote the differential of $f$ at $x_0$ by $D f_{x_0}$.
\end{notation}
\begin{proposition}
  Let
    $(M, U, \norm{\cdot}_U)$ and $(N, V, \norm{\cdot}_V)$
      be normed affine spaces,
    $X$ and $Y$ be open subsets of $M$ and $N$ respectively,
    $f \colon X \to Y$,
    $x_0 \in X$,
    $f$ be Fr\'{e}chet differentiable at $x_0$.
  Then $f$ is Gateux differentiable at $x_0$.
  Moreover any $u \in U$
  \begin{equation}
    D f_{x_0} u = \nabla_u f(x_0).
  \end{equation}
\end{proposition}
\begin{proof}
  Let $u \in U$, $\varepsilon > 0$.
  Take $\delta > 0$ so that
  $\set{x_0 + t u}{\abs{t} < \delta} \subseteq X$
  and for any $t \in (- \delta, \delta)$,
  \begin{equation}
    \norm{f(x_0 + t u) - f(x_0) - D f_{x_0}(t u))}
    \leq \varepsilon \norm{t u}
    = \varepsilon t \norm{u}.
  \end{equation}
  But the definition of derivative gives a similar estimate, without the term
  $\norm{u}$ which is bounded however and does not change the estimate.
  By the uniqueness of derivatives, we get the desired result.
\end{proof}
\begin{corollary}[Differential in coordinates]
  Let
    $(M, U, \norm{\cdot}_U)$ and $(N, V, \norm{\cdot}_V)$
      be normed affine spaces of dimensions $m$ and $n$ respectively,
    $X$ and $Y$ be open subsets of $M$ and $N$ respectively,
    $x_0 \in X$,
    $f \colon X \to Y$ be differentiable at $x_0$.
    $a = (a_1, ..., a_m)$ and $b = (b_1, ..., b_n)$ be bases of $U$ and $V$
      respectively,
    ${\bf A} \in M_{n \times m}(\R)$ be the matrix of $D f_{x_0}$ in
    those bases.
  Then for $i = 1, ..., m$ and $j = 1, ..., n$,
  \begin{equation}
    {\bf A}^j_i = b^j(\nabla_{a_i} f(x_0)).
  \end{equation}
\end{corollary}
\begin{proof}
  By definition of matrix of operator, if $A = D f_{x_0}$, then
  ${\bf A}^j_i = b^j(A a_i)$.
  But since for any $u \in U$, $D f_{x_0}(u) = \nabla_u f(x_0)$, we get
  $b^j(A a_i) = b^j(\nabla_{a_i} f(x_0))$, as wanted.
\end{proof}


\section{Systems of ordinary differential equations}
\label{section:systems_of_ordinary_differential_equations}
\begin{definition}
  Let $x \in \R$, $I \subseteq \R$.
  We say that $I$ is \textbf{an interval around $x$} if $I$ is a connected set
  containing an open neighbourhood of $x$.
\end{definition}
\begin{definition}
  Let
    $t_0 \in \R$,
    $n \in \N^+$,
    $x_0 \in \R^n$,
    $D \subseteq \R \times \R^n$ be an open neighbourhood of $(t_0, x_0)$,
    $F \colon D \to \R^n$ be continuous.
  The \textbf{initial value-problem for the system of ordinary differential
  equations for right-hand side $F$} is the following problem: find
    an interval $I$ around $t_0$,
    a neighbourhood $C \subseteq \R^n$ of $x_0$,
    and a curve $x \colon I \to C$,
  such that for any $t \in I$, $(t, x(t)) \in D$ and
  \begin{subequations}
    \label{equation:ordinary_differential_equation/initial_value_problem}
    \begin{alignat}{1}
      & \dot{x}(t) = F(t, x(t)),\ \forall t \in I, \\
      & x(t_0) = x_0.
    \end{alignat}
  \end{subequations}
\end{definition}
\begin{discussion}
  Under the assumptions of the above definition, assume that $x$ is a solution
  and let $t \in I$.
  Integrating over the segment $[t_0, t]$ the differential equation, we get
  \begin{equation}
    \int_{t_0}^t F(s, x(s))\, d s
    = \int_{t_0}^s \dot{x}(s)\, d s
    = x(t) - x(t_0)
    = x(t) - x_0,
  \end{equation}
  which can be restated as the system of integral equations
  \begin{equation}
    \label{equation:ordinary_differential_equation/initial_value_problem/integral}
    x(t) = x_0 + \int_{t_0}^t F(s, x(s)).
  \end{equation}
  Conversely, if $x$ solves
  \Cref{equation:ordinary_differential_equation/initial_value_problem/integral},
  then $x(t_0) = x_0$.
  Moreover, if $x$ is assumed to be differentiable, then differentiating it we
  get $\dot{x}(t) = F(t, x(t))$.

  In general the integral system
  \Cref{equation:ordinary_differential_equation/initial_value_problem/integral}
  is more general than the differential system
  \Cref{equation:ordinary_differential_equation/initial_value_problem}
  and may include non-smooth solutions.
  However they are equivalent under smoothness assumptions which will mostly
  interest us.
  Nevertheless, the analysis is still easier to perform on the integral one to
  which we will stick.
  We will show that under mild assumptions the integral formulation gives rise
  locally to a contraction operator to which the Banach fixed point theorem
  applies.
\end{discussion}
\begin{definition}
  Let
    $(X, d_X)$ and $(Y, d_Y)$ be metric spaces,
    $f \colon X \to Y$,
    $K \in \R$.
  We say that $f$ is \textbf{Lipschitz continuous with Lipschitz constant $L$}
  if for any $x_1, x_2 \in X$,
  \begin{equation}
    d_Y(f x_1, x_2) \leq L d_X(x_1, x_2).
  \end{equation}
  If the constant $L$ merely exists (but is not being fixed) we say that
  $f$ is Lipschitz continuous.
\end{definition}
\begin{definition}
  Let $n \in \N^+$, $D \subseteq \R \times \R^n$ be open, $S \subset D$.
  We say that $S$ is a \textbf{spatial surface} if there exists $t \in \R$ such
  that for any $x \in S$, the temporal (first) component of $S$ is $t$.
\end{definition}
\begin{definition}
  Let
    $n \in \N^+$,
    $D \subseteq \R \times \R^n$ be an open set,
    $F \colon D \to \R^n$ be continuous,
    $L \in \R^+$.
  We say that $F$ is \textbf{Lipschitz continuous with respect to the spatial
  arguments with Lipschitz constant $L$} if for any spatial surface
  $S \subset D$, $\restrict{F}{S}$ is Lipschitz continuous with Lipschitz
  constant $L$.

  If the constant $L$ merely exists (but is not being fixed) we say that
  $F$ is Lipschitz continuous with respect to the spatial arguments.
\end{definition}
\begin{notation}
  Let $n \in \N^+$, $x_0 \in \R^n$, $\varepsilon \in \R^+$.
  By $\mathcal{C}(x_0, \varepsilon)$ we will denote the closed ball centered at
  $x_0$ with
  radius $\varepsilon$, i.e.,
  \begin{equation}
    \mathcal{C}(x_0, \varepsilon)
    := \set{x \in \R^n}{\norm{x - x_0} \leq \varepsilon}.
  \end{equation}
\end{notation}
\begin{proposition}
  Let
    $t_0 \in \R$,
    $n \in \N^+$,
    $x_0 \in \R^n$,
    $D \subseteq \R \times \R^n$ be an open neighbourhood of $(t_0, x_0)$,
    $F \colon D \to \R^n$ be continuous and Lipschitz continuous with respect to
      the spatial arguments.
  Then there exist $\tau, h \in \R^+$ such that:
  \begin{itemize}
    \item
      for any
      $x \in C^0(\mathcal{C}(t_0, \tau), \mathcal{C}(x_0, h))$ and
      $t \in \mathcal{C}(t_0, \tau)$,
      \begin{equation}
        x_0 + \int_{t_0}^t F(s, x(s))\, d s \in \mathcal{C}(x_0, h),
      \end{equation}
      or equivalently,
      \begin{equation}
        \norm{\int_{t_0}^t F(s, x(s))\, d s} \leq h;
      \end{equation}
    \item
      the resulting operator
      \begin{equation}
        \label{equation:ordinary_differential_equation/initial_value_problem/operator}
        \begin{split}
          & T
            \colon C^0(\mathcal{C}(t_0, \tau), \mathcal{C}(x_0, h))
            \to C^0(\mathcal{C}(t_0, \tau), \mathcal{C}(x_0, h)), \\
          & (T x)(t) := x_0 + \int_{t_0}^t F(s, x(s))\, d s,
        \end{split}
      \end{equation}
      is a contraction.
  \end{itemize}
\end{proposition}
\begin{proof}
  For a compact (bounded and closed) subset $Y$ of $D$ denote
  \begin{equation}
    M_Y := \norm{F}_{Y, \infty} = \max_{y \in Y}(\abs{f(y)})
  \end{equation}
  (existing since $F$ is continuous, hence attaing maxima on compact subsets).
  Clearly if $Z \subseteq Y$ is a compact, then $M_Z \leq M_Y$.
  Let $L \in \R^+$ be a (spatial) Lipschitz constant for $f$.

  Let $q \in [0, 1)$ be arbitrary.
  ($q$ will be our final Lipschitz constant that will control how small we will
  choose our closed balls around $t_0$ and $x_0$.)
  Take any $\tilde{\tau} = q / L$ and choose $h \in \R^+$ sufficiently small so
  that
  \begin{equation}
    \mathcal{C}(t_0, \tau) \times \mathcal{C}(x_0, h) \subset D.
  \end{equation}
  Define $\tau := \min(\tilde{\tau}, q h / M_{\mathcal{C}(x_0, h)})$.
  We claim that $(\tau, h)$ satisfy the requirements of the proposition.
  Let
    $x \in C^0(\mathcal{C}(t_0, \tau), \mathcal{C}(x_0, h))$ and
    $t \in \mathcal{C}(t_0, \tau)$.
  \begin{itemize}
    \item
      \textbf{The integral operator $T$ is well defined.}
      \begin{equation}
        \begin{split}
          \norm{\int_{t_0}^t F(s, x(s))\, d s}
          \leq \int_{t_0}^{t} \norm{F(s, x(s))}\, d s
          \leq M_{\mathcal{C}(x_0, h)}) \abs{t - t_0}
          \leq M_{\mathcal{C}(x_0, h)}) \tau
          \leq q h
          < h.
        \end{split}
      \end{equation}
    \item
      \textbf{$T$ is a contraction with Lipschitz constant $q$.}
      Let $x, y \in C^0(\mathcal{C}(t_0, \tau), \mathcal{C}(x_0, h))$.
      Choose $t \in \mathcal{C}(t_0, \tau)$ such that
      \begin{equation}
        \norm{T x - T y}_{C^0(\mathcal{C}(t_0, \tau), \mathcal{C}(x_0, h))}
        = \norm{(T x - T y)(t)}_{\R^n}.
      \end{equation}
      Then
      \begin{equation}
        \begin{split}
          \norm{T x - T y}_\infty
          & = \norm{(T x - T y)(t)}_U \\
          & = \norm{\int_{t_0}^t (F(s, x(s)) - F(s, y(s)))\, d s}_U \\
          & \leq \int_{t_0}^t \norm{(F(s, x(s)) - F(s, y(s)))}_U\, d s \\
          & \leq L \int_{t_0}^t \norm{x(s) - y(s)}_U\, d s \\
          & \leq L \int_{t_0}^t \norm{x - y}_\infty\, d s \\
          & = L \abs{t - t_0} \norm{x - y}_\infty \\
          & \leq L \tau \norm{x - y}_\infty \\
          & \leq q \norm{x - y}_\infty.
        \end{split}
      \end{equation}
  \end{itemize}
  Hence, $T$ is a contraction with Lipschitz constant $q$.
\end{proof}
\begin{definition}
  Let
    $I$ be an open interval,
    $n \in \N^+$,
    $U$ be an open subset of $\R \times \R^n$.
    $V$ be a vector space,
    $k \in \N^+$,
    $m \geq k - 1$,
    $u \in C^m(U \to \R^n)$ (usually a flow velocity),
    $\partial_t \colon C^k(U, V) \to C^{k - 1}(U, V)$
      be the temporal partial derivative,
    $d_x \colon C^k(U, V) \to C^{k - 1}(U, V \times (\R^n)^*)$
      be the spatial differential (the differential when time is fixed),
    $i_u \colon C^{k - 1}(U, V \times (\R^n)^*) \to C^{k - 1}(U, V)$
      be the contraction with the flow field.
  The material derivative on $V$-valued functions is defined by
  \begin{equation}
    \frac{D}{D t} \colon C^k(U, V) \to C^{k - 1}(U, V),\
    \frac{D}{D t} := \partial_t + i_u \circ d_x.
  \end{equation}
\end{definition}
\begin{proposition}
  Let
    $I$ be an open interval,
    $n \in \N^+$,
    $A$ be an open subset of $\R^n$
    $c \colon I \to A$,
    $V$ be a vector space,
    $B$ be an open subset of $V$,
    $f \colon A \to B$.
  Then
  \begin{equation}
    \frac{d(f \circ c)}{d t} = i_{\dot{c}}(d f \circ c) \colon I \to V.
  \end{equation}
\end{proposition}
\begin{proposition}
  Let
    $I$ be an interval,
    $n \in \N^+$,
    $U$ be an open subset of $\R \times \R^n$.
    $V$ be a vector space,
    $k \in \N^+$,
    $u \in C^{k - 1}(U, \R^n)$ (usually a flow velocity),
    $x \colon I \to \R^n$ be an integral curve of $u$,
    $\tilde{x} := \id_I \odot x \colon I \to I \times \R^n$,
    $g \in C^k(I, \R^n)$.
  Then
  \begin{equation}
    \frac{d (g \circ \tilde{x})}{d t} = \frac{D g}{D t} \circ \tilde{x}.
  \end{equation}
\end{proposition}
\begin{proof}
  The flow velocity and motion $x$ are related by
  \begin{equation}
    \dot{x} = u \circ \tilde{x}.
  \end{equation}
  Hence,
  \begin{equation}
    \begin{split}
      \frac{d (g \circ \tilde{x})}{d t}
      & = i_{\dot{\tilde{x}}}(d g \circ \tilde{x}) \\
      & = i_{1 \oplus \dot{x}}((\partial_t g \oplus d_x g) \circ \tilde{x}) \\
      & = \partial_t g \circ \tilde{x}
        + i_{u \circ \tilde{x}} ((d_x g) \circ \tilde{x}) \\
      & = \partial_t g \circ \tilde{x} + (i_u(d_x g)) \circ \tilde{x}) \\
      & = ((\partial_t + i_u \circ d_x) g) \circ \tilde{x} \\
      & = \frac{D g}{D t} \circ \tilde{x}.
    \end{split}
  \end{equation}
\end{proof}
\begin{corollary}
  Let
    $I$ be an interval,
    $n \in \N^+$,
    $U$ be an open subset of $\R \times \R^n$.
    $V$ be a vector space,
    $k \in \N^+$,
    $u \in C^{k - 1}(U, \R^n)$ (usually a flow velocity),
    $x \colon I \to \R^n$ be an integral curve of $u$,
    $\tilde{x} := \id_I \odot x \colon I \to I \times \R^n$,
    $g \in C^k(I, \R^n)$.
  \begin{equation}
    \frac{d^k (g \circ \tilde{x})}{d t^k} = \frac{D^k g}{D t^k} \circ \tilde{x}.
  \end{equation}
\end{corollary}
\begin{proof}
  We proceed by the induction.
  In the base case $k = 0$ noth sides equal to $g \circ \tilde{x}$
  For the induction step assume the proposition is true for the integer $k$.
  Then for $k + 1$ we get
  \begin{equation}
    \begin{split}
      \frac{d^{k + 1} (g \circ \tilde{x})}{d t^{k + 1}}
      & = \frac{d}{d t}\left(\frac{d^k (g \circ \tilde{x})}{d t^k}\right)
      & \qquad (\text{Definition of operator power}) \\
      & = \frac{d}{d t}\left(\frac{D^k g}{D^k t} \circ \tilde{x}\right)
      & \qquad (\text{Induction hypothesis}) \\
      & = \frac{D}{D t}\left(\frac{D^k g}{D^k t}\right) \circ \tilde{x}
      & \qquad (\text{Pevious proposition}) \\
      & = \frac{D^{k + 1} g}{D t^{k + 1}} \circ \tilde{x}
      & \qquad (\text{Definition of operator power}).
    \end{split}
  \end{equation}
\end{proof}
\begin{theorem}[Picard-Lindel\"{o}f]
  Let
    $t_0 \in \R$,
    $n \in \N^+$,
    $x_0 \in \R^n$,
    $D \subseteq \R \times \R^n$ be an open neighbourhood of $(t_0, x_0)$,
    $k \in \N$,
    $F \in C^k(D, \R^n)$ be
      Lipschitz continuous with respect to the spatial arguments.
  Then there exist $\tau, h \in \R^+$ such that the problem
  \Cref{equation:ordinary_differential_equation/initial_value_problem}
  has a unique solution
  $x \in C^{k + 1}(\mathcal{C}(t_0, \tau), \mathcal{C}(x_0, h))$.

  In particular if $k = \infty$ so that $F \in C^\infty(D, \R^n)$, then
  $x \in C^\infty(\mathcal{C}(t_0, \tau), \mathcal{C}(x_0, h))$.
\end{theorem}
\begin{proof}
  Denote
    $I := \mathcal{C}(t_0, \tau)$,
    $V := \mathcal{C}(x_0, h)$,
  \begin{equation}
     u := \restrict{F}{I \times V} \in C^k(I \times V, \R^n).
   \end{equation} 
  Let $q \in [0, 1)$ be arbitrary.
  Take (from the proof of the previous proposition) $\tau, h \in \R^+$ such that
  the operator $T$ defined on
  $C^0(I, V)$ in
  \Cref{equation:ordinary_differential_equation/initial_value_problem/operator}
  is a contraction with Lipschitz constant $q$.
  According to the Banach fixed point theorem, since $C^0$ spaces between closed
  subsets of Euclidean spaces are complete, there exists a unique
  $x \in C^0(I, V)$ that is a fixed point
  of $T$.
  We need to show that the solution $x$ is in fact of class $C^{k + 1}$, not
  merely continuous.
  First we use the fact that
  \begin{equation}
    \frac{d}{d t} \int_{t_0}^t u(s, x(s))\, d s = u(t, x(t))
  \end{equation}
  to get $x'(t) = u(t, x(t))$.
  Define
  \begin{subequations}
    \begin{alignat}{1}
      & \tilde{x} \colon I \to I \times V, \\
      & \tilde{x}(t) := (t, x(t)),\ \forall t \in I,
    \end{alignat}
  \end{subequations}
  so that $\dot{x} = u \circ \tilde{x}$.
  From the previous propoition it follows that $u$ is $k$ time differentiable
  and
  \begin{equation}
    \frac{d^{k + 1} x}{d t^{k + 1}}
    = \frac{d^k (u \circ \tilde{x})}{d t^k}
    = \frac{D^k u}{D t^k} \circ \tilde{x}.
  \end{equation}
  Hence, $x \in C^{k + 1}(I, V)$, as wanted.
\end{proof}


\part{Smooth manifolds}

\section{Vector fields on manifolds}
\label{section:vector_fields_on_manifolds}
\begin{definition}
  Let
    $R$ be a commutative ring with unity,
    $V$ be an $R$-algebra with multiplication operation $[\cdot, \cdot]$.
  We say that $V$ is a \textbf{Lie algebra} if $[\cdot, \cdot]$ is alternating
  and satisfies the \textbf{Jacobi identity}: for any $x, y, z \in V$,
  \begin{equation}
    [x, [y, z]] + [y, [z, x]] + [z, [x, y]] = 0.
  \end{equation}
\end{definition}

\begin{proposition}
  Let $M$ be a smooth manifold, $X, Y \in \mathfrak{X} M$.
  Then the \textbf{commutator}
  \begin{equation}
    [X, Y] := X \circ Y - Y \circ X
  \end{equation}
  is also a vector field.
\end{proposition}

\begin{proof}
  We will show that $[X, Y]$ satisfies the Leibniz rule.
  Let $f, g \in \mathcal{F} M$.
  Then
  \begin{equation}
    X(Y(f g))
    = X(Y f \cdot g + f \cdot Y g)
    = X (Y f) \cdot g + Y f \cdot X g + X f \cdot Y g + f \cdot X (Y g).
  \end{equation}
  Analogously,
  \begin{equation}
    Y(X(f g))
    = Y (X f) \cdot g + X f \cdot Y g + Y f \cdot X g + f \cdot Y (X g).
  \end{equation}
  Hence,
  \begin{equation}
    [X, Y] (f g)
    = X(Y(f g)) - Y(X(f g))
    = X (Y f) \cdot g + f \cdot X (Y g) - Y (X f) \cdot g - f \cdot Y (X g)
    = [X, Y] f \cdot g + f \cdot [X, Y] g.
  \end{equation}
  Linearity of $[X, Y]$ is obvious.
  Hence, $[X, Y]$ is a vector field.
\end{proof}

\begin{discussion}
  Let $M$ be a smooth manifold of dimension $D$,
  $X, Y \in \mathfrak{X} M$,
  $(U, \varphi)$ be a chart on $M$ with local coordinates $x^1, ..., x^D$.
  Let $X$ and $Y$ be represented in local coordinates as
  \begin{equation}
    X = \sum_{p = 1}^D f^p \frac{\partial}{\partial x^p},\
    Y = \sum_{p = 1}^D g^p \frac{\partial}{\partial x^p}.
  \end{equation}
  We are going to derive expressions for $[X, Y]$.
  We have
  \begin{equation}
    X \circ Y
    = \sum_{p, q = 1}^D f^p \frac{\partial}{\partial x^p} \circ
      (g^q \frac{\partial}{\partial x^q})
    = \sum_{p, q = 1}^D f^p \frac{\partial g^q}{\partial x^p}
        \frac{\partial}{\partial x^q}
      + \sum_{p, q = 1}^D f^p g^q \frac{\partial^2}{\partial x^p \partial x^q}.
  \end{equation}
  Analogously,
  \begin{equation}
    Y \circ X
    = \sum_{p, q = 1}^D g^p \frac{\partial f^q}{\partial x^p}
        \frac{\partial}{\partial x^q}
      + \sum_{p, q = 1}^D f^p g^q \frac{\partial^2}{\partial x^q \partial x^p}.
  \end{equation}
  Because of the symmetry of second derivatives, the second order terms in
  $[X, Y]$ cancel.
  Hence,
  \begin{equation}
    [X, Y]
    = \sum_{p, q = 1}^D
        (f^p \frac{\partial g^q}{\partial x^p}
         - g^p \frac{\partial f^q}{\partial x^p})
        \frac{\partial}{\partial x^q}.
  \end{equation}
  This can also be written as
  \begin{equation}
    [X, Y] = \sum_{q = 1}^D (X g^q - Y f^q) \frac{\partial}{\partial x^q}.
  \end{equation}
\end{discussion}

\begin{definition}
  Let $M$ be a smooth manifold.
  Define the \textbf{Lie derivative on functions},
  $L \colon \mathfrak{X} M \to {\rm Der}_{\mathcal{F} M}(\mathcal{F} M)$,
  as the evaluation map, i.e.,
  for any $X \in \mathfrak{X} M$, $f \in \mathcal{F} M$,
  \begin{equation}
    L_X f := X f.
  \end{equation}
\end{definition}
\begin{definition}
  Let $M$ be a smooth manifold.
  Define the \textbf{Lie derivative on vector fields},
  $L \colon \mathfrak{X} M \to {\rm Der}_{\mathcal{F} M}(\mathfrak{X} M)$,
  as the adjoint map, i.e., for any $X, Y \in \mathfrak{X} M$,
  \begin{equation}
    L_X Y := {\rm adj}_X Y = [X, Y].
  \end{equation}
\end{definition}
\begin{definition}
  Let $M$ be a smooth manifold, $E$ and $F$ be bundles so that there are
  Lie derivatives on their spaces of sections, i.e., for $Z \in \{E, F\}$,
  \begin{equation}
    L^{(Z)} \colon \mathfrak{X} M \to {\rm Der}_{\mathcal{F} M}(\Gamma(Z)).
  \end{equation}
  Define the \textbf{tensor product Lie derivative}
  \begin{equation}
    L^{(E \otimes F)}
    \colon \mathfrak{X} M \to {\rm Der}_{\mathcal{F} M}(\Gamma(E \otimes F)),
  \end{equation}
  such that for any $X \in \mathfrak{X} M$, $V \in \Gamma E$, $W \in \Gamma F$,
  \begin{equation}
    L^{(E \otimes F)}_X(V \otimes W)
    := L^{(E)}_X V \otimes W + V \otimes L^{(F)}_X W.
  \end{equation}
\end{definition}
\begin{definition}
  Let $M$ be a smooth manifold, $E$ and $F$ be bundles so that there are
  Lie derivatives $L^{(Z)}$ on their spaces of sections, for $Z \in \{E, F\}$.
  Define the \textbf{Hom Lie derivative}
  \begin{equation}
    L^{\Hom(E, F)}
    \colon \mathfrak{X} M
    \to {\rm Der}_{\mathcal{F} M}(\Hom(\Gamma E, \Gamma F)),
  \end{equation}
  such that for any
    $X \in \mathfrak{X} M$,
    $\varphi \in \Hom(\Gamma E, \Gamma F)$,
    $V \in \Gamma E$,
  \begin{equation}
    L^{(F)}_X(\varphi V) = (L^{\Hom(E, F)}_X \varphi) V + \varphi(L^{(E)}_X V),
  \end{equation}
  which gives the explicit formula
  \begin{equation}
    (L^{\Hom(E, F)}_X \varphi) V = L^{(F)}_X(\varphi V) - \varphi(L^{(E)}_X V).
  \end{equation}
\end{definition}
\begin{definition}
  Let $M$ be a smooth manifold, $E$ be a bundle so that there is a
  Lie derivative $L^{(E)}$ on $\Gamma E$.
  Then, specialising the definition of the Lie derivative on Hom spaces for the
  trivial bundle as the codomain, we get the \textbf{dual Lie derivative}
  \begin{equation}
    L^{(E^*)} \colon \mathfrak{X} M \to {\rm Der}_{\mathcal{F} M}(\Gamma E^*),
  \end{equation}
  defined so that for any
    $X \in \mathfrak{X} M$,
    $\varphi \in \Gamma E^*$,
    $V \in \Gamma E$,
  \begin{equation}
    (L^{(E^*)}_X \varphi) V = X(\varphi V) - \varphi(L^{(E)}_X V).
  \end{equation}
\end{definition}
\begin{definition}
  Let $M$ be a smooth manifold.
  Consider the Lie derivative on the whole tensor algebra
  (tensor products of vector and covector fields).
  For any $X, Y \in \mathfrak{X} M$ denote the Lie bracket
  \begin{equation}
    [L_X , L_Y] = L_X \circ L_Y - L_Y \circ L_X.
  \end{equation}
\end{definition}
\begin{proposition}
  Let $M$ be a smooth manifold, $X, Y \in \mathfrak{X} M$.
  Then
  \begin{equation}
    [L_X , L_Y] = L_{[X, Y]}.
  \end{equation}
\end{proposition}
\begin{proof}
  We will consider $4$ cases.
  \begin{enumerate}
    \item
      On functions the claim is obvious since $L = \id$.
    \item
      On vector fields the claim is restatement of the homomorphism nature of
      the adjoint map.
    \item
      (Tensor products.)
      Assume that
      $[L_X , L_Y] V = L_{[X, Y]} V$ and
      $[L_X , L_Y] W = L_{[X, Y]} W$.
      Then
      \begin{equation}
        \begin{split}
          [L_X , L_Y](V \otimes W)
          & = (L_X \circ L_Y - L_Y \circ L_X)(V \otimes W) \\
          & = L_X(L_Y V \otimes W + V \otimes L_Y W)
            - L_Y(L_X V \otimes W + V \otimes L_X W) \\
          & \begin{split}
              & = L_X(L_Y V) \otimes W + L_Y V \otimes L_X W
                + L_X V \otimes L_Y W + V \otimes L_X(L_Y W) \\
              & \quad - L_Y(L_X V) \otimes W + L_X V \otimes L_Y W
                - L_Y V \otimes L_X W + V \otimes L_Y(L_X W)
            \end{split} \\
          & = (L_X \circ L_Y - L_Y \circ L_X) V \otimes W
            + V \otimes (L_X \circ L_Y - L_Y \circ L_X) W \\
          & = [L_X, L_Y] V \otimes W + V \otimes [L_X, L_Y] W \\
          & = L_{[X, Y]} V \otimes W + V \otimes L_{[X, Y]} V \\
          & = L_{[X, Y]}(V \otimes W).
        \end{split}
      \end{equation}
    \item
      (Homomorphisms.)
      Assume that
      $[L_X , L_Y] \varphi = L_{[X, Y]} \varphi$ and
      $[L_X , L_Y] V = L_{[X, Y]} V$.
      Then
      \begin{equation}
        \begin{split}
          ([L_X , L_Y] \phi) V
          & = ((L_X \circ L_Y - L_Y \circ L_X) \phi) V \\
          & = L_X((L_Y \phi) V) - (L_Y \phi) (L_X V)
            - (L_Y((L_X \phi) V) - (L_X \phi) (L_Y V)) \\
          & \begin{split}
              & = L_X(L_Y(\phi V) - \phi(L_Y V))
                - (L_Y(\phi (L_X V)) - \phi(L_Y(L_X V)) \\
              & \quad - (L_Y(L_X(\phi V) - \phi(L_X V))
                         - (L_X (\phi (L_Y V)) - \phi(L_X(L_Y V)))
            \end{split} \\
          & \begin{split}
              & = (L_X \circ L_Y)(\phi V) - L_X(\phi(L_Y V))
                - L_Y(\phi (L_X V)) + \phi((L_Y \circ L_X) V) \\
              & \quad - (L_Y \circ L_X) (\phi V) + L_Y (\phi(L_X V))
                         - L_X(\phi (L_Y V)) + \phi((L_X \circ L_Y) V)
            \end{split} \\
          & = [L_X, L_Y](\phi V) - \phi([L_X, L_Y] V) \\
          & = L_{[X, Y]}(\phi V) - \phi(L_{[X, Y]} V) \\
          & = (L_{[X, Y]} \phi) V.
        \end{split}
      \end{equation}
  \end{enumerate}
  By structural induction on the space of tensors, we get the desired result on
  the whole tensor algebra of $\mathfrak{X} M$.
\end{proof}
\begin{definition}
  Let
    $(M, g)$ be a (pseudo-)Riemannian manifold.
    $X \in \mathfrak{X} M$.
  We say that $X$ is a (global) \textbf{Killing vector field}
  (named after German mathematician Wilhem Killing) if
  \begin{equation}
    L_X g = 0.
  \end{equation}
  Denote by ${\rm Killing}(M, g)$ the space of all Killing vector fields on $M$.
\end{definition}
\begin{proposition}
  Let $(M, g)$ be a pseudo-Riemannian manifold.
  Then ${\rm Killing}(M, g)$ is a subalgebra of the Lie algebra of vector fields
  on $M$.
\end{proposition}
\begin{proof}
  Let $X, Y \in {\rm Killing}(M, g)$, i.e., $L_X g = L_Y g = 0$.
  Then
  \begin{equation}
    \begin{split}
      L_{[X, Y]} g
      = [L_X, L_Y] g = L_X(L_Y g) - L_Y(L_X g) = L_X(0) - L_Y(0) = 0.
      \qedhere
    \end{split}
  \end{equation}
\end{proof}
\begin{example}
  We are going to calculate the Lie algebra of the Killing vector fields on a
  pseudo-Euclidean space $(V, g)$ of dimension $D$.
  By Sylvester's law of inertia we can choose a basis in which the metric has a
  diagonal form with values
  \begin{equation}
    g_{i, j} = s_i \delta_{i, j},\ s_i \in \{-1, 1\}\ (i, j = 1, ..., D).
  \end{equation}
  In other words, in the chosen coordinate system
  \begin{equation}
    g = \sum_{i = 1}^D s_i\, d x^i \otimes d x^i.
  \end{equation}
  Let $X = \sum_{i = 1}^D f^i \partial_{x_i}$ be a Killing vector field.
  First, for any $i \in 1, ..., D$ we calculate
  \begin{equation}
    L_X(d x^i)
    = d(L_X x^i)
    = d f^i
    = \sum_{j = 1}^D (\partial_{x_j} f^i)\, d x^j.
  \end{equation}
  Then
  \begin{equation}
    0
    = L_X g
    = \sum_{i = 1}^D s_i L_X(d x^i) \otimes d x^i + d x^i \otimes L_X(d x^i)
    = \sum_{i, j = 1}^D
      (s_i \partial_{x_j} f^i + s_j \partial_{x_i} f^j)\, d x^i \otimes d x^j.
  \end{equation}
  Hence, for $i, j = 1, ..., D$,
  \begin{equation}
    s_i \partial_{x_j} f^i + s_j \partial_{x_i} f^j = 0.
  \end{equation}
  We can show that the above system of PDE leads to the following
  $D (D + 1) / 2$ linearly independent solutions:
  \begin{enumerate}
    \item
      the $D$ basis vectors fields
      \begin{equation}
        C_i := \partial_{x_i},\ i = 1, ..., D;
      \end{equation}
    \item
      the $D (D - 1) / 2$ linear vector fields (boosts)
      \begin{equation}
        B_{i, j}
        := s_j x_j \partial_{x_i} - s_i x_i \partial_{x_j},\
        1 \leq i < j \leq D.
      \end{equation}
  \end{enumerate}
\end{example}

\begin{definition}
  Let $M$ be a smooth manifold, $I$ be a real interval.
  A \textbf{smooth curve} on $M$ with domain $I$ is a smooth function
  $\gamma \colon I \to M$.
\end{definition}
\begin{definition}
  Let
    $M$ be a smooth manifold,
    $I$ be a real interval,
    $\gamma \colon I \to M$ be a smooth curve.
  We define the \textbf{derivative} of $\gamma$,
  $\dot{\gamma} \colon I \to T M$, by
  \begin{equation}
    \restrict{\dot{\gamma}}{t} \in T_{\gamma(t) M},\ t \in I,
  \end{equation}
  such that for any $f \in \mathcal{F} M$,
  \begin{equation}
    \restrict{\dot{\gamma}}{t} f := (f \circ \gamma)'(t).
  \end{equation}
  Here, $'$ denotes the standard derivative of the single-variable function
  $f \circ \gamma \colon I \to R$.
\end{definition}
\begin{definition}
  Let $M$ be a smooth manifold, $X \in \mathfrak{X} M$, $I$ be a real interval.
  We say that a smooth curve $\gamma \colon I \to M$ is an
  \textbf{integral curve} for $X$ if
  \begin{equation}
    \dot{\gamma} = X \circ \gamma.
  \end{equation}
  In other words, for any $t \in I$,
  \begin{equation}
    \restrict{\dot{\gamma}}{t} = \restrict{X}{\gamma(t)}.
  \end{equation}
\end{definition}
\begin{proposition}
  Let $M$ be a smooth manifold, $X \in \mathfrak{X} M$, $x_0 \in M$.
  Then
  \begin{enumerate}
    \item
      \textbf{Existence.}
      There exists an open interval $I$ containing $0$ and an integral curve
      $\gamma \colon I \to \R$ of $X$ with $\gamma(0) = x_0$.
    \item
      \textbf{Uniqueness.}
      In any open interval $I$ containing $0$ there exists at most one integral
      curve $\gamma \colon I \to \R$ of $X$ with $\gamma(0) = x_0$.
  \end{enumerate}
\end{proposition}
\begin{definition}
  Let $M$ be a smooth manifold, $X \in \mathfrak{X} M$, $x_0 \in M$.
  The \textbf{flow} of $X$ is the unique function
  (may not be defined everywhere)
  $\varphi \colon \R \to (M \to M)$
  satisfying the following: for any $x \in M$, 
  $\varphi_{\cdot}(x) \colon \R \to M$ is the integral curve of $X$ with
  $\varphi_0(x) = x$.
\end{definition}
\begin{example}
  Let $M := \R^2$ and
  $X := - y \frac{\partial}{\partial x} + x \frac{\partial}{\partial y}$.
  We will first find the integral curves of $X$.
  Let $\gamma \colon \R \to \R^2$ be an integral curve,
  $\gamma(t) = (u(t), v(t))$.
  Then $\dot{\gamma}(t) = (\dot{u}(t), \dot{v}(t))$
  and $(X \circ \gamma)(t) = (- v(t), u(t))$.
  Hence, we get the system
  \begin{equation}
    \dot{u} = - v,\ \dot{v} = u.
  \end{equation}
  Its soultions are of the form
  \begin{equation}
    (u(t), v(t)) = (A \cos t - B \sin t, A \sin t + B \cos t),\ A, B \in \R.
  \end{equation}
  Geometrically, they are circles centred at $0$ with radii $\sqrt{A^2 + B^2}$.

  If $\varphi \colon \R \to (\R^2 \to \R^2)$ is the flow of $X$, and has the
  above form, then for any $(x, y) \in \R^2$,
  \begin{equation}
    (x, y) = \varphi_0(x, y) = (A, B).
  \end{equation}
  Hence, the flow of $X$ is given by
  \begin{equation}
    \varphi_t(x, y)
    = (x \cos t - y \sin t, x \sin t + y \cos t)
    =
    \begin{pmatrix}
      \cos t & - \sin t \\
      \sin t & \cos t
    \end{pmatrix}
    \begin{pmatrix}
      x \\
      y
    \end{pmatrix}.
  \end{equation}
\end{example}
\begin{proposition}
  Let
    $M$ be a smooth manifold,
    $X \in \mathfrak{X} M$,
    $\varphi$ be the flow of $X$,
    $s \in \R$.
  Then $\varphi_t \colon M \to M$ is a diffeomorphism (an automorphism).
\end{proposition}
\begin{proposition}
  Let
    $M$ be a smooth manifold,
    $X \in \mathfrak{X} M$,
    $\varphi$ be the flow of $X$,
    $s, t \in \R$.
  Then
  \begin{equation}
    \varphi_{s + t} = \varphi_s \circ \varphi_t.
  \end{equation}
\end{proposition}
\begin{corollary}
  Let
    $M$ be a smooth manifold,
    $X \in \mathfrak{X} M$.
    $\varphi$ be the flow of $X$.
  If $\varphi$ is defined everywhere, then it is a group action of $(\R, +)$ on
  $\Aut M$.
\end{corollary}
\begin{remark}
  For this reason, the flow of a vector field on a manifold is also called a
  \textbf{one-parameter group of diffeomorphisms}.
\end{remark}


\section{Lie groups}
\label{section:lie_groups}
\begin{definition}
  Let
    $R$ be a ring,
    $V$ be a finite-dimensional $R$-module,
    $\omega \in \Lambda^2 V^*$.
  We say that $\omega$ is \textbf{non-degenerate} or \textbf{symplectic}
  if the associated map
  \begin{equation}
    \tilde{\omega} \colon V \to V^*,\
    X \in V \mapsto \tilde{\omega}(X) := i_X \omega \in V^*,
  \end{equation}
  is an isomorphism.

  The pair $(V, \omega)$ is called a \textbf{symplectic module}
  (or \textbf{symplectic vector space} if $R$ is a field).
\end{definition}
\begin{proposition}
  Let
    $R$ be a ring without,
    $(V, \omega)$ be a finite-dimensional symplectic module over $R$.
  Assume that for any $x \in R,\ x + x = 0 \Rightarrow x = 0$.
  Then $\dim V$ is an even number.
\end{proposition}
\begin{proof}
  Let $n = \dim V$.
  In a basis of $V$ $\omega$ is represented by an antisymmetric matrix $A$.
  But then
  \begin{equation}
    \det A = \det(A^T) = \det(-A) = (-1)^n \det A.
  \end{equation}
  If $n$ is odd, then $\det A + \det A = 0$.
  By assumption this means that $\det A = 0$
  which contradicts the nondegeneracy of $\omega$.
  Hence, $n$ is even.
\end{proof}
\begin{definition}
  Let $M$ be a smooth manifold, $\omega \in \Omega^\bullet M$.
  We say that:
  \begin{enumerate}
    \item
      $\omega$ is \textbf{closed} if $d \omega = 0$
    \item
      $\omega$ is \textbf{exact} if there exists $\eta \in \Omega^\bullet M$
      such that $d \eta = \omega$.
  \end{enumerate}
\end{definition}
\begin{proposition}
  Let $M$ be a smooth manifold, $\omega \in \Omega^\bullet M$.
  If $\omega$ is exact, then it is closed.
\end{proposition}
\begin{proof}
  Let $\eta \in \Omega^\bullet M$ be such that $d \eta = \omega$.
  Then $d \omega = d (d \eta) = 0$, i.e., $\omega$ is closed.
\end{proof}
\begin{definition}
  Let $M$ be a smooth manifold, $\omega \in \Omega^2 M$.
  We say that $\omega$ is a \textbf{symplectic form}
  if it is non-degenerate
  (with base module $\mathfrak{X} M$ over $\mathcal{F} M$) and closed.

  The pair $(M, \omega)$ is called a \textbf{symplectic manifold}.
\end{definition}
\begin{proposition}
  Let $(M, \omega)$ be a symplectic manifold.
  Then $M$ is even-dimensional.
\end{proposition}
\begin{definition}
  Let $Q$ be a smooth manifold.
  Consider the cotangent bundle $T^* Q$ with bundle projection
  $\pi \colon T^* Q \to Q$
  with differential $d \pi \colon T(T^* Q) \to T Q$.
  Define the \textbf{tautological one-form}
  $\theta \colon T^* Q \to T^* (T^* Q)$ as follows:
  for any $(q, p) \in T^*Q$ (i.e,. $q \in Q$, $p \in \Hom(T_q Q, \R)$),
  \begin{equation}
    \restrict{\theta}{(q, p)}
    := p \circ \restrict{d \pi}{(q, p)} \in T^*_{(q, p)}(T^* Q).
  \end{equation}
  In other words, if we denote $M := T^* Q$, then $\theta$ is a section of its
  cotangent bundle $T^* M$, i.e., an $1$-form on $M$. 
\end{definition}
\begin{discussion}
  Let $Q$ be a smooth manifold, $\pi \colon T^* Q \to Q$ be the projection.
  Then a $1$-form on $Q$ is a section of $\colon T^* Q$, i.e., a smooth map
  $\mu \colon Q \to \colon T^* Q$ such that $\pi \circ \mu = \id_Q$.
  As such it has a pullback
  $\mu^* \colon \Omega^\bullet(T^* Q) \to \Omega^\bullet Q$.
\end{discussion}
\begin{proposition}
  Let
    $Q$ be a smooth manifold,
    $\theta$ be the tautological one-form on $T^* Q$,
    $\mu \in \Omega^1 Q$.
  Then
  \begin{equation}
    \mu^* \theta = \mu.
  \end{equation}
\end{proposition}
\begin{proof}
  Let $q \in Q$.
  Then
  \begin{equation}
    \restrict{\mu^* \theta}{q}
    = \restrict{\theta}{\mu q} \circ \restrict{d \mu}{q}
    = \restrict{\mu}{q} \circ \restrict{(d \pi)}{\mu q}
      \circ \restrict{d \mu}{q}
    = \restrict{\mu}{q} \circ \restrict{d(\pi \circ \mu)}{q}
    = \restrict{\mu}{q}.
  \end{equation}
  Since $q$ is arbitrary, $\mu^* \theta = \mu$.
\end{proof}
\begin{definition}
  Let
    $Q$ be a smooth manifold,
    $\theta \in \Omega^1(T^* Q)$ be the tautological one-form.
  Define $\omega := - d \theta$.
  The pair $(T^* Q, \omega)$ is called the \textbf{phase space} of $Q$.
  (In this setting $Q$ is usually called the \textbf{configuration space}.)
\end{definition}
\begin{remark}
  Let $Q$ be a smooth manifold.
  The elements of $T^* Q$ are of the form $(q, p)$ where $q \in Q$ and
  $p \in T^*_q Q = \Hom(T_q Q, \R)$.
  $q$ is called \textbf{generalised position}, while $p$ is called
  \textbf{generalised momentum}.
\end{remark}
\begin{proposition}
  Let
    $Q$ be a smooth manifold,
    $(T^* Q, \omega)$ be its phase space.
  Then $(T^* Q, \omega)$ is a symplectic manifold.
\end{proposition}
\begin{definition}
  Let $Q$ be a smooth manifold of dimension $n$.
  Consider a point $q_0 \in Q$ and let $(U, \hat{\varphi})$ be a chart around
  $q_0$, i.e., $U$ is a neighbourhood of $q_0$ and
  $\hat{\varphi} \colon U \to \R^n$ is a diffeomorphism.
  Let $\{\hat{q}^i \colon U \to \R\}_{i = 1}^n$ be the corresponding local
  coordinates, i.e., if $\{\pi^i \colon \R^n \to \R\}_{i = 1}^n$ are the
  projection maps, then $\{\hat{q}^i = \pi^i \circ \hat{\varphi}\}_{i = 1}^n$.
  Let $i \in \{1, ..., n\}$.
  Define \textbf{position coordinate} $q^i \colon T^* U \to \R$ by
  \begin{equation}
    q^i := \hat{q}^i \circ \restrict{\pi}{U}.
  \end{equation}
  Also, define \textbf{momentum coordinate} $p_i \colon T^* U \to \R$
  as follows: for any $(q, p) \in T^* U$,
  \begin{equation}
    p_i(q, p)
    := p\left(\restrict{\frac{\partial}{\partial \hat{q}^i}}{q}\right).
  \end{equation}
\end{definition}
\begin{proposition}
  Let
    $Q$ be a smooth manifold of dimension $n$,
    $q_0 \in Q$,
    $(U, \hat{\varphi})$ be a chart around $q_0$,
    $\{\hat{q}^i \colon U \to \R\}_{i = 1}^n$ be the corresponding local
      coordinates,
    $\{q^i \colon T^* U \to \R\}_{i = 1}^n$ be the corresponding position
      coordinates,
    $\{p_i \colon T^* U \to \R\}_{i = 1}^n$ be the corresponding momentum
      coordinates.
  Then the map $\varphi \colon T^* U \to \R^{2 n}$ defined by
  \begin{equation}
    \varphi(q, p) = (q^1(q, p), ..., q^n(q, p), p_1(q, p), ..., p_n(q, p))
  \end{equation}
  is a diffeomorphism, i.e., $(T^* U, \varphi)$ is a chart around $(q_0, 0)$.
  (The covector in $T^*_{q_0}$ does not matter, so we make the trivial choice by
  taking zero.)

  These local coordinates are called \textbf{generalised coordinates}.
\end{proposition}
\begin{remark}
  From now on, given a manifold $Q$ and a chart $(U, \hat{\varphi})$, unless
  stated otherwise, we will fix the notation and use the objects defined above:
  the projection map $\pi \colon T^* Q \to Q$, the tautological one-form
  $\theta$ and the canonical symplectic form $\omega = - d \theta$;
  for $i = 1, ..., n$ the coordinate maps $\hat{q}^i$, $q^i$, and $p_i$;
  the chart $(T^* U, \varphi)$.
\end{remark}
\begin{proposition}
  Let
    $Q$ be a smooth manifold,
    $\xi \in \Omega^1(T^* Q)$ has the following property:
    for any $1$-form $\mu$ on $Q$, $\mu^* \xi= 0$.
  Then $\xi = 0$.
\end{proposition}
\begin{proof}
  Let
    $n := \dim Q$, $(U, \hat{\varphi})$ be a chart on $Q$ and
    $\{f_i, g^i \in \mathcal{F}(T^* U)\}_{i = 1}^n$ be such that
  \begin{equation}
    \restrict{\xi}{U} = \sum_{i = 1}^n f_i\, d q^i + \sum_{i = 1}^n g^i\, d p_i.
  \end{equation}
  Take arbitrary $\{h_j \in \mathcal{F} U\}_{j = 1}^n$ so that
  \begin{equation}
    \restrict{\mu}{U} = \sum_{j = 1}^n h_j\, d \hat{q}^j.
  \end{equation}
  Note that
  $q^i \circ \restrict{\mu}{U} = \hat{q}^i$ and
  $p_i \circ \restrict{\mu}{U} = h_i$.
  Hence,
  \begin{equation}
    0 
    = \restrict{(\mu^* \xi)}{U}
    = \sum_{i = 1}^n (f_i \circ \restrict{\mu}{U})\,
      d(q^i \circ \restrict{\mu}{U})
    + \sum_{i = 1}^n (g^i \circ \restrict{\mu}{U})\,
      d(p_i \circ \restrict{\mu}{U})
    = \sum_{i = 1}^n (f_i \circ \restrict{\mu}{U})\, d \hat{q}^i
    + \sum_{i = 1}^n (g^i \circ \restrict{\mu}{U})\, d h_i.
  \end{equation}
  Fix $q_0 \in U$, $p_0 \in T^*_{q_0} Q$ so that $(p_0, q_0) \in T^* U$.
  Denote
  \begin{equation}
    c_i
    :=
    p_0\left(\restrict{\frac{\partial}{\partial \hat{q}^i}}{q_0}\right),
    i = 1, ..., n,
  \end{equation}
  so that
  \begin{equation}
    p_0 = \sum_{i = 1}^n c_i \restrict{d \hat{q}^i}{q_0}.
  \end{equation}
  \begin{enumerate}
    \item
      We will first prove that
      for any $i \in \{1, ..., n\}$, $f_i(q_0, p_0) = 0$.
      Define the constant functions
      \begin{equation}
        h_i(q) := c_i,\ i \in \{1, ..., n\},\ q \in U.
      \end{equation}
      Then for any $i \in \{1, ..., n\}$, $d h_i = 0$.
      Hence,
      \begin{equation}
        \begin{split}
          0
          & = \restrict{(\mu^* \xi)}{q_0} \\
          & = \sum_{i = 1}^n
              f_i(q_0, \sum_{j = 1}^n h_j(q_0) \restrict{d \hat{q}^j}{q_0})\,
              \restrict{d \hat{q}^i}{q_0} \\
          & = \sum_{i = 1}^n
              f_i(q_0, \sum_{j = 1}^n c_j \restrict{d \hat{q}^j}{q_0})\,
              \restrict{d \hat{q}^i}{q_0} \\
          & = \sum_{i = 1}^n f_i(q_0, p_0)\, \restrict{d \hat{q}^i}{q_0}.
        \end{split}
      \end{equation}
      Therefore, for any $i \in \{1, ..., n\}$, $f_i(q_0, p_0) = 0$.
    \item
      We will now prove that
      for any $i \in \{1, ..., n\}$, $g^i(q_0, p_0) = 0$.
      Define the linear functions
      \begin{equation}
        h_i(q) := c_i + \hat{q}^i(q) - \hat{q}^i(q_0).
      \end{equation}
      Then for any $i \in \{1, ..., n\}$,
      $d h_i = d \hat{q}^i$ and $h_i(q) = c_i$.
      Hence,
      \begin{equation}
        \begin{split}
          0
          & = \restrict{(\mu^* \xi)}{q_0} \\
          & = \sum_{i = 1}^n
              g^i(q_0, \sum_{j = 1}^n h_j(q_0) \restrict{d \hat{q}^j}{q_0})\,
              \restrict{d h_i}{q_0} \\
          & = \sum_{i = 1}^n
              g^i(q_0, \sum_{j = 1}^n c_j \restrict{d \hat{q}^j}{q_0})\,
              \restrict{d \hat{q}^i}{q_0} \\
          & = \sum_{i = 1}^n g^i(q_0, p_0)\, \restrict{d \hat{q}^i}{q_0}.
        \end{split}
      \end{equation}
      Therefore, for any $i \in \{1, ..., n\}$, $g^i(q_0, p_0) = 0$.
  \end{enumerate}
  Since $(q_0, p_0) \in T^* U$ was arbitrary, we conclude that
  for any $i \in \{1, ..., n\}$, $f_i = g^i = 0$.
  Hence, $\restrict{\xi}{U} = 0$.
  Taking an atlas $\{(U_\alpha, \hat{\varphi}_\alpha)\}_{\alpha \in A}$ of $Q$
  (for some index set $A$), we conclude that $\xi = 0$.
\end{proof}
\begin{corollary}
  Let
    $Q$ be a smooth manifold,
    $\theta$ be the tautological one-form on $T^* Q$,
    $\eta \in \Omega^1(T^* Q)$ has the following property:
    for any $1$-form $\mu$ on $Q$, $\mu^* \eta = \mu$.
  Then $\eta = \theta$.
\end{corollary}
\begin{proof}
  Write $\eta = \theta + \xi$, i.e., $\xi := \eta - \theta$.
  Then, for any $\mu \in \Omega^1 Q$,
  \begin{equation}
    \mu
    = \mu^* \eta
    = \mu^* \theta + \mu^* \xi
    = \mu + \mu^* \xi
    \Rightarrow \mu^* \xi = 0.
  \end{equation}
  But from the previous proposition it follows that $\xi = 0$,
  and hence $\eta = \theta$.
\end{proof}
\begin{proposition}
  Let
    $Q$ be a smooth manifold of dimension $n$,
    $(U, \hat{\varphi})$ be a chart,
    $(q, p) \in T^* U$,
    $ i \in \{1, ..., n\}$.
  Then
  \begin{equation}
    \restrict{d \pi}{(q, p)}
    \left(\restrict{\frac{\partial}{\partial q^i}}{(q, p)}\right)
    = \restrict{\frac{\partial}{\partial \hat{q}^i}}{q}
  \end{equation}
  and
  \begin{equation}
    \restrict{d \pi}{(q, p)}
    \left(\restrict{\frac{\partial}{\partial p^i}}{(q, p)}\right)
    = 0.
  \end{equation}
\end{proposition}
\begin{proof}
  Let $f \colon Q \to \R$ be smooth.
  Define the functions
  $\hat{g} := f \circ \hat{\varphi}^{-1} \colon \R^n \to \R$ and
  $g := f \circ \pi \circ \varphi^{-1} \colon \R^{2 n} \to \R$.
  Let $(X^1, ..., X^n, Y^1, ..., Y^n) := \varphi(p, q) \in \R^n$.
  This means that $(X^1, ..., X^n) = \hat{\varphi}(q)$.
  Then
  \begin{equation}
    g(X^1, ..., X^n, Y^1, ..., Y^n)
    = f(\pi(q, p))
    = f(q)
    = \hat{g}(X^1, ..., X^n).
  \end{equation}
  Hence,
  \begin{equation}
    \begin{split}
      \frac{\partial g}{\partial x^i}(X^1, ..., X^n, Y^1, ..., Y^n)
      & = \lim_{h \to 0}
        \frac
        {g(X^1, ..., X^i + h, ..., X^n, Y^1, ..., Y^n)
         - g(X^1, ..., X^n, Y^1, ..., Y^n)}
        {h} \\
      & = \lim_{h \to 0}
        \frac{\hat{g}(X^1, ..., X^i + h, ..., X^n) - \hat{g}(X^1, ..., X^n)}{h}
        \\
      & = \frac{\partial \hat{g}}{\partial \hat{x}^i}(X^1, ..., X^n).
    \end{split}
  \end{equation}
  Similarly, since $g$ is constant with respect to the last $n$ coordinates,
  \begin{equation}
    \frac{\partial g}{\partial x^{n + i}}(X^1, ..., X^n, Y^1, ..., Y^n) = 0
  \end{equation}
  Take the standard coordinate systems (given by projections)
  $\{\hat{x}^k\}_{k = 1}^n$ on $\R^n$ and
  $\{x^k\}_{k = 1}^{2 n}$ on $\R^{2 n}$.
  Then, by the definitions of differential and partial derivative on manifold,
  \begin{equation}
    (\restrict{d \pi}{(q, p)}
      \left(\restrict{\frac{\partial}{\partial q^i}}{(q, p)}\right)) f
    = \restrict{\frac{\partial}{\partial q^i}}{(q, p)}(f \circ \pi)
    = \frac{\partial(f \circ \pi \circ \varphi^{-1})}{x^i}(\varphi(q, p))
    = \frac{\partial(f \circ \hat{\varphi}^{-1})}{\hat{x}^i}(\hat{\varphi}(q))
    = \restrict{\frac{\partial}{\partial \hat{q}^i}}{q} f,
  \end{equation}
  from which it follows that the first equality holds.
  Similarly,
  \begin{equation}
    (\restrict{d \pi}{(q, p)}
      \left(\restrict{\frac{\partial}{\partial p^i}}{(q, p)}\right)) f
    = \restrict{\frac{\partial}{\partial p^i}}{(q, p)}(f \circ \pi)
    = \frac{\partial(f \circ \pi \circ \varphi^{-1})}{x^{i + n}}(\varphi(q, p))
    = 0,
  \end{equation}
  from which it follows that the second equality holds.
\end{proof}
\begin{proposition}[Tautological one-form in generalised coordinates]
  Let
    $Q$ be a smooth manifold of dimension $n$,
    $(U, \hat{\varphi})$ be a chart.
  Then
  \begin{equation}
    \restrict{\theta}{U} = \sum_{i = 1}^n p_i\, d q^i.
  \end{equation}
\end{proposition}
\begin{proof}
  Let $(q, p) \in T^* U$.
  Recall that $\restrict{\theta}{(q, p)} = p \circ \restrict{d \pi}{(q, p)}$.
  Hence,
  \begin{equation}
    \restrict{\theta}{(q, p)}
    \left(\restrict{\frac{\partial}{\partial q^i}}{(q, p)}\right)
    = p\left(\restrict{\frac{\partial}{\partial \hat{q}^i}}{q}\right)
    = p_i(q, p),
  \end{equation}
  and
  \begin{equation}
    \restrict{\theta}{(q, p)}
    \left(\restrict{\frac{\partial}{\partial p^i}}{(q, p)}\right)
    = p(0)
    = 0.
  \end{equation}
  Therefore,
  \begin{equation}
    \restrict{\theta}{(q, p)}
    = \sum_{i = 1}^n
      \restrict{\theta}{(q, p)}
      \left(\restrict{\frac{\partial}{\partial q^i}}{(q, p)}\right)\,
      \restrict{d q^i}{(q, p)}
    + \sum_{i = 1}^n
      \restrict{\theta}{(q, p)}
      \left(\restrict{\frac{\partial}{\partial p^i}}{(q, p)}\right)\,
      \restrict{d p^i}{(q, p)}
    = \sum_{i = 1}^n p_i(q, p)\, \restrict{d q^i}{(q, p)},
  \end{equation}
  from which the proposition follows.
\end{proof}
\begin{corollary}[Canonical symplectic in generalised coordinates]
  Let
    $Q$ be a smooth manifold of dimension $n$,
    $(U, \hat{\varphi})$ be a chart.
  Then
  \begin{equation}
    \restrict{\omega}{U} = \sum_{i = 1}^n d q^i \wedge d p_i.
  \end{equation}
\end{corollary}
\begin{proof}
  Let $i \in \{1, ..., n\}$.
  Then
  \begin{equation}
    - d(p_i\, d q^i) = - d p_i \wedge d q^i = d q^i \wedge d p_i.
  \end{equation}
  Summing up for all $i$, we get the desired result.
\end{proof}
\begin{definition}
  Let $(M, \omega)$ be a symplectic manifold, $f \in \mathcal{F} M$.
  We say that $X \in \mathfrak{X} M$ is a \textbf{Hamiltonian vector field} for
  $f$ if
  \begin{equation}
    i_X \omega + d_0 f = 0.
  \end{equation}
\end{definition}
\begin{proposition}
  Let $(M, \omega)$ be a symplectic manifold, $f \in \mathcal{F} M$.
  Then there exists a unique Hamiltonian vector field for $f$.
\end{proposition}
\begin{proof}
  The non-degeneracy of $\omega$ means that we can interpret the symplectic form
  as the isomorphism $\tilde{\omega} \colon \mathfrak{X} M \to \Omega^1 M$,
  given by
  \begin{equation}
    (\tilde{\omega} X) := i_X \omega,\ X \in \mathfrak{X} M.
  \end{equation}
  Hence, the problem at hand has a unique solution
  $X = \tilde{\omega}^{-1}(- d_0 f)$.
\end{proof}
\begin{definition}
  Let $(M, \omega)$ be a symplectic manifold.
  Define the map $\hamiltonian \colon \mathcal{F} M \to \mathfrak{X} M$ by
  \begin{equation}
    \hamiltonian = - \tilde{\omega}^{-1} \circ d_0.
  \end{equation}
  It maps a function to its corresponding Hamiltonian vector field.
  We will write $\hamiltonian_f$ instead of $\hamiltonian(f)$
  for $f \in \mathcal{F} M$.
\end{definition}
\begin{proposition}
  Let
    $Q$ be a smooth manifold of dimension $n$,
    $(U, \hat{\varphi})$ be a chart on $Q$,
    $f \in \mathcal{F}(T^* Q)$.
  Then
  \begin{equation}
    \hamiltonian_f
    = \sum_{i = 1}^n
    \left(
      - \frac{\partial f}{\partial p^i} \frac{\partial}{\partial q^i}
      + \frac{\partial f}{\partial q^i} \frac{\partial}{\partial p^i}
    \right).
  \end{equation}
\end{proposition}
\begin{proof}
  First, note that
  $i_{\frac{\partial}{\partial q^i}} \omega = d p^i$ and
  $i_{\frac{\partial}{\partial p^i}} \omega = - d q^i$.
  Denote
  \begin{equation}
    X
    := \sum_{i = 1}^n
    \left(
      - \frac{\partial f}{\partial p^i} \frac{\partial}{\partial q^i}
      + \frac{\partial f}{\partial q^i} \frac{\partial}{\partial p^i}
    \right).
  \end{equation}
  Then
  \begin{equation}
    i_X \omega
    = \sum_{i}^n
    \left(
      - \frac{\partial f}{\partial p^i} i_{\frac{\partial}{\partial q^i}} \omega
      + \frac{\partial f}{\partial q^i} i_{\frac{\partial}{\partial p^i}} \omega
    \right)
    = \sum_{i}^n
    \left(
      - \frac{\partial f}{\partial p^i} d p^i
      - \frac{\partial f}{\partial q^i} d q^i
    \right)
    = - d f.
  \end{equation}
  Hence, $\hamiltonian_f = X$.
\end{proof}
\begin{proposition}
  Let $(M, \omega)$ be a symplectic manifold, $f, g \in \mathcal{F} M$.
  Then
  \begin{equation}
    \hamiltonian_{f g} = f \hamiltonian_g + g \hamiltonian_f.
  \end{equation}
\end{proposition}
\begin{proof}
  Follows directly from the Leibniz rule for $d_0$.
\end{proof}
\begin{definition}
  Let $(M, \omega)$ be a symplectic manifold, $X \in \mathfrak{X} M$.
  We say that $X$ is a \textbf{symplectic vector field} if $L_X \omega = 0$.
\end{definition}
\begin{remark}
  Since $L_{\lie{X}{Y}} = \lie{L_X}{L_Y} = L_X \circ L_Y - L_Y \circ L_X$,
  the symplectic vector fields form a Lie subalgebra of the Lie algebra of
  vector fields.
\end{remark}
\begin{proposition}
  Let $(M, \omega)$ be a symplectic manifold, $f \in \mathcal{F} M$.
  Then $\hamiltonian_f$ is a symplectic vector field.
\end{proposition}
\begin{proof}
  $
    L_{\hamiltonian_f} \omega
    = i_{\hamiltonian_f}(d \omega) + d(i_{\hamiltonian_f} \omega)
    = i_{\hamiltonian_f} 0 - d(d f)
    = 0.
  $
\end{proof}
\begin{proposition}
  Let
    $(M, \omega)$ be a symplectic manifold,
    $X \in \mathfrak{X} M$ be a symplectic vector fields.
  Then
  \begin{equation}
    d(i_X \omega) = 0.
  \end{equation}
\end{proposition}
\begin{proof}
  $
    d(i_X \omega)
    = L_X \omega - i_X(d \omega)
    = 0 - 0
    = 0.
  $
\end{proof}
\begin{proposition}
  Let $M$ be a smooth manifold, $X, Y \in \mathfrak{X} M$.
  Then
  \begin{equation}
    L_X \circ i_Y = i_{\lie{X}{Y}} + i_Y \circ L_X.
  \end{equation}
\end{proposition}
\begin{proposition}
  Let
    $(M, \omega)$ be a symplectic manifold,
    $X, Y \in \mathfrak{X} M$ be symplectic vector fields.
  Then
  \begin{equation}
    \lie{X}{Y} = \hamiltonian_{i_Y(i_X \omega)}.
  \end{equation}
\end{proposition}
\begin{proof}
  \begin{equation}
    i_{\lie{X}{Y}} \omega
    = (L_X \circ i_Y - i_Y \circ L_X) \omega
    = (L_X \circ i_Y) \omega
    = ((d \circ i_X + i_X \circ d) \circ i_Y) \omega
    = d(i_X(i_Y \omega))
    = - d(i_Y(i_X \omega)).
  \end{equation}
  We get the desired result from the definition of $\hamiltonian$.
\end{proof}
\begin{definition}
  Let $(M, \omega)$ be a symplectic manifold.
  Define the \textbf{Poisson bracket}
  $\poisson{\cdot}{\cdot} \colon \mathcal{F} M \to \mathcal{F} M$ by
  \begin{equation}
    \poisson{f}{g}
    := i_{\hamiltonian_g}(i_{\hamiltonian_f} \omega),\
    f, g \in \mathcal{F} M.
  \end{equation}
\end{definition}
\begin{corollary}
  Let $(M, \omega)$ be a symplectic manifold, $f, g \in \mathcal{F} M$.
  Then
  \begin{equation}
    \lie{\hamiltonian_f}{\hamiltonian_g}
    = \hamiltonian_{i_{\hamiltonian_g}(i_{\hamiltonian_f} \omega)}
    = \poisson{f}{g}.
  \end{equation}
\end{corollary}
\begin{proposition}[Leibniz rule holds for the Poisson bracket]
  Let $(M, \omega)$ be a symplectic manifold, $f, g, h \in \mathcal{F} M$.
  Then
  \begin{equation}
    \poisson{f}{g h} = \poisson{f}{g} h + g \poisson{f}{h}.
  \end{equation}
\end{proposition}
\begin{proof}
  $
    \poisson{f}{g h}
    = i_{\hamiltonian_{g h}}(i_{\hamiltonian_f} \omega)
    = i_{h \hamiltonian_g + g \hamiltonian_{h}}(i_{\hamiltonian_f} \omega)
    = (i_{\hamiltonian_g}(i_{\hamiltonian_f} \omega))\, h
      + g\, (i_{\hamiltonian_h}(i_{\hamiltonian_f} \omega)) 
    = \poisson{f}{g} h + g \poisson{f}{h}.
  $
\end{proof}
\begin{proposition}
  Let $(M, \omega)$ be a symplectic manifold, $f, g, h \in \mathcal{F} M$.
  Then
  \begin{equation}
    \poisson{f}{g} = L_{\hamiltonian_f} g.
  \end{equation}
\end{proposition}
\begin{proof}
  $
    \poisson{f}{g}
    = i_{\hamiltonian_g}(i_{\hamiltonian_f} \omega)
    = - i_{\hamiltonian_f} \circ i_{\hamiltonian_g} \omega
    = i_{\hamiltonian_f}(d g)
    = L_{\hamiltonian_f} g.
  $
\end{proof}
\begin{definition}
  Let $(M, \omega)$ be a symplectic manifold.
  Define
  ${\rm ad} \colon \mathcal{F} M \to (\mathcal{F} M \to \mathcal{F} M)$ by
  \begin{equation}
    {\rm ad}_f g := \poisson{f}{g},\ f, g \in \mathcal{F} M,
  \end{equation}
\end{definition}
\begin{proposition}
  Let $(M, \omega)$ be a symplectic manifold.
  Then
  \begin{equation}
    \lie{{\rm ad}_f}{{\rm ad}_g} = {\rm ad}_{\poisson{f}{g}}.
  \end{equation}
  (Here the bracket $\lie{\cdot}{\cdot}$ is the commutator of operators.)
\end{proposition}
\begin{proof}
  From the previous proposition it follows that
  \begin{equation}
    {\rm ad}_f = L_{X_f},\ f \in \mathcal{F} M.
  \end{equation}
  Hence,
  \begin{equation}
    \lie{{\rm ad}_f}{{\rm ad}_g}
    = \lie{L_{\hamiltonian_f}}{L_{\hamiltonian_g}}
    = L_{\lie{\hamiltonian_f}{\hamiltonian_g}}
    = L_{\hamiltonian_{\poisson{f}{g}}}
    = {\rm ad}_{\poisson{f}{g}}.
  \end{equation}
\end{proof}
\begin{corollary}
  Let $(M, \omega)$ be a symplectic manifold.
  Then $(\mathcal{F} M, \poisson{\cdot}{\cdot})$ is a Lie algebra over $\R$.
\end{corollary}
\begin{proof}
  Bilinearity and antisymmetry are trivial to check.
  The Jacobi identity is equivalent to the adjoint map being a Lie algebra
  homomorphism, which was the previous proposition.
\end{proof}
\begin{definition}
  Let
    $R$ be a commutative ring with unity ring,
    $(A, +, \cdot)$ be an $R$-module
    with additional structures of
    an associative algebra $(A, *)$ and
    a Lie algebra $(A, \poisson{\cdot}{\cdot})$.
  We say that $A$ is a Poisson algebra if the Lie bracket acts as a derivation,
  i.e., for all $f, g, h \in A$,
  \begin{equation}
    \poisson{f}{g * h} = \poisson{f}{g} * h + g * \poisson{f}{h}.
  \end{equation}
\end{definition}
\begin{corollary}
  Let $(M, \omega)$ be a symplectic manifold.
  Then $\mathcal{F} M$ is a Poisson algebra over $\R$.
  Here, addition, scalar multiplication, and multiplication are given by the
  corresponding pointwise operations, while the Lie bracket is given by the
  Poisson bracket.
\end{corollary}


\section{Exterior derivative}
\label{section:exterior_derivative}
\phantom{T}
\begin{example}
  The following are computations of the exterior derivative in flat space.
  \begin{enumerate}
    \item
      \textbf{Exterior derivative in $\R^2$.}
      Let $x^1, x^2 \colon \R^2 \to \R$ be the coordinate maps.
      \begin{enumerate}
        \item
          \textbf{Exterior derivative of $0$-forms.}
          Consider a $0$-form (a function)
          $f \in \Omega^1(\R^2) = \mathcal{F}(\R^2)$.
          Then
          \begin{equation}
            d_0 f
            = \frac{\partial f}{\partial x^1}\, d x^1
            + \frac{\partial f}{\partial x^2}\, d x^2.
          \end{equation}
        \item
          \textbf{Exterior derivative of $1$-forms.}
          Consider a $1$-form $\omega \in \Omega^1(\R^2)$ such that
          $\omega = f_1\, d x^1 + f_2\, d x^2$,
          for some $f_1, f_2 \in \mathcal{F}(\R^2)$.
          Then
          \begin{equation}
            \begin{split}
              d_1 \omega
              & = d f_1 \wedge d x^1 + d f_2 \wedge d x^2 \\
              & =
                \left(
                  \frac{\partial f_1}{\partial x^1}\, d x^1
                  + \frac{\partial f_1}{\partial x^2}\, d x^2
                \right)
                \wedge d x^1
                +
                \left(
                  \frac{\partial f_2}{\partial x^1}\, d x^1
                  + \frac{\partial f_2}{\partial x^2}\, d x^2
                \right)
                \wedge d x^2 \\
              & =
                \left(
                  \frac{\partial f_2}{\partial x^1}
                  - \frac{\partial f_1}{\partial x^2}
                \right)\,
                d x^1 \wedge d x^2.
            \end{split}
          \end{equation}
      \end{enumerate}
    \item
      \textbf{Exterior derivative in $\R^3$.}
      Let $x^1, x^2, x^3 \colon \R^3 \to \R$ be the coordinate maps.
      \begin{enumerate}
        \item
          \textbf{Exterior derivative of $0$-forms.}
          Consider a $0$-form (a function)
          $f \in \Omega^1(\R^3) = \mathcal{F}(\R^3)$.
          Then
          \begin{equation}
            d_0 f
            = \frac{\partial f}{\partial x^1}\, d x^1
            + \frac{\partial f}{\partial x^2}\, d x^2
            + \frac{\partial f}{\partial x^3}\, d x^3.
          \end{equation}
        \item
          \textbf{Exterior derivative of $1$-forms.}
          Consider a $1$-form $\omega \in \Omega^1(\R^3)$ such that
          $\omega = f_1\, d x^1 + f_2\, d x^2 + f_3\, d x^3$,
          for some $f_1, f_2, f_3 \in \mathcal{F}(\R^3)$.
          Then
          \begin{equation}
            \begin{split}
              d_1 \omega
              & = d f_1 \wedge d x^1 + d f_2 \wedge d x^2 + d f_3 \wedge d x^3
                \\
              & =
                \left(
                  \frac{\partial f_1}{\partial x^1}\, d x^1
                  + \frac{\partial f_1}{\partial x^2}\, d x^2
                  + \frac{\partial f_1}{\partial x^3}\, d x^3
                \right)
                \wedge d x^1 \\
              & \quad
                +
                \left(
                  \frac{\partial f_2}{\partial x^1}\, d x^1
                  + \frac{\partial f_2}{\partial x^2}\, d x^2
                  + \frac{\partial f_2}{\partial x^3}\, d x^3
                \right)
                \wedge d x^2 \\
              & \quad
                +
                \left(
                  \frac{\partial f_3}{\partial x^1}\, d x^1
                  + \frac{\partial f_3}{\partial x^2}\, d x^2
                  + \frac{\partial f_3}{\partial x^3}\, d x^3
                \right)
                \wedge d x^3
                \\
              & =
                \left(
                  \frac{\partial f_3}{\partial x^2}
                  - \frac{\partial f_2}{\partial x^3}
                \right)\,
                d x^2 \wedge d x^3
                +
                \left(
                  \frac{\partial f_1}{\partial x^3}
                  - \frac{\partial f_3}{\partial x^1}
                \right)\,
                d x^3 \wedge d x^1
                +
                \left(
                  \frac{\partial f_2}{\partial x^1}
                  - \frac{\partial f_1}{\partial x^2}
                \right)\,
                d x^1 \wedge d x^2.
            \end{split}
          \end{equation}
        \item
          \textbf{Exterior derivative of $1$-forms.}
          Consider a $2$-form $\omega \in \Omega^2(\R^3)$ such that
          \begin{equation}
            \omega
            = f_1\, d x^2 \wedge d x^3
            + f_2\, d x^3 \wedge d x^1
            + f_3\, d x^1 \wedge d x^2,
          \end{equation}
          for some $f_1, f_2, f_3 \in \mathcal{F}(\R^3)$.
          Then
          \begin{equation}
            \begin{split}
              d_2 \omega
              & = d f_1 \wedge d x^2 \wedge d x^3
                + d f_2 \wedge d x^3 \wedge d x^1
                + d f_3 \wedge d x^1 \wedge d x^2
                \\
              & =
                \left(
                  \frac{\partial f_1}{\partial x^1}\, d x^1
                  + \frac{\partial f_1}{\partial x^2}\, d x^2
                  + \frac{\partial f_1}{\partial x^3}\, d x^3
                \right)
                \wedge d x^2 \wedge d x^3 \\
              & \quad
                +
                \left(
                  \frac{\partial f_2}{\partial x^1}\, d x^1
                  + \frac{\partial f_2}{\partial x^2}\, d x^2
                  + \frac{\partial f_2}{\partial x^3}\, d x^3
                \right)
                \wedge d x^3 \wedge d x^1 \\
              & \quad
                +
                \left(
                  \frac{\partial f_3}{\partial x^1}\, d x^1
                  + \frac{\partial f_3}{\partial x^2}\, d x^2
                  + \frac{\partial f_3}{\partial x^3}\, d x^3
                \right)
                \wedge d x^1 \wedge d x^2 \\
              & =
                \left(
                  \frac{\partial f_1}{\partial x^1}
                  + \frac{\partial f_2}{\partial x^2}
                  + \frac{\partial f_3}{\partial x^3}
                \right)\,
                d x^1 \wedge d x^2 \wedge d x^3.
            \end{split}
          \end{equation}
      \end{enumerate}
  \end{enumerate}
\end{example}


\section{Symplectic manifolds}
\label{section:hamiltonian_systems}
\begin{definition}
  Let
    $R$ be a ring,
    $V$ be a finite-dimensional $R$-module,
    $\omega \in \Lambda^2 V^*$.
  We say that $\omega$ is \textbf{non-degenerate} or \textbf{symplectic}
  if the associated map
  \begin{equation}
    \tilde{\omega} \colon V \to V^*,\
    X \in V \mapsto \tilde{\omega}(X) := i_X \omega \in V^*,
  \end{equation}
  is an isomorphism.

  The pair $(V, \omega)$ is called a \textbf{symplectic module}
  (or \textbf{symplectic vector space} if $R$ is a field).
\end{definition}
\begin{proposition}
  Let
    $R$ be a ring without,
    $(V, \omega)$ be a finite-dimensional symplectic module over $R$.
  Assume that for any $x \in R,\ x + x = 0 \Rightarrow x = 0$.
  Then $\dim V$ is an even number.
\end{proposition}
\begin{proof}
  Let $n = \dim V$.
  In a basis of $V$ $\omega$ is represented by an antisymmetric matrix $A$.
  But then
  \begin{equation}
    \det A = \det(A^T) = \det(-A) = (-1)^n \det A.
  \end{equation}
  If $n$ is odd, then $\det A + \det A = 0$.
  By assumption this means that $\det A = 0$
  which contradicts the nondegeneracy of $\omega$.
  Hence, $n$ is even.
\end{proof}
\begin{definition}
  Let $M$ be a smooth manifold, $\omega \in \Omega^\bullet M$.
  We say that:
  \begin{enumerate}
    \item
      $\omega$ is \textbf{closed} if $d \omega = 0$
    \item
      $\omega$ is \textbf{exact} if there exists $\eta \in \Omega^\bullet M$
      such that $d \eta = \omega$.
  \end{enumerate}
\end{definition}
\begin{proposition}
  Let $M$ be a smooth manifold, $\omega \in \Omega^\bullet M$.
  If $\omega$ is exact, then it is closed.
\end{proposition}
\begin{proof}
  Let $\eta \in \Omega^\bullet M$ be such that $d \eta = \omega$.
  Then $d \omega = d (d \eta) = 0$, i.e., $\omega$ is closed.
\end{proof}
\begin{definition}
  Let $M$ be a smooth manifold, $\omega \in \Omega^2 M$.
  We say that $\omega$ is a \textbf{symplectic form}
  if it is non-degenerate
  (with base module $\mathfrak{X} M$ over $\mathcal{F} M$) and closed.

  The pair $(M, \omega)$ is called a \textbf{symplectic manifold}.
\end{definition}
\begin{proposition}
  Let $(M, \omega)$ be a symplectic manifold.
  Then $M$ is even-dimensional.
\end{proposition}
\begin{definition}
  Let $Q$ be a smooth manifold.
  Consider the cotangent bundle $T^* Q$ with bundle projection
  $\pi \colon T^* Q \to Q$
  with differential $d \pi \colon T(T^* Q) \to T Q$.
  Define the \textbf{tautological one-form}
  $\theta \colon T^* Q \to T^* (T^* Q)$ as follows:
  for any $(q, p) \in T^*Q$ (i.e,. $q \in Q$, $p \in \Hom(T_q Q, \R)$),
  \begin{equation}
    \restrict{\theta}{(q, p)}
    := p \circ \restrict{d \pi}{(q, p)} \in T^*_{(q, p)}(T^* Q).
  \end{equation}
  In other words, if we denote $M := T^* Q$, then $\theta$ is a section of its
  cotangent bundle $T^* M$, i.e., an $1$-form on $M$. 
\end{definition}
\begin{discussion}
  Let $Q$ be a smooth manifold, $\pi \colon T^* Q \to Q$ be the projection.
  Then a $1$-form on $Q$ is a section of $\colon T^* Q$, i.e., a smooth map
  $\mu \colon Q \to \colon T^* Q$ such that $\pi \circ \mu = \id_Q$.
  As such it has a pullback
  $\mu^* \colon \Omega^\bullet(T^* Q) \to \Omega^\bullet Q$.
\end{discussion}
\begin{proposition}
  Let
    $Q$ be a smooth manifold,
    $\theta$ be the tautological one-form on $T^* Q$,
    $\mu \in \Omega^1 Q$.
  Then
  \begin{equation}
    \mu^* \theta = \mu.
  \end{equation}
\end{proposition}
\begin{proof}
  Let $q \in Q$.
  Then
  \begin{equation}
    \restrict{\mu^* \theta}{q}
    = \restrict{\theta}{\mu q} \circ \restrict{d \mu}{q}
    = \restrict{\mu}{q} \circ \restrict{(d \pi)}{\mu q}
      \circ \restrict{d \mu}{q}
    = \restrict{\mu}{q} \circ \restrict{d(\pi \circ \mu)}{q}
    = \restrict{\mu}{q}.
  \end{equation}
  Since $q$ is arbitrary, $\mu^* \theta = \mu$.
\end{proof}
\begin{definition}
  Let
    $Q$ be a smooth manifold,
    $\theta \in \Omega^1(T^* Q)$ be the tautological one-form.
  Define $\omega := - d \theta$.
  The pair $(T^* Q, \omega)$ is called the \textbf{phase space} of $Q$.
  (In this setting $Q$ is usually called the \textbf{configuration space}.)
\end{definition}
\begin{remark}
  Let $Q$ be a smooth manifold.
  The elements of $T^* Q$ are of the form $(q, p)$ where $q \in Q$ and
  $p \in T^*_q Q = \Hom(T_q Q, \R)$.
  $q$ is called \textbf{generalised position}, while $p$ is called
  \textbf{generalised momentum}.
\end{remark}
\begin{proposition}
  Let
    $Q$ be a smooth manifold,
    $(T^* Q, \omega)$ be its phase space.
  Then $(T^* Q, \omega)$ is a symplectic manifold.
\end{proposition}
\begin{definition}
  Let $Q$ be a smooth manifold of dimension $n$.
  Consider a point $q_0 \in Q$ and let $(U, \hat{\varphi})$ be a chart around
  $q_0$, i.e., $U$ is a neighbourhood of $q_0$ and
  $\hat{\varphi} \colon U \to \R^n$ is a diffeomorphism.
  Let $\{\hat{q}^i \colon U \to \R\}_{i = 1}^n$ be the corresponding local
  coordinates, i.e., if $\{\pi^i \colon \R^n \to \R\}_{i = 1}^n$ are the
  projection maps, then $\{\hat{q}^i = \pi^i \circ \hat{\varphi}\}_{i = 1}^n$.
  Let $i \in \{1, ..., n\}$.
  Define \textbf{position coordinate} $q^i \colon T^* U \to \R$ by
  \begin{equation}
    q^i := \hat{q}^i \circ \restrict{\pi}{U}.
  \end{equation}
  Also, define \textbf{momentum coordinate} $p_i \colon T^* U \to \R$
  as follows: for any $(q, p) \in T^* U$,
  \begin{equation}
    p_i(q, p)
    := p\left(\restrict{\frac{\partial}{\partial \hat{q}^i}}{q}\right).
  \end{equation}
\end{definition}
\begin{proposition}
  Let
    $Q$ be a smooth manifold of dimension $n$,
    $q_0 \in Q$,
    $(U, \hat{\varphi})$ be a chart around $q_0$,
    $\{\hat{q}^i \colon U \to \R\}_{i = 1}^n$ be the corresponding local
      coordinates,
    $\{q^i \colon T^* U \to \R\}_{i = 1}^n$ be the corresponding position
      coordinates,
    $\{p_i \colon T^* U \to \R\}_{i = 1}^n$ be the corresponding momentum
      coordinates.
  Then the map $\varphi \colon T^* U \to \R^{2 n}$ defined by
  \begin{equation}
    \varphi(q, p) = (q^1(q, p), ..., q^n(q, p), p_1(q, p), ..., p_n(q, p))
  \end{equation}
  is a diffeomorphism, i.e., $(T^* U, \varphi)$ is a chart around $(q_0, 0)$.
  (The covector in $T^*_{q_0}$ does not matter, so we make the trivial choice by
  taking zero.)

  These local coordinates are called \textbf{generalised coordinates}.
\end{proposition}
\begin{remark}
  From now on, given a manifold $Q$ and a chart $(U, \hat{\varphi})$, unless
  stated otherwise, we will fix the notation and use the objects defined above:
  the projection map $\pi \colon T^* Q \to Q$, the tautological one-form
  $\theta$ and the canonical symplectic form $\omega = - d \theta$;
  for $i = 1, ..., n$ the coordinate maps $\hat{q}^i$, $q^i$, and $p_i$;
  the chart $(T^* U, \varphi)$.
\end{remark}
\begin{proposition}
  Let
    $Q$ be a smooth manifold,
    $\xi \in \Omega^1(T^* Q)$ has the following property:
    for any $1$-form $\mu$ on $Q$, $\mu^* \xi= 0$.
  Then $\xi = 0$.
\end{proposition}
\begin{proof}
  Let
    $n := \dim Q$, $(U, \hat{\varphi})$ be a chart on $Q$ and
    $\{f_i, g^i \in \mathcal{F}(T^* U)\}_{i = 1}^n$ be such that
  \begin{equation}
    \restrict{\xi}{U} = \sum_{i = 1}^n f_i\, d q^i + \sum_{i = 1}^n g^i\, d p_i.
  \end{equation}
  Take arbitrary $\{h_j \in \mathcal{F} U\}_{j = 1}^n$ so that
  \begin{equation}
    \restrict{\mu}{U} = \sum_{j = 1}^n h_j\, d \hat{q}^j.
  \end{equation}
  Note that
  $q^i \circ \restrict{\mu}{U} = \hat{q}^i$ and
  $p_i \circ \restrict{\mu}{U} = h_i$.
  Hence,
  \begin{equation}
    0 
    = \restrict{(\mu^* \xi)}{U}
    = \sum_{i = 1}^n (f_i \circ \restrict{\mu}{U})\,
      d(q^i \circ \restrict{\mu}{U})
    + \sum_{i = 1}^n (g^i \circ \restrict{\mu}{U})\,
      d(p_i \circ \restrict{\mu}{U})
    = \sum_{i = 1}^n (f_i \circ \restrict{\mu}{U})\, d \hat{q}^i
    + \sum_{i = 1}^n (g^i \circ \restrict{\mu}{U})\, d h_i.
  \end{equation}
  Fix $q_0 \in U$, $p_0 \in T^*_{q_0} Q$ so that $(p_0, q_0) \in T^* U$.
  Denote
  \begin{equation}
    c_i
    :=
    p_0\left(\restrict{\frac{\partial}{\partial \hat{q}^i}}{q_0}\right),
    i = 1, ..., n,
  \end{equation}
  so that
  \begin{equation}
    p_0 = \sum_{i = 1}^n c_i \restrict{d \hat{q}^i}{q_0}.
  \end{equation}
  \begin{enumerate}
    \item
      We will first prove that
      for any $i \in \{1, ..., n\}$, $f_i(q_0, p_0) = 0$.
      Define the constant functions
      \begin{equation}
        h_i(q) := c_i,\ i \in \{1, ..., n\},\ q \in U.
      \end{equation}
      Then for any $i \in \{1, ..., n\}$, $d h_i = 0$.
      Hence,
      \begin{equation}
        \begin{split}
          0
          & = \restrict{(\mu^* \xi)}{q_0} \\
          & = \sum_{i = 1}^n
              f_i(q_0, \sum_{j = 1}^n h_j(q_0) \restrict{d \hat{q}^j}{q_0})\,
              \restrict{d \hat{q}^i}{q_0} \\
          & = \sum_{i = 1}^n
              f_i(q_0, \sum_{j = 1}^n c_j \restrict{d \hat{q}^j}{q_0})\,
              \restrict{d \hat{q}^i}{q_0} \\
          & = \sum_{i = 1}^n f_i(q_0, p_0)\, \restrict{d \hat{q}^i}{q_0}.
        \end{split}
      \end{equation}
      Therefore, for any $i \in \{1, ..., n\}$, $f_i(q_0, p_0) = 0$.
    \item
      We will now prove that
      for any $i \in \{1, ..., n\}$, $g^i(q_0, p_0) = 0$.
      Define the linear functions
      \begin{equation}
        h_i(q) := c_i + \hat{q}^i(q) - \hat{q}^i(q_0).
      \end{equation}
      Then for any $i \in \{1, ..., n\}$,
      $d h_i = d \hat{q}^i$ and $h_i(q) = c_i$.
      Hence,
      \begin{equation}
        \begin{split}
          0
          & = \restrict{(\mu^* \xi)}{q_0} \\
          & = \sum_{i = 1}^n
              g^i(q_0, \sum_{j = 1}^n h_j(q_0) \restrict{d \hat{q}^j}{q_0})\,
              \restrict{d h_i}{q_0} \\
          & = \sum_{i = 1}^n
              g^i(q_0, \sum_{j = 1}^n c_j \restrict{d \hat{q}^j}{q_0})\,
              \restrict{d \hat{q}^i}{q_0} \\
          & = \sum_{i = 1}^n g^i(q_0, p_0)\, \restrict{d \hat{q}^i}{q_0}.
        \end{split}
      \end{equation}
      Therefore, for any $i \in \{1, ..., n\}$, $g^i(q_0, p_0) = 0$.
  \end{enumerate}
  Since $(q_0, p_0) \in T^* U$ was arbitrary, we conclude that
  for any $i \in \{1, ..., n\}$, $f_i = g^i = 0$.
  Hence, $\restrict{\xi}{U} = 0$.
  Taking an atlas $\{(U_\alpha, \hat{\varphi}_\alpha)\}_{\alpha \in A}$ of $Q$
  (for some index set $A$), we conclude that $\xi = 0$.
\end{proof}
\begin{corollary}
  Let
    $Q$ be a smooth manifold,
    $\theta$ be the tautological one-form on $T^* Q$,
    $\eta \in \Omega^1(T^* Q)$ has the following property:
    for any $1$-form $\mu$ on $Q$, $\mu^* \eta = \mu$.
  Then $\eta = \theta$.
\end{corollary}
\begin{proof}
  Write $\eta = \theta + \xi$, i.e., $\xi := \eta - \theta$.
  Then, for any $\mu \in \Omega^1 Q$,
  \begin{equation}
    \mu
    = \mu^* \eta
    = \mu^* \theta + \mu^* \xi
    = \mu + \mu^* \xi
    \Rightarrow \mu^* \xi = 0.
  \end{equation}
  But from the previous proposition it follows that $\xi = 0$,
  and hence $\eta = \theta$.
\end{proof}
\begin{proposition}
  Let
    $Q$ be a smooth manifold of dimension $n$,
    $(U, \hat{\varphi})$ be a chart,
    $(q, p) \in T^* U$,
    $ i \in \{1, ..., n\}$.
  Then
  \begin{equation}
    \restrict{d \pi}{(q, p)}
    \left(\restrict{\frac{\partial}{\partial q^i}}{(q, p)}\right)
    = \restrict{\frac{\partial}{\partial \hat{q}^i}}{q}
  \end{equation}
  and
  \begin{equation}
    \restrict{d \pi}{(q, p)}
    \left(\restrict{\frac{\partial}{\partial p^i}}{(q, p)}\right)
    = 0.
  \end{equation}
\end{proposition}
\begin{proof}
  Let $f \colon Q \to \R$ be smooth.
  Define the functions
  $\hat{g} := f \circ \hat{\varphi}^{-1} \colon \R^n \to \R$ and
  $g := f \circ \pi \circ \varphi^{-1} \colon \R^{2 n} \to \R$.
  Let $(X^1, ..., X^n, Y^1, ..., Y^n) := \varphi(p, q) \in \R^n$.
  This means that $(X^1, ..., X^n) = \hat{\varphi}(q)$.
  Then
  \begin{equation}
    g(X^1, ..., X^n, Y^1, ..., Y^n)
    = f(\pi(q, p))
    = f(q)
    = \hat{g}(X^1, ..., X^n).
  \end{equation}
  Hence,
  \begin{equation}
    \begin{split}
      \frac{\partial g}{\partial x^i}(X^1, ..., X^n, Y^1, ..., Y^n)
      & = \lim_{h \to 0}
        \frac
        {g(X^1, ..., X^i + h, ..., X^n, Y^1, ..., Y^n)
         - g(X^1, ..., X^n, Y^1, ..., Y^n)}
        {h} \\
      & = \lim_{h \to 0}
        \frac{\hat{g}(X^1, ..., X^i + h, ..., X^n) - \hat{g}(X^1, ..., X^n)}{h}
        \\
      & = \frac{\partial \hat{g}}{\partial \hat{x}^i}(X^1, ..., X^n).
    \end{split}
  \end{equation}
  Similarly, since $g$ is constant with respect to the last $n$ coordinates,
  \begin{equation}
    \frac{\partial g}{\partial x^{n + i}}(X^1, ..., X^n, Y^1, ..., Y^n) = 0
  \end{equation}
  Take the standard coordinate systems (given by projections)
  $\{\hat{x}^k\}_{k = 1}^n$ on $\R^n$ and
  $\{x^k\}_{k = 1}^{2 n}$ on $\R^{2 n}$.
  Then, by the definitions of differential and partial derivative on manifold,
  \begin{equation}
    (\restrict{d \pi}{(q, p)}
      \left(\restrict{\frac{\partial}{\partial q^i}}{(q, p)}\right)) f
    = \restrict{\frac{\partial}{\partial q^i}}{(q, p)}(f \circ \pi)
    = \frac{\partial(f \circ \pi \circ \varphi^{-1})}{x^i}(\varphi(q, p))
    = \frac{\partial(f \circ \hat{\varphi}^{-1})}{\hat{x}^i}(\hat{\varphi}(q))
    = \restrict{\frac{\partial}{\partial \hat{q}^i}}{q} f,
  \end{equation}
  from which it follows that the first equality holds.
  Similarly,
  \begin{equation}
    (\restrict{d \pi}{(q, p)}
      \left(\restrict{\frac{\partial}{\partial p^i}}{(q, p)}\right)) f
    = \restrict{\frac{\partial}{\partial p^i}}{(q, p)}(f \circ \pi)
    = \frac{\partial(f \circ \pi \circ \varphi^{-1})}{x^{i + n}}(\varphi(q, p))
    = 0,
  \end{equation}
  from which it follows that the second equality holds.
\end{proof}
\begin{proposition}[Tautological one-form in generalised coordinates]
  Let
    $Q$ be a smooth manifold of dimension $n$,
    $(U, \hat{\varphi})$ be a chart.
  Then
  \begin{equation}
    \restrict{\theta}{U} = \sum_{i = 1}^n p_i\, d q^i.
  \end{equation}
\end{proposition}
\begin{proof}
  Let $(q, p) \in T^* U$.
  Recall that $\restrict{\theta}{(q, p)} = p \circ \restrict{d \pi}{(q, p)}$.
  Hence,
  \begin{equation}
    \restrict{\theta}{(q, p)}
    \left(\restrict{\frac{\partial}{\partial q^i}}{(q, p)}\right)
    = p\left(\restrict{\frac{\partial}{\partial \hat{q}^i}}{q}\right)
    = p_i(q, p),
  \end{equation}
  and
  \begin{equation}
    \restrict{\theta}{(q, p)}
    \left(\restrict{\frac{\partial}{\partial p^i}}{(q, p)}\right)
    = p(0)
    = 0.
  \end{equation}
  Therefore,
  \begin{equation}
    \restrict{\theta}{(q, p)}
    = \sum_{i = 1}^n
      \restrict{\theta}{(q, p)}
      \left(\restrict{\frac{\partial}{\partial q^i}}{(q, p)}\right)\,
      \restrict{d q^i}{(q, p)}
    + \sum_{i = 1}^n
      \restrict{\theta}{(q, p)}
      \left(\restrict{\frac{\partial}{\partial p^i}}{(q, p)}\right)\,
      \restrict{d p^i}{(q, p)}
    = \sum_{i = 1}^n p_i(q, p)\, \restrict{d q^i}{(q, p)},
  \end{equation}
  from which the proposition follows.
\end{proof}
\begin{corollary}[Canonical symplectic in generalised coordinates]
  Let
    $Q$ be a smooth manifold of dimension $n$,
    $(U, \hat{\varphi})$ be a chart.
  Then
  \begin{equation}
    \restrict{\omega}{U} = \sum_{i = 1}^n d q^i \wedge d p_i.
  \end{equation}
\end{corollary}
\begin{proof}
  Let $i \in \{1, ..., n\}$.
  Then
  \begin{equation}
    - d(p_i\, d q^i) = - d p_i \wedge d q^i = d q^i \wedge d p_i.
  \end{equation}
  Summing up for all $i$, we get the desired result.
\end{proof}
\begin{definition}
  Let $(M, \omega)$ be a symplectic manifold, $f \in \mathcal{F} M$.
  We say that $X \in \mathfrak{X} M$ is a \textbf{Hamiltonian vector field} for
  $f$ if
  \begin{equation}
    i_X \omega + d_0 f = 0.
  \end{equation}
\end{definition}
\begin{proposition}
  Let $(M, \omega)$ be a symplectic manifold, $f \in \mathcal{F} M$.
  Then there exists a unique Hamiltonian vector field for $f$.
\end{proposition}
\begin{proof}
  The non-degeneracy of $\omega$ means that we can interpret the symplectic form
  as the isomorphism $\tilde{\omega} \colon \mathfrak{X} M \to \Omega^1 M$,
  given by
  \begin{equation}
    (\tilde{\omega} X) := i_X \omega,\ X \in \mathfrak{X} M.
  \end{equation}
  Hence, the problem at hand has a unique solution
  $X = \tilde{\omega}^{-1}(- d_0 f)$.
\end{proof}
\begin{definition}
  Let $(M, \omega)$ be a symplectic manifold.
  Define the map $\hamiltonian \colon \mathcal{F} M \to \mathfrak{X} M$ by
  \begin{equation}
    \hamiltonian = - \tilde{\omega}^{-1} \circ d_0.
  \end{equation}
  It maps a function to its corresponding Hamiltonian vector field.
  We will write $\hamiltonian_f$ instead of $\hamiltonian(f)$
  for $f \in \mathcal{F} M$.
\end{definition}
\begin{proposition}
  Let
    $Q$ be a smooth manifold of dimension $n$,
    $(U, \hat{\varphi})$ be a chart on $Q$,
    $f \in \mathcal{F}(T^* Q)$.
  Then
  \begin{equation}
    \hamiltonian_f
    = \sum_{i = 1}^n
    \left(
      - \frac{\partial f}{\partial p^i} \frac{\partial}{\partial q^i}
      + \frac{\partial f}{\partial q^i} \frac{\partial}{\partial p^i}
    \right).
  \end{equation}
\end{proposition}
\begin{proof}
  First, note that
  $i_{\frac{\partial}{\partial q^i}} \omega = d p^i$ and
  $i_{\frac{\partial}{\partial p^i}} \omega = - d q^i$.
  Denote
  \begin{equation}
    X
    := \sum_{i = 1}^n
    \left(
      - \frac{\partial f}{\partial p^i} \frac{\partial}{\partial q^i}
      + \frac{\partial f}{\partial q^i} \frac{\partial}{\partial p^i}
    \right).
  \end{equation}
  Then
  \begin{equation}
    i_X \omega
    = \sum_{i}^n
    \left(
      - \frac{\partial f}{\partial p^i} i_{\frac{\partial}{\partial q^i}} \omega
      + \frac{\partial f}{\partial q^i} i_{\frac{\partial}{\partial p^i}} \omega
    \right)
    = \sum_{i}^n
    \left(
      - \frac{\partial f}{\partial p^i} d p^i
      - \frac{\partial f}{\partial q^i} d q^i
    \right)
    = - d f.
  \end{equation}
  Hence, $\hamiltonian_f = X$.
\end{proof}
\begin{proposition}
  Let $(M, \omega)$ be a symplectic manifold, $f, g \in \mathcal{F} M$.
  Then
  \begin{equation}
    \hamiltonian_{f g} = f \hamiltonian_g + g \hamiltonian_f.
  \end{equation}
\end{proposition}
\begin{proof}
  Follows directly from the Leibniz rule for $d_0$.
\end{proof}
\begin{definition}
  Let $(M, \omega)$ be a symplectic manifold, $X \in \mathfrak{X} M$.
  We say that $X$ is a \textbf{symplectic vector field} if $L_X \omega = 0$.
\end{definition}
\begin{remark}
  Since $L_{\lie{X}{Y}} = \lie{L_X}{L_Y} = L_X \circ L_Y - L_Y \circ L_X$,
  the symplectic vector fields form a Lie subalgebra of the Lie algebra of
  vector fields.
\end{remark}
\begin{proposition}
  Let $(M, \omega)$ be a symplectic manifold, $f \in \mathcal{F} M$.
  Then $\hamiltonian_f$ is a symplectic vector field.
\end{proposition}
\begin{proof}
  $
    L_{\hamiltonian_f} \omega
    = i_{\hamiltonian_f}(d \omega) + d(i_{\hamiltonian_f} \omega)
    = i_{\hamiltonian_f} 0 - d(d f)
    = 0.
  $
\end{proof}
\begin{proposition}
  Let
    $(M, \omega)$ be a symplectic manifold,
    $X \in \mathfrak{X} M$ be a symplectic vector fields.
  Then
  \begin{equation}
    d(i_X \omega) = 0.
  \end{equation}
\end{proposition}
\begin{proof}
  $
    d(i_X \omega)
    = L_X \omega - i_X(d \omega)
    = 0 - 0
    = 0.
  $
\end{proof}
\begin{proposition}
  Let $M$ be a smooth manifold, $X, Y \in \mathfrak{X} M$.
  Then
  \begin{equation}
    L_X \circ i_Y = i_{\lie{X}{Y}} + i_Y \circ L_X.
  \end{equation}
\end{proposition}
\begin{proposition}
  Let
    $(M, \omega)$ be a symplectic manifold,
    $X, Y \in \mathfrak{X} M$ be symplectic vector fields.
  Then
  \begin{equation}
    \lie{X}{Y} = \hamiltonian_{i_Y(i_X \omega)}.
  \end{equation}
\end{proposition}
\begin{proof}
  \begin{equation}
    i_{\lie{X}{Y}} \omega
    = (L_X \circ i_Y - i_Y \circ L_X) \omega
    = (L_X \circ i_Y) \omega
    = ((d \circ i_X + i_X \circ d) \circ i_Y) \omega
    = d(i_X(i_Y \omega))
    = - d(i_Y(i_X \omega)).
  \end{equation}
  We get the desired result from the definition of $\hamiltonian$.
\end{proof}
\begin{definition}
  Let $(M, \omega)$ be a symplectic manifold.
  Define the \textbf{Poisson bracket}
  $\poisson{\cdot}{\cdot} \colon \mathcal{F} M \to \mathcal{F} M$ by
  \begin{equation}
    \poisson{f}{g}
    := i_{\hamiltonian_g}(i_{\hamiltonian_f} \omega),\
    f, g \in \mathcal{F} M.
  \end{equation}
\end{definition}
\begin{corollary}
  Let $(M, \omega)$ be a symplectic manifold, $f, g \in \mathcal{F} M$.
  Then
  \begin{equation}
    \lie{\hamiltonian_f}{\hamiltonian_g}
    = \hamiltonian_{i_{\hamiltonian_g}(i_{\hamiltonian_f} \omega)}
    = \poisson{f}{g}.
  \end{equation}
\end{corollary}
\begin{proposition}[Leibniz rule holds for the Poisson bracket]
  Let $(M, \omega)$ be a symplectic manifold, $f, g, h \in \mathcal{F} M$.
  Then
  \begin{equation}
    \poisson{f}{g h} = \poisson{f}{g} h + g \poisson{f}{h}.
  \end{equation}
\end{proposition}
\begin{proof}
  $
    \poisson{f}{g h}
    = i_{\hamiltonian_{g h}}(i_{\hamiltonian_f} \omega)
    = i_{h \hamiltonian_g + g \hamiltonian_{h}}(i_{\hamiltonian_f} \omega)
    = (i_{\hamiltonian_g}(i_{\hamiltonian_f} \omega))\, h
      + g\, (i_{\hamiltonian_h}(i_{\hamiltonian_f} \omega)) 
    = \poisson{f}{g} h + g \poisson{f}{h}.
  $
\end{proof}
\begin{proposition}
  Let $(M, \omega)$ be a symplectic manifold, $f, g, h \in \mathcal{F} M$.
  Then
  \begin{equation}
    \poisson{f}{g} = L_{\hamiltonian_f} g.
  \end{equation}
\end{proposition}
\begin{proof}
  $
    \poisson{f}{g}
    = i_{\hamiltonian_g}(i_{\hamiltonian_f} \omega)
    = - i_{\hamiltonian_f} \circ i_{\hamiltonian_g} \omega
    = i_{\hamiltonian_f}(d g)
    = L_{\hamiltonian_f} g.
  $
\end{proof}
\begin{definition}
  Let $(M, \omega)$ be a symplectic manifold.
  Define
  ${\rm ad} \colon \mathcal{F} M \to (\mathcal{F} M \to \mathcal{F} M)$ by
  \begin{equation}
    {\rm ad}_f g := \poisson{f}{g},\ f, g \in \mathcal{F} M,
  \end{equation}
\end{definition}
\begin{proposition}
  Let $(M, \omega)$ be a symplectic manifold.
  Then
  \begin{equation}
    \lie{{\rm ad}_f}{{\rm ad}_g} = {\rm ad}_{\poisson{f}{g}}.
  \end{equation}
  (Here the bracket $\lie{\cdot}{\cdot}$ is the commutator of operators.)
\end{proposition}
\begin{proof}
  From the previous proposition it follows that
  \begin{equation}
    {\rm ad}_f = L_{X_f},\ f \in \mathcal{F} M.
  \end{equation}
  Hence,
  \begin{equation}
    \lie{{\rm ad}_f}{{\rm ad}_g}
    = \lie{L_{\hamiltonian_f}}{L_{\hamiltonian_g}}
    = L_{\lie{\hamiltonian_f}{\hamiltonian_g}}
    = L_{\hamiltonian_{\poisson{f}{g}}}
    = {\rm ad}_{\poisson{f}{g}}.
  \end{equation}
\end{proof}
\begin{corollary}
  Let $(M, \omega)$ be a symplectic manifold.
  Then $(\mathcal{F} M, \poisson{\cdot}{\cdot})$ is a Lie algebra over $\R$.
\end{corollary}
\begin{proof}
  Bilinearity and antisymmetry are trivial to check.
  The Jacobi identity is equivalent to the adjoint map being a Lie algebra
  homomorphism, which was the previous proposition.
\end{proof}
\begin{definition}
  Let
    $R$ be a commutative ring with unity ring,
    $(A, +, \cdot)$ be an $R$-module
    with additional structures of
    an associative algebra $(A, *)$ and
    a Lie algebra $(A, \poisson{\cdot}{\cdot})$.
  We say that $A$ is a Poisson algebra if the Lie bracket acts as a derivation,
  i.e., for all $f, g, h \in A$,
  \begin{equation}
    \poisson{f}{g * h} = \poisson{f}{g} * h + g * \poisson{f}{h}.
  \end{equation}
\end{definition}
\begin{corollary}
  Let $(M, \omega)$ be a symplectic manifold.
  Then $\mathcal{F} M$ is a Poisson algebra over $\R$.
  Here, addition, scalar multiplication, and multiplication are given by the
  corresponding pointwise operations, while the Lie bracket is given by the
  Poisson bracket.
\end{corollary}


\section{Hamiltonian systems}
\label{section:hamiltonian_systems}
\begin{definition}
  Let $(M, \omega)$ be a symplectic manifold, $H \in \mathcal{F} M$.
  We say that $(M, \omega, H)$ is a \textbf{Hamiltonian system}. 
\end{definition}
\begin{definition}
  Let
    $M$ be a manifold,
    $H \in \mathcal{F} M$,
    $G$ be a Lie group,
    $\rho \colon G \to \Aut M$ be a group action.
  We say that $\rho$ \textbf{leaves $H$ invariant} if for any $g \in G$,
  \begin{equation}
    H \circ \rho_g = H.
  \end{equation}
\end{definition}
\begin{definition}
  Let
    $(M, \omega, H)$ be a Hamiltonian system,
    $G$ be a Lie group with Lie algebra $\mathfrak{g}$,
    $\rho \colon G \to \Aut M$ be a group action leaving $H$ invariant.
  Define $\gamma \colon \mathfrak{g} \to \mathfrak{X} M$ as follows:
  for $A \in \mathfrak{g}$, $x \in M$,
  \begin{equation}
    \gamma^A x
    := - \restrict{\frac{d}{d t}}{t = 0}(t \mapsto \rho_{(\exp(t A)} x).
  \end{equation}
  Let $\hat{J} \colon \mathfrak{g} \to \mathcal{F} M$ satisfies
  \begin{equation}
    \hamiltonian_H \circ \hat{J} = \gamma.
  \end{equation}
  Define the \textbf{momentum map} $J \colon M \to \mathfrak{g}^*$ as follows:
  for any $x \in M$, $A \in \mathfrak{g}$,
  \begin{equation}
    (J x) A = \hat{J}^A x.
  \end{equation}
\end{definition}
\begin{theorem}[Noether]
  Let
    $(M, \omega, H)$ be a Hamiltonian system,
    $G$ be a Lie group with Lie algebra $\mathfrak{g}$,
    $\rho \colon G \to \Aut(M)$ be a group action that leaves $H$ invariant,
    $\gamma \colon \mathfrak{g} \to \mathfrak{X} M$ and
    $\hat{J} \colon \mathfrak{g} \to \mathcal{F} M$ be constructed as above,
    $J \colon M \to \mathfrak{g}^*$ be the corresponding momentum map,
    $\varphi \colon \R \to \Aut M$ be the flow of $\hamiltonian_H$.
  Then $J$ leaves the flow of $\hamiltonian_H$ invariant.
  In other words, for any $t \in \R$,
  \begin{equation}
    J \circ \varphi_t = J.
  \end{equation}
\end{theorem}
\begin{proof}
  Let $A \in \mathfrak{g}$, $x \in M$ 
  \begin{equation}
    \poisson{\hat{J}^A}{H}(x)
    = {\restrict{\hamiltonian_{\hat{J}(A)}} H}(x)
    = (\gamma^A H)(x)
    = \restrict{\frac{d}{d t}}{t = 0} H(\rho_{\exp(t A)} x)
    = \restrict{\frac{d}{d t}}{t = 0} H(x)
    = 0.
  \end{equation}
  Hence, $\hat{J}^A \circ \varphi_t$ is constant.
  Therefore,
  \begin{equation}
    \hat{J}^A \circ \varphi_t = \hat{J}^A \circ \varphi_0 = \hat{J}^A.
  \end{equation}
  As a result,
  \begin{equation}
    (J(\varphi_t(x))) A
    = \hat{J}^A(\varphi_t(x))
    = \hat{J}^A(x)
    = (J x) A.
  \end{equation}
  Since $A$ and $x$ are arbitrary, $J \circ \varphi_t = J$, as wanted.
\end{proof}


\part{Meshes}

\section{Meshes}
\label{section:meshes}
\begin{definition}
  Let
    $R$ be a commutative ring with unity,
    $V$ be an $R$-algebra with multiplication operation $[\cdot, \cdot]$.
  We say that $V$ is a \textbf{Lie algebra} if $[\cdot, \cdot]$ is alternating
  and satisfies the \textbf{Jacobi identity}: for any $x, y, z \in V$,
  \begin{equation}
    [x, [y, z]] + [y, [z, x]] + [z, [x, y]] = 0.
  \end{equation}
\end{definition}


\section{Relative orientation on meshes}
\label{section:relative_orientation_on_meshes}
\input{mesh/diamond_property-theorem.tex}
\begin{definition}
  Let
    $R$ be a commutative ring with unity,
    $V$ be an $R$-algebra with multiplication operation $[\cdot, \cdot]$.
  We say that $V$ is a \textbf{Lie algebra} if $[\cdot, \cdot]$ is alternating
  and satisfies the \textbf{Jacobi identity}: for any $x, y, z \in V$,
  \begin{equation}
    [x, [y, z]] + [y, [z, x]] + [z, [x, y]] = 0.
  \end{equation}
\end{definition}

\begin{remark}
  When the space of scalars is a field, a module becomes a vector space.
  Hence, modules generalise vector spaces.
\end{remark}

\input{mesh/relative_orientation/existence-theorem.tex}

\section{Chains and boundary operator on meshes}
\label{section:chains_and_boundary_operator_on_meshes}
\input{mesh/chain/p-definition.tex}
\begin{definition}
  Let
    $R$ be a commutative ring with unity,
    $V$ be an $R$-algebra with multiplication operation $[\cdot, \cdot]$.
  We say that $V$ is a \textbf{Lie algebra} if $[\cdot, \cdot]$ is alternating
  and satisfies the \textbf{Jacobi identity}: for any $x, y, z \in V$,
  \begin{equation}
    [x, [y, z]] + [y, [z, x]] + [z, [x, y]] = 0.
  \end{equation}
\end{definition}

\begin{definition}
  Let
    $R$ be a commutative ring with unity,
    $V$ be an $R$-algebra with multiplication operation $[\cdot, \cdot]$.
  We say that $V$ is a \textbf{Lie algebra} if $[\cdot, \cdot]$ is alternating
  and satisfies the \textbf{Jacobi identity}: for any $x, y, z \in V$,
  \begin{equation}
    [x, [y, z]] + [y, [z, x]] + [z, [x, y]] = 0.
  \end{equation}
\end{definition}

\input{mesh/boundary_operator/is_chain_complex-proposition.tex}
\input{mesh/boundary_operator/is_chain_complex-proof.tex}
\input{mesh/boundary_operator/is_unique_up_to_isomorphism-proposition.tex}
\input{mesh/boundary_operator/is_unique_up_to_isomorphism-remark.tex}

\section{Cochains and coboundary operator on meshes}
\label{section:cochains_and_coboundary_operator_on_meshes}
\begin{definition}
  Let
    $R$ be a commutative ring with unity,
    $V$ be an $R$-algebra with multiplication operation $[\cdot, \cdot]$.
  We say that $V$ is a \textbf{Lie algebra} if $[\cdot, \cdot]$ is alternating
  and satisfies the \textbf{Jacobi identity}: for any $x, y, z \in V$,
  \begin{equation}
    [x, [y, z]] + [y, [z, x]] + [z, [x, y]] = 0.
  \end{equation}
\end{definition}

\begin{definition}
  Let
    $R$ be a commutative ring with unity,
    $V$ be an $R$-algebra with multiplication operation $[\cdot, \cdot]$.
  We say that $V$ is a \textbf{Lie algebra} if $[\cdot, \cdot]$ is alternating
  and satisfies the \textbf{Jacobi identity}: for any $x, y, z \in V$,
  \begin{equation}
    [x, [y, z]] + [y, [z, x]] + [z, [x, y]] = 0.
  \end{equation}
\end{definition}

\input{mesh/coboundary_operator/is_cochain_complex-proposition.tex}
\input{mesh/coboundary_operator/is_cochain_complex-proof.tex}

\section{Combinatorial differential forms and Forman subdivision}
\label{section:combinatorial_differential_forms_and_forman_subdivision}
\begin{definition}
  Let
    $R$ be a commutative ring with unity,
    $V$ be an $R$-algebra with multiplication operation $[\cdot, \cdot]$.
  We say that $V$ is a \textbf{Lie algebra} if $[\cdot, \cdot]$ is alternating
  and satisfies the \textbf{Jacobi identity}: for any $x, y, z \in V$,
  \begin{equation}
    [x, [y, z]] + [y, [z, x]] + [z, [x, y]] = 0.
  \end{equation}
\end{definition}

\input{mesh/combinatorial_differential_form/discrete_exterior_derivative-definition.tex}
\input{mesh/combinatorial_differential_form/is_cochain_complex-proposition.tex}
\input{mesh/combinatorial_differential_form/is_cochain_complex-proof.tex}
\begin{definition}
  Let
    $R$ be a commutative ring with unity,
    $V$ be an $R$-algebra with multiplication operation $[\cdot, \cdot]$.
  We say that $V$ is a \textbf{Lie algebra} if $[\cdot, \cdot]$ is alternating
  and satisfies the \textbf{Jacobi identity}: for any $x, y, z \in V$,
  \begin{equation}
    [x, [y, z]] + [y, [z, x]] + [z, [x, y]] = 0.
  \end{equation}
\end{definition}

\begin{example}
  Let $R$ be a commutative ring with unity.
  The following are examples of modules over $R$.
  \begin{enumerate}
    \item
      For any $n \in \N$, the space $R^n$ is a module over $R$ under
      component-wise addition and multiplication with a scalar.
    \item
      For any $m, n \in \N$, the space $M_{m \times n}(R)$ of $m \times n$
      matrices with elements in $R$ is a module over $R$ under under
      component-wise addition and multiplication with a scalar.
    \item
      For any set $X$, the ring $R^X$ can also be considered as a module over
      $R$ with pointwise addition and multiplication with a scalar.
      It generalises the previous two cases when $X = \{1, ..., n\}$ and
      $X = \{1, ..., m\} \times \{1, ..., n\}$ respectively.
  \end{enumerate}
\end{example}

\input{mesh/forman_subdivision/concept-figure.tex}
\input{mesh/forman_subdivision/relative_orientation-definition.tex}
\input{mesh/forman_subdivision/cochains_isomorphic_to_original_forms-theorem.tex}
\begin{definition}
  Let
    $R$ be a commutative ring with unity,
    $V$ be an $R$-algebra with multiplication operation $[\cdot, \cdot]$.
  We say that $V$ is a \textbf{Lie algebra} if $[\cdot, \cdot]$ is alternating
  and satisfies the \textbf{Jacobi identity}: for any $x, y, z \in V$,
  \begin{equation}
    [x, [y, z]] + [y, [z, x]] + [z, [x, y]] = 0.
  \end{equation}
\end{definition}

\begin{example}
  Let $R$ be a commutative ring with unity.
  The following are examples of modules over $R$.
  \begin{enumerate}
    \item
      For any $n \in \N$, the space $R^n$ is a module over $R$ under
      component-wise addition and multiplication with a scalar.
    \item
      For any $m, n \in \N$, the space $M_{m \times n}(R)$ of $m \times n$
      matrices with elements in $R$ is a module over $R$ under under
      component-wise addition and multiplication with a scalar.
    \item
      For any set $X$, the ring $R^X$ can also be considered as a module over
      $R$ with pointwise addition and multiplication with a scalar.
      It generalises the previous two cases when $X = \{1, ..., n\}$ and
      $X = \{1, ..., m\} \times \{1, ..., n\}$ respectively.
  \end{enumerate}
\end{example}

\begin{definition}
  Let
    $R$ be a commutative ring with unity,
    $V$ be an $R$-algebra with multiplication operation $[\cdot, \cdot]$.
  We say that $V$ is a \textbf{Lie algebra} if $[\cdot, \cdot]$ is alternating
  and satisfies the \textbf{Jacobi identity}: for any $x, y, z \in V$,
  \begin{equation}
    [x, [y, z]] + [y, [z, x]] + [z, [x, y]] = 0.
  \end{equation}
\end{definition}

\begin{definition}
  Let $\mathcal{K}$ be a quasi-cubical mesh,
    $p \in \{1, ..., \dim \mathcal{K}\}$,
    $a \in \mathcal{K}_p$,
    $b, c \in \mathcal{K}_{p - 1},\ b, c \prec a$.
  We say that $b$ and $c$ are \textbf{topologically parallel} (with respect to
  $a$) if their intersection is empty.
  In that case we will write
  \begin{equation}
    a \setminus b := c,\ a \setminus c := b.
  \end{equation}
\end{definition}

\input{partially_ordered_set/closed_interval-definition.tex}
\input{mesh/interval_simplicial-definition.tex}
\input{mesh/forman_subdivision/is_quasi_cubical_iff_original_is_interval_simplicial-proposition.tex}
\input{mesh/forman_subdivision/is_quasi_cubical_in_up_to_2d-proposition.tex}
\input{polytope/simple-definition.tex}
\input{mesh/forman_subdivision/is_quasi_cubical_in_3d_iff_original_is_simple-proposition.tex}
\begin{proposition}
  Let
    $D \in \N$,
    $K$ be a quasi-cubical mesh of dimension $D$,
    $\partial$ be the topological (unsigned) boundary operator on $K$,
    $\delta$ be the topological (unsigned) coboundary operator on $K$,
    $\perp$ be the perpendicularity operator on $K$,
    $p \in \{0, ..., D\}$.
  Then
  \begin{equation}
    \partial_{D - p} \circ \perp_p = \perp_{p + 1} \circ \delta_p.
  \end{equation}
\end{proposition}


\section{Metric-dependent calculus on quasi-cubical meshes}
\label{section:metric_dependent_calculus_on_quasi_cubical_meshes}
\input{mesh/interval_simplicial_are_not_uncommon-discussion.tex}
\begin{definition}
  Let
    $D \in \N$,
    $K$ be a quasi-cubical mesh of dimension $D$,
    $a_D \in M_D$,
    $p \in \{0, ..., D\}$,
    $b_p \in M_p$,
    $c_{D - p} \in M_{D - p}$.
  We say that $b_p$ and $c_{D - p}$ are \textbf{topologically orthogonal}
  with respect to $a_D$ if $b_p, c_{D - p} \preceq a_D$, and the intersection of
  $b_p$ with $c_{D - p}$ is a node in $a_D$.
  In this case we write
  \begin{equation}
    b_p \oplus c_{D - p} = a_p\
    \text{and}\
    b_p \perp c_{D - p}.
  \end{equation}
\end{definition}

\input{euclidean_measure-notation.tex}
\begin{definition}
  Let
    $R$ be a commutative ring with unity,
    $V$ be an $R$-algebra with multiplication operation $[\cdot, \cdot]$.
  We say that $V$ is a \textbf{Lie algebra} if $[\cdot, \cdot]$ is alternating
  and satisfies the \textbf{Jacobi identity}: for any $x, y, z \in V$,
  \begin{equation}
    [x, [y, z]] + [y, [z, x]] + [z, [x, y]] = 0.
  \end{equation}
\end{definition}

\input{mesh/quasi_cubical/inner_product/regular_cube-example.tex}
\begin{definition}
  Let
    $R$ be a commutative ring with unity,
    $V$ be an $R$-algebra with multiplication operation $[\cdot, \cdot]$.
  We say that $V$ is a \textbf{Lie algebra} if $[\cdot, \cdot]$ is alternating
  and satisfies the \textbf{Jacobi identity}: for any $x, y, z \in V$,
  \begin{equation}
    [x, [y, z]] + [y, [z, x]] + [z, [x, y]] = 0.
  \end{equation}
\end{definition}

\begin{proposition}
  Let
    $D \in \N$,
    $K$ be a compatibly oriented quasi-cubical
    \hyperref[cmc:mesh:definition]{mesh} of dimension $D$,
    $[K] := \sum_{c_D \in K_D} c^D$ be the fundamental class of $K$
    $\inner{\cdot}{\cdot}$ be an orthogonal inner product on $K$,
    $p \in \{0, ..., D\}$.
  The \hyperref[cmc/mesh/quasi_cubical/hodge_star/concept-definition]
               {Hodge star operator}
  $\star_p \colon C^p K \to C^{D - p} K$ has the following closed form:
  for any $\pi^p \in C^p K$ and any $c^{D - p} \in C^{D - p} K$,
  \begin{equation}
    (\star_p \pi^p)(b_{D - p})
    = \sum_{a_p \perp b_{D - p}}
      \frac{(a^p \smile b^{D - p})[K]}{\inner{b^{D - p}}{b^{D - p}}} \pi^p(a_p).
  \end{equation}
\end{proposition}

\input{mesh/quasi_cubical/adjoint_coboundary/regular_cube_closed_form-corollary.tex}
\begin{definition}
  Let
    $R$ be a commutative ring with unity,
    $V$ be an $R$-algebra with multiplication operation $[\cdot, \cdot]$.
  We say that $V$ is a \textbf{Lie algebra} if $[\cdot, \cdot]$ is alternating
  and satisfies the \textbf{Jacobi identity}: for any $x, y, z \in V$,
  \begin{equation}
    [x, [y, z]] + [y, [z, x]] + [z, [x, y]] = 0.
  \end{equation}
\end{definition}

\begin{proposition}
  Let
    $D \in \N$,
    $K$ be a compatibly oriented quasi-cubical
    \hyperref[cmc:mesh:definition]{mesh} of dimension $D$,
    $[K] := \sum_{c_D \in K_D} c^D$ be the fundamental class of $K$
    $\inner{\cdot}{\cdot}$ be an orthogonal inner product on $K$,
    $p \in \{0, ..., D\}$.
  The \hyperref[cmc/mesh/quasi_cubical/hodge_star/concept-definition]
               {Hodge star operator}
  $\star_p \colon C^p K \to C^{D - p} K$ has the following closed form:
  for any $\pi^p \in C^p K$ and any $c^{D - p} \in C^{D - p} K$,
  \begin{equation}
    (\star_p \pi^p)(b_{D - p})
    = \sum_{a_p \perp b_{D - p}}
      \frac{(a^p \smile b^{D - p})[K]}{\inner{b^{D - p}}{b^{D - p}}} \pi^p(a_p).
  \end{equation}
\end{proposition}


\section{Product meshes}
\label{section:product_meshes}
\begin{definition}
  Let $\mathcal{K}$ and $\mathcal{L}$ be meshes.
  The \textbf{product mesh} $\mathcal{K} \times \mathcal{L}$ is defined as the
  poset product of $\mathcal{K}$ and $\mathcal{L}$.
\end{definition}
\begin{proposition}
  Let $\mathcal{K}$ and $\mathcal{L}$ be meshes.
  Then $\mathcal{K} \times \mathcal{L}$ is also a mesh.
\end{proposition}
\begin{proposition}
  Let $\mathcal{K}$ and $\mathcal{L}$ be quasi-cubical meshes.
  Then $\mathcal{K} \times \mathcal{L}$ is also a quasi-cubical mesh.
\end{proposition}
\begin{proposition}
  Let $\mathcal{M}$ and $\mathcal{N}$ be combinatorial meshes.
  Then
  \begin{equation}
    {\rm Forman}(\mathcal{M} \times \mathcal{N})
    = {\rm Forman}(\mathcal{M}) \times {\rm Forman}(\mathcal{N}).
  \end{equation}
\end{proposition}
\begin{definition}
  Let $\varphi \colon \mathcal{K} \to \mathcal{P} K$ and
    $\psi \colon \mathcal{L} \to \mathcal{P} L$ be
  mesh embeddings.
  Define the \textbf{product embedding}
  \begin{equation}
    \varphi \times \psi \colon
    \mathcal{K} \times \mathcal{L} \to \mathcal{P} (K \times L)
  \end{equation}
  as follows: for any $(a, \alpha) \in \mathcal{K} \times \mathcal{L}$,
  \begin{equation}
    (\varphi \times \psi)(a, \alpha) = (\varphi a) \times (\psi \alpha).
  \end{equation}
\end{definition}
\begin{proposition}
  Let
    $(\mathcal{K}, \varepsilon^{\mathcal{K}})$ and
      $(\mathcal{K}, \varepsilon^{\mathcal{K}})$
      be meshes with relative orientations,
    $C_\bullet(\mathcal{K}, \partial^{\mathcal{K}})$ and
      $C_\bullet(\mathcal{L}, \partial^{\mathcal{L}})$
    be the corresponding chain complexes (of boundary operators).
  Then the chain complex
  \begin{equation}
    C_\bullet(\mathcal{K}, \partial^{\mathcal{K}}) \otimes
    C_\bullet(\mathcal{L}, \partial^{\mathcal{L}})
  \end{equation}
  induces a relative orientations on $\mathcal{K} \times \mathcal{L}$.
  Precisely, if $0, \leq p \leq \dim \mathcal{K}$,
  $0, \leq q \leq \dim \mathcal{L}$,
  $a \in \mathcal{K}_p, \alpha \in \mathcal{L}_{q}$,
  $(b, \beta) \in \partial_{p + q}(a, \alpha)$,
  \begin{equation}
    \varepsilon^{\mathcal{K} \times \mathcal{L}}_{p + q}(
      (a, \alpha), (b, \beta)) =
    \begin{cases}
      \varepsilon_p^{\mathcal{K}}(a, b),\ & \alpha = \beta \\
      (-1)^p \varepsilon_q^{\mathcal{L}}(\alpha, \beta),\ & a = b
    \end{cases}.
  \end{equation}
\end{proposition}
\begin{definition}
  Let $\mathcal{K}, \mu^{\mathcal{K}}$ and $\mathcal{L}, \mu^{\mathcal{L}}$ be
  Riemannian meshes.
  Define the \textbf{product measures}
  $\mu^{\mathcal{K}} \times \mu^{\mathcal{L}}$
  on $\mathcal{K} \times \mathcal{L}$ as follows:
  for any $a \in \mathcal{K},\ \alpha \in \mathcal{L}$,
  \begin{equation}
    (\mu^{\mathcal{K}} \times \mu^{\mathcal{L}})(a, \alpha)
    := \mu^{\mathcal{K}}(a)\, \mu^{\mathcal{L}}(\alpha).
  \end{equation}
\end{definition}
\begin{proposition}
  Let
    $\mathcal{K}$ and $\mathcal{L}$ be combinatorial meshes,
    $(K, g^K)$ and $(L, g^L)$ be Riemannian manifolds,
    $\varphi^{\mathcal{K}} \colon \mathcal{K} \to K$ and
      $\varphi^{\mathcal{L}} \colon \mathcal{L} \to L$
      be mesh embeddings,
    $\mu^{\mathcal{K}}$ and $\mu^{\mathcal{L}}$ be the induced measures.
  Then $\mu^{\mathcal{K}} \times \mu^{\mathcal{L}}$ is the measure induced by
  the embedding $\varphi^{\mathcal{K}} \times \varphi^{\mathcal{L}}$ of the mesh
  $\mathcal{K} \times \mathcal{L}$
  in the Riemannian manifold $(K \times L, g^K \times g^L)$.
\end{proposition}
\begin{proposition}
  Let
    $(\mathcal{K}, \mu^{\mathcal{K}})$ and $(\mathcal{L}, \mu^{\mathcal{L}})$
      be Riemannian meshes,
    $\inner{\cdot}{\cdot}_{\mathcal{K}}$ and
      $\inner{\cdot}{\cdot}_{\mathcal{L}}$
      be the respective induced inner products,
    $\inner{\cdot}{\cdot}_{\mathcal{K} \times \mathcal{L}}$
      be the inner product induced by
      $\mu^{\mathcal{L}} \times \mu^{\mathcal{L}}$,
    $(a, \alpha) \in \mathcal{K} \times \mathcal{L}$.
  Then
  \begin{equation}
    \inner{(a, \alpha)^\bullet}{(a, \alpha)^\bullet}
    _{\mathcal{K} \times \mathcal{L}}
    = \inner{a^\bullet}{a^\bullet}_\mathcal{K}\,
      \inner{\alpha^\bullet}{\alpha^\bullet}_\mathcal{L}.
  \end{equation}
\end{proposition}
\begin{proof}
  Let $D = \dim \mathcal{K}$, $d = \dim \mathcal{L}$.
  Then $\dim (\mathcal{K} \times \mathcal{L}) =  D + d$.
  Hence,
  \begin{equation}
    \begin{split}
      \inner{(a, \alpha)^\bullet}{(a, \alpha)^\bullet}
      _{\mathcal{K} \times \mathcal{L}}
      & = \frac{1}{2^{D + d} \mu^{\mathcal{K} \times \mathcal{L}}(a, \alpha)}
          \sum_{(b, \beta) \perp (a, \alpha)}
            \mu^{\mathcal{K} \times \mathcal{L}}(b, \beta) \\
      & = \frac{1}{2^D \mu^{\mathcal{K}}(a)}
          \frac{1}{2^d \mu^{\mathcal{L}}(\alpha)}
          \sum_{b \perp a,\ \beta \perp \alpha}
            \mu^{\mathcal{K}}(b)\, \mu^{\mathcal{L}}(\beta) \\
      & = \left(
            \frac{1}{2^D \mu^{\mathcal{K}}(a)}
            \sum_{b \perp a} \mu^{\mathcal{K}}(b)
          \right)\,
          \left(
            \frac{1}{2^d \mu^{\mathcal{L}}(\alpha)}
            \sum_{\beta \perp \alpha} \mu^{\mathcal{L}}(\beta)
          \right) \\
      & = \inner{a^\bullet}{a^\bullet}_{\mathcal{K}}\,
          \inner{\alpha^\bullet}{\alpha^\bullet}_{\mathcal{L}}.
    \end{split}
  \end{equation}
\end{proof}
\begin{proposition}
  Let
    $\mathcal{K}$ and $\mathcal{L}$ be combinatorial meshes,
    $(K, g^K)$ and $(L, g^L)$ be smooth Riemannian manifolds that realise them,
    $\inner{\cdot}{\cdot}_K$ and $\inner{\cdot}{\cdot}_L$ be the induced
      inner products of differential forms,
    $W^\mathcal{K} \colon C^\bullet \mathcal{K} \to H \Omega^\bullet K$ and
      $W^\mathcal{L} \colon C^\bullet \mathcal{L} \to H \Omega^\bullet L$
      be Whitney maps,
    $\inner{\cdot}{\cdot}_{\mathcal{K}}$, $\inner{\cdot}{\cdot}_{\mathcal{L}}$,
      and $\inner{\cdot}{\cdot}_{\mathcal{K} \times \mathcal{L}}$
      be the induced Whitney inner products
      (the last one is induced by $W^\mathcal{K} \otimes W^\mathcal{L}$),
    $\sigma, \tau \in C^\bullet \mathcal{K}$,
    $\omega, \eta \in C^\bullet \mathcal{L}$.
  Then
  \begin{equation}
    \inner{\sigma \otimes \omega}{\tau \otimes \eta}
    _{\mathcal{K} \times \mathcal{L}}
    = \inner{\sigma}{\tau}_{\mathcal{K}}\, \inner{\omega}{\eta}_{\mathcal{L}}.
  \end{equation}
\end{proposition}
\begin{proof}
  A direct computation:
  \begin{equation}
    \begin{split}
      \inner{\sigma \otimes \omega}{\tau \otimes \eta}
      _{\mathcal{K} \times \mathcal{L}}
      & = \inner{W^{\mathcal{K} \times \mathcal{L}}(\sigma \otimes \omega)}
          {W^{\mathcal{K} \times \mathcal{L}}(\tau \otimes \eta)}
          _{K \times L} \\
      & = \inner{W^{\mathcal{K}} \sigma \boxtimes W^{\mathcal{L}} \omega}
          {W^{\mathcal{K}} \tau \boxtimes W^{\mathcal{L}} \eta}
          _{K \times L} \\
      & = \int_{K \times L}
          g^{K \times L}(
            W^{\mathcal{K}} \sigma \boxtimes W^{\mathcal{L}} \omega,
            W^{\mathcal{K}} \tau \boxtimes W^{\mathcal{L}} \eta)\,
          \vol_{K \times L} \\
      & = \left(
            \int_K g^K(W^{\mathcal{K}} \sigma, W^{\mathcal{K}} \tau) \vol_K
          \right)\,
          \left(
            \int_L g^L(W^{\mathcal{L}} \omega, W^{\mathcal{L}} \eta) \vol_L
          \right) \\
      & = \inner{W^{\mathcal{K}} \sigma}{W^{\mathcal{K}} \tau}_K\,
          \inner{W^{\mathcal{L}} \omega}{W^{\mathcal{L}} \eta}_L \\
      & = \inner{\sigma}{\tau}_{\mathcal{K}}\,
          \inner{\omega}{\eta}_{\mathcal{L}}.
    \end{split}
  \end{equation}
\end{proof}


\section{Approximating vector fields with 1-cochains}
\label{section:approximating_vector_fields_with_1_cochains}
\begin{definition}
  Let
    $R$ be a commutative ring with unity,
    $V$ be an $R$-algebra with multiplication operation $[\cdot, \cdot]$.
  We say that $V$ is a \textbf{Lie algebra} if $[\cdot, \cdot]$ is alternating
  and satisfies the \textbf{Jacobi identity}: for any $x, y, z \in V$,
  \begin{equation}
    [x, [y, z]] + [y, [z, x]] + [z, [x, y]] = 0.
  \end{equation}
\end{definition}

\input{moore_penrose_inverse/physical_dimension-remark.tex}
\input{moore_penrose_inverse/existence_and_uniqueness-theorem.tex}
\input{moore_penrose_inverse/full_rank_closed_form-remark.tex}
\begin{definition}
  Let
    $R$ be a commutative ring with unity,
    $V$ be an $R$-algebra with multiplication operation $[\cdot, \cdot]$.
  We say that $V$ is a \textbf{Lie algebra} if $[\cdot, \cdot]$ is alternating
  and satisfies the \textbf{Jacobi identity}: for any $x, y, z \in V$,
  \begin{equation}
    [x, [y, z]] + [y, [z, x]] + [z, [x, y]] = 0.
  \end{equation}
\end{definition}

\input{mesh/flat/node_matrix-definition.tex}
\input{mesh/relative_orientation/neighbor_representation_of_1_cochain-definition.tex}
\input{mesh/flat/embedding_of_1_cochain-definition.tex}
\input{mesh/flat/embedding_of_1_cochain-example.tex}
\input{mesh/flat/approximation_of_vector_field-definition.tex}
\input{mesh/flat/approximation_of_vector_field-example.tex}
\input{mesh/flat/summary_of_operators-discussion.tex}

\section{Vector fields on combinatorial meshes}
\label{section:vector_fields_on_combinatorial_meshes}
\begin{definition}
  Let
    $R$ be a commutative ring with unity,
    $V$ be an $R$-algebra with multiplication operation $[\cdot, \cdot]$.
  We say that $V$ is a \textbf{Lie algebra} if $[\cdot, \cdot]$ is alternating
  and satisfies the \textbf{Jacobi identity}: for any $x, y, z \in V$,
  \begin{equation}
    [x, [y, z]] + [y, [z, x]] + [z, [x, y]] = 0.
  \end{equation}
\end{definition}

\begin{definition}
  Let $\mathcal{K}$ be a quasi-cubical mesh.
  The \textbf{discrete interior product on $1$-cochains} is defined as
  \begin{equation}
    i \colon \mathfrak{X} K \to \Hom(C^1 \mathcal{K}, C^0 \mathcal{K}),
    i_{\mathcal{X}} \sigma := \sigma \circ \mathcal{X} \in C^0 \mathcal{K},\
    \mathcal{X} \in \mathfrak{X} K,\ \sigma \in C^1 \mathcal{K}.
  \end{equation}
  In other words, for a node $\mathcal{N}$,
  \begin{equation}
    (i_{\mathcal{X}} \sigma)(\mathcal{N})
    = (\sigma \circ \mathcal{X})(\mathcal{N})
    = \sum_{\mathcal{E} \succ \mathcal{N}}
      \mathcal{X}^{\mathcal{E}}_{\mathcal{N}} \sigma_{\mathcal{E}}.
  \end{equation}
\end{definition}
\begin{definition}
  Let
    $D \in \N^+$,
    $K$ be an orthogonal parallelotope in $\R^D$,
    whose unit directions are the vectors $e_1, ..., e_D$.
  Define a regular mesh for $K$ as follows.
  Let
    $h_1, ..., h_D \in \R^+$,
    $\mathcal{K}$ be a grid of (orthogonal) parallelotopes with sides
      $h_1, ..., h_D$.
  Let $X \in \mathfrak{X} K$,
  $X = \sum_{p = 1}^D f^p \frac{\partial}{\partial x^p}$.
  Define the approximation
  \begin{equation}
    J \colon \mathfrak{X} K \to \mathfrak{X} \mathcal{K}
  \end{equation}
  as follows.
  Let $p \in \{1, ..., D\}$,
  $\mathcal{E}$ be an edge in the direction of the basis vector $e_p$,
  $\mathcal{N}$ be a node of $\mathcal{E}$ with coordinates
  $x = (x_1, ..., x_D)$.
  Then,
  \begin{equation}
    (J X)^{\mathcal{E}}_{\mathcal{N}} :=
    \begin{cases}
      \frac{f^p(x)}{2 h_p}, & \text{$\mathcal{N}$ is an interior node} \\
      \frac{f^p(x)}{h_p}, & \text{$\mathcal{N}$ is a boundary node} \\
    \end{cases}.
  \end{equation}
\end{definition}
\begin{remark}
  Consider the setup of the previous definition.
  We will show that the discrete interior product is a good approximation of the
  continuous one.
  Precisely, let $X \in \mathfrak{X} K$, $\omega \in \Omega^1 K$.
  We will compare $i_{J X}{R_1 \omega}$ with $R i_X \omega$.
  By definition, $i_X \omega = \omega(X)$.
  In coordinates,
  \begin{align}
    X & = \sum_{p = 1}^D {f^p} \frac{\partial}{\partial x^p},\
      f^1, ..., f^D \in \mathcal{F} K, \\
    \omega & = \sum_{p = 1}^D {g_p}\, d x^p,\
      g_1, ..., g_D \in \mathcal{F} K, \\
    i_X \omega & = \sum_{p = 1}^D f^p\; g_p.
  \end{align}
  Consider a node $\mathcal{N}$ with coordinates $x = (x_1, ..., x_D)$.
  Then
  \begin{equation}
    (R i_X \omega) \mathcal{N} = \sum_{p = 1}^D f^p(x_p)\; g_p(x_p).
  \end{equation}
  On the other hand,
  \begin{equation}
    (i_{J X}{R_1 \omega}) \mathcal{N}
    = \sum_{\mathcal{E} \succ \mathcal{N}}
      (J X)^{\mathcal{E}}_{\mathcal{N}} \int_{\mathcal{E}} \omega
    = \sum_{p = 1}^D
      \sum_{\mathcal{E} \succ \mathcal{N},\ \mathcal{E} \parallel e_p}
        (J X)^{\mathcal{E}}_{\mathcal{N}} \int_{\mathcal{E}} \omega.
  \end{equation}
  In the above equation, for any $p \in \{1, ..., D\}$
  the internal sum consists of $1$ or $2$ terms:
  $1$ when $\mathcal{N}$ is on the boundary of the direction of $e_p$,
  and $2$ elsewhere.
  Define the this internal sum as $A_p$.
  We need to show it is close to $f^p(x_p)\; g_p(x_p)$.
  
  First, assume that $\mathcal{N}$ is in the interior.
  Then it is the boundary of two edges parallel to $e_p$:
  $\mathcal{E}_1$ and $\mathcal{E}_2$.
  Combined, they give the segment $\mathcal{E}$ connecting
  $x - h_p e_p$ with $x + h_p e_p$.
  Then
  \begin{equation}
    A_p = \frac{f^p(x)}{2 h_p} \int_{\mathcal{E}} \omega
    = f^p(x_p) \frac{1}{2 h_p}
      \int_{x_p - h_p}^{x_p + h_p}
      g_p(x_1, ..., x_{p - 1}, t, x_{p + 1}, ..., x_D)\, d t.
  \end{equation}
  However, from $1$D numerical analysis it follows that the approximation
  (the lowest order Gauss quadrature)
  \begin{equation}
    \frac{1}{b - a} \int_a^b g(t)\, d t \approx g\left(\frac{a + b}{2}\right)
  \end{equation}
  is exact for polynomials of degree $\leq 1$.
  In our case, substituting $a = x_p - h_p,\ b = x_p + h_p$
  implies that $f^p(x_p)\; g_p(x_p)$ is an $O(h_p^2)$ approximation of $A_p$.

  Second, assume that $\mathcal{E}$ is a boundary node.
  Fix it to be on the left.
  Then, using an analogous argument, we need to compare
  $g(a)$ with $\frac{1}{b - a} \int_a^b g(t)\, dt$
  which is the formula of left rectangles, this time exact only for constants.
  This gives total error of $O(h_p)$ in the direction of $e_p$ with $O(h^2)$ in
  the interior.
  Hence, the approximation formula is of order $O(h_1 + ... + h_D)$ with order
  $O(h_1^2 + ... + h_D^2)$ in the interior.
\end{remark}
\begin{remark}
  \begin{equation}
    i_{f \mathcal{X}} \sigma = f \smile i_\mathcal{X} \sigma.
  \end{equation}
  Indeed, for any $\mathcal{N} \in \mathcal{K}_0$,
  \begin{equation}
    (i_{f \mathcal{X}} \sigma) \mathcal{N}_\bullet
    = (\sigma \circ {f \mathcal{X}}) \mathcal{N}_\bullet
    = \sigma (f \mathcal{N}_\bullet\, \mathcal{X} \mathcal{N}_\bullet)
    = f \mathcal{N}_\bullet\, \sigma(\mathcal{X} \mathcal{N}_\bullet)
    = (f \smile i_\mathcal{X} \sigma) \mathcal{N}_\bullet.
  \end{equation}
\end{remark}

\begin{definition}
  Let $\mathcal{K}$ be a quasi-cubical mesh.
  The \textbf{Lie derivative on $0$-cochains} is defined as
  \begin{equation}
    L \colon \mathfrak{X} K \to \Hom(C^0 \mathcal{K}, C^0 \mathcal{K}),
    L_{\mathcal{X}} := i_{\mathcal{X}} \circ \delta,\
    \mathcal{X} \in \mathfrak{X} K.
  \end{equation}
  The above expression applied to a function $f \in C^0 \mathcal{K}$ equals to
  \begin{equation}
    L_{\mathcal{X}} f
    = i_{\mathcal{X}} (\delta f)
    = (\delta_0 f) \circ \mathcal{X}
    = f \circ \partial_1 \circ X.
  \end{equation}
\end{definition}
\begin{remark}
  No we are going to analyse to what extent the Leibniz rule is reproduced in
  the discrete setting.
  Namely, for
  $\mathcal{X} \in \mathfrak{X} K$, $f, g \in \mathcal{F} K$
  we are estimating
  \begin{equation}
    \Delta_\mathcal{X}(f, g) :=
    (L_{\mathcal{X}} f) \smile g
    + f \smile (L_{\mathcal{X}} g)
    - L_{\mathcal{X}} (f \smile g).
  \end{equation}
  Let $\mathcal{N} \in \mathcal{K}_0$.
  Then
  \begin{equation}
    \begin{split}
        \Delta_\mathcal{X}(f, g) \mathcal{N}_\bullet
      & =
        ((L_{\mathcal{X}} f) \smile g + f \smile (L_{\mathcal{X}} g)
        - L_{\mathcal{X}} (f \smile g)) \mathcal{N}_\bullet \\
      & = f (\partial(X \mathcal{N}_\bullet))\, g \mathcal{N}_\bullet
        + f \mathcal{N}_\bullet\, g (\partial(X \mathcal{N}_\bullet))
        - (f \smile g)(\partial(X \mathcal{N}_\bullet)) \\
      & = \sum_{\mathcal{E} \succ \mathcal{N}}
        \mathcal{X}_\mathcal{N}^\mathcal{E}
        ( (\sum_{\mathcal{N}' \prec \mathcal{E}}
          \varepsilon(\mathcal{E}, \mathcal{N}') f_{\mathcal{N}'})
          g_{\mathcal{N}}
          + f_{\mathcal{N}} (\sum_{\mathcal{N}' \prec \mathcal{E}}
          \varepsilon(\mathcal{E}, \mathcal{N}') g_{\mathcal{N}'})
          - \sum_{\mathcal{N}' \prec \mathcal{E}}
          \varepsilon(\mathcal{E}, \mathcal{N}')
          f_{\mathcal{N}'} g_{\mathcal{N}'}
        ) \\
      & = \sum_{\mathcal{E} \succ \mathcal{N}}
        \mathcal{X}_\mathcal{N}^\mathcal{E}\,
        \varepsilon(\mathcal{E}, \mathcal{N})\,
        \, (\delta_0 f)(\mathcal{E})\,
        (\delta_0 g)(\mathcal{E}).
    \end{split}
  \end{equation}
\end{remark}

\begin{definition}
  Let $\mathcal{K}$ be a quasi-cubical mesh,
    $\varepsilon$ be relative orientations on $\mathcal{K}$,
    $X, Y \in \mathcal{X} K$.
  Define the commutator $[X, Y] \in \mathcal{X} K$ as follows:
  for any edge $\mathcal{E}$ and node $\mathcal{N} \prec \mathcal{E}$,
  \begin{equation}
    [X, Y]^{\mathcal{E}}_{\mathcal{N}}
    := 2 \varepsilon(\mathcal{E}, \mathcal{N})
      (   X^{\mathcal{E}}_{\mathcal{E} \setminus \mathcal{N}}
          Y^{\mathcal{E}}_{\mathcal{N}}
        - X^{\mathcal{E}}_{\mathcal{N}}
          Y^{\mathcal{E}}_{\mathcal{E} \setminus \mathcal{N}}
      )
      + \sum_{\mathcal{F} \succ_{2, 0} \mathcal{N}}
        \sum_{\mathcal{E}' \perp_{\mathcal{F}} \mathcal{E},\
              \mathcal{N} \in \mathcal{E}'}
        \varepsilon(\mathcal{E}', \mathcal{N})
        ( X^{\mathcal{E}'}_{\mathcal{N}}
          (Y^{\mathcal{E}}_{\mathcal{N}}
           - Y^{\mathcal{F} \setminus \mathcal{E}}
            _{\mathcal{E}' \setminus \mathcal{N}})
          +
          Y^{\mathcal{E}'}_{\mathcal{N}}
          (X^{\mathcal{E}}_{\mathcal{N}}
           - X^{\mathcal{F} \setminus \mathcal{E}}
            _{\mathcal{E}' \setminus \mathcal{N}})
        ).
  \end{equation}
\end{definition}

\begin{discussion}
  Let us calculate the discrete Lie derivative of the commutator on a regular
  quasic-cubical mesh and compare it to the continuum.
  Let $D \in \N^+$, $\mathcal{K}$ be a regular quasi-cubical of dimension $D$
  with unit orthogonal directions $e_1, ..., e_D \in \R^D$ and lengths
  $h_1, ..., h_D \in \R^+$.
  We assume that for any $p \in \{0, ..., D\}$, any $p$-element ordered set
  $I \subset (1, ..., D)$, and any $p$-cell with directions
  $e_{I_1}, ..., e_{I_p}$, its embedded orientation is induced by
  the multivector $e_{I_1} \wedge ... \wedge e_{I_p}$.
  For any $\alpha = (\alpha_1, ..., \alpha_D) \in \N^D$ by $\mathcal{N}_\alpha$
  we will denote the node with coordinates $(\alpha_1 h_1, ..., \alpha_D h_D)$.

  Let $X, Y \in \mathcal{X} K$.
  Let $p \in \{1, ..., D\}$, $\alpha = (\alpha_1, ..., \alpha_D) \in \N^D$.
  $s \in \{0, 1\}$ and take the edge
  $\mathcal{E}_{p, \alpha}$ with endpoints $\mathcal{N}_\alpha$ and
  $\mathcal{N}_{\alpha + e_p}$.
  Then for $s \in {0, 1}$, assuming $\mathcal{N}_{\alpha + s e_p}$
  is an interior node,
  \begin{equation}
    \begin{split}
      [X, Y]^{\mathcal{E}_{p, \alpha}}_{\mathcal{N}_{\alpha + s e_p}} =\
      & 2 (-1)^{1 - s}
      ( X^{\mathcal{E}_{p, \alpha}}_{\mathcal{N}_{\alpha + (1 - s) e_p}}
        Y^{\mathcal{E}_{p, \alpha}}_{\mathcal{N}_{\alpha + s e_p}}
        -
        Y^{\mathcal{E}_{p, \alpha}}_{\mathcal{N}_{\alpha + (1 - s) e_p}}
        X^{\mathcal{E}_{p, \alpha}}_{\mathcal{N}_{\alpha + s e_p}}
      ) \\
      & +
      \sum_{q \neq p}
        \sum_{t \in \{-1, 0\}}
          (-1)^{t + 1} \\
          & \qquad
          ( X^{\mathcal{E}_{q, \alpha + s e_p + t e_q}}
             _{\mathcal{N}_{\alpha + s e_p}}
            (Y^{\mathcal{E}_{p, \alpha}}_{\mathcal{N}_{\alpha + s e_p}}
             - Y^{\mathcal{E}_{p, \alpha + (2 t + 1) e_q}}
                _{\mathcal{N}_{\alpha + (2 t + 1) e_q + s e_p }})
            -
            Y^{\mathcal{E}_{q, \alpha + s e_p + t e_q}}
             _{\mathcal{N}_{\alpha + s e_p}}
            (X^{\mathcal{E}_{p, \alpha}}_{\mathcal{N}_{\alpha + s e_p}}
             - X^{\mathcal{E}_{p, \alpha + (2 t + 1) e_q}}
                _{\mathcal{N}_{\alpha + (2 t + 1) e_q + s e_p }})
          ).
    \end{split}
  \end{equation}
  Now, assume these discrete vector fields come from continuous ones:
  \begin{equation}
    X = J {\bf X}
      = J\left(\sum_{p = 1}^D f^p \frac{\partial}{\partial x^p}\right),\
    Y = J {\bf Y}
      = J\left(\sum_{p = 1}^D g^p \frac{\partial}{\partial x^p}\right).
  \end{equation}
  Then
  \begin{equation}
    \begin{split}
      [X, Y]^{\mathcal{E}_{p, \alpha}}_{\mathcal{N}_{\alpha + s e_p}} = \
      &
      (-1)^{1 - s}
        \frac{f^p(\alpha + (1 - s) h_p e_p) g^p(\alpha + s h_p e_p)
          - f^p(\alpha + s h_p e_p) g^p(\alpha + (1 - s) h_p e_p)}{2 h_p^2} \\
      &
      +
      \sum_{q \neq p}
        \sum_{t \in \{-1, 0\}}
          (-1)^{t + 1} \\
      & \qquad
          \left(
            \frac{f^q(\alpha + s h_p e_p)}{2 h_q}
            \left(
              \frac{g^p(\alpha + s h_p e_p)
              - g^p(\alpha + s h_p e_p + (2 t + 1) h_q e_q)}{2 h_p}
            \right)
          \right. \\
      & \qquad \
            -
          \left.
            \frac{g^q(\alpha + s h_p e_p)}{2 h_q}
            \left(
              \frac{f^p(\alpha + s h_p e_p)
              - f^p(\alpha + s h_p e_p + (2 t + 1) h_q e_q)}{2 h_p}
            \right)
          \right).
    \end{split}
  \end{equation}
  For simplicity, let $s = 0$.
  Then,
  \begin{equation}
    \begin{split}
        [X, Y]^{\mathcal{E}_{p, \alpha}}_{\mathcal{N}_{\alpha}} = \
      & -
        \frac
        {f^p(\alpha + h_p e_p) g^p(\alpha) - f^p(\alpha) g^p(\alpha + h_p e_p)}
        {2 h_p^2} \\
      & +
        \sum_{q \neq p}
        \sum_{t \in \{-1, 0\}}
          (-1)^{t + 1} \\
      & \qquad
          \left(
            \frac{f^q(\alpha)}{2 h_q}
            \left(
              \frac{g^p(\alpha)
              - g^p(\alpha + (2 t + 1) h_q e_q)}{2 h_p}
            \right)
            -
            \frac{g^q(\alpha)}{2 h_q}
            \left(
              \frac{f^p(\alpha)
              - f^p(\alpha + (2 t + 1) h_q e_q)}{2 h_p}
            \right)
          \right).
    \end{split}
  \end{equation}
  Hence,
  \begin{equation}
    \begin{split}
        [X, Y]^{\mathcal{E}_{p, \alpha}}_{\mathcal{N}_{\alpha}} = \
      & \frac
        {f^p(\alpha) g^p(\alpha + h_p e_p) - f^p(\alpha + h_p e_p) g^p(\alpha)}
        {2 h_p^2} \\
      & +
        \sum_{q \neq p}
          \left(
            \frac{f^q(\alpha)}{2 h_q}
            \left(\frac{g^p(\alpha + h_q e_q) - g^p(\alpha)}{2 h_p}\right)
            -
            \frac{g^q(\alpha)}{2 h_q}
            \left(\frac{f^p(\alpha + h_q e_q) - f^p(\alpha)}{2 h_p}\right)
          \right) \\
      & +
        \sum_{q \neq p}
          \left(
            \frac{f^q(\alpha)}{2 h_q}
            \left(\frac{g^p(\alpha) - g^p(\alpha - h_q e_q)}{2 h_p}\right)
            -
            \frac{g^q(\alpha)}{2 h_q}
            \left(\frac{f^p(\alpha) - f^p(\alpha - h_q e_q)}{2 h_p}\right)
          \right)
    \end{split}.
  \end{equation}
  Up to a first order approximation, the last expression equals to
  \begin{equation}
    \begin{split}
        [X, Y]^{\mathcal{E}_{p, \alpha}}_{\mathcal{N}_{\alpha}} \approx \
      & \frac
        {\left(f^p \frac{\partial g^p}{\partial x^p}
         - g^p \frac{\partial f^p}{\partial x^p}\right)(\alpha)}
        {2 h_p} \\
      & +
        \sum_{q \neq p}
          \frac
          {
            \left(
              f^q \frac{\partial g^p}{\partial x^q}
              - g^q \frac{\partial f^p}{\partial x^q}
            \right)
            (\alpha)
          }
          {4 h_p} \\
      & +
        \sum_{q \neq p}
          \frac
          {
            \left(
              f^q \frac{\partial g^p}{\partial x^q}
              - g^q \frac{\partial f^p}{\partial x^q}
            \right)
            (\alpha)
          }
          {4 h_p},
    \end{split}
  \end{equation}
  which simplifies to
  \begin{equation}
    [X, Y]^{\mathcal{E}_{p, \alpha}}_{\mathcal{N}_{\alpha}}
    \approx
    \sum_{q = 1}^D
      \frac
      {
        \left(
          f^q \frac{\partial g^p}{\partial x^q}
          - g^q \frac{\partial f^p}{\partial x^q}
        \right)
        (\alpha)
      }
      {2 h_p}.
  \end{equation}
  If we denote ${\bf Z} := [{\bf X}, {\bf Y}]$, its expression in coordinates is
  \begin{align}
    {\bf Z} & = \sum_{p = 1}^D z^p \frac{\partial}{\partial x^p}, \\
    z^p & = \sum_{q = 1}^D ( f^q \frac{\partial g^p}{\partial x^q}
                             - g^q \frac{\partial f^p}{\partial x^q}).
  \end{align}
  Hence, the discrete Lie bracket further simplifies to
  \begin{equation}
    [X, Y]^{\mathcal{E}_{p, \alpha}}_{\mathcal{N}_{\alpha}}
    \approx
    \frac{z^p(\alpha)}{2 h_p}.
  \end{equation}
  However, by definition, the right hand side expresses the coefficients of the
  discrete vector field that approximates ${\bf Z}$.
  Hence,
  \begin{equation}
    J([{\bf X}, {\bf Y}]) \approx [J {\bf X}, J {\bf Y}],
  \end{equation}
  where the left bracket is the continuum Lie bracket,
  while the right bracket is the discrete Lie bracket.

  This calculation is the reason for the definition we gave earlier.
\end{discussion}


\section{Discrete vector bundles}
\label{section:discrete_vector_bundles}
\begin{definition}
  Let
    $R$ be a ring,
    $V$ be a finite-dimensional $R$-module,
    $\omega \in \Lambda^2 V^*$.
  We say that $\omega$ is \textbf{non-degenerate} or \textbf{symplectic}
  if the associated map
  \begin{equation}
    \tilde{\omega} \colon V \to V^*,\
    X \in V \mapsto \tilde{\omega}(X) := i_X \omega \in V^*,
  \end{equation}
  is an isomorphism.

  The pair $(V, \omega)$ is called a \textbf{symplectic module}
  (or \textbf{symplectic vector space} if $R$ is a field).
\end{definition}
\begin{proposition}
  Let
    $R$ be a ring without,
    $(V, \omega)$ be a finite-dimensional symplectic module over $R$.
  Assume that for any $x \in R,\ x + x = 0 \Rightarrow x = 0$.
  Then $\dim V$ is an even number.
\end{proposition}
\begin{proof}
  Let $n = \dim V$.
  In a basis of $V$ $\omega$ is represented by an antisymmetric matrix $A$.
  But then
  \begin{equation}
    \det A = \det(A^T) = \det(-A) = (-1)^n \det A.
  \end{equation}
  If $n$ is odd, then $\det A + \det A = 0$.
  By assumption this means that $\det A = 0$
  which contradicts the nondegeneracy of $\omega$.
  Hence, $n$ is even.
\end{proof}
\begin{definition}
  Let $M$ be a smooth manifold, $\omega \in \Omega^\bullet M$.
  We say that:
  \begin{enumerate}
    \item
      $\omega$ is \textbf{closed} if $d \omega = 0$
    \item
      $\omega$ is \textbf{exact} if there exists $\eta \in \Omega^\bullet M$
      such that $d \eta = \omega$.
  \end{enumerate}
\end{definition}
\begin{proposition}
  Let $M$ be a smooth manifold, $\omega \in \Omega^\bullet M$.
  If $\omega$ is exact, then it is closed.
\end{proposition}
\begin{proof}
  Let $\eta \in \Omega^\bullet M$ be such that $d \eta = \omega$.
  Then $d \omega = d (d \eta) = 0$, i.e., $\omega$ is closed.
\end{proof}
\begin{definition}
  Let $M$ be a smooth manifold, $\omega \in \Omega^2 M$.
  We say that $\omega$ is a \textbf{symplectic form}
  if it is non-degenerate
  (with base module $\mathfrak{X} M$ over $\mathcal{F} M$) and closed.

  The pair $(M, \omega)$ is called a \textbf{symplectic manifold}.
\end{definition}
\begin{proposition}
  Let $(M, \omega)$ be a symplectic manifold.
  Then $M$ is even-dimensional.
\end{proposition}
\begin{definition}
  Let $Q$ be a smooth manifold.
  Consider the cotangent bundle $T^* Q$ with bundle projection
  $\pi \colon T^* Q \to Q$
  with differential $d \pi \colon T(T^* Q) \to T Q$.
  Define the \textbf{tautological one-form}
  $\theta \colon T^* Q \to T^* (T^* Q)$ as follows:
  for any $(q, p) \in T^*Q$ (i.e,. $q \in Q$, $p \in \Hom(T_q Q, \R)$),
  \begin{equation}
    \restrict{\theta}{(q, p)}
    := p \circ \restrict{d \pi}{(q, p)} \in T^*_{(q, p)}(T^* Q).
  \end{equation}
  In other words, if we denote $M := T^* Q$, then $\theta$ is a section of its
  cotangent bundle $T^* M$, i.e., an $1$-form on $M$. 
\end{definition}
\begin{discussion}
  Let $Q$ be a smooth manifold, $\pi \colon T^* Q \to Q$ be the projection.
  Then a $1$-form on $Q$ is a section of $\colon T^* Q$, i.e., a smooth map
  $\mu \colon Q \to \colon T^* Q$ such that $\pi \circ \mu = \id_Q$.
  As such it has a pullback
  $\mu^* \colon \Omega^\bullet(T^* Q) \to \Omega^\bullet Q$.
\end{discussion}
\begin{proposition}
  Let
    $Q$ be a smooth manifold,
    $\theta$ be the tautological one-form on $T^* Q$,
    $\mu \in \Omega^1 Q$.
  Then
  \begin{equation}
    \mu^* \theta = \mu.
  \end{equation}
\end{proposition}
\begin{proof}
  Let $q \in Q$.
  Then
  \begin{equation}
    \restrict{\mu^* \theta}{q}
    = \restrict{\theta}{\mu q} \circ \restrict{d \mu}{q}
    = \restrict{\mu}{q} \circ \restrict{(d \pi)}{\mu q}
      \circ \restrict{d \mu}{q}
    = \restrict{\mu}{q} \circ \restrict{d(\pi \circ \mu)}{q}
    = \restrict{\mu}{q}.
  \end{equation}
  Since $q$ is arbitrary, $\mu^* \theta = \mu$.
\end{proof}
\begin{definition}
  Let
    $Q$ be a smooth manifold,
    $\theta \in \Omega^1(T^* Q)$ be the tautological one-form.
  Define $\omega := - d \theta$.
  The pair $(T^* Q, \omega)$ is called the \textbf{phase space} of $Q$.
  (In this setting $Q$ is usually called the \textbf{configuration space}.)
\end{definition}
\begin{remark}
  Let $Q$ be a smooth manifold.
  The elements of $T^* Q$ are of the form $(q, p)$ where $q \in Q$ and
  $p \in T^*_q Q = \Hom(T_q Q, \R)$.
  $q$ is called \textbf{generalised position}, while $p$ is called
  \textbf{generalised momentum}.
\end{remark}
\begin{proposition}
  Let
    $Q$ be a smooth manifold,
    $(T^* Q, \omega)$ be its phase space.
  Then $(T^* Q, \omega)$ is a symplectic manifold.
\end{proposition}
\begin{definition}
  Let $Q$ be a smooth manifold of dimension $n$.
  Consider a point $q_0 \in Q$ and let $(U, \hat{\varphi})$ be a chart around
  $q_0$, i.e., $U$ is a neighbourhood of $q_0$ and
  $\hat{\varphi} \colon U \to \R^n$ is a diffeomorphism.
  Let $\{\hat{q}^i \colon U \to \R\}_{i = 1}^n$ be the corresponding local
  coordinates, i.e., if $\{\pi^i \colon \R^n \to \R\}_{i = 1}^n$ are the
  projection maps, then $\{\hat{q}^i = \pi^i \circ \hat{\varphi}\}_{i = 1}^n$.
  Let $i \in \{1, ..., n\}$.
  Define \textbf{position coordinate} $q^i \colon T^* U \to \R$ by
  \begin{equation}
    q^i := \hat{q}^i \circ \restrict{\pi}{U}.
  \end{equation}
  Also, define \textbf{momentum coordinate} $p_i \colon T^* U \to \R$
  as follows: for any $(q, p) \in T^* U$,
  \begin{equation}
    p_i(q, p)
    := p\left(\restrict{\frac{\partial}{\partial \hat{q}^i}}{q}\right).
  \end{equation}
\end{definition}
\begin{proposition}
  Let
    $Q$ be a smooth manifold of dimension $n$,
    $q_0 \in Q$,
    $(U, \hat{\varphi})$ be a chart around $q_0$,
    $\{\hat{q}^i \colon U \to \R\}_{i = 1}^n$ be the corresponding local
      coordinates,
    $\{q^i \colon T^* U \to \R\}_{i = 1}^n$ be the corresponding position
      coordinates,
    $\{p_i \colon T^* U \to \R\}_{i = 1}^n$ be the corresponding momentum
      coordinates.
  Then the map $\varphi \colon T^* U \to \R^{2 n}$ defined by
  \begin{equation}
    \varphi(q, p) = (q^1(q, p), ..., q^n(q, p), p_1(q, p), ..., p_n(q, p))
  \end{equation}
  is a diffeomorphism, i.e., $(T^* U, \varphi)$ is a chart around $(q_0, 0)$.
  (The covector in $T^*_{q_0}$ does not matter, so we make the trivial choice by
  taking zero.)

  These local coordinates are called \textbf{generalised coordinates}.
\end{proposition}
\begin{remark}
  From now on, given a manifold $Q$ and a chart $(U, \hat{\varphi})$, unless
  stated otherwise, we will fix the notation and use the objects defined above:
  the projection map $\pi \colon T^* Q \to Q$, the tautological one-form
  $\theta$ and the canonical symplectic form $\omega = - d \theta$;
  for $i = 1, ..., n$ the coordinate maps $\hat{q}^i$, $q^i$, and $p_i$;
  the chart $(T^* U, \varphi)$.
\end{remark}
\begin{proposition}
  Let
    $Q$ be a smooth manifold,
    $\xi \in \Omega^1(T^* Q)$ has the following property:
    for any $1$-form $\mu$ on $Q$, $\mu^* \xi= 0$.
  Then $\xi = 0$.
\end{proposition}
\begin{proof}
  Let
    $n := \dim Q$, $(U, \hat{\varphi})$ be a chart on $Q$ and
    $\{f_i, g^i \in \mathcal{F}(T^* U)\}_{i = 1}^n$ be such that
  \begin{equation}
    \restrict{\xi}{U} = \sum_{i = 1}^n f_i\, d q^i + \sum_{i = 1}^n g^i\, d p_i.
  \end{equation}
  Take arbitrary $\{h_j \in \mathcal{F} U\}_{j = 1}^n$ so that
  \begin{equation}
    \restrict{\mu}{U} = \sum_{j = 1}^n h_j\, d \hat{q}^j.
  \end{equation}
  Note that
  $q^i \circ \restrict{\mu}{U} = \hat{q}^i$ and
  $p_i \circ \restrict{\mu}{U} = h_i$.
  Hence,
  \begin{equation}
    0 
    = \restrict{(\mu^* \xi)}{U}
    = \sum_{i = 1}^n (f_i \circ \restrict{\mu}{U})\,
      d(q^i \circ \restrict{\mu}{U})
    + \sum_{i = 1}^n (g^i \circ \restrict{\mu}{U})\,
      d(p_i \circ \restrict{\mu}{U})
    = \sum_{i = 1}^n (f_i \circ \restrict{\mu}{U})\, d \hat{q}^i
    + \sum_{i = 1}^n (g^i \circ \restrict{\mu}{U})\, d h_i.
  \end{equation}
  Fix $q_0 \in U$, $p_0 \in T^*_{q_0} Q$ so that $(p_0, q_0) \in T^* U$.
  Denote
  \begin{equation}
    c_i
    :=
    p_0\left(\restrict{\frac{\partial}{\partial \hat{q}^i}}{q_0}\right),
    i = 1, ..., n,
  \end{equation}
  so that
  \begin{equation}
    p_0 = \sum_{i = 1}^n c_i \restrict{d \hat{q}^i}{q_0}.
  \end{equation}
  \begin{enumerate}
    \item
      We will first prove that
      for any $i \in \{1, ..., n\}$, $f_i(q_0, p_0) = 0$.
      Define the constant functions
      \begin{equation}
        h_i(q) := c_i,\ i \in \{1, ..., n\},\ q \in U.
      \end{equation}
      Then for any $i \in \{1, ..., n\}$, $d h_i = 0$.
      Hence,
      \begin{equation}
        \begin{split}
          0
          & = \restrict{(\mu^* \xi)}{q_0} \\
          & = \sum_{i = 1}^n
              f_i(q_0, \sum_{j = 1}^n h_j(q_0) \restrict{d \hat{q}^j}{q_0})\,
              \restrict{d \hat{q}^i}{q_0} \\
          & = \sum_{i = 1}^n
              f_i(q_0, \sum_{j = 1}^n c_j \restrict{d \hat{q}^j}{q_0})\,
              \restrict{d \hat{q}^i}{q_0} \\
          & = \sum_{i = 1}^n f_i(q_0, p_0)\, \restrict{d \hat{q}^i}{q_0}.
        \end{split}
      \end{equation}
      Therefore, for any $i \in \{1, ..., n\}$, $f_i(q_0, p_0) = 0$.
    \item
      We will now prove that
      for any $i \in \{1, ..., n\}$, $g^i(q_0, p_0) = 0$.
      Define the linear functions
      \begin{equation}
        h_i(q) := c_i + \hat{q}^i(q) - \hat{q}^i(q_0).
      \end{equation}
      Then for any $i \in \{1, ..., n\}$,
      $d h_i = d \hat{q}^i$ and $h_i(q) = c_i$.
      Hence,
      \begin{equation}
        \begin{split}
          0
          & = \restrict{(\mu^* \xi)}{q_0} \\
          & = \sum_{i = 1}^n
              g^i(q_0, \sum_{j = 1}^n h_j(q_0) \restrict{d \hat{q}^j}{q_0})\,
              \restrict{d h_i}{q_0} \\
          & = \sum_{i = 1}^n
              g^i(q_0, \sum_{j = 1}^n c_j \restrict{d \hat{q}^j}{q_0})\,
              \restrict{d \hat{q}^i}{q_0} \\
          & = \sum_{i = 1}^n g^i(q_0, p_0)\, \restrict{d \hat{q}^i}{q_0}.
        \end{split}
      \end{equation}
      Therefore, for any $i \in \{1, ..., n\}$, $g^i(q_0, p_0) = 0$.
  \end{enumerate}
  Since $(q_0, p_0) \in T^* U$ was arbitrary, we conclude that
  for any $i \in \{1, ..., n\}$, $f_i = g^i = 0$.
  Hence, $\restrict{\xi}{U} = 0$.
  Taking an atlas $\{(U_\alpha, \hat{\varphi}_\alpha)\}_{\alpha \in A}$ of $Q$
  (for some index set $A$), we conclude that $\xi = 0$.
\end{proof}
\begin{corollary}
  Let
    $Q$ be a smooth manifold,
    $\theta$ be the tautological one-form on $T^* Q$,
    $\eta \in \Omega^1(T^* Q)$ has the following property:
    for any $1$-form $\mu$ on $Q$, $\mu^* \eta = \mu$.
  Then $\eta = \theta$.
\end{corollary}
\begin{proof}
  Write $\eta = \theta + \xi$, i.e., $\xi := \eta - \theta$.
  Then, for any $\mu \in \Omega^1 Q$,
  \begin{equation}
    \mu
    = \mu^* \eta
    = \mu^* \theta + \mu^* \xi
    = \mu + \mu^* \xi
    \Rightarrow \mu^* \xi = 0.
  \end{equation}
  But from the previous proposition it follows that $\xi = 0$,
  and hence $\eta = \theta$.
\end{proof}
\begin{proposition}
  Let
    $Q$ be a smooth manifold of dimension $n$,
    $(U, \hat{\varphi})$ be a chart,
    $(q, p) \in T^* U$,
    $ i \in \{1, ..., n\}$.
  Then
  \begin{equation}
    \restrict{d \pi}{(q, p)}
    \left(\restrict{\frac{\partial}{\partial q^i}}{(q, p)}\right)
    = \restrict{\frac{\partial}{\partial \hat{q}^i}}{q}
  \end{equation}
  and
  \begin{equation}
    \restrict{d \pi}{(q, p)}
    \left(\restrict{\frac{\partial}{\partial p^i}}{(q, p)}\right)
    = 0.
  \end{equation}
\end{proposition}
\begin{proof}
  Let $f \colon Q \to \R$ be smooth.
  Define the functions
  $\hat{g} := f \circ \hat{\varphi}^{-1} \colon \R^n \to \R$ and
  $g := f \circ \pi \circ \varphi^{-1} \colon \R^{2 n} \to \R$.
  Let $(X^1, ..., X^n, Y^1, ..., Y^n) := \varphi(p, q) \in \R^n$.
  This means that $(X^1, ..., X^n) = \hat{\varphi}(q)$.
  Then
  \begin{equation}
    g(X^1, ..., X^n, Y^1, ..., Y^n)
    = f(\pi(q, p))
    = f(q)
    = \hat{g}(X^1, ..., X^n).
  \end{equation}
  Hence,
  \begin{equation}
    \begin{split}
      \frac{\partial g}{\partial x^i}(X^1, ..., X^n, Y^1, ..., Y^n)
      & = \lim_{h \to 0}
        \frac
        {g(X^1, ..., X^i + h, ..., X^n, Y^1, ..., Y^n)
         - g(X^1, ..., X^n, Y^1, ..., Y^n)}
        {h} \\
      & = \lim_{h \to 0}
        \frac{\hat{g}(X^1, ..., X^i + h, ..., X^n) - \hat{g}(X^1, ..., X^n)}{h}
        \\
      & = \frac{\partial \hat{g}}{\partial \hat{x}^i}(X^1, ..., X^n).
    \end{split}
  \end{equation}
  Similarly, since $g$ is constant with respect to the last $n$ coordinates,
  \begin{equation}
    \frac{\partial g}{\partial x^{n + i}}(X^1, ..., X^n, Y^1, ..., Y^n) = 0
  \end{equation}
  Take the standard coordinate systems (given by projections)
  $\{\hat{x}^k\}_{k = 1}^n$ on $\R^n$ and
  $\{x^k\}_{k = 1}^{2 n}$ on $\R^{2 n}$.
  Then, by the definitions of differential and partial derivative on manifold,
  \begin{equation}
    (\restrict{d \pi}{(q, p)}
      \left(\restrict{\frac{\partial}{\partial q^i}}{(q, p)}\right)) f
    = \restrict{\frac{\partial}{\partial q^i}}{(q, p)}(f \circ \pi)
    = \frac{\partial(f \circ \pi \circ \varphi^{-1})}{x^i}(\varphi(q, p))
    = \frac{\partial(f \circ \hat{\varphi}^{-1})}{\hat{x}^i}(\hat{\varphi}(q))
    = \restrict{\frac{\partial}{\partial \hat{q}^i}}{q} f,
  \end{equation}
  from which it follows that the first equality holds.
  Similarly,
  \begin{equation}
    (\restrict{d \pi}{(q, p)}
      \left(\restrict{\frac{\partial}{\partial p^i}}{(q, p)}\right)) f
    = \restrict{\frac{\partial}{\partial p^i}}{(q, p)}(f \circ \pi)
    = \frac{\partial(f \circ \pi \circ \varphi^{-1})}{x^{i + n}}(\varphi(q, p))
    = 0,
  \end{equation}
  from which it follows that the second equality holds.
\end{proof}
\begin{proposition}[Tautological one-form in generalised coordinates]
  Let
    $Q$ be a smooth manifold of dimension $n$,
    $(U, \hat{\varphi})$ be a chart.
  Then
  \begin{equation}
    \restrict{\theta}{U} = \sum_{i = 1}^n p_i\, d q^i.
  \end{equation}
\end{proposition}
\begin{proof}
  Let $(q, p) \in T^* U$.
  Recall that $\restrict{\theta}{(q, p)} = p \circ \restrict{d \pi}{(q, p)}$.
  Hence,
  \begin{equation}
    \restrict{\theta}{(q, p)}
    \left(\restrict{\frac{\partial}{\partial q^i}}{(q, p)}\right)
    = p\left(\restrict{\frac{\partial}{\partial \hat{q}^i}}{q}\right)
    = p_i(q, p),
  \end{equation}
  and
  \begin{equation}
    \restrict{\theta}{(q, p)}
    \left(\restrict{\frac{\partial}{\partial p^i}}{(q, p)}\right)
    = p(0)
    = 0.
  \end{equation}
  Therefore,
  \begin{equation}
    \restrict{\theta}{(q, p)}
    = \sum_{i = 1}^n
      \restrict{\theta}{(q, p)}
      \left(\restrict{\frac{\partial}{\partial q^i}}{(q, p)}\right)\,
      \restrict{d q^i}{(q, p)}
    + \sum_{i = 1}^n
      \restrict{\theta}{(q, p)}
      \left(\restrict{\frac{\partial}{\partial p^i}}{(q, p)}\right)\,
      \restrict{d p^i}{(q, p)}
    = \sum_{i = 1}^n p_i(q, p)\, \restrict{d q^i}{(q, p)},
  \end{equation}
  from which the proposition follows.
\end{proof}
\begin{corollary}[Canonical symplectic in generalised coordinates]
  Let
    $Q$ be a smooth manifold of dimension $n$,
    $(U, \hat{\varphi})$ be a chart.
  Then
  \begin{equation}
    \restrict{\omega}{U} = \sum_{i = 1}^n d q^i \wedge d p_i.
  \end{equation}
\end{corollary}
\begin{proof}
  Let $i \in \{1, ..., n\}$.
  Then
  \begin{equation}
    - d(p_i\, d q^i) = - d p_i \wedge d q^i = d q^i \wedge d p_i.
  \end{equation}
  Summing up for all $i$, we get the desired result.
\end{proof}
\begin{definition}
  Let $(M, \omega)$ be a symplectic manifold, $f \in \mathcal{F} M$.
  We say that $X \in \mathfrak{X} M$ is a \textbf{Hamiltonian vector field} for
  $f$ if
  \begin{equation}
    i_X \omega + d_0 f = 0.
  \end{equation}
\end{definition}
\begin{proposition}
  Let $(M, \omega)$ be a symplectic manifold, $f \in \mathcal{F} M$.
  Then there exists a unique Hamiltonian vector field for $f$.
\end{proposition}
\begin{proof}
  The non-degeneracy of $\omega$ means that we can interpret the symplectic form
  as the isomorphism $\tilde{\omega} \colon \mathfrak{X} M \to \Omega^1 M$,
  given by
  \begin{equation}
    (\tilde{\omega} X) := i_X \omega,\ X \in \mathfrak{X} M.
  \end{equation}
  Hence, the problem at hand has a unique solution
  $X = \tilde{\omega}^{-1}(- d_0 f)$.
\end{proof}
\begin{definition}
  Let $(M, \omega)$ be a symplectic manifold.
  Define the map $\hamiltonian \colon \mathcal{F} M \to \mathfrak{X} M$ by
  \begin{equation}
    \hamiltonian = - \tilde{\omega}^{-1} \circ d_0.
  \end{equation}
  It maps a function to its corresponding Hamiltonian vector field.
  We will write $\hamiltonian_f$ instead of $\hamiltonian(f)$
  for $f \in \mathcal{F} M$.
\end{definition}
\begin{proposition}
  Let
    $Q$ be a smooth manifold of dimension $n$,
    $(U, \hat{\varphi})$ be a chart on $Q$,
    $f \in \mathcal{F}(T^* Q)$.
  Then
  \begin{equation}
    \hamiltonian_f
    = \sum_{i = 1}^n
    \left(
      - \frac{\partial f}{\partial p^i} \frac{\partial}{\partial q^i}
      + \frac{\partial f}{\partial q^i} \frac{\partial}{\partial p^i}
    \right).
  \end{equation}
\end{proposition}
\begin{proof}
  First, note that
  $i_{\frac{\partial}{\partial q^i}} \omega = d p^i$ and
  $i_{\frac{\partial}{\partial p^i}} \omega = - d q^i$.
  Denote
  \begin{equation}
    X
    := \sum_{i = 1}^n
    \left(
      - \frac{\partial f}{\partial p^i} \frac{\partial}{\partial q^i}
      + \frac{\partial f}{\partial q^i} \frac{\partial}{\partial p^i}
    \right).
  \end{equation}
  Then
  \begin{equation}
    i_X \omega
    = \sum_{i}^n
    \left(
      - \frac{\partial f}{\partial p^i} i_{\frac{\partial}{\partial q^i}} \omega
      + \frac{\partial f}{\partial q^i} i_{\frac{\partial}{\partial p^i}} \omega
    \right)
    = \sum_{i}^n
    \left(
      - \frac{\partial f}{\partial p^i} d p^i
      - \frac{\partial f}{\partial q^i} d q^i
    \right)
    = - d f.
  \end{equation}
  Hence, $\hamiltonian_f = X$.
\end{proof}
\begin{proposition}
  Let $(M, \omega)$ be a symplectic manifold, $f, g \in \mathcal{F} M$.
  Then
  \begin{equation}
    \hamiltonian_{f g} = f \hamiltonian_g + g \hamiltonian_f.
  \end{equation}
\end{proposition}
\begin{proof}
  Follows directly from the Leibniz rule for $d_0$.
\end{proof}
\begin{definition}
  Let $(M, \omega)$ be a symplectic manifold, $X \in \mathfrak{X} M$.
  We say that $X$ is a \textbf{symplectic vector field} if $L_X \omega = 0$.
\end{definition}
\begin{remark}
  Since $L_{\lie{X}{Y}} = \lie{L_X}{L_Y} = L_X \circ L_Y - L_Y \circ L_X$,
  the symplectic vector fields form a Lie subalgebra of the Lie algebra of
  vector fields.
\end{remark}
\begin{proposition}
  Let $(M, \omega)$ be a symplectic manifold, $f \in \mathcal{F} M$.
  Then $\hamiltonian_f$ is a symplectic vector field.
\end{proposition}
\begin{proof}
  $
    L_{\hamiltonian_f} \omega
    = i_{\hamiltonian_f}(d \omega) + d(i_{\hamiltonian_f} \omega)
    = i_{\hamiltonian_f} 0 - d(d f)
    = 0.
  $
\end{proof}
\begin{proposition}
  Let
    $(M, \omega)$ be a symplectic manifold,
    $X \in \mathfrak{X} M$ be a symplectic vector fields.
  Then
  \begin{equation}
    d(i_X \omega) = 0.
  \end{equation}
\end{proposition}
\begin{proof}
  $
    d(i_X \omega)
    = L_X \omega - i_X(d \omega)
    = 0 - 0
    = 0.
  $
\end{proof}
\begin{proposition}
  Let $M$ be a smooth manifold, $X, Y \in \mathfrak{X} M$.
  Then
  \begin{equation}
    L_X \circ i_Y = i_{\lie{X}{Y}} + i_Y \circ L_X.
  \end{equation}
\end{proposition}
\begin{proposition}
  Let
    $(M, \omega)$ be a symplectic manifold,
    $X, Y \in \mathfrak{X} M$ be symplectic vector fields.
  Then
  \begin{equation}
    \lie{X}{Y} = \hamiltonian_{i_Y(i_X \omega)}.
  \end{equation}
\end{proposition}
\begin{proof}
  \begin{equation}
    i_{\lie{X}{Y}} \omega
    = (L_X \circ i_Y - i_Y \circ L_X) \omega
    = (L_X \circ i_Y) \omega
    = ((d \circ i_X + i_X \circ d) \circ i_Y) \omega
    = d(i_X(i_Y \omega))
    = - d(i_Y(i_X \omega)).
  \end{equation}
  We get the desired result from the definition of $\hamiltonian$.
\end{proof}
\begin{definition}
  Let $(M, \omega)$ be a symplectic manifold.
  Define the \textbf{Poisson bracket}
  $\poisson{\cdot}{\cdot} \colon \mathcal{F} M \to \mathcal{F} M$ by
  \begin{equation}
    \poisson{f}{g}
    := i_{\hamiltonian_g}(i_{\hamiltonian_f} \omega),\
    f, g \in \mathcal{F} M.
  \end{equation}
\end{definition}
\begin{corollary}
  Let $(M, \omega)$ be a symplectic manifold, $f, g \in \mathcal{F} M$.
  Then
  \begin{equation}
    \lie{\hamiltonian_f}{\hamiltonian_g}
    = \hamiltonian_{i_{\hamiltonian_g}(i_{\hamiltonian_f} \omega)}
    = \poisson{f}{g}.
  \end{equation}
\end{corollary}
\begin{proposition}[Leibniz rule holds for the Poisson bracket]
  Let $(M, \omega)$ be a symplectic manifold, $f, g, h \in \mathcal{F} M$.
  Then
  \begin{equation}
    \poisson{f}{g h} = \poisson{f}{g} h + g \poisson{f}{h}.
  \end{equation}
\end{proposition}
\begin{proof}
  $
    \poisson{f}{g h}
    = i_{\hamiltonian_{g h}}(i_{\hamiltonian_f} \omega)
    = i_{h \hamiltonian_g + g \hamiltonian_{h}}(i_{\hamiltonian_f} \omega)
    = (i_{\hamiltonian_g}(i_{\hamiltonian_f} \omega))\, h
      + g\, (i_{\hamiltonian_h}(i_{\hamiltonian_f} \omega)) 
    = \poisson{f}{g} h + g \poisson{f}{h}.
  $
\end{proof}
\begin{proposition}
  Let $(M, \omega)$ be a symplectic manifold, $f, g, h \in \mathcal{F} M$.
  Then
  \begin{equation}
    \poisson{f}{g} = L_{\hamiltonian_f} g.
  \end{equation}
\end{proposition}
\begin{proof}
  $
    \poisson{f}{g}
    = i_{\hamiltonian_g}(i_{\hamiltonian_f} \omega)
    = - i_{\hamiltonian_f} \circ i_{\hamiltonian_g} \omega
    = i_{\hamiltonian_f}(d g)
    = L_{\hamiltonian_f} g.
  $
\end{proof}
\begin{definition}
  Let $(M, \omega)$ be a symplectic manifold.
  Define
  ${\rm ad} \colon \mathcal{F} M \to (\mathcal{F} M \to \mathcal{F} M)$ by
  \begin{equation}
    {\rm ad}_f g := \poisson{f}{g},\ f, g \in \mathcal{F} M,
  \end{equation}
\end{definition}
\begin{proposition}
  Let $(M, \omega)$ be a symplectic manifold.
  Then
  \begin{equation}
    \lie{{\rm ad}_f}{{\rm ad}_g} = {\rm ad}_{\poisson{f}{g}}.
  \end{equation}
  (Here the bracket $\lie{\cdot}{\cdot}$ is the commutator of operators.)
\end{proposition}
\begin{proof}
  From the previous proposition it follows that
  \begin{equation}
    {\rm ad}_f = L_{X_f},\ f \in \mathcal{F} M.
  \end{equation}
  Hence,
  \begin{equation}
    \lie{{\rm ad}_f}{{\rm ad}_g}
    = \lie{L_{\hamiltonian_f}}{L_{\hamiltonian_g}}
    = L_{\lie{\hamiltonian_f}{\hamiltonian_g}}
    = L_{\hamiltonian_{\poisson{f}{g}}}
    = {\rm ad}_{\poisson{f}{g}}.
  \end{equation}
\end{proof}
\begin{corollary}
  Let $(M, \omega)$ be a symplectic manifold.
  Then $(\mathcal{F} M, \poisson{\cdot}{\cdot})$ is a Lie algebra over $\R$.
\end{corollary}
\begin{proof}
  Bilinearity and antisymmetry are trivial to check.
  The Jacobi identity is equivalent to the adjoint map being a Lie algebra
  homomorphism, which was the previous proposition.
\end{proof}
\begin{definition}
  Let
    $R$ be a commutative ring with unity ring,
    $(A, +, \cdot)$ be an $R$-module
    with additional structures of
    an associative algebra $(A, *)$ and
    a Lie algebra $(A, \poisson{\cdot}{\cdot})$.
  We say that $A$ is a Poisson algebra if the Lie bracket acts as a derivation,
  i.e., for all $f, g, h \in A$,
  \begin{equation}
    \poisson{f}{g * h} = \poisson{f}{g} * h + g * \poisson{f}{h}.
  \end{equation}
\end{definition}
\begin{corollary}
  Let $(M, \omega)$ be a symplectic manifold.
  Then $\mathcal{F} M$ is a Poisson algebra over $\R$.
  Here, addition, scalar multiplication, and multiplication are given by the
  corresponding pointwise operations, while the Lie bracket is given by the
  Poisson bracket.
\end{corollary}


\section{Sections on discrete vector bundles}
\label{section:sections_on_discrete_vector bundles}
\begin{definition}
  Let
    $R$ be a ring,
    $V$ be a finite-dimensional $R$-module,
    $\omega \in \Lambda^2 V^*$.
  We say that $\omega$ is \textbf{non-degenerate} or \textbf{symplectic}
  if the associated map
  \begin{equation}
    \tilde{\omega} \colon V \to V^*,\
    X \in V \mapsto \tilde{\omega}(X) := i_X \omega \in V^*,
  \end{equation}
  is an isomorphism.

  The pair $(V, \omega)$ is called a \textbf{symplectic module}
  (or \textbf{symplectic vector space} if $R$ is a field).
\end{definition}
\begin{proposition}
  Let
    $R$ be a ring without,
    $(V, \omega)$ be a finite-dimensional symplectic module over $R$.
  Assume that for any $x \in R,\ x + x = 0 \Rightarrow x = 0$.
  Then $\dim V$ is an even number.
\end{proposition}
\begin{proof}
  Let $n = \dim V$.
  In a basis of $V$ $\omega$ is represented by an antisymmetric matrix $A$.
  But then
  \begin{equation}
    \det A = \det(A^T) = \det(-A) = (-1)^n \det A.
  \end{equation}
  If $n$ is odd, then $\det A + \det A = 0$.
  By assumption this means that $\det A = 0$
  which contradicts the nondegeneracy of $\omega$.
  Hence, $n$ is even.
\end{proof}
\begin{definition}
  Let $M$ be a smooth manifold, $\omega \in \Omega^\bullet M$.
  We say that:
  \begin{enumerate}
    \item
      $\omega$ is \textbf{closed} if $d \omega = 0$
    \item
      $\omega$ is \textbf{exact} if there exists $\eta \in \Omega^\bullet M$
      such that $d \eta = \omega$.
  \end{enumerate}
\end{definition}
\begin{proposition}
  Let $M$ be a smooth manifold, $\omega \in \Omega^\bullet M$.
  If $\omega$ is exact, then it is closed.
\end{proposition}
\begin{proof}
  Let $\eta \in \Omega^\bullet M$ be such that $d \eta = \omega$.
  Then $d \omega = d (d \eta) = 0$, i.e., $\omega$ is closed.
\end{proof}
\begin{definition}
  Let $M$ be a smooth manifold, $\omega \in \Omega^2 M$.
  We say that $\omega$ is a \textbf{symplectic form}
  if it is non-degenerate
  (with base module $\mathfrak{X} M$ over $\mathcal{F} M$) and closed.

  The pair $(M, \omega)$ is called a \textbf{symplectic manifold}.
\end{definition}
\begin{proposition}
  Let $(M, \omega)$ be a symplectic manifold.
  Then $M$ is even-dimensional.
\end{proposition}
\begin{definition}
  Let $Q$ be a smooth manifold.
  Consider the cotangent bundle $T^* Q$ with bundle projection
  $\pi \colon T^* Q \to Q$
  with differential $d \pi \colon T(T^* Q) \to T Q$.
  Define the \textbf{tautological one-form}
  $\theta \colon T^* Q \to T^* (T^* Q)$ as follows:
  for any $(q, p) \in T^*Q$ (i.e,. $q \in Q$, $p \in \Hom(T_q Q, \R)$),
  \begin{equation}
    \restrict{\theta}{(q, p)}
    := p \circ \restrict{d \pi}{(q, p)} \in T^*_{(q, p)}(T^* Q).
  \end{equation}
  In other words, if we denote $M := T^* Q$, then $\theta$ is a section of its
  cotangent bundle $T^* M$, i.e., an $1$-form on $M$. 
\end{definition}
\begin{discussion}
  Let $Q$ be a smooth manifold, $\pi \colon T^* Q \to Q$ be the projection.
  Then a $1$-form on $Q$ is a section of $\colon T^* Q$, i.e., a smooth map
  $\mu \colon Q \to \colon T^* Q$ such that $\pi \circ \mu = \id_Q$.
  As such it has a pullback
  $\mu^* \colon \Omega^\bullet(T^* Q) \to \Omega^\bullet Q$.
\end{discussion}
\begin{proposition}
  Let
    $Q$ be a smooth manifold,
    $\theta$ be the tautological one-form on $T^* Q$,
    $\mu \in \Omega^1 Q$.
  Then
  \begin{equation}
    \mu^* \theta = \mu.
  \end{equation}
\end{proposition}
\begin{proof}
  Let $q \in Q$.
  Then
  \begin{equation}
    \restrict{\mu^* \theta}{q}
    = \restrict{\theta}{\mu q} \circ \restrict{d \mu}{q}
    = \restrict{\mu}{q} \circ \restrict{(d \pi)}{\mu q}
      \circ \restrict{d \mu}{q}
    = \restrict{\mu}{q} \circ \restrict{d(\pi \circ \mu)}{q}
    = \restrict{\mu}{q}.
  \end{equation}
  Since $q$ is arbitrary, $\mu^* \theta = \mu$.
\end{proof}
\begin{definition}
  Let
    $Q$ be a smooth manifold,
    $\theta \in \Omega^1(T^* Q)$ be the tautological one-form.
  Define $\omega := - d \theta$.
  The pair $(T^* Q, \omega)$ is called the \textbf{phase space} of $Q$.
  (In this setting $Q$ is usually called the \textbf{configuration space}.)
\end{definition}
\begin{remark}
  Let $Q$ be a smooth manifold.
  The elements of $T^* Q$ are of the form $(q, p)$ where $q \in Q$ and
  $p \in T^*_q Q = \Hom(T_q Q, \R)$.
  $q$ is called \textbf{generalised position}, while $p$ is called
  \textbf{generalised momentum}.
\end{remark}
\begin{proposition}
  Let
    $Q$ be a smooth manifold,
    $(T^* Q, \omega)$ be its phase space.
  Then $(T^* Q, \omega)$ is a symplectic manifold.
\end{proposition}
\begin{definition}
  Let $Q$ be a smooth manifold of dimension $n$.
  Consider a point $q_0 \in Q$ and let $(U, \hat{\varphi})$ be a chart around
  $q_0$, i.e., $U$ is a neighbourhood of $q_0$ and
  $\hat{\varphi} \colon U \to \R^n$ is a diffeomorphism.
  Let $\{\hat{q}^i \colon U \to \R\}_{i = 1}^n$ be the corresponding local
  coordinates, i.e., if $\{\pi^i \colon \R^n \to \R\}_{i = 1}^n$ are the
  projection maps, then $\{\hat{q}^i = \pi^i \circ \hat{\varphi}\}_{i = 1}^n$.
  Let $i \in \{1, ..., n\}$.
  Define \textbf{position coordinate} $q^i \colon T^* U \to \R$ by
  \begin{equation}
    q^i := \hat{q}^i \circ \restrict{\pi}{U}.
  \end{equation}
  Also, define \textbf{momentum coordinate} $p_i \colon T^* U \to \R$
  as follows: for any $(q, p) \in T^* U$,
  \begin{equation}
    p_i(q, p)
    := p\left(\restrict{\frac{\partial}{\partial \hat{q}^i}}{q}\right).
  \end{equation}
\end{definition}
\begin{proposition}
  Let
    $Q$ be a smooth manifold of dimension $n$,
    $q_0 \in Q$,
    $(U, \hat{\varphi})$ be a chart around $q_0$,
    $\{\hat{q}^i \colon U \to \R\}_{i = 1}^n$ be the corresponding local
      coordinates,
    $\{q^i \colon T^* U \to \R\}_{i = 1}^n$ be the corresponding position
      coordinates,
    $\{p_i \colon T^* U \to \R\}_{i = 1}^n$ be the corresponding momentum
      coordinates.
  Then the map $\varphi \colon T^* U \to \R^{2 n}$ defined by
  \begin{equation}
    \varphi(q, p) = (q^1(q, p), ..., q^n(q, p), p_1(q, p), ..., p_n(q, p))
  \end{equation}
  is a diffeomorphism, i.e., $(T^* U, \varphi)$ is a chart around $(q_0, 0)$.
  (The covector in $T^*_{q_0}$ does not matter, so we make the trivial choice by
  taking zero.)

  These local coordinates are called \textbf{generalised coordinates}.
\end{proposition}
\begin{remark}
  From now on, given a manifold $Q$ and a chart $(U, \hat{\varphi})$, unless
  stated otherwise, we will fix the notation and use the objects defined above:
  the projection map $\pi \colon T^* Q \to Q$, the tautological one-form
  $\theta$ and the canonical symplectic form $\omega = - d \theta$;
  for $i = 1, ..., n$ the coordinate maps $\hat{q}^i$, $q^i$, and $p_i$;
  the chart $(T^* U, \varphi)$.
\end{remark}
\begin{proposition}
  Let
    $Q$ be a smooth manifold,
    $\xi \in \Omega^1(T^* Q)$ has the following property:
    for any $1$-form $\mu$ on $Q$, $\mu^* \xi= 0$.
  Then $\xi = 0$.
\end{proposition}
\begin{proof}
  Let
    $n := \dim Q$, $(U, \hat{\varphi})$ be a chart on $Q$ and
    $\{f_i, g^i \in \mathcal{F}(T^* U)\}_{i = 1}^n$ be such that
  \begin{equation}
    \restrict{\xi}{U} = \sum_{i = 1}^n f_i\, d q^i + \sum_{i = 1}^n g^i\, d p_i.
  \end{equation}
  Take arbitrary $\{h_j \in \mathcal{F} U\}_{j = 1}^n$ so that
  \begin{equation}
    \restrict{\mu}{U} = \sum_{j = 1}^n h_j\, d \hat{q}^j.
  \end{equation}
  Note that
  $q^i \circ \restrict{\mu}{U} = \hat{q}^i$ and
  $p_i \circ \restrict{\mu}{U} = h_i$.
  Hence,
  \begin{equation}
    0 
    = \restrict{(\mu^* \xi)}{U}
    = \sum_{i = 1}^n (f_i \circ \restrict{\mu}{U})\,
      d(q^i \circ \restrict{\mu}{U})
    + \sum_{i = 1}^n (g^i \circ \restrict{\mu}{U})\,
      d(p_i \circ \restrict{\mu}{U})
    = \sum_{i = 1}^n (f_i \circ \restrict{\mu}{U})\, d \hat{q}^i
    + \sum_{i = 1}^n (g^i \circ \restrict{\mu}{U})\, d h_i.
  \end{equation}
  Fix $q_0 \in U$, $p_0 \in T^*_{q_0} Q$ so that $(p_0, q_0) \in T^* U$.
  Denote
  \begin{equation}
    c_i
    :=
    p_0\left(\restrict{\frac{\partial}{\partial \hat{q}^i}}{q_0}\right),
    i = 1, ..., n,
  \end{equation}
  so that
  \begin{equation}
    p_0 = \sum_{i = 1}^n c_i \restrict{d \hat{q}^i}{q_0}.
  \end{equation}
  \begin{enumerate}
    \item
      We will first prove that
      for any $i \in \{1, ..., n\}$, $f_i(q_0, p_0) = 0$.
      Define the constant functions
      \begin{equation}
        h_i(q) := c_i,\ i \in \{1, ..., n\},\ q \in U.
      \end{equation}
      Then for any $i \in \{1, ..., n\}$, $d h_i = 0$.
      Hence,
      \begin{equation}
        \begin{split}
          0
          & = \restrict{(\mu^* \xi)}{q_0} \\
          & = \sum_{i = 1}^n
              f_i(q_0, \sum_{j = 1}^n h_j(q_0) \restrict{d \hat{q}^j}{q_0})\,
              \restrict{d \hat{q}^i}{q_0} \\
          & = \sum_{i = 1}^n
              f_i(q_0, \sum_{j = 1}^n c_j \restrict{d \hat{q}^j}{q_0})\,
              \restrict{d \hat{q}^i}{q_0} \\
          & = \sum_{i = 1}^n f_i(q_0, p_0)\, \restrict{d \hat{q}^i}{q_0}.
        \end{split}
      \end{equation}
      Therefore, for any $i \in \{1, ..., n\}$, $f_i(q_0, p_0) = 0$.
    \item
      We will now prove that
      for any $i \in \{1, ..., n\}$, $g^i(q_0, p_0) = 0$.
      Define the linear functions
      \begin{equation}
        h_i(q) := c_i + \hat{q}^i(q) - \hat{q}^i(q_0).
      \end{equation}
      Then for any $i \in \{1, ..., n\}$,
      $d h_i = d \hat{q}^i$ and $h_i(q) = c_i$.
      Hence,
      \begin{equation}
        \begin{split}
          0
          & = \restrict{(\mu^* \xi)}{q_0} \\
          & = \sum_{i = 1}^n
              g^i(q_0, \sum_{j = 1}^n h_j(q_0) \restrict{d \hat{q}^j}{q_0})\,
              \restrict{d h_i}{q_0} \\
          & = \sum_{i = 1}^n
              g^i(q_0, \sum_{j = 1}^n c_j \restrict{d \hat{q}^j}{q_0})\,
              \restrict{d \hat{q}^i}{q_0} \\
          & = \sum_{i = 1}^n g^i(q_0, p_0)\, \restrict{d \hat{q}^i}{q_0}.
        \end{split}
      \end{equation}
      Therefore, for any $i \in \{1, ..., n\}$, $g^i(q_0, p_0) = 0$.
  \end{enumerate}
  Since $(q_0, p_0) \in T^* U$ was arbitrary, we conclude that
  for any $i \in \{1, ..., n\}$, $f_i = g^i = 0$.
  Hence, $\restrict{\xi}{U} = 0$.
  Taking an atlas $\{(U_\alpha, \hat{\varphi}_\alpha)\}_{\alpha \in A}$ of $Q$
  (for some index set $A$), we conclude that $\xi = 0$.
\end{proof}
\begin{corollary}
  Let
    $Q$ be a smooth manifold,
    $\theta$ be the tautological one-form on $T^* Q$,
    $\eta \in \Omega^1(T^* Q)$ has the following property:
    for any $1$-form $\mu$ on $Q$, $\mu^* \eta = \mu$.
  Then $\eta = \theta$.
\end{corollary}
\begin{proof}
  Write $\eta = \theta + \xi$, i.e., $\xi := \eta - \theta$.
  Then, for any $\mu \in \Omega^1 Q$,
  \begin{equation}
    \mu
    = \mu^* \eta
    = \mu^* \theta + \mu^* \xi
    = \mu + \mu^* \xi
    \Rightarrow \mu^* \xi = 0.
  \end{equation}
  But from the previous proposition it follows that $\xi = 0$,
  and hence $\eta = \theta$.
\end{proof}
\begin{proposition}
  Let
    $Q$ be a smooth manifold of dimension $n$,
    $(U, \hat{\varphi})$ be a chart,
    $(q, p) \in T^* U$,
    $ i \in \{1, ..., n\}$.
  Then
  \begin{equation}
    \restrict{d \pi}{(q, p)}
    \left(\restrict{\frac{\partial}{\partial q^i}}{(q, p)}\right)
    = \restrict{\frac{\partial}{\partial \hat{q}^i}}{q}
  \end{equation}
  and
  \begin{equation}
    \restrict{d \pi}{(q, p)}
    \left(\restrict{\frac{\partial}{\partial p^i}}{(q, p)}\right)
    = 0.
  \end{equation}
\end{proposition}
\begin{proof}
  Let $f \colon Q \to \R$ be smooth.
  Define the functions
  $\hat{g} := f \circ \hat{\varphi}^{-1} \colon \R^n \to \R$ and
  $g := f \circ \pi \circ \varphi^{-1} \colon \R^{2 n} \to \R$.
  Let $(X^1, ..., X^n, Y^1, ..., Y^n) := \varphi(p, q) \in \R^n$.
  This means that $(X^1, ..., X^n) = \hat{\varphi}(q)$.
  Then
  \begin{equation}
    g(X^1, ..., X^n, Y^1, ..., Y^n)
    = f(\pi(q, p))
    = f(q)
    = \hat{g}(X^1, ..., X^n).
  \end{equation}
  Hence,
  \begin{equation}
    \begin{split}
      \frac{\partial g}{\partial x^i}(X^1, ..., X^n, Y^1, ..., Y^n)
      & = \lim_{h \to 0}
        \frac
        {g(X^1, ..., X^i + h, ..., X^n, Y^1, ..., Y^n)
         - g(X^1, ..., X^n, Y^1, ..., Y^n)}
        {h} \\
      & = \lim_{h \to 0}
        \frac{\hat{g}(X^1, ..., X^i + h, ..., X^n) - \hat{g}(X^1, ..., X^n)}{h}
        \\
      & = \frac{\partial \hat{g}}{\partial \hat{x}^i}(X^1, ..., X^n).
    \end{split}
  \end{equation}
  Similarly, since $g$ is constant with respect to the last $n$ coordinates,
  \begin{equation}
    \frac{\partial g}{\partial x^{n + i}}(X^1, ..., X^n, Y^1, ..., Y^n) = 0
  \end{equation}
  Take the standard coordinate systems (given by projections)
  $\{\hat{x}^k\}_{k = 1}^n$ on $\R^n$ and
  $\{x^k\}_{k = 1}^{2 n}$ on $\R^{2 n}$.
  Then, by the definitions of differential and partial derivative on manifold,
  \begin{equation}
    (\restrict{d \pi}{(q, p)}
      \left(\restrict{\frac{\partial}{\partial q^i}}{(q, p)}\right)) f
    = \restrict{\frac{\partial}{\partial q^i}}{(q, p)}(f \circ \pi)
    = \frac{\partial(f \circ \pi \circ \varphi^{-1})}{x^i}(\varphi(q, p))
    = \frac{\partial(f \circ \hat{\varphi}^{-1})}{\hat{x}^i}(\hat{\varphi}(q))
    = \restrict{\frac{\partial}{\partial \hat{q}^i}}{q} f,
  \end{equation}
  from which it follows that the first equality holds.
  Similarly,
  \begin{equation}
    (\restrict{d \pi}{(q, p)}
      \left(\restrict{\frac{\partial}{\partial p^i}}{(q, p)}\right)) f
    = \restrict{\frac{\partial}{\partial p^i}}{(q, p)}(f \circ \pi)
    = \frac{\partial(f \circ \pi \circ \varphi^{-1})}{x^{i + n}}(\varphi(q, p))
    = 0,
  \end{equation}
  from which it follows that the second equality holds.
\end{proof}
\begin{proposition}[Tautological one-form in generalised coordinates]
  Let
    $Q$ be a smooth manifold of dimension $n$,
    $(U, \hat{\varphi})$ be a chart.
  Then
  \begin{equation}
    \restrict{\theta}{U} = \sum_{i = 1}^n p_i\, d q^i.
  \end{equation}
\end{proposition}
\begin{proof}
  Let $(q, p) \in T^* U$.
  Recall that $\restrict{\theta}{(q, p)} = p \circ \restrict{d \pi}{(q, p)}$.
  Hence,
  \begin{equation}
    \restrict{\theta}{(q, p)}
    \left(\restrict{\frac{\partial}{\partial q^i}}{(q, p)}\right)
    = p\left(\restrict{\frac{\partial}{\partial \hat{q}^i}}{q}\right)
    = p_i(q, p),
  \end{equation}
  and
  \begin{equation}
    \restrict{\theta}{(q, p)}
    \left(\restrict{\frac{\partial}{\partial p^i}}{(q, p)}\right)
    = p(0)
    = 0.
  \end{equation}
  Therefore,
  \begin{equation}
    \restrict{\theta}{(q, p)}
    = \sum_{i = 1}^n
      \restrict{\theta}{(q, p)}
      \left(\restrict{\frac{\partial}{\partial q^i}}{(q, p)}\right)\,
      \restrict{d q^i}{(q, p)}
    + \sum_{i = 1}^n
      \restrict{\theta}{(q, p)}
      \left(\restrict{\frac{\partial}{\partial p^i}}{(q, p)}\right)\,
      \restrict{d p^i}{(q, p)}
    = \sum_{i = 1}^n p_i(q, p)\, \restrict{d q^i}{(q, p)},
  \end{equation}
  from which the proposition follows.
\end{proof}
\begin{corollary}[Canonical symplectic in generalised coordinates]
  Let
    $Q$ be a smooth manifold of dimension $n$,
    $(U, \hat{\varphi})$ be a chart.
  Then
  \begin{equation}
    \restrict{\omega}{U} = \sum_{i = 1}^n d q^i \wedge d p_i.
  \end{equation}
\end{corollary}
\begin{proof}
  Let $i \in \{1, ..., n\}$.
  Then
  \begin{equation}
    - d(p_i\, d q^i) = - d p_i \wedge d q^i = d q^i \wedge d p_i.
  \end{equation}
  Summing up for all $i$, we get the desired result.
\end{proof}
\begin{definition}
  Let $(M, \omega)$ be a symplectic manifold, $f \in \mathcal{F} M$.
  We say that $X \in \mathfrak{X} M$ is a \textbf{Hamiltonian vector field} for
  $f$ if
  \begin{equation}
    i_X \omega + d_0 f = 0.
  \end{equation}
\end{definition}
\begin{proposition}
  Let $(M, \omega)$ be a symplectic manifold, $f \in \mathcal{F} M$.
  Then there exists a unique Hamiltonian vector field for $f$.
\end{proposition}
\begin{proof}
  The non-degeneracy of $\omega$ means that we can interpret the symplectic form
  as the isomorphism $\tilde{\omega} \colon \mathfrak{X} M \to \Omega^1 M$,
  given by
  \begin{equation}
    (\tilde{\omega} X) := i_X \omega,\ X \in \mathfrak{X} M.
  \end{equation}
  Hence, the problem at hand has a unique solution
  $X = \tilde{\omega}^{-1}(- d_0 f)$.
\end{proof}
\begin{definition}
  Let $(M, \omega)$ be a symplectic manifold.
  Define the map $\hamiltonian \colon \mathcal{F} M \to \mathfrak{X} M$ by
  \begin{equation}
    \hamiltonian = - \tilde{\omega}^{-1} \circ d_0.
  \end{equation}
  It maps a function to its corresponding Hamiltonian vector field.
  We will write $\hamiltonian_f$ instead of $\hamiltonian(f)$
  for $f \in \mathcal{F} M$.
\end{definition}
\begin{proposition}
  Let
    $Q$ be a smooth manifold of dimension $n$,
    $(U, \hat{\varphi})$ be a chart on $Q$,
    $f \in \mathcal{F}(T^* Q)$.
  Then
  \begin{equation}
    \hamiltonian_f
    = \sum_{i = 1}^n
    \left(
      - \frac{\partial f}{\partial p^i} \frac{\partial}{\partial q^i}
      + \frac{\partial f}{\partial q^i} \frac{\partial}{\partial p^i}
    \right).
  \end{equation}
\end{proposition}
\begin{proof}
  First, note that
  $i_{\frac{\partial}{\partial q^i}} \omega = d p^i$ and
  $i_{\frac{\partial}{\partial p^i}} \omega = - d q^i$.
  Denote
  \begin{equation}
    X
    := \sum_{i = 1}^n
    \left(
      - \frac{\partial f}{\partial p^i} \frac{\partial}{\partial q^i}
      + \frac{\partial f}{\partial q^i} \frac{\partial}{\partial p^i}
    \right).
  \end{equation}
  Then
  \begin{equation}
    i_X \omega
    = \sum_{i}^n
    \left(
      - \frac{\partial f}{\partial p^i} i_{\frac{\partial}{\partial q^i}} \omega
      + \frac{\partial f}{\partial q^i} i_{\frac{\partial}{\partial p^i}} \omega
    \right)
    = \sum_{i}^n
    \left(
      - \frac{\partial f}{\partial p^i} d p^i
      - \frac{\partial f}{\partial q^i} d q^i
    \right)
    = - d f.
  \end{equation}
  Hence, $\hamiltonian_f = X$.
\end{proof}
\begin{proposition}
  Let $(M, \omega)$ be a symplectic manifold, $f, g \in \mathcal{F} M$.
  Then
  \begin{equation}
    \hamiltonian_{f g} = f \hamiltonian_g + g \hamiltonian_f.
  \end{equation}
\end{proposition}
\begin{proof}
  Follows directly from the Leibniz rule for $d_0$.
\end{proof}
\begin{definition}
  Let $(M, \omega)$ be a symplectic manifold, $X \in \mathfrak{X} M$.
  We say that $X$ is a \textbf{symplectic vector field} if $L_X \omega = 0$.
\end{definition}
\begin{remark}
  Since $L_{\lie{X}{Y}} = \lie{L_X}{L_Y} = L_X \circ L_Y - L_Y \circ L_X$,
  the symplectic vector fields form a Lie subalgebra of the Lie algebra of
  vector fields.
\end{remark}
\begin{proposition}
  Let $(M, \omega)$ be a symplectic manifold, $f \in \mathcal{F} M$.
  Then $\hamiltonian_f$ is a symplectic vector field.
\end{proposition}
\begin{proof}
  $
    L_{\hamiltonian_f} \omega
    = i_{\hamiltonian_f}(d \omega) + d(i_{\hamiltonian_f} \omega)
    = i_{\hamiltonian_f} 0 - d(d f)
    = 0.
  $
\end{proof}
\begin{proposition}
  Let
    $(M, \omega)$ be a symplectic manifold,
    $X \in \mathfrak{X} M$ be a symplectic vector fields.
  Then
  \begin{equation}
    d(i_X \omega) = 0.
  \end{equation}
\end{proposition}
\begin{proof}
  $
    d(i_X \omega)
    = L_X \omega - i_X(d \omega)
    = 0 - 0
    = 0.
  $
\end{proof}
\begin{proposition}
  Let $M$ be a smooth manifold, $X, Y \in \mathfrak{X} M$.
  Then
  \begin{equation}
    L_X \circ i_Y = i_{\lie{X}{Y}} + i_Y \circ L_X.
  \end{equation}
\end{proposition}
\begin{proposition}
  Let
    $(M, \omega)$ be a symplectic manifold,
    $X, Y \in \mathfrak{X} M$ be symplectic vector fields.
  Then
  \begin{equation}
    \lie{X}{Y} = \hamiltonian_{i_Y(i_X \omega)}.
  \end{equation}
\end{proposition}
\begin{proof}
  \begin{equation}
    i_{\lie{X}{Y}} \omega
    = (L_X \circ i_Y - i_Y \circ L_X) \omega
    = (L_X \circ i_Y) \omega
    = ((d \circ i_X + i_X \circ d) \circ i_Y) \omega
    = d(i_X(i_Y \omega))
    = - d(i_Y(i_X \omega)).
  \end{equation}
  We get the desired result from the definition of $\hamiltonian$.
\end{proof}
\begin{definition}
  Let $(M, \omega)$ be a symplectic manifold.
  Define the \textbf{Poisson bracket}
  $\poisson{\cdot}{\cdot} \colon \mathcal{F} M \to \mathcal{F} M$ by
  \begin{equation}
    \poisson{f}{g}
    := i_{\hamiltonian_g}(i_{\hamiltonian_f} \omega),\
    f, g \in \mathcal{F} M.
  \end{equation}
\end{definition}
\begin{corollary}
  Let $(M, \omega)$ be a symplectic manifold, $f, g \in \mathcal{F} M$.
  Then
  \begin{equation}
    \lie{\hamiltonian_f}{\hamiltonian_g}
    = \hamiltonian_{i_{\hamiltonian_g}(i_{\hamiltonian_f} \omega)}
    = \poisson{f}{g}.
  \end{equation}
\end{corollary}
\begin{proposition}[Leibniz rule holds for the Poisson bracket]
  Let $(M, \omega)$ be a symplectic manifold, $f, g, h \in \mathcal{F} M$.
  Then
  \begin{equation}
    \poisson{f}{g h} = \poisson{f}{g} h + g \poisson{f}{h}.
  \end{equation}
\end{proposition}
\begin{proof}
  $
    \poisson{f}{g h}
    = i_{\hamiltonian_{g h}}(i_{\hamiltonian_f} \omega)
    = i_{h \hamiltonian_g + g \hamiltonian_{h}}(i_{\hamiltonian_f} \omega)
    = (i_{\hamiltonian_g}(i_{\hamiltonian_f} \omega))\, h
      + g\, (i_{\hamiltonian_h}(i_{\hamiltonian_f} \omega)) 
    = \poisson{f}{g} h + g \poisson{f}{h}.
  $
\end{proof}
\begin{proposition}
  Let $(M, \omega)$ be a symplectic manifold, $f, g, h \in \mathcal{F} M$.
  Then
  \begin{equation}
    \poisson{f}{g} = L_{\hamiltonian_f} g.
  \end{equation}
\end{proposition}
\begin{proof}
  $
    \poisson{f}{g}
    = i_{\hamiltonian_g}(i_{\hamiltonian_f} \omega)
    = - i_{\hamiltonian_f} \circ i_{\hamiltonian_g} \omega
    = i_{\hamiltonian_f}(d g)
    = L_{\hamiltonian_f} g.
  $
\end{proof}
\begin{definition}
  Let $(M, \omega)$ be a symplectic manifold.
  Define
  ${\rm ad} \colon \mathcal{F} M \to (\mathcal{F} M \to \mathcal{F} M)$ by
  \begin{equation}
    {\rm ad}_f g := \poisson{f}{g},\ f, g \in \mathcal{F} M,
  \end{equation}
\end{definition}
\begin{proposition}
  Let $(M, \omega)$ be a symplectic manifold.
  Then
  \begin{equation}
    \lie{{\rm ad}_f}{{\rm ad}_g} = {\rm ad}_{\poisson{f}{g}}.
  \end{equation}
  (Here the bracket $\lie{\cdot}{\cdot}$ is the commutator of operators.)
\end{proposition}
\begin{proof}
  From the previous proposition it follows that
  \begin{equation}
    {\rm ad}_f = L_{X_f},\ f \in \mathcal{F} M.
  \end{equation}
  Hence,
  \begin{equation}
    \lie{{\rm ad}_f}{{\rm ad}_g}
    = \lie{L_{\hamiltonian_f}}{L_{\hamiltonian_g}}
    = L_{\lie{\hamiltonian_f}{\hamiltonian_g}}
    = L_{\hamiltonian_{\poisson{f}{g}}}
    = {\rm ad}_{\poisson{f}{g}}.
  \end{equation}
\end{proof}
\begin{corollary}
  Let $(M, \omega)$ be a symplectic manifold.
  Then $(\mathcal{F} M, \poisson{\cdot}{\cdot})$ is a Lie algebra over $\R$.
\end{corollary}
\begin{proof}
  Bilinearity and antisymmetry are trivial to check.
  The Jacobi identity is equivalent to the adjoint map being a Lie algebra
  homomorphism, which was the previous proposition.
\end{proof}
\begin{definition}
  Let
    $R$ be a commutative ring with unity ring,
    $(A, +, \cdot)$ be an $R$-module
    with additional structures of
    an associative algebra $(A, *)$ and
    a Lie algebra $(A, \poisson{\cdot}{\cdot})$.
  We say that $A$ is a Poisson algebra if the Lie bracket acts as a derivation,
  i.e., for all $f, g, h \in A$,
  \begin{equation}
    \poisson{f}{g * h} = \poisson{f}{g} * h + g * \poisson{f}{h}.
  \end{equation}
\end{definition}
\begin{corollary}
  Let $(M, \omega)$ be a symplectic manifold.
  Then $\mathcal{F} M$ is a Poisson algebra over $\R$.
  Here, addition, scalar multiplication, and multiplication are given by the
  corresponding pointwise operations, while the Lie bracket is given by the
  Poisson bracket.
\end{corollary}


\section{Chain and cochain spaces on quasi-cubical meshes}
\label{section:chain_and_cochain_spaces_on_quasi-cubical_meshes}
\begin{definition}
  Let
    $R$ be a ring,
    $V$ be a finite-dimensional $R$-module,
    $\omega \in \Lambda^2 V^*$.
  We say that $\omega$ is \textbf{non-degenerate} or \textbf{symplectic}
  if the associated map
  \begin{equation}
    \tilde{\omega} \colon V \to V^*,\
    X \in V \mapsto \tilde{\omega}(X) := i_X \omega \in V^*,
  \end{equation}
  is an isomorphism.

  The pair $(V, \omega)$ is called a \textbf{symplectic module}
  (or \textbf{symplectic vector space} if $R$ is a field).
\end{definition}
\begin{proposition}
  Let
    $R$ be a ring without,
    $(V, \omega)$ be a finite-dimensional symplectic module over $R$.
  Assume that for any $x \in R,\ x + x = 0 \Rightarrow x = 0$.
  Then $\dim V$ is an even number.
\end{proposition}
\begin{proof}
  Let $n = \dim V$.
  In a basis of $V$ $\omega$ is represented by an antisymmetric matrix $A$.
  But then
  \begin{equation}
    \det A = \det(A^T) = \det(-A) = (-1)^n \det A.
  \end{equation}
  If $n$ is odd, then $\det A + \det A = 0$.
  By assumption this means that $\det A = 0$
  which contradicts the nondegeneracy of $\omega$.
  Hence, $n$ is even.
\end{proof}
\begin{definition}
  Let $M$ be a smooth manifold, $\omega \in \Omega^\bullet M$.
  We say that:
  \begin{enumerate}
    \item
      $\omega$ is \textbf{closed} if $d \omega = 0$
    \item
      $\omega$ is \textbf{exact} if there exists $\eta \in \Omega^\bullet M$
      such that $d \eta = \omega$.
  \end{enumerate}
\end{definition}
\begin{proposition}
  Let $M$ be a smooth manifold, $\omega \in \Omega^\bullet M$.
  If $\omega$ is exact, then it is closed.
\end{proposition}
\begin{proof}
  Let $\eta \in \Omega^\bullet M$ be such that $d \eta = \omega$.
  Then $d \omega = d (d \eta) = 0$, i.e., $\omega$ is closed.
\end{proof}
\begin{definition}
  Let $M$ be a smooth manifold, $\omega \in \Omega^2 M$.
  We say that $\omega$ is a \textbf{symplectic form}
  if it is non-degenerate
  (with base module $\mathfrak{X} M$ over $\mathcal{F} M$) and closed.

  The pair $(M, \omega)$ is called a \textbf{symplectic manifold}.
\end{definition}
\begin{proposition}
  Let $(M, \omega)$ be a symplectic manifold.
  Then $M$ is even-dimensional.
\end{proposition}
\begin{definition}
  Let $Q$ be a smooth manifold.
  Consider the cotangent bundle $T^* Q$ with bundle projection
  $\pi \colon T^* Q \to Q$
  with differential $d \pi \colon T(T^* Q) \to T Q$.
  Define the \textbf{tautological one-form}
  $\theta \colon T^* Q \to T^* (T^* Q)$ as follows:
  for any $(q, p) \in T^*Q$ (i.e,. $q \in Q$, $p \in \Hom(T_q Q, \R)$),
  \begin{equation}
    \restrict{\theta}{(q, p)}
    := p \circ \restrict{d \pi}{(q, p)} \in T^*_{(q, p)}(T^* Q).
  \end{equation}
  In other words, if we denote $M := T^* Q$, then $\theta$ is a section of its
  cotangent bundle $T^* M$, i.e., an $1$-form on $M$. 
\end{definition}
\begin{discussion}
  Let $Q$ be a smooth manifold, $\pi \colon T^* Q \to Q$ be the projection.
  Then a $1$-form on $Q$ is a section of $\colon T^* Q$, i.e., a smooth map
  $\mu \colon Q \to \colon T^* Q$ such that $\pi \circ \mu = \id_Q$.
  As such it has a pullback
  $\mu^* \colon \Omega^\bullet(T^* Q) \to \Omega^\bullet Q$.
\end{discussion}
\begin{proposition}
  Let
    $Q$ be a smooth manifold,
    $\theta$ be the tautological one-form on $T^* Q$,
    $\mu \in \Omega^1 Q$.
  Then
  \begin{equation}
    \mu^* \theta = \mu.
  \end{equation}
\end{proposition}
\begin{proof}
  Let $q \in Q$.
  Then
  \begin{equation}
    \restrict{\mu^* \theta}{q}
    = \restrict{\theta}{\mu q} \circ \restrict{d \mu}{q}
    = \restrict{\mu}{q} \circ \restrict{(d \pi)}{\mu q}
      \circ \restrict{d \mu}{q}
    = \restrict{\mu}{q} \circ \restrict{d(\pi \circ \mu)}{q}
    = \restrict{\mu}{q}.
  \end{equation}
  Since $q$ is arbitrary, $\mu^* \theta = \mu$.
\end{proof}
\begin{definition}
  Let
    $Q$ be a smooth manifold,
    $\theta \in \Omega^1(T^* Q)$ be the tautological one-form.
  Define $\omega := - d \theta$.
  The pair $(T^* Q, \omega)$ is called the \textbf{phase space} of $Q$.
  (In this setting $Q$ is usually called the \textbf{configuration space}.)
\end{definition}
\begin{remark}
  Let $Q$ be a smooth manifold.
  The elements of $T^* Q$ are of the form $(q, p)$ where $q \in Q$ and
  $p \in T^*_q Q = \Hom(T_q Q, \R)$.
  $q$ is called \textbf{generalised position}, while $p$ is called
  \textbf{generalised momentum}.
\end{remark}
\begin{proposition}
  Let
    $Q$ be a smooth manifold,
    $(T^* Q, \omega)$ be its phase space.
  Then $(T^* Q, \omega)$ is a symplectic manifold.
\end{proposition}
\begin{definition}
  Let $Q$ be a smooth manifold of dimension $n$.
  Consider a point $q_0 \in Q$ and let $(U, \hat{\varphi})$ be a chart around
  $q_0$, i.e., $U$ is a neighbourhood of $q_0$ and
  $\hat{\varphi} \colon U \to \R^n$ is a diffeomorphism.
  Let $\{\hat{q}^i \colon U \to \R\}_{i = 1}^n$ be the corresponding local
  coordinates, i.e., if $\{\pi^i \colon \R^n \to \R\}_{i = 1}^n$ are the
  projection maps, then $\{\hat{q}^i = \pi^i \circ \hat{\varphi}\}_{i = 1}^n$.
  Let $i \in \{1, ..., n\}$.
  Define \textbf{position coordinate} $q^i \colon T^* U \to \R$ by
  \begin{equation}
    q^i := \hat{q}^i \circ \restrict{\pi}{U}.
  \end{equation}
  Also, define \textbf{momentum coordinate} $p_i \colon T^* U \to \R$
  as follows: for any $(q, p) \in T^* U$,
  \begin{equation}
    p_i(q, p)
    := p\left(\restrict{\frac{\partial}{\partial \hat{q}^i}}{q}\right).
  \end{equation}
\end{definition}
\begin{proposition}
  Let
    $Q$ be a smooth manifold of dimension $n$,
    $q_0 \in Q$,
    $(U, \hat{\varphi})$ be a chart around $q_0$,
    $\{\hat{q}^i \colon U \to \R\}_{i = 1}^n$ be the corresponding local
      coordinates,
    $\{q^i \colon T^* U \to \R\}_{i = 1}^n$ be the corresponding position
      coordinates,
    $\{p_i \colon T^* U \to \R\}_{i = 1}^n$ be the corresponding momentum
      coordinates.
  Then the map $\varphi \colon T^* U \to \R^{2 n}$ defined by
  \begin{equation}
    \varphi(q, p) = (q^1(q, p), ..., q^n(q, p), p_1(q, p), ..., p_n(q, p))
  \end{equation}
  is a diffeomorphism, i.e., $(T^* U, \varphi)$ is a chart around $(q_0, 0)$.
  (The covector in $T^*_{q_0}$ does not matter, so we make the trivial choice by
  taking zero.)

  These local coordinates are called \textbf{generalised coordinates}.
\end{proposition}
\begin{remark}
  From now on, given a manifold $Q$ and a chart $(U, \hat{\varphi})$, unless
  stated otherwise, we will fix the notation and use the objects defined above:
  the projection map $\pi \colon T^* Q \to Q$, the tautological one-form
  $\theta$ and the canonical symplectic form $\omega = - d \theta$;
  for $i = 1, ..., n$ the coordinate maps $\hat{q}^i$, $q^i$, and $p_i$;
  the chart $(T^* U, \varphi)$.
\end{remark}
\begin{proposition}
  Let
    $Q$ be a smooth manifold,
    $\xi \in \Omega^1(T^* Q)$ has the following property:
    for any $1$-form $\mu$ on $Q$, $\mu^* \xi= 0$.
  Then $\xi = 0$.
\end{proposition}
\begin{proof}
  Let
    $n := \dim Q$, $(U, \hat{\varphi})$ be a chart on $Q$ and
    $\{f_i, g^i \in \mathcal{F}(T^* U)\}_{i = 1}^n$ be such that
  \begin{equation}
    \restrict{\xi}{U} = \sum_{i = 1}^n f_i\, d q^i + \sum_{i = 1}^n g^i\, d p_i.
  \end{equation}
  Take arbitrary $\{h_j \in \mathcal{F} U\}_{j = 1}^n$ so that
  \begin{equation}
    \restrict{\mu}{U} = \sum_{j = 1}^n h_j\, d \hat{q}^j.
  \end{equation}
  Note that
  $q^i \circ \restrict{\mu}{U} = \hat{q}^i$ and
  $p_i \circ \restrict{\mu}{U} = h_i$.
  Hence,
  \begin{equation}
    0 
    = \restrict{(\mu^* \xi)}{U}
    = \sum_{i = 1}^n (f_i \circ \restrict{\mu}{U})\,
      d(q^i \circ \restrict{\mu}{U})
    + \sum_{i = 1}^n (g^i \circ \restrict{\mu}{U})\,
      d(p_i \circ \restrict{\mu}{U})
    = \sum_{i = 1}^n (f_i \circ \restrict{\mu}{U})\, d \hat{q}^i
    + \sum_{i = 1}^n (g^i \circ \restrict{\mu}{U})\, d h_i.
  \end{equation}
  Fix $q_0 \in U$, $p_0 \in T^*_{q_0} Q$ so that $(p_0, q_0) \in T^* U$.
  Denote
  \begin{equation}
    c_i
    :=
    p_0\left(\restrict{\frac{\partial}{\partial \hat{q}^i}}{q_0}\right),
    i = 1, ..., n,
  \end{equation}
  so that
  \begin{equation}
    p_0 = \sum_{i = 1}^n c_i \restrict{d \hat{q}^i}{q_0}.
  \end{equation}
  \begin{enumerate}
    \item
      We will first prove that
      for any $i \in \{1, ..., n\}$, $f_i(q_0, p_0) = 0$.
      Define the constant functions
      \begin{equation}
        h_i(q) := c_i,\ i \in \{1, ..., n\},\ q \in U.
      \end{equation}
      Then for any $i \in \{1, ..., n\}$, $d h_i = 0$.
      Hence,
      \begin{equation}
        \begin{split}
          0
          & = \restrict{(\mu^* \xi)}{q_0} \\
          & = \sum_{i = 1}^n
              f_i(q_0, \sum_{j = 1}^n h_j(q_0) \restrict{d \hat{q}^j}{q_0})\,
              \restrict{d \hat{q}^i}{q_0} \\
          & = \sum_{i = 1}^n
              f_i(q_0, \sum_{j = 1}^n c_j \restrict{d \hat{q}^j}{q_0})\,
              \restrict{d \hat{q}^i}{q_0} \\
          & = \sum_{i = 1}^n f_i(q_0, p_0)\, \restrict{d \hat{q}^i}{q_0}.
        \end{split}
      \end{equation}
      Therefore, for any $i \in \{1, ..., n\}$, $f_i(q_0, p_0) = 0$.
    \item
      We will now prove that
      for any $i \in \{1, ..., n\}$, $g^i(q_0, p_0) = 0$.
      Define the linear functions
      \begin{equation}
        h_i(q) := c_i + \hat{q}^i(q) - \hat{q}^i(q_0).
      \end{equation}
      Then for any $i \in \{1, ..., n\}$,
      $d h_i = d \hat{q}^i$ and $h_i(q) = c_i$.
      Hence,
      \begin{equation}
        \begin{split}
          0
          & = \restrict{(\mu^* \xi)}{q_0} \\
          & = \sum_{i = 1}^n
              g^i(q_0, \sum_{j = 1}^n h_j(q_0) \restrict{d \hat{q}^j}{q_0})\,
              \restrict{d h_i}{q_0} \\
          & = \sum_{i = 1}^n
              g^i(q_0, \sum_{j = 1}^n c_j \restrict{d \hat{q}^j}{q_0})\,
              \restrict{d \hat{q}^i}{q_0} \\
          & = \sum_{i = 1}^n g^i(q_0, p_0)\, \restrict{d \hat{q}^i}{q_0}.
        \end{split}
      \end{equation}
      Therefore, for any $i \in \{1, ..., n\}$, $g^i(q_0, p_0) = 0$.
  \end{enumerate}
  Since $(q_0, p_0) \in T^* U$ was arbitrary, we conclude that
  for any $i \in \{1, ..., n\}$, $f_i = g^i = 0$.
  Hence, $\restrict{\xi}{U} = 0$.
  Taking an atlas $\{(U_\alpha, \hat{\varphi}_\alpha)\}_{\alpha \in A}$ of $Q$
  (for some index set $A$), we conclude that $\xi = 0$.
\end{proof}
\begin{corollary}
  Let
    $Q$ be a smooth manifold,
    $\theta$ be the tautological one-form on $T^* Q$,
    $\eta \in \Omega^1(T^* Q)$ has the following property:
    for any $1$-form $\mu$ on $Q$, $\mu^* \eta = \mu$.
  Then $\eta = \theta$.
\end{corollary}
\begin{proof}
  Write $\eta = \theta + \xi$, i.e., $\xi := \eta - \theta$.
  Then, for any $\mu \in \Omega^1 Q$,
  \begin{equation}
    \mu
    = \mu^* \eta
    = \mu^* \theta + \mu^* \xi
    = \mu + \mu^* \xi
    \Rightarrow \mu^* \xi = 0.
  \end{equation}
  But from the previous proposition it follows that $\xi = 0$,
  and hence $\eta = \theta$.
\end{proof}
\begin{proposition}
  Let
    $Q$ be a smooth manifold of dimension $n$,
    $(U, \hat{\varphi})$ be a chart,
    $(q, p) \in T^* U$,
    $ i \in \{1, ..., n\}$.
  Then
  \begin{equation}
    \restrict{d \pi}{(q, p)}
    \left(\restrict{\frac{\partial}{\partial q^i}}{(q, p)}\right)
    = \restrict{\frac{\partial}{\partial \hat{q}^i}}{q}
  \end{equation}
  and
  \begin{equation}
    \restrict{d \pi}{(q, p)}
    \left(\restrict{\frac{\partial}{\partial p^i}}{(q, p)}\right)
    = 0.
  \end{equation}
\end{proposition}
\begin{proof}
  Let $f \colon Q \to \R$ be smooth.
  Define the functions
  $\hat{g} := f \circ \hat{\varphi}^{-1} \colon \R^n \to \R$ and
  $g := f \circ \pi \circ \varphi^{-1} \colon \R^{2 n} \to \R$.
  Let $(X^1, ..., X^n, Y^1, ..., Y^n) := \varphi(p, q) \in \R^n$.
  This means that $(X^1, ..., X^n) = \hat{\varphi}(q)$.
  Then
  \begin{equation}
    g(X^1, ..., X^n, Y^1, ..., Y^n)
    = f(\pi(q, p))
    = f(q)
    = \hat{g}(X^1, ..., X^n).
  \end{equation}
  Hence,
  \begin{equation}
    \begin{split}
      \frac{\partial g}{\partial x^i}(X^1, ..., X^n, Y^1, ..., Y^n)
      & = \lim_{h \to 0}
        \frac
        {g(X^1, ..., X^i + h, ..., X^n, Y^1, ..., Y^n)
         - g(X^1, ..., X^n, Y^1, ..., Y^n)}
        {h} \\
      & = \lim_{h \to 0}
        \frac{\hat{g}(X^1, ..., X^i + h, ..., X^n) - \hat{g}(X^1, ..., X^n)}{h}
        \\
      & = \frac{\partial \hat{g}}{\partial \hat{x}^i}(X^1, ..., X^n).
    \end{split}
  \end{equation}
  Similarly, since $g$ is constant with respect to the last $n$ coordinates,
  \begin{equation}
    \frac{\partial g}{\partial x^{n + i}}(X^1, ..., X^n, Y^1, ..., Y^n) = 0
  \end{equation}
  Take the standard coordinate systems (given by projections)
  $\{\hat{x}^k\}_{k = 1}^n$ on $\R^n$ and
  $\{x^k\}_{k = 1}^{2 n}$ on $\R^{2 n}$.
  Then, by the definitions of differential and partial derivative on manifold,
  \begin{equation}
    (\restrict{d \pi}{(q, p)}
      \left(\restrict{\frac{\partial}{\partial q^i}}{(q, p)}\right)) f
    = \restrict{\frac{\partial}{\partial q^i}}{(q, p)}(f \circ \pi)
    = \frac{\partial(f \circ \pi \circ \varphi^{-1})}{x^i}(\varphi(q, p))
    = \frac{\partial(f \circ \hat{\varphi}^{-1})}{\hat{x}^i}(\hat{\varphi}(q))
    = \restrict{\frac{\partial}{\partial \hat{q}^i}}{q} f,
  \end{equation}
  from which it follows that the first equality holds.
  Similarly,
  \begin{equation}
    (\restrict{d \pi}{(q, p)}
      \left(\restrict{\frac{\partial}{\partial p^i}}{(q, p)}\right)) f
    = \restrict{\frac{\partial}{\partial p^i}}{(q, p)}(f \circ \pi)
    = \frac{\partial(f \circ \pi \circ \varphi^{-1})}{x^{i + n}}(\varphi(q, p))
    = 0,
  \end{equation}
  from which it follows that the second equality holds.
\end{proof}
\begin{proposition}[Tautological one-form in generalised coordinates]
  Let
    $Q$ be a smooth manifold of dimension $n$,
    $(U, \hat{\varphi})$ be a chart.
  Then
  \begin{equation}
    \restrict{\theta}{U} = \sum_{i = 1}^n p_i\, d q^i.
  \end{equation}
\end{proposition}
\begin{proof}
  Let $(q, p) \in T^* U$.
  Recall that $\restrict{\theta}{(q, p)} = p \circ \restrict{d \pi}{(q, p)}$.
  Hence,
  \begin{equation}
    \restrict{\theta}{(q, p)}
    \left(\restrict{\frac{\partial}{\partial q^i}}{(q, p)}\right)
    = p\left(\restrict{\frac{\partial}{\partial \hat{q}^i}}{q}\right)
    = p_i(q, p),
  \end{equation}
  and
  \begin{equation}
    \restrict{\theta}{(q, p)}
    \left(\restrict{\frac{\partial}{\partial p^i}}{(q, p)}\right)
    = p(0)
    = 0.
  \end{equation}
  Therefore,
  \begin{equation}
    \restrict{\theta}{(q, p)}
    = \sum_{i = 1}^n
      \restrict{\theta}{(q, p)}
      \left(\restrict{\frac{\partial}{\partial q^i}}{(q, p)}\right)\,
      \restrict{d q^i}{(q, p)}
    + \sum_{i = 1}^n
      \restrict{\theta}{(q, p)}
      \left(\restrict{\frac{\partial}{\partial p^i}}{(q, p)}\right)\,
      \restrict{d p^i}{(q, p)}
    = \sum_{i = 1}^n p_i(q, p)\, \restrict{d q^i}{(q, p)},
  \end{equation}
  from which the proposition follows.
\end{proof}
\begin{corollary}[Canonical symplectic in generalised coordinates]
  Let
    $Q$ be a smooth manifold of dimension $n$,
    $(U, \hat{\varphi})$ be a chart.
  Then
  \begin{equation}
    \restrict{\omega}{U} = \sum_{i = 1}^n d q^i \wedge d p_i.
  \end{equation}
\end{corollary}
\begin{proof}
  Let $i \in \{1, ..., n\}$.
  Then
  \begin{equation}
    - d(p_i\, d q^i) = - d p_i \wedge d q^i = d q^i \wedge d p_i.
  \end{equation}
  Summing up for all $i$, we get the desired result.
\end{proof}
\begin{definition}
  Let $(M, \omega)$ be a symplectic manifold, $f \in \mathcal{F} M$.
  We say that $X \in \mathfrak{X} M$ is a \textbf{Hamiltonian vector field} for
  $f$ if
  \begin{equation}
    i_X \omega + d_0 f = 0.
  \end{equation}
\end{definition}
\begin{proposition}
  Let $(M, \omega)$ be a symplectic manifold, $f \in \mathcal{F} M$.
  Then there exists a unique Hamiltonian vector field for $f$.
\end{proposition}
\begin{proof}
  The non-degeneracy of $\omega$ means that we can interpret the symplectic form
  as the isomorphism $\tilde{\omega} \colon \mathfrak{X} M \to \Omega^1 M$,
  given by
  \begin{equation}
    (\tilde{\omega} X) := i_X \omega,\ X \in \mathfrak{X} M.
  \end{equation}
  Hence, the problem at hand has a unique solution
  $X = \tilde{\omega}^{-1}(- d_0 f)$.
\end{proof}
\begin{definition}
  Let $(M, \omega)$ be a symplectic manifold.
  Define the map $\hamiltonian \colon \mathcal{F} M \to \mathfrak{X} M$ by
  \begin{equation}
    \hamiltonian = - \tilde{\omega}^{-1} \circ d_0.
  \end{equation}
  It maps a function to its corresponding Hamiltonian vector field.
  We will write $\hamiltonian_f$ instead of $\hamiltonian(f)$
  for $f \in \mathcal{F} M$.
\end{definition}
\begin{proposition}
  Let
    $Q$ be a smooth manifold of dimension $n$,
    $(U, \hat{\varphi})$ be a chart on $Q$,
    $f \in \mathcal{F}(T^* Q)$.
  Then
  \begin{equation}
    \hamiltonian_f
    = \sum_{i = 1}^n
    \left(
      - \frac{\partial f}{\partial p^i} \frac{\partial}{\partial q^i}
      + \frac{\partial f}{\partial q^i} \frac{\partial}{\partial p^i}
    \right).
  \end{equation}
\end{proposition}
\begin{proof}
  First, note that
  $i_{\frac{\partial}{\partial q^i}} \omega = d p^i$ and
  $i_{\frac{\partial}{\partial p^i}} \omega = - d q^i$.
  Denote
  \begin{equation}
    X
    := \sum_{i = 1}^n
    \left(
      - \frac{\partial f}{\partial p^i} \frac{\partial}{\partial q^i}
      + \frac{\partial f}{\partial q^i} \frac{\partial}{\partial p^i}
    \right).
  \end{equation}
  Then
  \begin{equation}
    i_X \omega
    = \sum_{i}^n
    \left(
      - \frac{\partial f}{\partial p^i} i_{\frac{\partial}{\partial q^i}} \omega
      + \frac{\partial f}{\partial q^i} i_{\frac{\partial}{\partial p^i}} \omega
    \right)
    = \sum_{i}^n
    \left(
      - \frac{\partial f}{\partial p^i} d p^i
      - \frac{\partial f}{\partial q^i} d q^i
    \right)
    = - d f.
  \end{equation}
  Hence, $\hamiltonian_f = X$.
\end{proof}
\begin{proposition}
  Let $(M, \omega)$ be a symplectic manifold, $f, g \in \mathcal{F} M$.
  Then
  \begin{equation}
    \hamiltonian_{f g} = f \hamiltonian_g + g \hamiltonian_f.
  \end{equation}
\end{proposition}
\begin{proof}
  Follows directly from the Leibniz rule for $d_0$.
\end{proof}
\begin{definition}
  Let $(M, \omega)$ be a symplectic manifold, $X \in \mathfrak{X} M$.
  We say that $X$ is a \textbf{symplectic vector field} if $L_X \omega = 0$.
\end{definition}
\begin{remark}
  Since $L_{\lie{X}{Y}} = \lie{L_X}{L_Y} = L_X \circ L_Y - L_Y \circ L_X$,
  the symplectic vector fields form a Lie subalgebra of the Lie algebra of
  vector fields.
\end{remark}
\begin{proposition}
  Let $(M, \omega)$ be a symplectic manifold, $f \in \mathcal{F} M$.
  Then $\hamiltonian_f$ is a symplectic vector field.
\end{proposition}
\begin{proof}
  $
    L_{\hamiltonian_f} \omega
    = i_{\hamiltonian_f}(d \omega) + d(i_{\hamiltonian_f} \omega)
    = i_{\hamiltonian_f} 0 - d(d f)
    = 0.
  $
\end{proof}
\begin{proposition}
  Let
    $(M, \omega)$ be a symplectic manifold,
    $X \in \mathfrak{X} M$ be a symplectic vector fields.
  Then
  \begin{equation}
    d(i_X \omega) = 0.
  \end{equation}
\end{proposition}
\begin{proof}
  $
    d(i_X \omega)
    = L_X \omega - i_X(d \omega)
    = 0 - 0
    = 0.
  $
\end{proof}
\begin{proposition}
  Let $M$ be a smooth manifold, $X, Y \in \mathfrak{X} M$.
  Then
  \begin{equation}
    L_X \circ i_Y = i_{\lie{X}{Y}} + i_Y \circ L_X.
  \end{equation}
\end{proposition}
\begin{proposition}
  Let
    $(M, \omega)$ be a symplectic manifold,
    $X, Y \in \mathfrak{X} M$ be symplectic vector fields.
  Then
  \begin{equation}
    \lie{X}{Y} = \hamiltonian_{i_Y(i_X \omega)}.
  \end{equation}
\end{proposition}
\begin{proof}
  \begin{equation}
    i_{\lie{X}{Y}} \omega
    = (L_X \circ i_Y - i_Y \circ L_X) \omega
    = (L_X \circ i_Y) \omega
    = ((d \circ i_X + i_X \circ d) \circ i_Y) \omega
    = d(i_X(i_Y \omega))
    = - d(i_Y(i_X \omega)).
  \end{equation}
  We get the desired result from the definition of $\hamiltonian$.
\end{proof}
\begin{definition}
  Let $(M, \omega)$ be a symplectic manifold.
  Define the \textbf{Poisson bracket}
  $\poisson{\cdot}{\cdot} \colon \mathcal{F} M \to \mathcal{F} M$ by
  \begin{equation}
    \poisson{f}{g}
    := i_{\hamiltonian_g}(i_{\hamiltonian_f} \omega),\
    f, g \in \mathcal{F} M.
  \end{equation}
\end{definition}
\begin{corollary}
  Let $(M, \omega)$ be a symplectic manifold, $f, g \in \mathcal{F} M$.
  Then
  \begin{equation}
    \lie{\hamiltonian_f}{\hamiltonian_g}
    = \hamiltonian_{i_{\hamiltonian_g}(i_{\hamiltonian_f} \omega)}
    = \poisson{f}{g}.
  \end{equation}
\end{corollary}
\begin{proposition}[Leibniz rule holds for the Poisson bracket]
  Let $(M, \omega)$ be a symplectic manifold, $f, g, h \in \mathcal{F} M$.
  Then
  \begin{equation}
    \poisson{f}{g h} = \poisson{f}{g} h + g \poisson{f}{h}.
  \end{equation}
\end{proposition}
\begin{proof}
  $
    \poisson{f}{g h}
    = i_{\hamiltonian_{g h}}(i_{\hamiltonian_f} \omega)
    = i_{h \hamiltonian_g + g \hamiltonian_{h}}(i_{\hamiltonian_f} \omega)
    = (i_{\hamiltonian_g}(i_{\hamiltonian_f} \omega))\, h
      + g\, (i_{\hamiltonian_h}(i_{\hamiltonian_f} \omega)) 
    = \poisson{f}{g} h + g \poisson{f}{h}.
  $
\end{proof}
\begin{proposition}
  Let $(M, \omega)$ be a symplectic manifold, $f, g, h \in \mathcal{F} M$.
  Then
  \begin{equation}
    \poisson{f}{g} = L_{\hamiltonian_f} g.
  \end{equation}
\end{proposition}
\begin{proof}
  $
    \poisson{f}{g}
    = i_{\hamiltonian_g}(i_{\hamiltonian_f} \omega)
    = - i_{\hamiltonian_f} \circ i_{\hamiltonian_g} \omega
    = i_{\hamiltonian_f}(d g)
    = L_{\hamiltonian_f} g.
  $
\end{proof}
\begin{definition}
  Let $(M, \omega)$ be a symplectic manifold.
  Define
  ${\rm ad} \colon \mathcal{F} M \to (\mathcal{F} M \to \mathcal{F} M)$ by
  \begin{equation}
    {\rm ad}_f g := \poisson{f}{g},\ f, g \in \mathcal{F} M,
  \end{equation}
\end{definition}
\begin{proposition}
  Let $(M, \omega)$ be a symplectic manifold.
  Then
  \begin{equation}
    \lie{{\rm ad}_f}{{\rm ad}_g} = {\rm ad}_{\poisson{f}{g}}.
  \end{equation}
  (Here the bracket $\lie{\cdot}{\cdot}$ is the commutator of operators.)
\end{proposition}
\begin{proof}
  From the previous proposition it follows that
  \begin{equation}
    {\rm ad}_f = L_{X_f},\ f \in \mathcal{F} M.
  \end{equation}
  Hence,
  \begin{equation}
    \lie{{\rm ad}_f}{{\rm ad}_g}
    = \lie{L_{\hamiltonian_f}}{L_{\hamiltonian_g}}
    = L_{\lie{\hamiltonian_f}{\hamiltonian_g}}
    = L_{\hamiltonian_{\poisson{f}{g}}}
    = {\rm ad}_{\poisson{f}{g}}.
  \end{equation}
\end{proof}
\begin{corollary}
  Let $(M, \omega)$ be a symplectic manifold.
  Then $(\mathcal{F} M, \poisson{\cdot}{\cdot})$ is a Lie algebra over $\R$.
\end{corollary}
\begin{proof}
  Bilinearity and antisymmetry are trivial to check.
  The Jacobi identity is equivalent to the adjoint map being a Lie algebra
  homomorphism, which was the previous proposition.
\end{proof}
\begin{definition}
  Let
    $R$ be a commutative ring with unity ring,
    $(A, +, \cdot)$ be an $R$-module
    with additional structures of
    an associative algebra $(A, *)$ and
    a Lie algebra $(A, \poisson{\cdot}{\cdot})$.
  We say that $A$ is a Poisson algebra if the Lie bracket acts as a derivation,
  i.e., for all $f, g, h \in A$,
  \begin{equation}
    \poisson{f}{g * h} = \poisson{f}{g} * h + g * \poisson{f}{h}.
  \end{equation}
\end{definition}
\begin{corollary}
  Let $(M, \omega)$ be a symplectic manifold.
  Then $\mathcal{F} M$ is a Poisson algebra over $\R$.
  Here, addition, scalar multiplication, and multiplication are given by the
  corresponding pointwise operations, while the Lie bracket is given by the
  Poisson bracket.
\end{corollary}


\section{Discrete bundle-valued differential forms}
\label{section:discrete_bundle-valued_differential_forms}
\begin{definition}
  Let
    $R$ be a ring,
    $V$ be a finite-dimensional $R$-module,
    $\omega \in \Lambda^2 V^*$.
  We say that $\omega$ is \textbf{non-degenerate} or \textbf{symplectic}
  if the associated map
  \begin{equation}
    \tilde{\omega} \colon V \to V^*,\
    X \in V \mapsto \tilde{\omega}(X) := i_X \omega \in V^*,
  \end{equation}
  is an isomorphism.

  The pair $(V, \omega)$ is called a \textbf{symplectic module}
  (or \textbf{symplectic vector space} if $R$ is a field).
\end{definition}
\begin{proposition}
  Let
    $R$ be a ring without,
    $(V, \omega)$ be a finite-dimensional symplectic module over $R$.
  Assume that for any $x \in R,\ x + x = 0 \Rightarrow x = 0$.
  Then $\dim V$ is an even number.
\end{proposition}
\begin{proof}
  Let $n = \dim V$.
  In a basis of $V$ $\omega$ is represented by an antisymmetric matrix $A$.
  But then
  \begin{equation}
    \det A = \det(A^T) = \det(-A) = (-1)^n \det A.
  \end{equation}
  If $n$ is odd, then $\det A + \det A = 0$.
  By assumption this means that $\det A = 0$
  which contradicts the nondegeneracy of $\omega$.
  Hence, $n$ is even.
\end{proof}
\begin{definition}
  Let $M$ be a smooth manifold, $\omega \in \Omega^\bullet M$.
  We say that:
  \begin{enumerate}
    \item
      $\omega$ is \textbf{closed} if $d \omega = 0$
    \item
      $\omega$ is \textbf{exact} if there exists $\eta \in \Omega^\bullet M$
      such that $d \eta = \omega$.
  \end{enumerate}
\end{definition}
\begin{proposition}
  Let $M$ be a smooth manifold, $\omega \in \Omega^\bullet M$.
  If $\omega$ is exact, then it is closed.
\end{proposition}
\begin{proof}
  Let $\eta \in \Omega^\bullet M$ be such that $d \eta = \omega$.
  Then $d \omega = d (d \eta) = 0$, i.e., $\omega$ is closed.
\end{proof}
\begin{definition}
  Let $M$ be a smooth manifold, $\omega \in \Omega^2 M$.
  We say that $\omega$ is a \textbf{symplectic form}
  if it is non-degenerate
  (with base module $\mathfrak{X} M$ over $\mathcal{F} M$) and closed.

  The pair $(M, \omega)$ is called a \textbf{symplectic manifold}.
\end{definition}
\begin{proposition}
  Let $(M, \omega)$ be a symplectic manifold.
  Then $M$ is even-dimensional.
\end{proposition}
\begin{definition}
  Let $Q$ be a smooth manifold.
  Consider the cotangent bundle $T^* Q$ with bundle projection
  $\pi \colon T^* Q \to Q$
  with differential $d \pi \colon T(T^* Q) \to T Q$.
  Define the \textbf{tautological one-form}
  $\theta \colon T^* Q \to T^* (T^* Q)$ as follows:
  for any $(q, p) \in T^*Q$ (i.e,. $q \in Q$, $p \in \Hom(T_q Q, \R)$),
  \begin{equation}
    \restrict{\theta}{(q, p)}
    := p \circ \restrict{d \pi}{(q, p)} \in T^*_{(q, p)}(T^* Q).
  \end{equation}
  In other words, if we denote $M := T^* Q$, then $\theta$ is a section of its
  cotangent bundle $T^* M$, i.e., an $1$-form on $M$. 
\end{definition}
\begin{discussion}
  Let $Q$ be a smooth manifold, $\pi \colon T^* Q \to Q$ be the projection.
  Then a $1$-form on $Q$ is a section of $\colon T^* Q$, i.e., a smooth map
  $\mu \colon Q \to \colon T^* Q$ such that $\pi \circ \mu = \id_Q$.
  As such it has a pullback
  $\mu^* \colon \Omega^\bullet(T^* Q) \to \Omega^\bullet Q$.
\end{discussion}
\begin{proposition}
  Let
    $Q$ be a smooth manifold,
    $\theta$ be the tautological one-form on $T^* Q$,
    $\mu \in \Omega^1 Q$.
  Then
  \begin{equation}
    \mu^* \theta = \mu.
  \end{equation}
\end{proposition}
\begin{proof}
  Let $q \in Q$.
  Then
  \begin{equation}
    \restrict{\mu^* \theta}{q}
    = \restrict{\theta}{\mu q} \circ \restrict{d \mu}{q}
    = \restrict{\mu}{q} \circ \restrict{(d \pi)}{\mu q}
      \circ \restrict{d \mu}{q}
    = \restrict{\mu}{q} \circ \restrict{d(\pi \circ \mu)}{q}
    = \restrict{\mu}{q}.
  \end{equation}
  Since $q$ is arbitrary, $\mu^* \theta = \mu$.
\end{proof}
\begin{definition}
  Let
    $Q$ be a smooth manifold,
    $\theta \in \Omega^1(T^* Q)$ be the tautological one-form.
  Define $\omega := - d \theta$.
  The pair $(T^* Q, \omega)$ is called the \textbf{phase space} of $Q$.
  (In this setting $Q$ is usually called the \textbf{configuration space}.)
\end{definition}
\begin{remark}
  Let $Q$ be a smooth manifold.
  The elements of $T^* Q$ are of the form $(q, p)$ where $q \in Q$ and
  $p \in T^*_q Q = \Hom(T_q Q, \R)$.
  $q$ is called \textbf{generalised position}, while $p$ is called
  \textbf{generalised momentum}.
\end{remark}
\begin{proposition}
  Let
    $Q$ be a smooth manifold,
    $(T^* Q, \omega)$ be its phase space.
  Then $(T^* Q, \omega)$ is a symplectic manifold.
\end{proposition}
\begin{definition}
  Let $Q$ be a smooth manifold of dimension $n$.
  Consider a point $q_0 \in Q$ and let $(U, \hat{\varphi})$ be a chart around
  $q_0$, i.e., $U$ is a neighbourhood of $q_0$ and
  $\hat{\varphi} \colon U \to \R^n$ is a diffeomorphism.
  Let $\{\hat{q}^i \colon U \to \R\}_{i = 1}^n$ be the corresponding local
  coordinates, i.e., if $\{\pi^i \colon \R^n \to \R\}_{i = 1}^n$ are the
  projection maps, then $\{\hat{q}^i = \pi^i \circ \hat{\varphi}\}_{i = 1}^n$.
  Let $i \in \{1, ..., n\}$.
  Define \textbf{position coordinate} $q^i \colon T^* U \to \R$ by
  \begin{equation}
    q^i := \hat{q}^i \circ \restrict{\pi}{U}.
  \end{equation}
  Also, define \textbf{momentum coordinate} $p_i \colon T^* U \to \R$
  as follows: for any $(q, p) \in T^* U$,
  \begin{equation}
    p_i(q, p)
    := p\left(\restrict{\frac{\partial}{\partial \hat{q}^i}}{q}\right).
  \end{equation}
\end{definition}
\begin{proposition}
  Let
    $Q$ be a smooth manifold of dimension $n$,
    $q_0 \in Q$,
    $(U, \hat{\varphi})$ be a chart around $q_0$,
    $\{\hat{q}^i \colon U \to \R\}_{i = 1}^n$ be the corresponding local
      coordinates,
    $\{q^i \colon T^* U \to \R\}_{i = 1}^n$ be the corresponding position
      coordinates,
    $\{p_i \colon T^* U \to \R\}_{i = 1}^n$ be the corresponding momentum
      coordinates.
  Then the map $\varphi \colon T^* U \to \R^{2 n}$ defined by
  \begin{equation}
    \varphi(q, p) = (q^1(q, p), ..., q^n(q, p), p_1(q, p), ..., p_n(q, p))
  \end{equation}
  is a diffeomorphism, i.e., $(T^* U, \varphi)$ is a chart around $(q_0, 0)$.
  (The covector in $T^*_{q_0}$ does not matter, so we make the trivial choice by
  taking zero.)

  These local coordinates are called \textbf{generalised coordinates}.
\end{proposition}
\begin{remark}
  From now on, given a manifold $Q$ and a chart $(U, \hat{\varphi})$, unless
  stated otherwise, we will fix the notation and use the objects defined above:
  the projection map $\pi \colon T^* Q \to Q$, the tautological one-form
  $\theta$ and the canonical symplectic form $\omega = - d \theta$;
  for $i = 1, ..., n$ the coordinate maps $\hat{q}^i$, $q^i$, and $p_i$;
  the chart $(T^* U, \varphi)$.
\end{remark}
\begin{proposition}
  Let
    $Q$ be a smooth manifold,
    $\xi \in \Omega^1(T^* Q)$ has the following property:
    for any $1$-form $\mu$ on $Q$, $\mu^* \xi= 0$.
  Then $\xi = 0$.
\end{proposition}
\begin{proof}
  Let
    $n := \dim Q$, $(U, \hat{\varphi})$ be a chart on $Q$ and
    $\{f_i, g^i \in \mathcal{F}(T^* U)\}_{i = 1}^n$ be such that
  \begin{equation}
    \restrict{\xi}{U} = \sum_{i = 1}^n f_i\, d q^i + \sum_{i = 1}^n g^i\, d p_i.
  \end{equation}
  Take arbitrary $\{h_j \in \mathcal{F} U\}_{j = 1}^n$ so that
  \begin{equation}
    \restrict{\mu}{U} = \sum_{j = 1}^n h_j\, d \hat{q}^j.
  \end{equation}
  Note that
  $q^i \circ \restrict{\mu}{U} = \hat{q}^i$ and
  $p_i \circ \restrict{\mu}{U} = h_i$.
  Hence,
  \begin{equation}
    0 
    = \restrict{(\mu^* \xi)}{U}
    = \sum_{i = 1}^n (f_i \circ \restrict{\mu}{U})\,
      d(q^i \circ \restrict{\mu}{U})
    + \sum_{i = 1}^n (g^i \circ \restrict{\mu}{U})\,
      d(p_i \circ \restrict{\mu}{U})
    = \sum_{i = 1}^n (f_i \circ \restrict{\mu}{U})\, d \hat{q}^i
    + \sum_{i = 1}^n (g^i \circ \restrict{\mu}{U})\, d h_i.
  \end{equation}
  Fix $q_0 \in U$, $p_0 \in T^*_{q_0} Q$ so that $(p_0, q_0) \in T^* U$.
  Denote
  \begin{equation}
    c_i
    :=
    p_0\left(\restrict{\frac{\partial}{\partial \hat{q}^i}}{q_0}\right),
    i = 1, ..., n,
  \end{equation}
  so that
  \begin{equation}
    p_0 = \sum_{i = 1}^n c_i \restrict{d \hat{q}^i}{q_0}.
  \end{equation}
  \begin{enumerate}
    \item
      We will first prove that
      for any $i \in \{1, ..., n\}$, $f_i(q_0, p_0) = 0$.
      Define the constant functions
      \begin{equation}
        h_i(q) := c_i,\ i \in \{1, ..., n\},\ q \in U.
      \end{equation}
      Then for any $i \in \{1, ..., n\}$, $d h_i = 0$.
      Hence,
      \begin{equation}
        \begin{split}
          0
          & = \restrict{(\mu^* \xi)}{q_0} \\
          & = \sum_{i = 1}^n
              f_i(q_0, \sum_{j = 1}^n h_j(q_0) \restrict{d \hat{q}^j}{q_0})\,
              \restrict{d \hat{q}^i}{q_0} \\
          & = \sum_{i = 1}^n
              f_i(q_0, \sum_{j = 1}^n c_j \restrict{d \hat{q}^j}{q_0})\,
              \restrict{d \hat{q}^i}{q_0} \\
          & = \sum_{i = 1}^n f_i(q_0, p_0)\, \restrict{d \hat{q}^i}{q_0}.
        \end{split}
      \end{equation}
      Therefore, for any $i \in \{1, ..., n\}$, $f_i(q_0, p_0) = 0$.
    \item
      We will now prove that
      for any $i \in \{1, ..., n\}$, $g^i(q_0, p_0) = 0$.
      Define the linear functions
      \begin{equation}
        h_i(q) := c_i + \hat{q}^i(q) - \hat{q}^i(q_0).
      \end{equation}
      Then for any $i \in \{1, ..., n\}$,
      $d h_i = d \hat{q}^i$ and $h_i(q) = c_i$.
      Hence,
      \begin{equation}
        \begin{split}
          0
          & = \restrict{(\mu^* \xi)}{q_0} \\
          & = \sum_{i = 1}^n
              g^i(q_0, \sum_{j = 1}^n h_j(q_0) \restrict{d \hat{q}^j}{q_0})\,
              \restrict{d h_i}{q_0} \\
          & = \sum_{i = 1}^n
              g^i(q_0, \sum_{j = 1}^n c_j \restrict{d \hat{q}^j}{q_0})\,
              \restrict{d \hat{q}^i}{q_0} \\
          & = \sum_{i = 1}^n g^i(q_0, p_0)\, \restrict{d \hat{q}^i}{q_0}.
        \end{split}
      \end{equation}
      Therefore, for any $i \in \{1, ..., n\}$, $g^i(q_0, p_0) = 0$.
  \end{enumerate}
  Since $(q_0, p_0) \in T^* U$ was arbitrary, we conclude that
  for any $i \in \{1, ..., n\}$, $f_i = g^i = 0$.
  Hence, $\restrict{\xi}{U} = 0$.
  Taking an atlas $\{(U_\alpha, \hat{\varphi}_\alpha)\}_{\alpha \in A}$ of $Q$
  (for some index set $A$), we conclude that $\xi = 0$.
\end{proof}
\begin{corollary}
  Let
    $Q$ be a smooth manifold,
    $\theta$ be the tautological one-form on $T^* Q$,
    $\eta \in \Omega^1(T^* Q)$ has the following property:
    for any $1$-form $\mu$ on $Q$, $\mu^* \eta = \mu$.
  Then $\eta = \theta$.
\end{corollary}
\begin{proof}
  Write $\eta = \theta + \xi$, i.e., $\xi := \eta - \theta$.
  Then, for any $\mu \in \Omega^1 Q$,
  \begin{equation}
    \mu
    = \mu^* \eta
    = \mu^* \theta + \mu^* \xi
    = \mu + \mu^* \xi
    \Rightarrow \mu^* \xi = 0.
  \end{equation}
  But from the previous proposition it follows that $\xi = 0$,
  and hence $\eta = \theta$.
\end{proof}
\begin{proposition}
  Let
    $Q$ be a smooth manifold of dimension $n$,
    $(U, \hat{\varphi})$ be a chart,
    $(q, p) \in T^* U$,
    $ i \in \{1, ..., n\}$.
  Then
  \begin{equation}
    \restrict{d \pi}{(q, p)}
    \left(\restrict{\frac{\partial}{\partial q^i}}{(q, p)}\right)
    = \restrict{\frac{\partial}{\partial \hat{q}^i}}{q}
  \end{equation}
  and
  \begin{equation}
    \restrict{d \pi}{(q, p)}
    \left(\restrict{\frac{\partial}{\partial p^i}}{(q, p)}\right)
    = 0.
  \end{equation}
\end{proposition}
\begin{proof}
  Let $f \colon Q \to \R$ be smooth.
  Define the functions
  $\hat{g} := f \circ \hat{\varphi}^{-1} \colon \R^n \to \R$ and
  $g := f \circ \pi \circ \varphi^{-1} \colon \R^{2 n} \to \R$.
  Let $(X^1, ..., X^n, Y^1, ..., Y^n) := \varphi(p, q) \in \R^n$.
  This means that $(X^1, ..., X^n) = \hat{\varphi}(q)$.
  Then
  \begin{equation}
    g(X^1, ..., X^n, Y^1, ..., Y^n)
    = f(\pi(q, p))
    = f(q)
    = \hat{g}(X^1, ..., X^n).
  \end{equation}
  Hence,
  \begin{equation}
    \begin{split}
      \frac{\partial g}{\partial x^i}(X^1, ..., X^n, Y^1, ..., Y^n)
      & = \lim_{h \to 0}
        \frac
        {g(X^1, ..., X^i + h, ..., X^n, Y^1, ..., Y^n)
         - g(X^1, ..., X^n, Y^1, ..., Y^n)}
        {h} \\
      & = \lim_{h \to 0}
        \frac{\hat{g}(X^1, ..., X^i + h, ..., X^n) - \hat{g}(X^1, ..., X^n)}{h}
        \\
      & = \frac{\partial \hat{g}}{\partial \hat{x}^i}(X^1, ..., X^n).
    \end{split}
  \end{equation}
  Similarly, since $g$ is constant with respect to the last $n$ coordinates,
  \begin{equation}
    \frac{\partial g}{\partial x^{n + i}}(X^1, ..., X^n, Y^1, ..., Y^n) = 0
  \end{equation}
  Take the standard coordinate systems (given by projections)
  $\{\hat{x}^k\}_{k = 1}^n$ on $\R^n$ and
  $\{x^k\}_{k = 1}^{2 n}$ on $\R^{2 n}$.
  Then, by the definitions of differential and partial derivative on manifold,
  \begin{equation}
    (\restrict{d \pi}{(q, p)}
      \left(\restrict{\frac{\partial}{\partial q^i}}{(q, p)}\right)) f
    = \restrict{\frac{\partial}{\partial q^i}}{(q, p)}(f \circ \pi)
    = \frac{\partial(f \circ \pi \circ \varphi^{-1})}{x^i}(\varphi(q, p))
    = \frac{\partial(f \circ \hat{\varphi}^{-1})}{\hat{x}^i}(\hat{\varphi}(q))
    = \restrict{\frac{\partial}{\partial \hat{q}^i}}{q} f,
  \end{equation}
  from which it follows that the first equality holds.
  Similarly,
  \begin{equation}
    (\restrict{d \pi}{(q, p)}
      \left(\restrict{\frac{\partial}{\partial p^i}}{(q, p)}\right)) f
    = \restrict{\frac{\partial}{\partial p^i}}{(q, p)}(f \circ \pi)
    = \frac{\partial(f \circ \pi \circ \varphi^{-1})}{x^{i + n}}(\varphi(q, p))
    = 0,
  \end{equation}
  from which it follows that the second equality holds.
\end{proof}
\begin{proposition}[Tautological one-form in generalised coordinates]
  Let
    $Q$ be a smooth manifold of dimension $n$,
    $(U, \hat{\varphi})$ be a chart.
  Then
  \begin{equation}
    \restrict{\theta}{U} = \sum_{i = 1}^n p_i\, d q^i.
  \end{equation}
\end{proposition}
\begin{proof}
  Let $(q, p) \in T^* U$.
  Recall that $\restrict{\theta}{(q, p)} = p \circ \restrict{d \pi}{(q, p)}$.
  Hence,
  \begin{equation}
    \restrict{\theta}{(q, p)}
    \left(\restrict{\frac{\partial}{\partial q^i}}{(q, p)}\right)
    = p\left(\restrict{\frac{\partial}{\partial \hat{q}^i}}{q}\right)
    = p_i(q, p),
  \end{equation}
  and
  \begin{equation}
    \restrict{\theta}{(q, p)}
    \left(\restrict{\frac{\partial}{\partial p^i}}{(q, p)}\right)
    = p(0)
    = 0.
  \end{equation}
  Therefore,
  \begin{equation}
    \restrict{\theta}{(q, p)}
    = \sum_{i = 1}^n
      \restrict{\theta}{(q, p)}
      \left(\restrict{\frac{\partial}{\partial q^i}}{(q, p)}\right)\,
      \restrict{d q^i}{(q, p)}
    + \sum_{i = 1}^n
      \restrict{\theta}{(q, p)}
      \left(\restrict{\frac{\partial}{\partial p^i}}{(q, p)}\right)\,
      \restrict{d p^i}{(q, p)}
    = \sum_{i = 1}^n p_i(q, p)\, \restrict{d q^i}{(q, p)},
  \end{equation}
  from which the proposition follows.
\end{proof}
\begin{corollary}[Canonical symplectic in generalised coordinates]
  Let
    $Q$ be a smooth manifold of dimension $n$,
    $(U, \hat{\varphi})$ be a chart.
  Then
  \begin{equation}
    \restrict{\omega}{U} = \sum_{i = 1}^n d q^i \wedge d p_i.
  \end{equation}
\end{corollary}
\begin{proof}
  Let $i \in \{1, ..., n\}$.
  Then
  \begin{equation}
    - d(p_i\, d q^i) = - d p_i \wedge d q^i = d q^i \wedge d p_i.
  \end{equation}
  Summing up for all $i$, we get the desired result.
\end{proof}
\begin{definition}
  Let $(M, \omega)$ be a symplectic manifold, $f \in \mathcal{F} M$.
  We say that $X \in \mathfrak{X} M$ is a \textbf{Hamiltonian vector field} for
  $f$ if
  \begin{equation}
    i_X \omega + d_0 f = 0.
  \end{equation}
\end{definition}
\begin{proposition}
  Let $(M, \omega)$ be a symplectic manifold, $f \in \mathcal{F} M$.
  Then there exists a unique Hamiltonian vector field for $f$.
\end{proposition}
\begin{proof}
  The non-degeneracy of $\omega$ means that we can interpret the symplectic form
  as the isomorphism $\tilde{\omega} \colon \mathfrak{X} M \to \Omega^1 M$,
  given by
  \begin{equation}
    (\tilde{\omega} X) := i_X \omega,\ X \in \mathfrak{X} M.
  \end{equation}
  Hence, the problem at hand has a unique solution
  $X = \tilde{\omega}^{-1}(- d_0 f)$.
\end{proof}
\begin{definition}
  Let $(M, \omega)$ be a symplectic manifold.
  Define the map $\hamiltonian \colon \mathcal{F} M \to \mathfrak{X} M$ by
  \begin{equation}
    \hamiltonian = - \tilde{\omega}^{-1} \circ d_0.
  \end{equation}
  It maps a function to its corresponding Hamiltonian vector field.
  We will write $\hamiltonian_f$ instead of $\hamiltonian(f)$
  for $f \in \mathcal{F} M$.
\end{definition}
\begin{proposition}
  Let
    $Q$ be a smooth manifold of dimension $n$,
    $(U, \hat{\varphi})$ be a chart on $Q$,
    $f \in \mathcal{F}(T^* Q)$.
  Then
  \begin{equation}
    \hamiltonian_f
    = \sum_{i = 1}^n
    \left(
      - \frac{\partial f}{\partial p^i} \frac{\partial}{\partial q^i}
      + \frac{\partial f}{\partial q^i} \frac{\partial}{\partial p^i}
    \right).
  \end{equation}
\end{proposition}
\begin{proof}
  First, note that
  $i_{\frac{\partial}{\partial q^i}} \omega = d p^i$ and
  $i_{\frac{\partial}{\partial p^i}} \omega = - d q^i$.
  Denote
  \begin{equation}
    X
    := \sum_{i = 1}^n
    \left(
      - \frac{\partial f}{\partial p^i} \frac{\partial}{\partial q^i}
      + \frac{\partial f}{\partial q^i} \frac{\partial}{\partial p^i}
    \right).
  \end{equation}
  Then
  \begin{equation}
    i_X \omega
    = \sum_{i}^n
    \left(
      - \frac{\partial f}{\partial p^i} i_{\frac{\partial}{\partial q^i}} \omega
      + \frac{\partial f}{\partial q^i} i_{\frac{\partial}{\partial p^i}} \omega
    \right)
    = \sum_{i}^n
    \left(
      - \frac{\partial f}{\partial p^i} d p^i
      - \frac{\partial f}{\partial q^i} d q^i
    \right)
    = - d f.
  \end{equation}
  Hence, $\hamiltonian_f = X$.
\end{proof}
\begin{proposition}
  Let $(M, \omega)$ be a symplectic manifold, $f, g \in \mathcal{F} M$.
  Then
  \begin{equation}
    \hamiltonian_{f g} = f \hamiltonian_g + g \hamiltonian_f.
  \end{equation}
\end{proposition}
\begin{proof}
  Follows directly from the Leibniz rule for $d_0$.
\end{proof}
\begin{definition}
  Let $(M, \omega)$ be a symplectic manifold, $X \in \mathfrak{X} M$.
  We say that $X$ is a \textbf{symplectic vector field} if $L_X \omega = 0$.
\end{definition}
\begin{remark}
  Since $L_{\lie{X}{Y}} = \lie{L_X}{L_Y} = L_X \circ L_Y - L_Y \circ L_X$,
  the symplectic vector fields form a Lie subalgebra of the Lie algebra of
  vector fields.
\end{remark}
\begin{proposition}
  Let $(M, \omega)$ be a symplectic manifold, $f \in \mathcal{F} M$.
  Then $\hamiltonian_f$ is a symplectic vector field.
\end{proposition}
\begin{proof}
  $
    L_{\hamiltonian_f} \omega
    = i_{\hamiltonian_f}(d \omega) + d(i_{\hamiltonian_f} \omega)
    = i_{\hamiltonian_f} 0 - d(d f)
    = 0.
  $
\end{proof}
\begin{proposition}
  Let
    $(M, \omega)$ be a symplectic manifold,
    $X \in \mathfrak{X} M$ be a symplectic vector fields.
  Then
  \begin{equation}
    d(i_X \omega) = 0.
  \end{equation}
\end{proposition}
\begin{proof}
  $
    d(i_X \omega)
    = L_X \omega - i_X(d \omega)
    = 0 - 0
    = 0.
  $
\end{proof}
\begin{proposition}
  Let $M$ be a smooth manifold, $X, Y \in \mathfrak{X} M$.
  Then
  \begin{equation}
    L_X \circ i_Y = i_{\lie{X}{Y}} + i_Y \circ L_X.
  \end{equation}
\end{proposition}
\begin{proposition}
  Let
    $(M, \omega)$ be a symplectic manifold,
    $X, Y \in \mathfrak{X} M$ be symplectic vector fields.
  Then
  \begin{equation}
    \lie{X}{Y} = \hamiltonian_{i_Y(i_X \omega)}.
  \end{equation}
\end{proposition}
\begin{proof}
  \begin{equation}
    i_{\lie{X}{Y}} \omega
    = (L_X \circ i_Y - i_Y \circ L_X) \omega
    = (L_X \circ i_Y) \omega
    = ((d \circ i_X + i_X \circ d) \circ i_Y) \omega
    = d(i_X(i_Y \omega))
    = - d(i_Y(i_X \omega)).
  \end{equation}
  We get the desired result from the definition of $\hamiltonian$.
\end{proof}
\begin{definition}
  Let $(M, \omega)$ be a symplectic manifold.
  Define the \textbf{Poisson bracket}
  $\poisson{\cdot}{\cdot} \colon \mathcal{F} M \to \mathcal{F} M$ by
  \begin{equation}
    \poisson{f}{g}
    := i_{\hamiltonian_g}(i_{\hamiltonian_f} \omega),\
    f, g \in \mathcal{F} M.
  \end{equation}
\end{definition}
\begin{corollary}
  Let $(M, \omega)$ be a symplectic manifold, $f, g \in \mathcal{F} M$.
  Then
  \begin{equation}
    \lie{\hamiltonian_f}{\hamiltonian_g}
    = \hamiltonian_{i_{\hamiltonian_g}(i_{\hamiltonian_f} \omega)}
    = \poisson{f}{g}.
  \end{equation}
\end{corollary}
\begin{proposition}[Leibniz rule holds for the Poisson bracket]
  Let $(M, \omega)$ be a symplectic manifold, $f, g, h \in \mathcal{F} M$.
  Then
  \begin{equation}
    \poisson{f}{g h} = \poisson{f}{g} h + g \poisson{f}{h}.
  \end{equation}
\end{proposition}
\begin{proof}
  $
    \poisson{f}{g h}
    = i_{\hamiltonian_{g h}}(i_{\hamiltonian_f} \omega)
    = i_{h \hamiltonian_g + g \hamiltonian_{h}}(i_{\hamiltonian_f} \omega)
    = (i_{\hamiltonian_g}(i_{\hamiltonian_f} \omega))\, h
      + g\, (i_{\hamiltonian_h}(i_{\hamiltonian_f} \omega)) 
    = \poisson{f}{g} h + g \poisson{f}{h}.
  $
\end{proof}
\begin{proposition}
  Let $(M, \omega)$ be a symplectic manifold, $f, g, h \in \mathcal{F} M$.
  Then
  \begin{equation}
    \poisson{f}{g} = L_{\hamiltonian_f} g.
  \end{equation}
\end{proposition}
\begin{proof}
  $
    \poisson{f}{g}
    = i_{\hamiltonian_g}(i_{\hamiltonian_f} \omega)
    = - i_{\hamiltonian_f} \circ i_{\hamiltonian_g} \omega
    = i_{\hamiltonian_f}(d g)
    = L_{\hamiltonian_f} g.
  $
\end{proof}
\begin{definition}
  Let $(M, \omega)$ be a symplectic manifold.
  Define
  ${\rm ad} \colon \mathcal{F} M \to (\mathcal{F} M \to \mathcal{F} M)$ by
  \begin{equation}
    {\rm ad}_f g := \poisson{f}{g},\ f, g \in \mathcal{F} M,
  \end{equation}
\end{definition}
\begin{proposition}
  Let $(M, \omega)$ be a symplectic manifold.
  Then
  \begin{equation}
    \lie{{\rm ad}_f}{{\rm ad}_g} = {\rm ad}_{\poisson{f}{g}}.
  \end{equation}
  (Here the bracket $\lie{\cdot}{\cdot}$ is the commutator of operators.)
\end{proposition}
\begin{proof}
  From the previous proposition it follows that
  \begin{equation}
    {\rm ad}_f = L_{X_f},\ f \in \mathcal{F} M.
  \end{equation}
  Hence,
  \begin{equation}
    \lie{{\rm ad}_f}{{\rm ad}_g}
    = \lie{L_{\hamiltonian_f}}{L_{\hamiltonian_g}}
    = L_{\lie{\hamiltonian_f}{\hamiltonian_g}}
    = L_{\hamiltonian_{\poisson{f}{g}}}
    = {\rm ad}_{\poisson{f}{g}}.
  \end{equation}
\end{proof}
\begin{corollary}
  Let $(M, \omega)$ be a symplectic manifold.
  Then $(\mathcal{F} M, \poisson{\cdot}{\cdot})$ is a Lie algebra over $\R$.
\end{corollary}
\begin{proof}
  Bilinearity and antisymmetry are trivial to check.
  The Jacobi identity is equivalent to the adjoint map being a Lie algebra
  homomorphism, which was the previous proposition.
\end{proof}
\begin{definition}
  Let
    $R$ be a commutative ring with unity ring,
    $(A, +, \cdot)$ be an $R$-module
    with additional structures of
    an associative algebra $(A, *)$ and
    a Lie algebra $(A, \poisson{\cdot}{\cdot})$.
  We say that $A$ is a Poisson algebra if the Lie bracket acts as a derivation,
  i.e., for all $f, g, h \in A$,
  \begin{equation}
    \poisson{f}{g * h} = \poisson{f}{g} * h + g * \poisson{f}{h}.
  \end{equation}
\end{definition}
\begin{corollary}
  Let $(M, \omega)$ be a symplectic manifold.
  Then $\mathcal{F} M$ is a Poisson algebra over $\R$.
  Here, addition, scalar multiplication, and multiplication are given by the
  corresponding pointwise operations, while the Lie bracket is given by the
  Poisson bracket.
\end{corollary}


\section{Connections on discrete vector bundles}
\label{section:connections_on_discrete_vector_bundles}
\begin{definition}
  Let
    $R$ be a ring,
    $V$ be a finite-dimensional $R$-module,
    $\omega \in \Lambda^2 V^*$.
  We say that $\omega$ is \textbf{non-degenerate} or \textbf{symplectic}
  if the associated map
  \begin{equation}
    \tilde{\omega} \colon V \to V^*,\
    X \in V \mapsto \tilde{\omega}(X) := i_X \omega \in V^*,
  \end{equation}
  is an isomorphism.

  The pair $(V, \omega)$ is called a \textbf{symplectic module}
  (or \textbf{symplectic vector space} if $R$ is a field).
\end{definition}
\begin{proposition}
  Let
    $R$ be a ring without,
    $(V, \omega)$ be a finite-dimensional symplectic module over $R$.
  Assume that for any $x \in R,\ x + x = 0 \Rightarrow x = 0$.
  Then $\dim V$ is an even number.
\end{proposition}
\begin{proof}
  Let $n = \dim V$.
  In a basis of $V$ $\omega$ is represented by an antisymmetric matrix $A$.
  But then
  \begin{equation}
    \det A = \det(A^T) = \det(-A) = (-1)^n \det A.
  \end{equation}
  If $n$ is odd, then $\det A + \det A = 0$.
  By assumption this means that $\det A = 0$
  which contradicts the nondegeneracy of $\omega$.
  Hence, $n$ is even.
\end{proof}
\begin{definition}
  Let $M$ be a smooth manifold, $\omega \in \Omega^\bullet M$.
  We say that:
  \begin{enumerate}
    \item
      $\omega$ is \textbf{closed} if $d \omega = 0$
    \item
      $\omega$ is \textbf{exact} if there exists $\eta \in \Omega^\bullet M$
      such that $d \eta = \omega$.
  \end{enumerate}
\end{definition}
\begin{proposition}
  Let $M$ be a smooth manifold, $\omega \in \Omega^\bullet M$.
  If $\omega$ is exact, then it is closed.
\end{proposition}
\begin{proof}
  Let $\eta \in \Omega^\bullet M$ be such that $d \eta = \omega$.
  Then $d \omega = d (d \eta) = 0$, i.e., $\omega$ is closed.
\end{proof}
\begin{definition}
  Let $M$ be a smooth manifold, $\omega \in \Omega^2 M$.
  We say that $\omega$ is a \textbf{symplectic form}
  if it is non-degenerate
  (with base module $\mathfrak{X} M$ over $\mathcal{F} M$) and closed.

  The pair $(M, \omega)$ is called a \textbf{symplectic manifold}.
\end{definition}
\begin{proposition}
  Let $(M, \omega)$ be a symplectic manifold.
  Then $M$ is even-dimensional.
\end{proposition}
\begin{definition}
  Let $Q$ be a smooth manifold.
  Consider the cotangent bundle $T^* Q$ with bundle projection
  $\pi \colon T^* Q \to Q$
  with differential $d \pi \colon T(T^* Q) \to T Q$.
  Define the \textbf{tautological one-form}
  $\theta \colon T^* Q \to T^* (T^* Q)$ as follows:
  for any $(q, p) \in T^*Q$ (i.e,. $q \in Q$, $p \in \Hom(T_q Q, \R)$),
  \begin{equation}
    \restrict{\theta}{(q, p)}
    := p \circ \restrict{d \pi}{(q, p)} \in T^*_{(q, p)}(T^* Q).
  \end{equation}
  In other words, if we denote $M := T^* Q$, then $\theta$ is a section of its
  cotangent bundle $T^* M$, i.e., an $1$-form on $M$. 
\end{definition}
\begin{discussion}
  Let $Q$ be a smooth manifold, $\pi \colon T^* Q \to Q$ be the projection.
  Then a $1$-form on $Q$ is a section of $\colon T^* Q$, i.e., a smooth map
  $\mu \colon Q \to \colon T^* Q$ such that $\pi \circ \mu = \id_Q$.
  As such it has a pullback
  $\mu^* \colon \Omega^\bullet(T^* Q) \to \Omega^\bullet Q$.
\end{discussion}
\begin{proposition}
  Let
    $Q$ be a smooth manifold,
    $\theta$ be the tautological one-form on $T^* Q$,
    $\mu \in \Omega^1 Q$.
  Then
  \begin{equation}
    \mu^* \theta = \mu.
  \end{equation}
\end{proposition}
\begin{proof}
  Let $q \in Q$.
  Then
  \begin{equation}
    \restrict{\mu^* \theta}{q}
    = \restrict{\theta}{\mu q} \circ \restrict{d \mu}{q}
    = \restrict{\mu}{q} \circ \restrict{(d \pi)}{\mu q}
      \circ \restrict{d \mu}{q}
    = \restrict{\mu}{q} \circ \restrict{d(\pi \circ \mu)}{q}
    = \restrict{\mu}{q}.
  \end{equation}
  Since $q$ is arbitrary, $\mu^* \theta = \mu$.
\end{proof}
\begin{definition}
  Let
    $Q$ be a smooth manifold,
    $\theta \in \Omega^1(T^* Q)$ be the tautological one-form.
  Define $\omega := - d \theta$.
  The pair $(T^* Q, \omega)$ is called the \textbf{phase space} of $Q$.
  (In this setting $Q$ is usually called the \textbf{configuration space}.)
\end{definition}
\begin{remark}
  Let $Q$ be a smooth manifold.
  The elements of $T^* Q$ are of the form $(q, p)$ where $q \in Q$ and
  $p \in T^*_q Q = \Hom(T_q Q, \R)$.
  $q$ is called \textbf{generalised position}, while $p$ is called
  \textbf{generalised momentum}.
\end{remark}
\begin{proposition}
  Let
    $Q$ be a smooth manifold,
    $(T^* Q, \omega)$ be its phase space.
  Then $(T^* Q, \omega)$ is a symplectic manifold.
\end{proposition}
\begin{definition}
  Let $Q$ be a smooth manifold of dimension $n$.
  Consider a point $q_0 \in Q$ and let $(U, \hat{\varphi})$ be a chart around
  $q_0$, i.e., $U$ is a neighbourhood of $q_0$ and
  $\hat{\varphi} \colon U \to \R^n$ is a diffeomorphism.
  Let $\{\hat{q}^i \colon U \to \R\}_{i = 1}^n$ be the corresponding local
  coordinates, i.e., if $\{\pi^i \colon \R^n \to \R\}_{i = 1}^n$ are the
  projection maps, then $\{\hat{q}^i = \pi^i \circ \hat{\varphi}\}_{i = 1}^n$.
  Let $i \in \{1, ..., n\}$.
  Define \textbf{position coordinate} $q^i \colon T^* U \to \R$ by
  \begin{equation}
    q^i := \hat{q}^i \circ \restrict{\pi}{U}.
  \end{equation}
  Also, define \textbf{momentum coordinate} $p_i \colon T^* U \to \R$
  as follows: for any $(q, p) \in T^* U$,
  \begin{equation}
    p_i(q, p)
    := p\left(\restrict{\frac{\partial}{\partial \hat{q}^i}}{q}\right).
  \end{equation}
\end{definition}
\begin{proposition}
  Let
    $Q$ be a smooth manifold of dimension $n$,
    $q_0 \in Q$,
    $(U, \hat{\varphi})$ be a chart around $q_0$,
    $\{\hat{q}^i \colon U \to \R\}_{i = 1}^n$ be the corresponding local
      coordinates,
    $\{q^i \colon T^* U \to \R\}_{i = 1}^n$ be the corresponding position
      coordinates,
    $\{p_i \colon T^* U \to \R\}_{i = 1}^n$ be the corresponding momentum
      coordinates.
  Then the map $\varphi \colon T^* U \to \R^{2 n}$ defined by
  \begin{equation}
    \varphi(q, p) = (q^1(q, p), ..., q^n(q, p), p_1(q, p), ..., p_n(q, p))
  \end{equation}
  is a diffeomorphism, i.e., $(T^* U, \varphi)$ is a chart around $(q_0, 0)$.
  (The covector in $T^*_{q_0}$ does not matter, so we make the trivial choice by
  taking zero.)

  These local coordinates are called \textbf{generalised coordinates}.
\end{proposition}
\begin{remark}
  From now on, given a manifold $Q$ and a chart $(U, \hat{\varphi})$, unless
  stated otherwise, we will fix the notation and use the objects defined above:
  the projection map $\pi \colon T^* Q \to Q$, the tautological one-form
  $\theta$ and the canonical symplectic form $\omega = - d \theta$;
  for $i = 1, ..., n$ the coordinate maps $\hat{q}^i$, $q^i$, and $p_i$;
  the chart $(T^* U, \varphi)$.
\end{remark}
\begin{proposition}
  Let
    $Q$ be a smooth manifold,
    $\xi \in \Omega^1(T^* Q)$ has the following property:
    for any $1$-form $\mu$ on $Q$, $\mu^* \xi= 0$.
  Then $\xi = 0$.
\end{proposition}
\begin{proof}
  Let
    $n := \dim Q$, $(U, \hat{\varphi})$ be a chart on $Q$ and
    $\{f_i, g^i \in \mathcal{F}(T^* U)\}_{i = 1}^n$ be such that
  \begin{equation}
    \restrict{\xi}{U} = \sum_{i = 1}^n f_i\, d q^i + \sum_{i = 1}^n g^i\, d p_i.
  \end{equation}
  Take arbitrary $\{h_j \in \mathcal{F} U\}_{j = 1}^n$ so that
  \begin{equation}
    \restrict{\mu}{U} = \sum_{j = 1}^n h_j\, d \hat{q}^j.
  \end{equation}
  Note that
  $q^i \circ \restrict{\mu}{U} = \hat{q}^i$ and
  $p_i \circ \restrict{\mu}{U} = h_i$.
  Hence,
  \begin{equation}
    0 
    = \restrict{(\mu^* \xi)}{U}
    = \sum_{i = 1}^n (f_i \circ \restrict{\mu}{U})\,
      d(q^i \circ \restrict{\mu}{U})
    + \sum_{i = 1}^n (g^i \circ \restrict{\mu}{U})\,
      d(p_i \circ \restrict{\mu}{U})
    = \sum_{i = 1}^n (f_i \circ \restrict{\mu}{U})\, d \hat{q}^i
    + \sum_{i = 1}^n (g^i \circ \restrict{\mu}{U})\, d h_i.
  \end{equation}
  Fix $q_0 \in U$, $p_0 \in T^*_{q_0} Q$ so that $(p_0, q_0) \in T^* U$.
  Denote
  \begin{equation}
    c_i
    :=
    p_0\left(\restrict{\frac{\partial}{\partial \hat{q}^i}}{q_0}\right),
    i = 1, ..., n,
  \end{equation}
  so that
  \begin{equation}
    p_0 = \sum_{i = 1}^n c_i \restrict{d \hat{q}^i}{q_0}.
  \end{equation}
  \begin{enumerate}
    \item
      We will first prove that
      for any $i \in \{1, ..., n\}$, $f_i(q_0, p_0) = 0$.
      Define the constant functions
      \begin{equation}
        h_i(q) := c_i,\ i \in \{1, ..., n\},\ q \in U.
      \end{equation}
      Then for any $i \in \{1, ..., n\}$, $d h_i = 0$.
      Hence,
      \begin{equation}
        \begin{split}
          0
          & = \restrict{(\mu^* \xi)}{q_0} \\
          & = \sum_{i = 1}^n
              f_i(q_0, \sum_{j = 1}^n h_j(q_0) \restrict{d \hat{q}^j}{q_0})\,
              \restrict{d \hat{q}^i}{q_0} \\
          & = \sum_{i = 1}^n
              f_i(q_0, \sum_{j = 1}^n c_j \restrict{d \hat{q}^j}{q_0})\,
              \restrict{d \hat{q}^i}{q_0} \\
          & = \sum_{i = 1}^n f_i(q_0, p_0)\, \restrict{d \hat{q}^i}{q_0}.
        \end{split}
      \end{equation}
      Therefore, for any $i \in \{1, ..., n\}$, $f_i(q_0, p_0) = 0$.
    \item
      We will now prove that
      for any $i \in \{1, ..., n\}$, $g^i(q_0, p_0) = 0$.
      Define the linear functions
      \begin{equation}
        h_i(q) := c_i + \hat{q}^i(q) - \hat{q}^i(q_0).
      \end{equation}
      Then for any $i \in \{1, ..., n\}$,
      $d h_i = d \hat{q}^i$ and $h_i(q) = c_i$.
      Hence,
      \begin{equation}
        \begin{split}
          0
          & = \restrict{(\mu^* \xi)}{q_0} \\
          & = \sum_{i = 1}^n
              g^i(q_0, \sum_{j = 1}^n h_j(q_0) \restrict{d \hat{q}^j}{q_0})\,
              \restrict{d h_i}{q_0} \\
          & = \sum_{i = 1}^n
              g^i(q_0, \sum_{j = 1}^n c_j \restrict{d \hat{q}^j}{q_0})\,
              \restrict{d \hat{q}^i}{q_0} \\
          & = \sum_{i = 1}^n g^i(q_0, p_0)\, \restrict{d \hat{q}^i}{q_0}.
        \end{split}
      \end{equation}
      Therefore, for any $i \in \{1, ..., n\}$, $g^i(q_0, p_0) = 0$.
  \end{enumerate}
  Since $(q_0, p_0) \in T^* U$ was arbitrary, we conclude that
  for any $i \in \{1, ..., n\}$, $f_i = g^i = 0$.
  Hence, $\restrict{\xi}{U} = 0$.
  Taking an atlas $\{(U_\alpha, \hat{\varphi}_\alpha)\}_{\alpha \in A}$ of $Q$
  (for some index set $A$), we conclude that $\xi = 0$.
\end{proof}
\begin{corollary}
  Let
    $Q$ be a smooth manifold,
    $\theta$ be the tautological one-form on $T^* Q$,
    $\eta \in \Omega^1(T^* Q)$ has the following property:
    for any $1$-form $\mu$ on $Q$, $\mu^* \eta = \mu$.
  Then $\eta = \theta$.
\end{corollary}
\begin{proof}
  Write $\eta = \theta + \xi$, i.e., $\xi := \eta - \theta$.
  Then, for any $\mu \in \Omega^1 Q$,
  \begin{equation}
    \mu
    = \mu^* \eta
    = \mu^* \theta + \mu^* \xi
    = \mu + \mu^* \xi
    \Rightarrow \mu^* \xi = 0.
  \end{equation}
  But from the previous proposition it follows that $\xi = 0$,
  and hence $\eta = \theta$.
\end{proof}
\begin{proposition}
  Let
    $Q$ be a smooth manifold of dimension $n$,
    $(U, \hat{\varphi})$ be a chart,
    $(q, p) \in T^* U$,
    $ i \in \{1, ..., n\}$.
  Then
  \begin{equation}
    \restrict{d \pi}{(q, p)}
    \left(\restrict{\frac{\partial}{\partial q^i}}{(q, p)}\right)
    = \restrict{\frac{\partial}{\partial \hat{q}^i}}{q}
  \end{equation}
  and
  \begin{equation}
    \restrict{d \pi}{(q, p)}
    \left(\restrict{\frac{\partial}{\partial p^i}}{(q, p)}\right)
    = 0.
  \end{equation}
\end{proposition}
\begin{proof}
  Let $f \colon Q \to \R$ be smooth.
  Define the functions
  $\hat{g} := f \circ \hat{\varphi}^{-1} \colon \R^n \to \R$ and
  $g := f \circ \pi \circ \varphi^{-1} \colon \R^{2 n} \to \R$.
  Let $(X^1, ..., X^n, Y^1, ..., Y^n) := \varphi(p, q) \in \R^n$.
  This means that $(X^1, ..., X^n) = \hat{\varphi}(q)$.
  Then
  \begin{equation}
    g(X^1, ..., X^n, Y^1, ..., Y^n)
    = f(\pi(q, p))
    = f(q)
    = \hat{g}(X^1, ..., X^n).
  \end{equation}
  Hence,
  \begin{equation}
    \begin{split}
      \frac{\partial g}{\partial x^i}(X^1, ..., X^n, Y^1, ..., Y^n)
      & = \lim_{h \to 0}
        \frac
        {g(X^1, ..., X^i + h, ..., X^n, Y^1, ..., Y^n)
         - g(X^1, ..., X^n, Y^1, ..., Y^n)}
        {h} \\
      & = \lim_{h \to 0}
        \frac{\hat{g}(X^1, ..., X^i + h, ..., X^n) - \hat{g}(X^1, ..., X^n)}{h}
        \\
      & = \frac{\partial \hat{g}}{\partial \hat{x}^i}(X^1, ..., X^n).
    \end{split}
  \end{equation}
  Similarly, since $g$ is constant with respect to the last $n$ coordinates,
  \begin{equation}
    \frac{\partial g}{\partial x^{n + i}}(X^1, ..., X^n, Y^1, ..., Y^n) = 0
  \end{equation}
  Take the standard coordinate systems (given by projections)
  $\{\hat{x}^k\}_{k = 1}^n$ on $\R^n$ and
  $\{x^k\}_{k = 1}^{2 n}$ on $\R^{2 n}$.
  Then, by the definitions of differential and partial derivative on manifold,
  \begin{equation}
    (\restrict{d \pi}{(q, p)}
      \left(\restrict{\frac{\partial}{\partial q^i}}{(q, p)}\right)) f
    = \restrict{\frac{\partial}{\partial q^i}}{(q, p)}(f \circ \pi)
    = \frac{\partial(f \circ \pi \circ \varphi^{-1})}{x^i}(\varphi(q, p))
    = \frac{\partial(f \circ \hat{\varphi}^{-1})}{\hat{x}^i}(\hat{\varphi}(q))
    = \restrict{\frac{\partial}{\partial \hat{q}^i}}{q} f,
  \end{equation}
  from which it follows that the first equality holds.
  Similarly,
  \begin{equation}
    (\restrict{d \pi}{(q, p)}
      \left(\restrict{\frac{\partial}{\partial p^i}}{(q, p)}\right)) f
    = \restrict{\frac{\partial}{\partial p^i}}{(q, p)}(f \circ \pi)
    = \frac{\partial(f \circ \pi \circ \varphi^{-1})}{x^{i + n}}(\varphi(q, p))
    = 0,
  \end{equation}
  from which it follows that the second equality holds.
\end{proof}
\begin{proposition}[Tautological one-form in generalised coordinates]
  Let
    $Q$ be a smooth manifold of dimension $n$,
    $(U, \hat{\varphi})$ be a chart.
  Then
  \begin{equation}
    \restrict{\theta}{U} = \sum_{i = 1}^n p_i\, d q^i.
  \end{equation}
\end{proposition}
\begin{proof}
  Let $(q, p) \in T^* U$.
  Recall that $\restrict{\theta}{(q, p)} = p \circ \restrict{d \pi}{(q, p)}$.
  Hence,
  \begin{equation}
    \restrict{\theta}{(q, p)}
    \left(\restrict{\frac{\partial}{\partial q^i}}{(q, p)}\right)
    = p\left(\restrict{\frac{\partial}{\partial \hat{q}^i}}{q}\right)
    = p_i(q, p),
  \end{equation}
  and
  \begin{equation}
    \restrict{\theta}{(q, p)}
    \left(\restrict{\frac{\partial}{\partial p^i}}{(q, p)}\right)
    = p(0)
    = 0.
  \end{equation}
  Therefore,
  \begin{equation}
    \restrict{\theta}{(q, p)}
    = \sum_{i = 1}^n
      \restrict{\theta}{(q, p)}
      \left(\restrict{\frac{\partial}{\partial q^i}}{(q, p)}\right)\,
      \restrict{d q^i}{(q, p)}
    + \sum_{i = 1}^n
      \restrict{\theta}{(q, p)}
      \left(\restrict{\frac{\partial}{\partial p^i}}{(q, p)}\right)\,
      \restrict{d p^i}{(q, p)}
    = \sum_{i = 1}^n p_i(q, p)\, \restrict{d q^i}{(q, p)},
  \end{equation}
  from which the proposition follows.
\end{proof}
\begin{corollary}[Canonical symplectic in generalised coordinates]
  Let
    $Q$ be a smooth manifold of dimension $n$,
    $(U, \hat{\varphi})$ be a chart.
  Then
  \begin{equation}
    \restrict{\omega}{U} = \sum_{i = 1}^n d q^i \wedge d p_i.
  \end{equation}
\end{corollary}
\begin{proof}
  Let $i \in \{1, ..., n\}$.
  Then
  \begin{equation}
    - d(p_i\, d q^i) = - d p_i \wedge d q^i = d q^i \wedge d p_i.
  \end{equation}
  Summing up for all $i$, we get the desired result.
\end{proof}
\begin{definition}
  Let $(M, \omega)$ be a symplectic manifold, $f \in \mathcal{F} M$.
  We say that $X \in \mathfrak{X} M$ is a \textbf{Hamiltonian vector field} for
  $f$ if
  \begin{equation}
    i_X \omega + d_0 f = 0.
  \end{equation}
\end{definition}
\begin{proposition}
  Let $(M, \omega)$ be a symplectic manifold, $f \in \mathcal{F} M$.
  Then there exists a unique Hamiltonian vector field for $f$.
\end{proposition}
\begin{proof}
  The non-degeneracy of $\omega$ means that we can interpret the symplectic form
  as the isomorphism $\tilde{\omega} \colon \mathfrak{X} M \to \Omega^1 M$,
  given by
  \begin{equation}
    (\tilde{\omega} X) := i_X \omega,\ X \in \mathfrak{X} M.
  \end{equation}
  Hence, the problem at hand has a unique solution
  $X = \tilde{\omega}^{-1}(- d_0 f)$.
\end{proof}
\begin{definition}
  Let $(M, \omega)$ be a symplectic manifold.
  Define the map $\hamiltonian \colon \mathcal{F} M \to \mathfrak{X} M$ by
  \begin{equation}
    \hamiltonian = - \tilde{\omega}^{-1} \circ d_0.
  \end{equation}
  It maps a function to its corresponding Hamiltonian vector field.
  We will write $\hamiltonian_f$ instead of $\hamiltonian(f)$
  for $f \in \mathcal{F} M$.
\end{definition}
\begin{proposition}
  Let
    $Q$ be a smooth manifold of dimension $n$,
    $(U, \hat{\varphi})$ be a chart on $Q$,
    $f \in \mathcal{F}(T^* Q)$.
  Then
  \begin{equation}
    \hamiltonian_f
    = \sum_{i = 1}^n
    \left(
      - \frac{\partial f}{\partial p^i} \frac{\partial}{\partial q^i}
      + \frac{\partial f}{\partial q^i} \frac{\partial}{\partial p^i}
    \right).
  \end{equation}
\end{proposition}
\begin{proof}
  First, note that
  $i_{\frac{\partial}{\partial q^i}} \omega = d p^i$ and
  $i_{\frac{\partial}{\partial p^i}} \omega = - d q^i$.
  Denote
  \begin{equation}
    X
    := \sum_{i = 1}^n
    \left(
      - \frac{\partial f}{\partial p^i} \frac{\partial}{\partial q^i}
      + \frac{\partial f}{\partial q^i} \frac{\partial}{\partial p^i}
    \right).
  \end{equation}
  Then
  \begin{equation}
    i_X \omega
    = \sum_{i}^n
    \left(
      - \frac{\partial f}{\partial p^i} i_{\frac{\partial}{\partial q^i}} \omega
      + \frac{\partial f}{\partial q^i} i_{\frac{\partial}{\partial p^i}} \omega
    \right)
    = \sum_{i}^n
    \left(
      - \frac{\partial f}{\partial p^i} d p^i
      - \frac{\partial f}{\partial q^i} d q^i
    \right)
    = - d f.
  \end{equation}
  Hence, $\hamiltonian_f = X$.
\end{proof}
\begin{proposition}
  Let $(M, \omega)$ be a symplectic manifold, $f, g \in \mathcal{F} M$.
  Then
  \begin{equation}
    \hamiltonian_{f g} = f \hamiltonian_g + g \hamiltonian_f.
  \end{equation}
\end{proposition}
\begin{proof}
  Follows directly from the Leibniz rule for $d_0$.
\end{proof}
\begin{definition}
  Let $(M, \omega)$ be a symplectic manifold, $X \in \mathfrak{X} M$.
  We say that $X$ is a \textbf{symplectic vector field} if $L_X \omega = 0$.
\end{definition}
\begin{remark}
  Since $L_{\lie{X}{Y}} = \lie{L_X}{L_Y} = L_X \circ L_Y - L_Y \circ L_X$,
  the symplectic vector fields form a Lie subalgebra of the Lie algebra of
  vector fields.
\end{remark}
\begin{proposition}
  Let $(M, \omega)$ be a symplectic manifold, $f \in \mathcal{F} M$.
  Then $\hamiltonian_f$ is a symplectic vector field.
\end{proposition}
\begin{proof}
  $
    L_{\hamiltonian_f} \omega
    = i_{\hamiltonian_f}(d \omega) + d(i_{\hamiltonian_f} \omega)
    = i_{\hamiltonian_f} 0 - d(d f)
    = 0.
  $
\end{proof}
\begin{proposition}
  Let
    $(M, \omega)$ be a symplectic manifold,
    $X \in \mathfrak{X} M$ be a symplectic vector fields.
  Then
  \begin{equation}
    d(i_X \omega) = 0.
  \end{equation}
\end{proposition}
\begin{proof}
  $
    d(i_X \omega)
    = L_X \omega - i_X(d \omega)
    = 0 - 0
    = 0.
  $
\end{proof}
\begin{proposition}
  Let $M$ be a smooth manifold, $X, Y \in \mathfrak{X} M$.
  Then
  \begin{equation}
    L_X \circ i_Y = i_{\lie{X}{Y}} + i_Y \circ L_X.
  \end{equation}
\end{proposition}
\begin{proposition}
  Let
    $(M, \omega)$ be a symplectic manifold,
    $X, Y \in \mathfrak{X} M$ be symplectic vector fields.
  Then
  \begin{equation}
    \lie{X}{Y} = \hamiltonian_{i_Y(i_X \omega)}.
  \end{equation}
\end{proposition}
\begin{proof}
  \begin{equation}
    i_{\lie{X}{Y}} \omega
    = (L_X \circ i_Y - i_Y \circ L_X) \omega
    = (L_X \circ i_Y) \omega
    = ((d \circ i_X + i_X \circ d) \circ i_Y) \omega
    = d(i_X(i_Y \omega))
    = - d(i_Y(i_X \omega)).
  \end{equation}
  We get the desired result from the definition of $\hamiltonian$.
\end{proof}
\begin{definition}
  Let $(M, \omega)$ be a symplectic manifold.
  Define the \textbf{Poisson bracket}
  $\poisson{\cdot}{\cdot} \colon \mathcal{F} M \to \mathcal{F} M$ by
  \begin{equation}
    \poisson{f}{g}
    := i_{\hamiltonian_g}(i_{\hamiltonian_f} \omega),\
    f, g \in \mathcal{F} M.
  \end{equation}
\end{definition}
\begin{corollary}
  Let $(M, \omega)$ be a symplectic manifold, $f, g \in \mathcal{F} M$.
  Then
  \begin{equation}
    \lie{\hamiltonian_f}{\hamiltonian_g}
    = \hamiltonian_{i_{\hamiltonian_g}(i_{\hamiltonian_f} \omega)}
    = \poisson{f}{g}.
  \end{equation}
\end{corollary}
\begin{proposition}[Leibniz rule holds for the Poisson bracket]
  Let $(M, \omega)$ be a symplectic manifold, $f, g, h \in \mathcal{F} M$.
  Then
  \begin{equation}
    \poisson{f}{g h} = \poisson{f}{g} h + g \poisson{f}{h}.
  \end{equation}
\end{proposition}
\begin{proof}
  $
    \poisson{f}{g h}
    = i_{\hamiltonian_{g h}}(i_{\hamiltonian_f} \omega)
    = i_{h \hamiltonian_g + g \hamiltonian_{h}}(i_{\hamiltonian_f} \omega)
    = (i_{\hamiltonian_g}(i_{\hamiltonian_f} \omega))\, h
      + g\, (i_{\hamiltonian_h}(i_{\hamiltonian_f} \omega)) 
    = \poisson{f}{g} h + g \poisson{f}{h}.
  $
\end{proof}
\begin{proposition}
  Let $(M, \omega)$ be a symplectic manifold, $f, g, h \in \mathcal{F} M$.
  Then
  \begin{equation}
    \poisson{f}{g} = L_{\hamiltonian_f} g.
  \end{equation}
\end{proposition}
\begin{proof}
  $
    \poisson{f}{g}
    = i_{\hamiltonian_g}(i_{\hamiltonian_f} \omega)
    = - i_{\hamiltonian_f} \circ i_{\hamiltonian_g} \omega
    = i_{\hamiltonian_f}(d g)
    = L_{\hamiltonian_f} g.
  $
\end{proof}
\begin{definition}
  Let $(M, \omega)$ be a symplectic manifold.
  Define
  ${\rm ad} \colon \mathcal{F} M \to (\mathcal{F} M \to \mathcal{F} M)$ by
  \begin{equation}
    {\rm ad}_f g := \poisson{f}{g},\ f, g \in \mathcal{F} M,
  \end{equation}
\end{definition}
\begin{proposition}
  Let $(M, \omega)$ be a symplectic manifold.
  Then
  \begin{equation}
    \lie{{\rm ad}_f}{{\rm ad}_g} = {\rm ad}_{\poisson{f}{g}}.
  \end{equation}
  (Here the bracket $\lie{\cdot}{\cdot}$ is the commutator of operators.)
\end{proposition}
\begin{proof}
  From the previous proposition it follows that
  \begin{equation}
    {\rm ad}_f = L_{X_f},\ f \in \mathcal{F} M.
  \end{equation}
  Hence,
  \begin{equation}
    \lie{{\rm ad}_f}{{\rm ad}_g}
    = \lie{L_{\hamiltonian_f}}{L_{\hamiltonian_g}}
    = L_{\lie{\hamiltonian_f}{\hamiltonian_g}}
    = L_{\hamiltonian_{\poisson{f}{g}}}
    = {\rm ad}_{\poisson{f}{g}}.
  \end{equation}
\end{proof}
\begin{corollary}
  Let $(M, \omega)$ be a symplectic manifold.
  Then $(\mathcal{F} M, \poisson{\cdot}{\cdot})$ is a Lie algebra over $\R$.
\end{corollary}
\begin{proof}
  Bilinearity and antisymmetry are trivial to check.
  The Jacobi identity is equivalent to the adjoint map being a Lie algebra
  homomorphism, which was the previous proposition.
\end{proof}
\begin{definition}
  Let
    $R$ be a commutative ring with unity ring,
    $(A, +, \cdot)$ be an $R$-module
    with additional structures of
    an associative algebra $(A, *)$ and
    a Lie algebra $(A, \poisson{\cdot}{\cdot})$.
  We say that $A$ is a Poisson algebra if the Lie bracket acts as a derivation,
  i.e., for all $f, g, h \in A$,
  \begin{equation}
    \poisson{f}{g * h} = \poisson{f}{g} * h + g * \poisson{f}{h}.
  \end{equation}
\end{definition}
\begin{corollary}
  Let $(M, \omega)$ be a symplectic manifold.
  Then $\mathcal{F} M$ is a Poisson algebra over $\R$.
  Here, addition, scalar multiplication, and multiplication are given by the
  corresponding pointwise operations, while the Lie bracket is given by the
  Poisson bracket.
\end{corollary}


\section{Discerete covariant exterior derivative}
\label{section:discrete_covariant_exterior_derivative}
\begin{definition}
  Let
    $R$ be a ring,
    $V$ be a finite-dimensional $R$-module,
    $\omega \in \Lambda^2 V^*$.
  We say that $\omega$ is \textbf{non-degenerate} or \textbf{symplectic}
  if the associated map
  \begin{equation}
    \tilde{\omega} \colon V \to V^*,\
    X \in V \mapsto \tilde{\omega}(X) := i_X \omega \in V^*,
  \end{equation}
  is an isomorphism.

  The pair $(V, \omega)$ is called a \textbf{symplectic module}
  (or \textbf{symplectic vector space} if $R$ is a field).
\end{definition}
\begin{proposition}
  Let
    $R$ be a ring without,
    $(V, \omega)$ be a finite-dimensional symplectic module over $R$.
  Assume that for any $x \in R,\ x + x = 0 \Rightarrow x = 0$.
  Then $\dim V$ is an even number.
\end{proposition}
\begin{proof}
  Let $n = \dim V$.
  In a basis of $V$ $\omega$ is represented by an antisymmetric matrix $A$.
  But then
  \begin{equation}
    \det A = \det(A^T) = \det(-A) = (-1)^n \det A.
  \end{equation}
  If $n$ is odd, then $\det A + \det A = 0$.
  By assumption this means that $\det A = 0$
  which contradicts the nondegeneracy of $\omega$.
  Hence, $n$ is even.
\end{proof}
\begin{definition}
  Let $M$ be a smooth manifold, $\omega \in \Omega^\bullet M$.
  We say that:
  \begin{enumerate}
    \item
      $\omega$ is \textbf{closed} if $d \omega = 0$
    \item
      $\omega$ is \textbf{exact} if there exists $\eta \in \Omega^\bullet M$
      such that $d \eta = \omega$.
  \end{enumerate}
\end{definition}
\begin{proposition}
  Let $M$ be a smooth manifold, $\omega \in \Omega^\bullet M$.
  If $\omega$ is exact, then it is closed.
\end{proposition}
\begin{proof}
  Let $\eta \in \Omega^\bullet M$ be such that $d \eta = \omega$.
  Then $d \omega = d (d \eta) = 0$, i.e., $\omega$ is closed.
\end{proof}
\begin{definition}
  Let $M$ be a smooth manifold, $\omega \in \Omega^2 M$.
  We say that $\omega$ is a \textbf{symplectic form}
  if it is non-degenerate
  (with base module $\mathfrak{X} M$ over $\mathcal{F} M$) and closed.

  The pair $(M, \omega)$ is called a \textbf{symplectic manifold}.
\end{definition}
\begin{proposition}
  Let $(M, \omega)$ be a symplectic manifold.
  Then $M$ is even-dimensional.
\end{proposition}
\begin{definition}
  Let $Q$ be a smooth manifold.
  Consider the cotangent bundle $T^* Q$ with bundle projection
  $\pi \colon T^* Q \to Q$
  with differential $d \pi \colon T(T^* Q) \to T Q$.
  Define the \textbf{tautological one-form}
  $\theta \colon T^* Q \to T^* (T^* Q)$ as follows:
  for any $(q, p) \in T^*Q$ (i.e,. $q \in Q$, $p \in \Hom(T_q Q, \R)$),
  \begin{equation}
    \restrict{\theta}{(q, p)}
    := p \circ \restrict{d \pi}{(q, p)} \in T^*_{(q, p)}(T^* Q).
  \end{equation}
  In other words, if we denote $M := T^* Q$, then $\theta$ is a section of its
  cotangent bundle $T^* M$, i.e., an $1$-form on $M$. 
\end{definition}
\begin{discussion}
  Let $Q$ be a smooth manifold, $\pi \colon T^* Q \to Q$ be the projection.
  Then a $1$-form on $Q$ is a section of $\colon T^* Q$, i.e., a smooth map
  $\mu \colon Q \to \colon T^* Q$ such that $\pi \circ \mu = \id_Q$.
  As such it has a pullback
  $\mu^* \colon \Omega^\bullet(T^* Q) \to \Omega^\bullet Q$.
\end{discussion}
\begin{proposition}
  Let
    $Q$ be a smooth manifold,
    $\theta$ be the tautological one-form on $T^* Q$,
    $\mu \in \Omega^1 Q$.
  Then
  \begin{equation}
    \mu^* \theta = \mu.
  \end{equation}
\end{proposition}
\begin{proof}
  Let $q \in Q$.
  Then
  \begin{equation}
    \restrict{\mu^* \theta}{q}
    = \restrict{\theta}{\mu q} \circ \restrict{d \mu}{q}
    = \restrict{\mu}{q} \circ \restrict{(d \pi)}{\mu q}
      \circ \restrict{d \mu}{q}
    = \restrict{\mu}{q} \circ \restrict{d(\pi \circ \mu)}{q}
    = \restrict{\mu}{q}.
  \end{equation}
  Since $q$ is arbitrary, $\mu^* \theta = \mu$.
\end{proof}
\begin{definition}
  Let
    $Q$ be a smooth manifold,
    $\theta \in \Omega^1(T^* Q)$ be the tautological one-form.
  Define $\omega := - d \theta$.
  The pair $(T^* Q, \omega)$ is called the \textbf{phase space} of $Q$.
  (In this setting $Q$ is usually called the \textbf{configuration space}.)
\end{definition}
\begin{remark}
  Let $Q$ be a smooth manifold.
  The elements of $T^* Q$ are of the form $(q, p)$ where $q \in Q$ and
  $p \in T^*_q Q = \Hom(T_q Q, \R)$.
  $q$ is called \textbf{generalised position}, while $p$ is called
  \textbf{generalised momentum}.
\end{remark}
\begin{proposition}
  Let
    $Q$ be a smooth manifold,
    $(T^* Q, \omega)$ be its phase space.
  Then $(T^* Q, \omega)$ is a symplectic manifold.
\end{proposition}
\begin{definition}
  Let $Q$ be a smooth manifold of dimension $n$.
  Consider a point $q_0 \in Q$ and let $(U, \hat{\varphi})$ be a chart around
  $q_0$, i.e., $U$ is a neighbourhood of $q_0$ and
  $\hat{\varphi} \colon U \to \R^n$ is a diffeomorphism.
  Let $\{\hat{q}^i \colon U \to \R\}_{i = 1}^n$ be the corresponding local
  coordinates, i.e., if $\{\pi^i \colon \R^n \to \R\}_{i = 1}^n$ are the
  projection maps, then $\{\hat{q}^i = \pi^i \circ \hat{\varphi}\}_{i = 1}^n$.
  Let $i \in \{1, ..., n\}$.
  Define \textbf{position coordinate} $q^i \colon T^* U \to \R$ by
  \begin{equation}
    q^i := \hat{q}^i \circ \restrict{\pi}{U}.
  \end{equation}
  Also, define \textbf{momentum coordinate} $p_i \colon T^* U \to \R$
  as follows: for any $(q, p) \in T^* U$,
  \begin{equation}
    p_i(q, p)
    := p\left(\restrict{\frac{\partial}{\partial \hat{q}^i}}{q}\right).
  \end{equation}
\end{definition}
\begin{proposition}
  Let
    $Q$ be a smooth manifold of dimension $n$,
    $q_0 \in Q$,
    $(U, \hat{\varphi})$ be a chart around $q_0$,
    $\{\hat{q}^i \colon U \to \R\}_{i = 1}^n$ be the corresponding local
      coordinates,
    $\{q^i \colon T^* U \to \R\}_{i = 1}^n$ be the corresponding position
      coordinates,
    $\{p_i \colon T^* U \to \R\}_{i = 1}^n$ be the corresponding momentum
      coordinates.
  Then the map $\varphi \colon T^* U \to \R^{2 n}$ defined by
  \begin{equation}
    \varphi(q, p) = (q^1(q, p), ..., q^n(q, p), p_1(q, p), ..., p_n(q, p))
  \end{equation}
  is a diffeomorphism, i.e., $(T^* U, \varphi)$ is a chart around $(q_0, 0)$.
  (The covector in $T^*_{q_0}$ does not matter, so we make the trivial choice by
  taking zero.)

  These local coordinates are called \textbf{generalised coordinates}.
\end{proposition}
\begin{remark}
  From now on, given a manifold $Q$ and a chart $(U, \hat{\varphi})$, unless
  stated otherwise, we will fix the notation and use the objects defined above:
  the projection map $\pi \colon T^* Q \to Q$, the tautological one-form
  $\theta$ and the canonical symplectic form $\omega = - d \theta$;
  for $i = 1, ..., n$ the coordinate maps $\hat{q}^i$, $q^i$, and $p_i$;
  the chart $(T^* U, \varphi)$.
\end{remark}
\begin{proposition}
  Let
    $Q$ be a smooth manifold,
    $\xi \in \Omega^1(T^* Q)$ has the following property:
    for any $1$-form $\mu$ on $Q$, $\mu^* \xi= 0$.
  Then $\xi = 0$.
\end{proposition}
\begin{proof}
  Let
    $n := \dim Q$, $(U, \hat{\varphi})$ be a chart on $Q$ and
    $\{f_i, g^i \in \mathcal{F}(T^* U)\}_{i = 1}^n$ be such that
  \begin{equation}
    \restrict{\xi}{U} = \sum_{i = 1}^n f_i\, d q^i + \sum_{i = 1}^n g^i\, d p_i.
  \end{equation}
  Take arbitrary $\{h_j \in \mathcal{F} U\}_{j = 1}^n$ so that
  \begin{equation}
    \restrict{\mu}{U} = \sum_{j = 1}^n h_j\, d \hat{q}^j.
  \end{equation}
  Note that
  $q^i \circ \restrict{\mu}{U} = \hat{q}^i$ and
  $p_i \circ \restrict{\mu}{U} = h_i$.
  Hence,
  \begin{equation}
    0 
    = \restrict{(\mu^* \xi)}{U}
    = \sum_{i = 1}^n (f_i \circ \restrict{\mu}{U})\,
      d(q^i \circ \restrict{\mu}{U})
    + \sum_{i = 1}^n (g^i \circ \restrict{\mu}{U})\,
      d(p_i \circ \restrict{\mu}{U})
    = \sum_{i = 1}^n (f_i \circ \restrict{\mu}{U})\, d \hat{q}^i
    + \sum_{i = 1}^n (g^i \circ \restrict{\mu}{U})\, d h_i.
  \end{equation}
  Fix $q_0 \in U$, $p_0 \in T^*_{q_0} Q$ so that $(p_0, q_0) \in T^* U$.
  Denote
  \begin{equation}
    c_i
    :=
    p_0\left(\restrict{\frac{\partial}{\partial \hat{q}^i}}{q_0}\right),
    i = 1, ..., n,
  \end{equation}
  so that
  \begin{equation}
    p_0 = \sum_{i = 1}^n c_i \restrict{d \hat{q}^i}{q_0}.
  \end{equation}
  \begin{enumerate}
    \item
      We will first prove that
      for any $i \in \{1, ..., n\}$, $f_i(q_0, p_0) = 0$.
      Define the constant functions
      \begin{equation}
        h_i(q) := c_i,\ i \in \{1, ..., n\},\ q \in U.
      \end{equation}
      Then for any $i \in \{1, ..., n\}$, $d h_i = 0$.
      Hence,
      \begin{equation}
        \begin{split}
          0
          & = \restrict{(\mu^* \xi)}{q_0} \\
          & = \sum_{i = 1}^n
              f_i(q_0, \sum_{j = 1}^n h_j(q_0) \restrict{d \hat{q}^j}{q_0})\,
              \restrict{d \hat{q}^i}{q_0} \\
          & = \sum_{i = 1}^n
              f_i(q_0, \sum_{j = 1}^n c_j \restrict{d \hat{q}^j}{q_0})\,
              \restrict{d \hat{q}^i}{q_0} \\
          & = \sum_{i = 1}^n f_i(q_0, p_0)\, \restrict{d \hat{q}^i}{q_0}.
        \end{split}
      \end{equation}
      Therefore, for any $i \in \{1, ..., n\}$, $f_i(q_0, p_0) = 0$.
    \item
      We will now prove that
      for any $i \in \{1, ..., n\}$, $g^i(q_0, p_0) = 0$.
      Define the linear functions
      \begin{equation}
        h_i(q) := c_i + \hat{q}^i(q) - \hat{q}^i(q_0).
      \end{equation}
      Then for any $i \in \{1, ..., n\}$,
      $d h_i = d \hat{q}^i$ and $h_i(q) = c_i$.
      Hence,
      \begin{equation}
        \begin{split}
          0
          & = \restrict{(\mu^* \xi)}{q_0} \\
          & = \sum_{i = 1}^n
              g^i(q_0, \sum_{j = 1}^n h_j(q_0) \restrict{d \hat{q}^j}{q_0})\,
              \restrict{d h_i}{q_0} \\
          & = \sum_{i = 1}^n
              g^i(q_0, \sum_{j = 1}^n c_j \restrict{d \hat{q}^j}{q_0})\,
              \restrict{d \hat{q}^i}{q_0} \\
          & = \sum_{i = 1}^n g^i(q_0, p_0)\, \restrict{d \hat{q}^i}{q_0}.
        \end{split}
      \end{equation}
      Therefore, for any $i \in \{1, ..., n\}$, $g^i(q_0, p_0) = 0$.
  \end{enumerate}
  Since $(q_0, p_0) \in T^* U$ was arbitrary, we conclude that
  for any $i \in \{1, ..., n\}$, $f_i = g^i = 0$.
  Hence, $\restrict{\xi}{U} = 0$.
  Taking an atlas $\{(U_\alpha, \hat{\varphi}_\alpha)\}_{\alpha \in A}$ of $Q$
  (for some index set $A$), we conclude that $\xi = 0$.
\end{proof}
\begin{corollary}
  Let
    $Q$ be a smooth manifold,
    $\theta$ be the tautological one-form on $T^* Q$,
    $\eta \in \Omega^1(T^* Q)$ has the following property:
    for any $1$-form $\mu$ on $Q$, $\mu^* \eta = \mu$.
  Then $\eta = \theta$.
\end{corollary}
\begin{proof}
  Write $\eta = \theta + \xi$, i.e., $\xi := \eta - \theta$.
  Then, for any $\mu \in \Omega^1 Q$,
  \begin{equation}
    \mu
    = \mu^* \eta
    = \mu^* \theta + \mu^* \xi
    = \mu + \mu^* \xi
    \Rightarrow \mu^* \xi = 0.
  \end{equation}
  But from the previous proposition it follows that $\xi = 0$,
  and hence $\eta = \theta$.
\end{proof}
\begin{proposition}
  Let
    $Q$ be a smooth manifold of dimension $n$,
    $(U, \hat{\varphi})$ be a chart,
    $(q, p) \in T^* U$,
    $ i \in \{1, ..., n\}$.
  Then
  \begin{equation}
    \restrict{d \pi}{(q, p)}
    \left(\restrict{\frac{\partial}{\partial q^i}}{(q, p)}\right)
    = \restrict{\frac{\partial}{\partial \hat{q}^i}}{q}
  \end{equation}
  and
  \begin{equation}
    \restrict{d \pi}{(q, p)}
    \left(\restrict{\frac{\partial}{\partial p^i}}{(q, p)}\right)
    = 0.
  \end{equation}
\end{proposition}
\begin{proof}
  Let $f \colon Q \to \R$ be smooth.
  Define the functions
  $\hat{g} := f \circ \hat{\varphi}^{-1} \colon \R^n \to \R$ and
  $g := f \circ \pi \circ \varphi^{-1} \colon \R^{2 n} \to \R$.
  Let $(X^1, ..., X^n, Y^1, ..., Y^n) := \varphi(p, q) \in \R^n$.
  This means that $(X^1, ..., X^n) = \hat{\varphi}(q)$.
  Then
  \begin{equation}
    g(X^1, ..., X^n, Y^1, ..., Y^n)
    = f(\pi(q, p))
    = f(q)
    = \hat{g}(X^1, ..., X^n).
  \end{equation}
  Hence,
  \begin{equation}
    \begin{split}
      \frac{\partial g}{\partial x^i}(X^1, ..., X^n, Y^1, ..., Y^n)
      & = \lim_{h \to 0}
        \frac
        {g(X^1, ..., X^i + h, ..., X^n, Y^1, ..., Y^n)
         - g(X^1, ..., X^n, Y^1, ..., Y^n)}
        {h} \\
      & = \lim_{h \to 0}
        \frac{\hat{g}(X^1, ..., X^i + h, ..., X^n) - \hat{g}(X^1, ..., X^n)}{h}
        \\
      & = \frac{\partial \hat{g}}{\partial \hat{x}^i}(X^1, ..., X^n).
    \end{split}
  \end{equation}
  Similarly, since $g$ is constant with respect to the last $n$ coordinates,
  \begin{equation}
    \frac{\partial g}{\partial x^{n + i}}(X^1, ..., X^n, Y^1, ..., Y^n) = 0
  \end{equation}
  Take the standard coordinate systems (given by projections)
  $\{\hat{x}^k\}_{k = 1}^n$ on $\R^n$ and
  $\{x^k\}_{k = 1}^{2 n}$ on $\R^{2 n}$.
  Then, by the definitions of differential and partial derivative on manifold,
  \begin{equation}
    (\restrict{d \pi}{(q, p)}
      \left(\restrict{\frac{\partial}{\partial q^i}}{(q, p)}\right)) f
    = \restrict{\frac{\partial}{\partial q^i}}{(q, p)}(f \circ \pi)
    = \frac{\partial(f \circ \pi \circ \varphi^{-1})}{x^i}(\varphi(q, p))
    = \frac{\partial(f \circ \hat{\varphi}^{-1})}{\hat{x}^i}(\hat{\varphi}(q))
    = \restrict{\frac{\partial}{\partial \hat{q}^i}}{q} f,
  \end{equation}
  from which it follows that the first equality holds.
  Similarly,
  \begin{equation}
    (\restrict{d \pi}{(q, p)}
      \left(\restrict{\frac{\partial}{\partial p^i}}{(q, p)}\right)) f
    = \restrict{\frac{\partial}{\partial p^i}}{(q, p)}(f \circ \pi)
    = \frac{\partial(f \circ \pi \circ \varphi^{-1})}{x^{i + n}}(\varphi(q, p))
    = 0,
  \end{equation}
  from which it follows that the second equality holds.
\end{proof}
\begin{proposition}[Tautological one-form in generalised coordinates]
  Let
    $Q$ be a smooth manifold of dimension $n$,
    $(U, \hat{\varphi})$ be a chart.
  Then
  \begin{equation}
    \restrict{\theta}{U} = \sum_{i = 1}^n p_i\, d q^i.
  \end{equation}
\end{proposition}
\begin{proof}
  Let $(q, p) \in T^* U$.
  Recall that $\restrict{\theta}{(q, p)} = p \circ \restrict{d \pi}{(q, p)}$.
  Hence,
  \begin{equation}
    \restrict{\theta}{(q, p)}
    \left(\restrict{\frac{\partial}{\partial q^i}}{(q, p)}\right)
    = p\left(\restrict{\frac{\partial}{\partial \hat{q}^i}}{q}\right)
    = p_i(q, p),
  \end{equation}
  and
  \begin{equation}
    \restrict{\theta}{(q, p)}
    \left(\restrict{\frac{\partial}{\partial p^i}}{(q, p)}\right)
    = p(0)
    = 0.
  \end{equation}
  Therefore,
  \begin{equation}
    \restrict{\theta}{(q, p)}
    = \sum_{i = 1}^n
      \restrict{\theta}{(q, p)}
      \left(\restrict{\frac{\partial}{\partial q^i}}{(q, p)}\right)\,
      \restrict{d q^i}{(q, p)}
    + \sum_{i = 1}^n
      \restrict{\theta}{(q, p)}
      \left(\restrict{\frac{\partial}{\partial p^i}}{(q, p)}\right)\,
      \restrict{d p^i}{(q, p)}
    = \sum_{i = 1}^n p_i(q, p)\, \restrict{d q^i}{(q, p)},
  \end{equation}
  from which the proposition follows.
\end{proof}
\begin{corollary}[Canonical symplectic in generalised coordinates]
  Let
    $Q$ be a smooth manifold of dimension $n$,
    $(U, \hat{\varphi})$ be a chart.
  Then
  \begin{equation}
    \restrict{\omega}{U} = \sum_{i = 1}^n d q^i \wedge d p_i.
  \end{equation}
\end{corollary}
\begin{proof}
  Let $i \in \{1, ..., n\}$.
  Then
  \begin{equation}
    - d(p_i\, d q^i) = - d p_i \wedge d q^i = d q^i \wedge d p_i.
  \end{equation}
  Summing up for all $i$, we get the desired result.
\end{proof}
\begin{definition}
  Let $(M, \omega)$ be a symplectic manifold, $f \in \mathcal{F} M$.
  We say that $X \in \mathfrak{X} M$ is a \textbf{Hamiltonian vector field} for
  $f$ if
  \begin{equation}
    i_X \omega + d_0 f = 0.
  \end{equation}
\end{definition}
\begin{proposition}
  Let $(M, \omega)$ be a symplectic manifold, $f \in \mathcal{F} M$.
  Then there exists a unique Hamiltonian vector field for $f$.
\end{proposition}
\begin{proof}
  The non-degeneracy of $\omega$ means that we can interpret the symplectic form
  as the isomorphism $\tilde{\omega} \colon \mathfrak{X} M \to \Omega^1 M$,
  given by
  \begin{equation}
    (\tilde{\omega} X) := i_X \omega,\ X \in \mathfrak{X} M.
  \end{equation}
  Hence, the problem at hand has a unique solution
  $X = \tilde{\omega}^{-1}(- d_0 f)$.
\end{proof}
\begin{definition}
  Let $(M, \omega)$ be a symplectic manifold.
  Define the map $\hamiltonian \colon \mathcal{F} M \to \mathfrak{X} M$ by
  \begin{equation}
    \hamiltonian = - \tilde{\omega}^{-1} \circ d_0.
  \end{equation}
  It maps a function to its corresponding Hamiltonian vector field.
  We will write $\hamiltonian_f$ instead of $\hamiltonian(f)$
  for $f \in \mathcal{F} M$.
\end{definition}
\begin{proposition}
  Let
    $Q$ be a smooth manifold of dimension $n$,
    $(U, \hat{\varphi})$ be a chart on $Q$,
    $f \in \mathcal{F}(T^* Q)$.
  Then
  \begin{equation}
    \hamiltonian_f
    = \sum_{i = 1}^n
    \left(
      - \frac{\partial f}{\partial p^i} \frac{\partial}{\partial q^i}
      + \frac{\partial f}{\partial q^i} \frac{\partial}{\partial p^i}
    \right).
  \end{equation}
\end{proposition}
\begin{proof}
  First, note that
  $i_{\frac{\partial}{\partial q^i}} \omega = d p^i$ and
  $i_{\frac{\partial}{\partial p^i}} \omega = - d q^i$.
  Denote
  \begin{equation}
    X
    := \sum_{i = 1}^n
    \left(
      - \frac{\partial f}{\partial p^i} \frac{\partial}{\partial q^i}
      + \frac{\partial f}{\partial q^i} \frac{\partial}{\partial p^i}
    \right).
  \end{equation}
  Then
  \begin{equation}
    i_X \omega
    = \sum_{i}^n
    \left(
      - \frac{\partial f}{\partial p^i} i_{\frac{\partial}{\partial q^i}} \omega
      + \frac{\partial f}{\partial q^i} i_{\frac{\partial}{\partial p^i}} \omega
    \right)
    = \sum_{i}^n
    \left(
      - \frac{\partial f}{\partial p^i} d p^i
      - \frac{\partial f}{\partial q^i} d q^i
    \right)
    = - d f.
  \end{equation}
  Hence, $\hamiltonian_f = X$.
\end{proof}
\begin{proposition}
  Let $(M, \omega)$ be a symplectic manifold, $f, g \in \mathcal{F} M$.
  Then
  \begin{equation}
    \hamiltonian_{f g} = f \hamiltonian_g + g \hamiltonian_f.
  \end{equation}
\end{proposition}
\begin{proof}
  Follows directly from the Leibniz rule for $d_0$.
\end{proof}
\begin{definition}
  Let $(M, \omega)$ be a symplectic manifold, $X \in \mathfrak{X} M$.
  We say that $X$ is a \textbf{symplectic vector field} if $L_X \omega = 0$.
\end{definition}
\begin{remark}
  Since $L_{\lie{X}{Y}} = \lie{L_X}{L_Y} = L_X \circ L_Y - L_Y \circ L_X$,
  the symplectic vector fields form a Lie subalgebra of the Lie algebra of
  vector fields.
\end{remark}
\begin{proposition}
  Let $(M, \omega)$ be a symplectic manifold, $f \in \mathcal{F} M$.
  Then $\hamiltonian_f$ is a symplectic vector field.
\end{proposition}
\begin{proof}
  $
    L_{\hamiltonian_f} \omega
    = i_{\hamiltonian_f}(d \omega) + d(i_{\hamiltonian_f} \omega)
    = i_{\hamiltonian_f} 0 - d(d f)
    = 0.
  $
\end{proof}
\begin{proposition}
  Let
    $(M, \omega)$ be a symplectic manifold,
    $X \in \mathfrak{X} M$ be a symplectic vector fields.
  Then
  \begin{equation}
    d(i_X \omega) = 0.
  \end{equation}
\end{proposition}
\begin{proof}
  $
    d(i_X \omega)
    = L_X \omega - i_X(d \omega)
    = 0 - 0
    = 0.
  $
\end{proof}
\begin{proposition}
  Let $M$ be a smooth manifold, $X, Y \in \mathfrak{X} M$.
  Then
  \begin{equation}
    L_X \circ i_Y = i_{\lie{X}{Y}} + i_Y \circ L_X.
  \end{equation}
\end{proposition}
\begin{proposition}
  Let
    $(M, \omega)$ be a symplectic manifold,
    $X, Y \in \mathfrak{X} M$ be symplectic vector fields.
  Then
  \begin{equation}
    \lie{X}{Y} = \hamiltonian_{i_Y(i_X \omega)}.
  \end{equation}
\end{proposition}
\begin{proof}
  \begin{equation}
    i_{\lie{X}{Y}} \omega
    = (L_X \circ i_Y - i_Y \circ L_X) \omega
    = (L_X \circ i_Y) \omega
    = ((d \circ i_X + i_X \circ d) \circ i_Y) \omega
    = d(i_X(i_Y \omega))
    = - d(i_Y(i_X \omega)).
  \end{equation}
  We get the desired result from the definition of $\hamiltonian$.
\end{proof}
\begin{definition}
  Let $(M, \omega)$ be a symplectic manifold.
  Define the \textbf{Poisson bracket}
  $\poisson{\cdot}{\cdot} \colon \mathcal{F} M \to \mathcal{F} M$ by
  \begin{equation}
    \poisson{f}{g}
    := i_{\hamiltonian_g}(i_{\hamiltonian_f} \omega),\
    f, g \in \mathcal{F} M.
  \end{equation}
\end{definition}
\begin{corollary}
  Let $(M, \omega)$ be a symplectic manifold, $f, g \in \mathcal{F} M$.
  Then
  \begin{equation}
    \lie{\hamiltonian_f}{\hamiltonian_g}
    = \hamiltonian_{i_{\hamiltonian_g}(i_{\hamiltonian_f} \omega)}
    = \poisson{f}{g}.
  \end{equation}
\end{corollary}
\begin{proposition}[Leibniz rule holds for the Poisson bracket]
  Let $(M, \omega)$ be a symplectic manifold, $f, g, h \in \mathcal{F} M$.
  Then
  \begin{equation}
    \poisson{f}{g h} = \poisson{f}{g} h + g \poisson{f}{h}.
  \end{equation}
\end{proposition}
\begin{proof}
  $
    \poisson{f}{g h}
    = i_{\hamiltonian_{g h}}(i_{\hamiltonian_f} \omega)
    = i_{h \hamiltonian_g + g \hamiltonian_{h}}(i_{\hamiltonian_f} \omega)
    = (i_{\hamiltonian_g}(i_{\hamiltonian_f} \omega))\, h
      + g\, (i_{\hamiltonian_h}(i_{\hamiltonian_f} \omega)) 
    = \poisson{f}{g} h + g \poisson{f}{h}.
  $
\end{proof}
\begin{proposition}
  Let $(M, \omega)$ be a symplectic manifold, $f, g, h \in \mathcal{F} M$.
  Then
  \begin{equation}
    \poisson{f}{g} = L_{\hamiltonian_f} g.
  \end{equation}
\end{proposition}
\begin{proof}
  $
    \poisson{f}{g}
    = i_{\hamiltonian_g}(i_{\hamiltonian_f} \omega)
    = - i_{\hamiltonian_f} \circ i_{\hamiltonian_g} \omega
    = i_{\hamiltonian_f}(d g)
    = L_{\hamiltonian_f} g.
  $
\end{proof}
\begin{definition}
  Let $(M, \omega)$ be a symplectic manifold.
  Define
  ${\rm ad} \colon \mathcal{F} M \to (\mathcal{F} M \to \mathcal{F} M)$ by
  \begin{equation}
    {\rm ad}_f g := \poisson{f}{g},\ f, g \in \mathcal{F} M,
  \end{equation}
\end{definition}
\begin{proposition}
  Let $(M, \omega)$ be a symplectic manifold.
  Then
  \begin{equation}
    \lie{{\rm ad}_f}{{\rm ad}_g} = {\rm ad}_{\poisson{f}{g}}.
  \end{equation}
  (Here the bracket $\lie{\cdot}{\cdot}$ is the commutator of operators.)
\end{proposition}
\begin{proof}
  From the previous proposition it follows that
  \begin{equation}
    {\rm ad}_f = L_{X_f},\ f \in \mathcal{F} M.
  \end{equation}
  Hence,
  \begin{equation}
    \lie{{\rm ad}_f}{{\rm ad}_g}
    = \lie{L_{\hamiltonian_f}}{L_{\hamiltonian_g}}
    = L_{\lie{\hamiltonian_f}{\hamiltonian_g}}
    = L_{\hamiltonian_{\poisson{f}{g}}}
    = {\rm ad}_{\poisson{f}{g}}.
  \end{equation}
\end{proof}
\begin{corollary}
  Let $(M, \omega)$ be a symplectic manifold.
  Then $(\mathcal{F} M, \poisson{\cdot}{\cdot})$ is a Lie algebra over $\R$.
\end{corollary}
\begin{proof}
  Bilinearity and antisymmetry are trivial to check.
  The Jacobi identity is equivalent to the adjoint map being a Lie algebra
  homomorphism, which was the previous proposition.
\end{proof}
\begin{definition}
  Let
    $R$ be a commutative ring with unity ring,
    $(A, +, \cdot)$ be an $R$-module
    with additional structures of
    an associative algebra $(A, *)$ and
    a Lie algebra $(A, \poisson{\cdot}{\cdot})$.
  We say that $A$ is a Poisson algebra if the Lie bracket acts as a derivation,
  i.e., for all $f, g, h \in A$,
  \begin{equation}
    \poisson{f}{g * h} = \poisson{f}{g} * h + g * \poisson{f}{h}.
  \end{equation}
\end{definition}
\begin{corollary}
  Let $(M, \omega)$ be a symplectic manifold.
  Then $\mathcal{F} M$ is a Poisson algebra over $\R$.
  Here, addition, scalar multiplication, and multiplication are given by the
  corresponding pointwise operations, while the Lie bracket is given by the
  Poisson bracket.
\end{corollary}


\part{Physics}

\section{Continuous heat transport}
\label{section:continuous_diffusion}
\input{diffusion/continuous/introduction-discussion.tex}
\begin{discussion}
  We are going to state the governing laws for the discrete heat transport
  phenomenon in the strong formulation.

  Let:
  \begin{itemize}
    \item $D$ be a positive integer (physical dimension);
    \item $M$ be a manifold-like flat mesh of dimension $D$;
    \item $K$ be the Forman subdivision of $M$;
    \item $t_0 \in \R$ be the initial time, $I = [t_0, \infty)$.
  \end{itemize}
  Physical quantities in our model are:
  \begin{itemize}
    \item
      temperature $u [\Theta] \colon I \to C^0 K$;
    \item
      heat energy density $\tilde{Q} [E L^{-D}] \colon I \to C^0 K$;
    \item
      dual heat flow rate $\tilde{q} [E L^{1 - D} T^{-1}] \colon I \to C^1 K$;
  \end{itemize}
  The governing laws are the following.
  \begin{itemize}
    \item
      Let
        $K' := K \setminus \partial K$ be the interior of $K$,
        $\tilde{f} [E L^{-D} T^{-1}]\in C^0 K'$
        be the dual internal production rate;
      \textbf{Conservation of heat energy} is modeled by the equation
      \begin{equation}
        \restrict{\frac{\partial \tilde{Q}}{\partial t}}{K'_0} =
        \restrict{(\delta_1^\star \tilde{q})}{K'_0} + \tilde{f}.
      \end{equation}
    \item
      Let $u_0 [\Theta] \in C^0 K$ be the initial temperature.
      The \textbf{initial condition} is given by prescribed initial temperature:
      \begin{equation}
        u(t_0, \cdot) = u_0.
      \end{equation}
    \item
      Let $\tilde{\pi} [E L^{-D} \Theta^{-1}] \colon C^0 K \to C^0 K$
      be the dual volumetric heat capacity
      (its matrix in the standard basis is diagonal).
      The \textbf{relation between temperature change and heat energy change}
      is given by
      \begin{equation}
        \frac{\partial \tilde{Q}}{\partial t}
        = \tilde{\pi} \frac{\partial u}{\partial t}.
      \end{equation}
    \item
      Let
      $\tilde{\kappa} [E L^{2 - D} T^{-1} \Theta^{-1}] \colon C^1 K \to C^1 K$
      be the dual thermal conductivity
      (its matrix in the standard basis is diagonal).
      The \textbf{Fourier's constitutive relation} is given by
      \begin{equation}
        \tilde{q} = - \tilde{\kappa} (\delta_0 u).
      \end{equation}
  \end{itemize}
  We complete our model with boundary conditions.
  Let $\Gamma_D, \Gamma_N$ form a partition of $\partial K$
  into Dirichlet and Neumann boundary.
  \begin{itemize}
    \item
      Let $g_D [\Theta] \colon I \to C^0 \Gamma_D$
      be the prescribed temperature on the Dirichlet boundary $\Gamma_D$.
      The \textbf{Dirichlet boundary condition} is given by
      \begin{equation}
        \restrict{u}{\Gamma_D} = g_D.
      \end{equation}
    \item
      Let
        ${\bf n} [L^{-1}] \colon \Gamma_N \to \R^d$
          be the generalized exterior unit normal,
        $\widetilde{g_N} [E L^{1 - D} T^{-1}] \colon I \to C^0 \Gamma_N$
          of physical dimension $[E L^{1 - D} T^{-1}]$
          be the prescribed dual flow rate through the Neumann boundary.
      The \textbf{Neumann boundary condition} is given by
      \begin{equation}
        \restrict{\overline{\tilde{q}}}{\Gamma_N} \cdot {\bf n}
        = \widetilde{g_N}.
      \end{equation}
  \end{itemize}
\end{discussion}

\subsection{Primal strong formulation}
\subsubsection{Transient}
\phantom{T}
\begin{formulation}
  \label{cmc/diffusion/continuous/steady_state/primal_strong-formulation}
  [Primal strong formulation for the steady-state continuous heat
  equation using differential forms]
  Let:
  \begin{itemize}
    \item
      $D$ be a positive integer (space dimension);
    \item
      $X$ be an open region in $\R^D$ (the space region);
    \item
      $\tilde{f} [E L^{-D} T^{-1}] \in \Omega^0 X$
      be the dual internal production rate;
    \item
      $\Gamma_D, \Gamma_N$ form a partition of $\partial X$;
    \item
      $g_D [\Theta] \in \Omega^0 \Gamma_D$
      be the prescribed temperature on the Dirchlet boundary.
    \item
      $\widetilde{g_N} [E L^{1 - D} T^{-1}] \in \Omega^0 \Gamma_N$
      be the prescribed flow rate density through the Neumann boundary;
    \item
      $\tilde{\kappa} [E L^{-1} T^{-1} \Theta^{-1}]
      \colon \Omega^1 X \to \Omega^1 X$
      be the dual thermal conductivity.
  \end{itemize}
  We are solving the following problem for the unknown temperature
  $u [\Theta] \in \Omega^0 X$:
  \begin{subequations}
    \begin{alignat}{3}
      & (d_1^\star \circ \tilde{\kappa} \circ d_0) u
      && = \tilde{f} \qquad
      && [E L^{-D} T^{-1}], \\
      %
      & \tr_{\Gamma_D, 0} u
      && = g_D \qquad
      && [\Theta], \\
      %
      & - (\star_{\Gamma_N, D - 1} \circ \tr_{\Gamma_N, D - 1}
        \circ \star_1 \circ \tilde{\kappa} \circ d_0) u
      && = \widetilde{g_N} \qquad
      && [E L^{1 - D} T^{-1}].
    \end{alignat}
  \end{subequations}
\end{formulation}

\subsubsection{Steady-state}
\begin{formulation}
  \label{cmc/diffusion/continuous/steady_state/primal_strong-formulation}
  [Primal strong formulation for the steady-state continuous heat
  equation using differential forms]
  Let:
  \begin{itemize}
    \item
      $D$ be a positive integer (space dimension);
    \item
      $X$ be an open region in $\R^D$ (the space region);
    \item
      $\tilde{f} [E L^{-D} T^{-1}] \in \Omega^0 X$
      be the dual internal production rate;
    \item
      $\Gamma_D, \Gamma_N$ form a partition of $\partial X$;
    \item
      $g_D [\Theta] \in \Omega^0 \Gamma_D$
      be the prescribed temperature on the Dirchlet boundary.
    \item
      $\widetilde{g_N} [E L^{1 - D} T^{-1}] \in \Omega^0 \Gamma_N$
      be the prescribed flow rate density through the Neumann boundary;
    \item
      $\tilde{\kappa} [E L^{-1} T^{-1} \Theta^{-1}]
      \colon \Omega^1 X \to \Omega^1 X$
      be the dual thermal conductivity.
  \end{itemize}
  We are solving the following problem for the unknown temperature
  $u [\Theta] \in \Omega^0 X$:
  \begin{subequations}
    \begin{alignat}{3}
      & (d_1^\star \circ \tilde{\kappa} \circ d_0) u
      && = \tilde{f} \qquad
      && [E L^{-D} T^{-1}], \\
      %
      & \tr_{\Gamma_D, 0} u
      && = g_D \qquad
      && [\Theta], \\
      %
      & - (\star_{\Gamma_N, D - 1} \circ \tr_{\Gamma_N, D - 1}
        \circ \star_1 \circ \tilde{\kappa} \circ d_0) u
      && = \widetilde{g_N} \qquad
      && [E L^{1 - D} T^{-1}].
    \end{alignat}
  \end{subequations}
\end{formulation}

\subsection{Primal weak formulation}
\subsubsection{Transient}
\begin{discussion}
  Using the variables from
  \Cref{cmc/diffusion/continuous/transient/primal_strong-formulation}
  we are going to introduce an alternative (primal weak) formulation.
  Let $w \in \Ker \tr_{\Gamma_D, 0}$ be a test function
  (later on the differentiablity assumptions on $w$ can be weakened).
  Multiply the conservation of energy with $w$ and integrate over $X$:
  \begin{equation}
    \begin{split}
      \int_X w \wedge \frac{\partial Q}{\partial t}
      & = - \int_X (w \wedge d_{D - 1} q) + \int_X (w \wedge f) \\
      & = - \int_{\partial X} \tr_{\partial X, D - 1} (w \wedge q)
        + \int_X (d_0 w \wedge q)
        + \int_X (w \wedge f) \\
      & =
        - \int_{\Gamma_N}
          (\tr_{\Gamma_N, 0} w \wedge \tr_{\Gamma_N, D - 1} q)
        - \int_X (d_0 w \wedge \star_1 \tilde{\kappa} d_0 u)
        + \int_X (w \wedge f) \\
      & = - \int_{\Gamma_N} (\tr_{\Gamma_N, 0} w \wedge g_N)
        - \int_X (d_0 w \wedge \star_1 \tilde{\kappa} d_0 u)
        + \int_X (w \wedge f) \\
      & = - \int_{\Gamma_N} (\tr_{\Gamma_N, 0} w \wedge g_N)
        - \inner{d_0 w}{\tilde{\kappa} d_0 u}_{X, 1}
        + \int_X (w \wedge f).
    \end{split}
  \end{equation}
  We also have:
  \begin{equation}
    \int_X w \wedge \frac{\partial Q}{\partial t}
    = \int_X w \wedge
      \left(\star_0 \tilde{\pi} \frac{\partial u}{\partial t}\right)
    =  \inner{w}{\tilde{\pi} \frac{\partial u}{\partial t}}_{X, 0}.
  \end{equation}
  Equating both equations leads to the following (primal weak) formulation.
\end{discussion}

\begin{formulation}
  \label{cmc/diffusion/discrete/steady_state/primal_weak-formulation}
  [Primal weak formulation for the steady-state discrete heat equation
    with discrete differential forms]
  The following formulation is a discrete version of
  \Cref{cmc/diffusion/continuous/steady_state/primal_weak-formulation}.
  Let:
  \begin{itemize}
    \item
      Let $D$ be a positive integer (space dimension);
    \item
      $K$ be an oriented quasi-cubical \hyperref[cmc:mesh:definition]{mesh} of
      dimension $D$ representing the material body;
    \item
      $[K]$ be the fundamental class of $K$;
    \item
      $\tilde{\kappa} [E L^{2 - D} T^{-1} \Theta^{-1}]
      \colon C^1 K \to C^1 K$
      be the thermal conductivity of the material, such that for any edge
      $c_1 \in K_1$, there exists some $\lambda > 0$ such that
      $\tilde{\kappa}(c^1) = \lambda c^1$;
    \item
      $f [E T^{-1}] \in C^D K$ be the internal production rate;
    \item
      $\partial K = \Gamma_D \cup \Gamma_N$ be the partition of the boundary of
      $K$ into Dirichlet ($\Gamma_D$) and Neumann ($\Gamma_N$) regions;
    \item
      $[\Gamma_N]$ be the fundamental class of $\Gamma_N$, where $\Gamma_N$
      has the boundary orientation induced from $K$;
    \item
      $g_D [\Theta] \in C^0 \Gamma_D$
      be the prescribed temperature on the Dirichlet boundary;
    \item
      $g_N [E T^{-1}] \in C^{D - 1} \Gamma_N$
      be the prescribed flow rate on the Neumann boundary.
  \end{itemize}
  Our unknown is temperature $u [\Theta] \in C^0 K$.
  We are solving the following problem for $u$:
  \begin{subequations}
    \begin{alignat}{4}
      & \forall v \in \Ker \tr_{\Gamma_D, 0}, \enspace
      && \inner{\delta_0 v}{\tilde{\kappa} \delta_0 u}_{K, 1}
      && = (v \smile f)[K]
         - (\tr_{\Gamma_N, 0} v \smile g_N)[\Gamma_N] \qquad
      && [E T^{-1} \Theta], \\
      %
      &
      && \tr_{\Gamma_D, 0} u
      && = g_D \qquad
      && [\Theta].
    \end{alignat}
  \end{subequations}
  The flow rate $q [E T^{-1}] \in C^{D - 1} K$
  is calculated in the post-processing phase by the formula
  \begin{equation}
    q(c_{D - 1}) :=
    \begin{cases}
      (- \star_1 \tilde{\kappa} \delta_0 u)(c_{D - 1}),
        & c_{D - 1} \in K_{D - 1} \setminus (\Gamma_N)_{D - 1} \\
      g_N(c_{D - 1}), & c_{D - 1} \in (\Gamma_N)_{D - 1}
    \end{cases}.
  \end{equation}
\end{formulation}

\subsubsection{Steady-state}
\begin{formulation}
  \label{cmc/diffusion/discrete/steady_state/primal_weak-formulation}
  [Primal weak formulation for the steady-state discrete heat equation
    with discrete differential forms]
  The following formulation is a discrete version of
  \Cref{cmc/diffusion/continuous/steady_state/primal_weak-formulation}.
  Let:
  \begin{itemize}
    \item
      Let $D$ be a positive integer (space dimension);
    \item
      $K$ be an oriented quasi-cubical \hyperref[cmc:mesh:definition]{mesh} of
      dimension $D$ representing the material body;
    \item
      $[K]$ be the fundamental class of $K$;
    \item
      $\tilde{\kappa} [E L^{2 - D} T^{-1} \Theta^{-1}]
      \colon C^1 K \to C^1 K$
      be the thermal conductivity of the material, such that for any edge
      $c_1 \in K_1$, there exists some $\lambda > 0$ such that
      $\tilde{\kappa}(c^1) = \lambda c^1$;
    \item
      $f [E T^{-1}] \in C^D K$ be the internal production rate;
    \item
      $\partial K = \Gamma_D \cup \Gamma_N$ be the partition of the boundary of
      $K$ into Dirichlet ($\Gamma_D$) and Neumann ($\Gamma_N$) regions;
    \item
      $[\Gamma_N]$ be the fundamental class of $\Gamma_N$, where $\Gamma_N$
      has the boundary orientation induced from $K$;
    \item
      $g_D [\Theta] \in C^0 \Gamma_D$
      be the prescribed temperature on the Dirichlet boundary;
    \item
      $g_N [E T^{-1}] \in C^{D - 1} \Gamma_N$
      be the prescribed flow rate on the Neumann boundary.
  \end{itemize}
  Our unknown is temperature $u [\Theta] \in C^0 K$.
  We are solving the following problem for $u$:
  \begin{subequations}
    \begin{alignat}{4}
      & \forall v \in \Ker \tr_{\Gamma_D, 0}, \enspace
      && \inner{\delta_0 v}{\tilde{\kappa} \delta_0 u}_{K, 1}
      && = (v \smile f)[K]
         - (\tr_{\Gamma_N, 0} v \smile g_N)[\Gamma_N] \qquad
      && [E T^{-1} \Theta], \\
      %
      &
      && \tr_{\Gamma_D, 0} u
      && = g_D \qquad
      && [\Theta].
    \end{alignat}
  \end{subequations}
  The flow rate $q [E T^{-1}] \in C^{D - 1} K$
  is calculated in the post-processing phase by the formula
  \begin{equation}
    q(c_{D - 1}) :=
    \begin{cases}
      (- \star_1 \tilde{\kappa} \delta_0 u)(c_{D - 1}),
        & c_{D - 1} \in K_{D - 1} \setminus (\Gamma_N)_{D - 1} \\
      g_N(c_{D - 1}), & c_{D - 1} \in (\Gamma_N)_{D - 1}
    \end{cases}.
  \end{equation}
\end{formulation}

\subsection{Mixed weak formulation}
\subsubsection{Transient}
\begin{discussion}
  We are going to formulate the \textbf{mixed weak formulation for continuous
  heat transport with differential forms}.
  Consider the model
  \Cref{cmc/diffusion/continuous/model_with_differential_forms-discussion}
  with the same domains and variable names.
  Let $r [E T^{-1}] \in \Ker \left(\tr_{\Gamma_N, D - 1}\right)$.
  Then
  \begin{equation}
    \kappa^{-1} q = d_\star^D \tilde{u} = - \star_1 d_0 u.
  \end{equation}
  Hence,
  \begin{equation}
    \begin{split}
      \inner{r}{\kappa^{-1} q}_{X, D - 1}
      & = \inner{r}{- \star_1 d_0 u}_{X, D - 1} \\
      & = - \int_X d_0 u \wedge r \\
      & = - \int_{\partial X}
        \tr_{\partial X, 0} u \wedge \tr_{\partial X, D - 1} r
        + \int_{X} (u \wedge d_{D - 1} r) \\
      & = - \int_{\Gamma_D} g_D \wedge \tr_{\Gamma_D, D - 1} r
        + \inner{\star_0 u}{d_{D - 1} r} \\
      & = - \int_{\Gamma_D} g_D \wedge \tr_{\Gamma_D, D - 1} r
        + \inner{\tilde{u}}{d_{D - 1} r}.
    \end{split}
  \end{equation}
  Let $\tilde{w} [\Theta] \in \Omega^D X$.
  Taking the inner product of the conservation law with $\tilde{w}$ gives
  \begin{equation}
   \inner{\pi \frac{\partial \tilde{u}}{\partial t}}{\tilde{w}}_{X, D}
    = - \inner{d_{D - 1} q}{\tilde{w}}_{X, D} + \inner{f}{\tilde{w}}.
  \end{equation}
  This leads to the following (mixed weak) formulation.
\end{discussion}

\begin{formulation}
  \label{cmc/diffusion/continuous/steady_state/mixed_weak-formulation}
  [Mixed weak formulation for the steady-state continuous heat
  equation with differential forms]
  Let:
  \begin{itemize}
    \item
      $D$ be a positive integer (space dimension);
    \item
      $X$ be a $D$-dimensional open region, representing a material body;
    \item
      $f [E T^{-1}] \in \Omega^D X$ be the internal production rate;
    \item
      $\kappa [E L^{2 - D} T^{-1} \Theta^{-1}]
      \colon \Omega^{D - 1} X \to \Omega^{D - 1} X$
      be the thermal conductivity of the material;
    \item
      $\partial X = \Gamma_D \cup \Gamma_N$ be the partition of the boundary of
      $X$ into Dirichlet ($\Gamma_D$) and Neumann ($\Gamma_N$) regions;
    \item
      $g_D [\Theta] \in \Omega^0 \Gamma_D$
      be the prescribed temperature on the Dirichlet boundary;
    \item
      $g_N [E T^{-1}] \in \Omega^{D - 1} \Gamma_N$
      be the prescribed flow rate on the Neumann boundary.
  \end{itemize}
  Define the following operators:
  \begin{subequations}
    \begin{alignat}{3}
      & A \colon \Omega^{D - 1} X \times \Omega^{D - 1} X \to \R, \quad
      && A(r, s)
        := \inner{r}{\kappa^{-1} s}_{X, D - 1} \qquad
      && [E^{-1} T \Theta], \\
      %
      & B \colon \Omega^D X \times \Omega^{D - 1} X \to \R, \quad
      && B(\tilde{w}, r)
        := \inner{d_{D - 1} r}{\tilde{w}}_{X, D} \qquad
      && [L^{-D}], \\
      %
      & G \colon \Omega^{D - 1} X \to \R, \quad
      && G(r)
        := \int_{\Gamma_D} (\tr_{\Gamma_D, D - 1} r \wedge g_D) \qquad
      && [\Theta], \\
      %
      & F \colon \Omega^D X \to \R, \quad
      && F(\tilde{w}) := \inner{f}{\tilde{w}}_{X, D} \qquad
      && [E T^{-1} L^{-D}].
    \end{alignat}
  \end{subequations}
  Our unknowns are:
  \begin{itemize}
    \item $q [E T^{-1}] \in \Omega^{D - 1} X$ (heat flow rate);
    \item $\tilde{u} [\Theta L^D] \in \Omega^D X$ (dual temperature).
  \end{itemize}
  We are solving the following problem for $q$ and $u$:
  \begin{subequations}
    \begin{alignat}{4}
      & \forall r [E T^{-1}] \in \Ker \tr_{\Gamma_N, D - 1}, \quad
      && A(r, q) - B^T(r, \tilde{u})
      && = - G(r) \qquad
      && [E T^{-1} \Theta], \\
      %
      & \forall \tilde{w} [\Theta L^D] \in \Omega^D X, \quad
      && - B(\tilde{w}, q)
      && = - F(\tilde{w}) \qquad
      && [E T^{-1} \Theta], \\
      %
      &
      && \tr_{\Gamma_N, D - 1} q
      && = g_N \qquad
      && [E T^{-1}].
    \end{alignat}
  \end{subequations}
  The temperature $u [\Theta] \in \Omega^0 X$ is calculated in the
  post-processing phase by the formula
  \begin{equation}
    u(x) :=
    \begin{cases}
      (\star_D \tilde{u})(x), & x \notin \Gamma_D \\
      g_D(x), & x \in \Gamma_D
    \end{cases}.
  \end{equation}
\end{formulation}

\subsubsection{Steady-state}
\begin{formulation}
  \label{cmc/diffusion/continuous/steady_state/mixed_weak-formulation}
  [Mixed weak formulation for the steady-state continuous heat
  equation with differential forms]
  Let:
  \begin{itemize}
    \item
      $D$ be a positive integer (space dimension);
    \item
      $X$ be a $D$-dimensional open region, representing a material body;
    \item
      $f [E T^{-1}] \in \Omega^D X$ be the internal production rate;
    \item
      $\kappa [E L^{2 - D} T^{-1} \Theta^{-1}]
      \colon \Omega^{D - 1} X \to \Omega^{D - 1} X$
      be the thermal conductivity of the material;
    \item
      $\partial X = \Gamma_D \cup \Gamma_N$ be the partition of the boundary of
      $X$ into Dirichlet ($\Gamma_D$) and Neumann ($\Gamma_N$) regions;
    \item
      $g_D [\Theta] \in \Omega^0 \Gamma_D$
      be the prescribed temperature on the Dirichlet boundary;
    \item
      $g_N [E T^{-1}] \in \Omega^{D - 1} \Gamma_N$
      be the prescribed flow rate on the Neumann boundary.
  \end{itemize}
  Define the following operators:
  \begin{subequations}
    \begin{alignat}{3}
      & A \colon \Omega^{D - 1} X \times \Omega^{D - 1} X \to \R, \quad
      && A(r, s)
        := \inner{r}{\kappa^{-1} s}_{X, D - 1} \qquad
      && [E^{-1} T \Theta], \\
      %
      & B \colon \Omega^D X \times \Omega^{D - 1} X \to \R, \quad
      && B(\tilde{w}, r)
        := \inner{d_{D - 1} r}{\tilde{w}}_{X, D} \qquad
      && [L^{-D}], \\
      %
      & G \colon \Omega^{D - 1} X \to \R, \quad
      && G(r)
        := \int_{\Gamma_D} (\tr_{\Gamma_D, D - 1} r \wedge g_D) \qquad
      && [\Theta], \\
      %
      & F \colon \Omega^D X \to \R, \quad
      && F(\tilde{w}) := \inner{f}{\tilde{w}}_{X, D} \qquad
      && [E T^{-1} L^{-D}].
    \end{alignat}
  \end{subequations}
  Our unknowns are:
  \begin{itemize}
    \item $q [E T^{-1}] \in \Omega^{D - 1} X$ (heat flow rate);
    \item $\tilde{u} [\Theta L^D] \in \Omega^D X$ (dual temperature).
  \end{itemize}
  We are solving the following problem for $q$ and $u$:
  \begin{subequations}
    \begin{alignat}{4}
      & \forall r [E T^{-1}] \in \Ker \tr_{\Gamma_N, D - 1}, \quad
      && A(r, q) - B^T(r, \tilde{u})
      && = - G(r) \qquad
      && [E T^{-1} \Theta], \\
      %
      & \forall \tilde{w} [\Theta L^D] \in \Omega^D X, \quad
      && - B(\tilde{w}, q)
      && = - F(\tilde{w}) \qquad
      && [E T^{-1} \Theta], \\
      %
      &
      && \tr_{\Gamma_N, D - 1} q
      && = g_N \qquad
      && [E T^{-1}].
    \end{alignat}
  \end{subequations}
  The temperature $u [\Theta] \in \Omega^0 X$ is calculated in the
  post-processing phase by the formula
  \begin{equation}
    u(x) :=
    \begin{cases}
      (\star_D \tilde{u})(x), & x \notin \Gamma_D \\
      g_D(x), & x \in \Gamma_D
    \end{cases}.
  \end{equation}
\end{formulation}


\section{Discrete heat transport}
\label{section:discrete_diffusion}
\input{restriction_of_free_vector_space-notation.tex}
\input{generalized-external_normal-notation.tex}
\begin{discussion}
  We are going to state the governing laws for the discrete heat transport
  phenomenon in the strong formulation.

  Let:
  \begin{itemize}
    \item $D$ be a positive integer (physical dimension);
    \item $M$ be a manifold-like flat mesh of dimension $D$;
    \item $K$ be the Forman subdivision of $M$;
    \item $t_0 \in \R$ be the initial time, $I = [t_0, \infty)$.
  \end{itemize}
  Physical quantities in our model are:
  \begin{itemize}
    \item
      temperature $u [\Theta] \colon I \to C^0 K$;
    \item
      heat energy density $\tilde{Q} [E L^{-D}] \colon I \to C^0 K$;
    \item
      dual heat flow rate $\tilde{q} [E L^{1 - D} T^{-1}] \colon I \to C^1 K$;
  \end{itemize}
  The governing laws are the following.
  \begin{itemize}
    \item
      Let
        $K' := K \setminus \partial K$ be the interior of $K$,
        $\tilde{f} [E L^{-D} T^{-1}]\in C^0 K'$
        be the dual internal production rate;
      \textbf{Conservation of heat energy} is modeled by the equation
      \begin{equation}
        \restrict{\frac{\partial \tilde{Q}}{\partial t}}{K'_0} =
        \restrict{(\delta_1^\star \tilde{q})}{K'_0} + \tilde{f}.
      \end{equation}
    \item
      Let $u_0 [\Theta] \in C^0 K$ be the initial temperature.
      The \textbf{initial condition} is given by prescribed initial temperature:
      \begin{equation}
        u(t_0, \cdot) = u_0.
      \end{equation}
    \item
      Let $\tilde{\pi} [E L^{-D} \Theta^{-1}] \colon C^0 K \to C^0 K$
      be the dual volumetric heat capacity
      (its matrix in the standard basis is diagonal).
      The \textbf{relation between temperature change and heat energy change}
      is given by
      \begin{equation}
        \frac{\partial \tilde{Q}}{\partial t}
        = \tilde{\pi} \frac{\partial u}{\partial t}.
      \end{equation}
    \item
      Let
      $\tilde{\kappa} [E L^{2 - D} T^{-1} \Theta^{-1}] \colon C^1 K \to C^1 K$
      be the dual thermal conductivity
      (its matrix in the standard basis is diagonal).
      The \textbf{Fourier's constitutive relation} is given by
      \begin{equation}
        \tilde{q} = - \tilde{\kappa} (\delta_0 u).
      \end{equation}
  \end{itemize}
  We complete our model with boundary conditions.
  Let $\Gamma_D, \Gamma_N$ form a partition of $\partial K$
  into Dirichlet and Neumann boundary.
  \begin{itemize}
    \item
      Let $g_D [\Theta] \colon I \to C^0 \Gamma_D$
      be the prescribed temperature on the Dirichlet boundary $\Gamma_D$.
      The \textbf{Dirichlet boundary condition} is given by
      \begin{equation}
        \restrict{u}{\Gamma_D} = g_D.
      \end{equation}
    \item
      Let
        ${\bf n} [L^{-1}] \colon \Gamma_N \to \R^d$
          be the generalized exterior unit normal,
        $\widetilde{g_N} [E L^{1 - D} T^{-1}] \colon I \to C^0 \Gamma_N$
          of physical dimension $[E L^{1 - D} T^{-1}]$
          be the prescribed dual flow rate through the Neumann boundary.
      The \textbf{Neumann boundary condition} is given by
      \begin{equation}
        \restrict{\overline{\tilde{q}}}{\Gamma_N} \cdot {\bf n}
        = \widetilde{g_N}.
      \end{equation}
  \end{itemize}
\end{discussion}

\subsection{Primal strong formulation}
\subsubsection{Steady-state}
\begin{formulation}
  \label{cmc/diffusion/discrete/steady_state/primal_strong_with_normals-discussion}
  [Primal strong formulation for the steady-state discrete heat equation
    with discrete differential forms]
  Let:
  \begin{itemize}
    \item
      $D \in \N$;
    \item
      $M$ be a manifold-like flat mesh of dimension $D$;
    \item
      $K$ be the Forman subdivision of $M$;
    \item
      $K' := K \setminus \partial K$ be the interior of $K$;
    \item
      $\tilde{f} [E L^{-D} T^{-1}] \in C^0 K$
      be the dual internal production rate;
    \item
      $\Gamma_D, \Gamma_N$ form a partition of $(\partial K)_0$;
    \item
      ${\bf n} [L^{-1}] \colon \Gamma_N \to \R^D$
      be the generalized exterior unit normal;
    \item
      $g_D [\Theta] \in C^0 \Gamma_D$
      be the prescribed temperature on the Dirichlet boundary;
    \item
      $\widetilde{g_N} [E L^{1 - D} T^{-1}] \in C^0 \Gamma_N$
      be the prescribed flow rate density through the Neumann Boundary;
    \item
      $\tilde{\pi} [E L^{-D} \Theta^{-1}] \colon C^0 K \to C^0 K$
      be the dual heat capacity (its matrix in the standard basis is diagonal);
    \item
      $\tilde{\kappa} [E L^{-D} T^{-1} \Theta^{-1}] \colon C^1 K \to C^1 K$
      be the dual conductivity (its matrix in the standard basis is diagonal);
  \end{itemize}
  We are solving the following problem for $u [\Theta] \in C^0 K$:
  \begin{subequations}
    \begin{alignat}{4}
      & (\tr_{K'_0} \circ \delta_1^\star \circ \tilde{\kappa} \circ \delta_0) u
      && = \tr_{K'_0} \tilde{f} \quad
      && (\text{balance of heat energy}) \quad
      && [E L^{-D} T^{-1}], \\
      %
      & \tr_{\Gamma_D, 0} u
      && = g_D \quad
      && (\text{Dirichlet boundary condition}) \quad
      && [\Theta], \\
      %
      & - \restrict{\overline{(\tilde{\kappa} \circ \delta_0) u}}{\Gamma_N}
        \cdot {\bf n}
      && = \widetilde{g_N} \quad
      && (\text{Neumann boundary condition}) \quad
      && [E L^{1 - D} T^{-1}].
    \end{alignat}
  \end{subequations}
\end{formulation}

\subsubsection{Transient}
\begin{formulation}
  \label{cmc/diffusion/discrete/steady_state/primal_strong_with_normals-discussion}
  [Primal strong formulation for the steady-state discrete heat equation
    with discrete differential forms]
  Let:
  \begin{itemize}
    \item
      $D \in \N$;
    \item
      $M$ be a manifold-like flat mesh of dimension $D$;
    \item
      $K$ be the Forman subdivision of $M$;
    \item
      $K' := K \setminus \partial K$ be the interior of $K$;
    \item
      $\tilde{f} [E L^{-D} T^{-1}] \in C^0 K$
      be the dual internal production rate;
    \item
      $\Gamma_D, \Gamma_N$ form a partition of $(\partial K)_0$;
    \item
      ${\bf n} [L^{-1}] \colon \Gamma_N \to \R^D$
      be the generalized exterior unit normal;
    \item
      $g_D [\Theta] \in C^0 \Gamma_D$
      be the prescribed temperature on the Dirichlet boundary;
    \item
      $\widetilde{g_N} [E L^{1 - D} T^{-1}] \in C^0 \Gamma_N$
      be the prescribed flow rate density through the Neumann Boundary;
    \item
      $\tilde{\pi} [E L^{-D} \Theta^{-1}] \colon C^0 K \to C^0 K$
      be the dual heat capacity (its matrix in the standard basis is diagonal);
    \item
      $\tilde{\kappa} [E L^{-D} T^{-1} \Theta^{-1}] \colon C^1 K \to C^1 K$
      be the dual conductivity (its matrix in the standard basis is diagonal);
  \end{itemize}
  We are solving the following problem for $u [\Theta] \in C^0 K$:
  \begin{subequations}
    \begin{alignat}{4}
      & (\tr_{K'_0} \circ \delta_1^\star \circ \tilde{\kappa} \circ \delta_0) u
      && = \tr_{K'_0} \tilde{f} \quad
      && (\text{balance of heat energy}) \quad
      && [E L^{-D} T^{-1}], \\
      %
      & \tr_{\Gamma_D, 0} u
      && = g_D \quad
      && (\text{Dirichlet boundary condition}) \quad
      && [\Theta], \\
      %
      & - \restrict{\overline{(\tilde{\kappa} \circ \delta_0) u}}{\Gamma_N}
        \cdot {\bf n}
      && = \widetilde{g_N} \quad
      && (\text{Neumann boundary condition}) \quad
      && [E L^{1 - D} T^{-1}].
    \end{alignat}
  \end{subequations}
\end{formulation}

\begin{discussion}
  Consider \Cref{cmc/diffusion/discrete/transient/primal_strong_with_normals-formulation}.
  This formulation is discrete in space but continuous in time.
  In order to numerically solve it we need to discretize the time variable.
  We will use the trapezoidal (Crank-Nicolson) method.

  Let
    $\tau [T] \in \R^+$ be the time step,
    $i \in \N$,
    $t_i := t_0 + i \tau$,
    $y_i [\Theta] := u(t_i, \cdot) \in C^0 K$,
    $\rho_0 [E L^{-D} T^{-1} \Theta^{-1}]
      := \delta_1^\star \circ \kappa_1 \circ \delta_0$.
  Integrating the conservation of heat energy in $[t_i, t_{i + 1}]$, we get
  \begin{equation}
    \restrict{(\pi_0 y_{i + 1} - \pi_0 y_i)}{K'_0}
    = - \int_{t_i}^{t_{i + 1}} \restrict{(\rho_0 u(t, \cdot))}{K'_0}\, d t
    + \int_{t_i}^{t_{i + 1}} f\, d t
    \approx
    - \frac{\tau}{2} \restrict{(\rho_0 y_i + \rho_0 y_{i + 1})}{K'_0} + \tau f.
  \end{equation}
  Rearranging, we get the discretized equation
  \begin{equation}
    \restrict{((\pi_0 + \frac{\tau}{2} \rho_0) y_{i + 1})}{K'_0}
    = \restrict{((\pi_0 - \frac{\tau}{2} \rho_0) y_i)}{K'_0} + \tau f.
  \end{equation}
  Define the $\dim C^0 K \times \dim C^0 K$ matrices of physical dimension
  $[E L^{-D} \Theta^{-1}]$
  \begin{equation}
    A := \pi_0 + \frac{\tau}{2} \rho_0,\ B := \pi_0 - \frac{\tau}{2} \rho_0.
  \end{equation}
  The discretized in time (space is already discrete) temperature
  $y [\Theta] \colon \N \to C^0 K$ is find iteratively as follows.
  $y_0 = u_0$ and for any $i > 0$, $y_i$ is solution to the following problem:
  \begin{subequations}
    \begin{alignat}{4}
      & \restrict{(A y_i)}{K'_0}
      && = \restrict{(B y_{i - 1})}{K'_0} + \tau f \quad
      && (\text{balance of heat energy}) \quad
      && [E L^{-D}], \\
      %
      & \restrict{y_i}{\Gamma_D}
      && = g_D \quad
      && (\text{Dirichlet boundary condition}) \quad
      && [\Theta], \\
      %
      & - \restrict{\overline{(\kappa_1 \circ \delta_0) y_i}}{\Gamma_N}
        \cdot {\bf n}
      && = g_N \quad
      && (\text{Neumann boundary condition}) \quad
      && [E L^{1 - D} T^{-1}].
    \end{alignat}
  \end{subequations}
  Of course, in practice we solve it for a finite number of time steps.
  Usually, we compare tho adjacent solutions $y_i$ and $y_{i + 1}$ and stop when
  the relative error is sufficiently small, i.e., until we reach a steady state.
  % Moreover, if $\pi_0$ and $\kappa_1$ are constants, and $K$ has mesh size
  % $h$, then $h^2 / \tau$ should be close to the thermal diffusivity
  % $\kappa_1 / \pi_0$ (both of physical dimension $[L^2 T^{-1}]$).
\end{discussion}

\subsection{Primal weak formulation}
\subsubsection{Steady-state}
\begin{formulation}
  \label{cmc/diffusion/discrete/steady_state/primal_weak-formulation}
  [Primal weak formulation for the steady-state discrete heat equation
    with discrete differential forms]
  The following formulation is a discrete version of
  \Cref{cmc/diffusion/continuous/steady_state/primal_weak-formulation}.
  Let:
  \begin{itemize}
    \item
      Let $D$ be a positive integer (space dimension);
    \item
      $K$ be an oriented quasi-cubical \hyperref[cmc:mesh:definition]{mesh} of
      dimension $D$ representing the material body;
    \item
      $[K]$ be the fundamental class of $K$;
    \item
      $\tilde{\kappa} [E L^{2 - D} T^{-1} \Theta^{-1}]
      \colon C^1 K \to C^1 K$
      be the thermal conductivity of the material, such that for any edge
      $c_1 \in K_1$, there exists some $\lambda > 0$ such that
      $\tilde{\kappa}(c^1) = \lambda c^1$;
    \item
      $f [E T^{-1}] \in C^D K$ be the internal production rate;
    \item
      $\partial K = \Gamma_D \cup \Gamma_N$ be the partition of the boundary of
      $K$ into Dirichlet ($\Gamma_D$) and Neumann ($\Gamma_N$) regions;
    \item
      $[\Gamma_N]$ be the fundamental class of $\Gamma_N$, where $\Gamma_N$
      has the boundary orientation induced from $K$;
    \item
      $g_D [\Theta] \in C^0 \Gamma_D$
      be the prescribed temperature on the Dirichlet boundary;
    \item
      $g_N [E T^{-1}] \in C^{D - 1} \Gamma_N$
      be the prescribed flow rate on the Neumann boundary.
  \end{itemize}
  Our unknown is temperature $u [\Theta] \in C^0 K$.
  We are solving the following problem for $u$:
  \begin{subequations}
    \begin{alignat}{4}
      & \forall v \in \Ker \tr_{\Gamma_D, 0}, \enspace
      && \inner{\delta_0 v}{\tilde{\kappa} \delta_0 u}_{K, 1}
      && = (v \smile f)[K]
         - (\tr_{\Gamma_N, 0} v \smile g_N)[\Gamma_N] \qquad
      && [E T^{-1} \Theta], \\
      %
      &
      && \tr_{\Gamma_D, 0} u
      && = g_D \qquad
      && [\Theta].
    \end{alignat}
  \end{subequations}
  The flow rate $q [E T^{-1}] \in C^{D - 1} K$
  is calculated in the post-processing phase by the formula
  \begin{equation}
    q(c_{D - 1}) :=
    \begin{cases}
      (- \star_1 \tilde{\kappa} \delta_0 u)(c_{D - 1}),
        & c_{D - 1} \in K_{D - 1} \setminus (\Gamma_N)_{D - 1} \\
      g_N(c_{D - 1}), & c_{D - 1} \in (\Gamma_N)_{D - 1}
    \end{cases}.
  \end{equation}
\end{formulation}

\begin{discussion}
  \label{cmc/diffusion/discrete/steady_state/primal_weak_solve-discussion}
  We are going to derive a solution to
  \Cref{cmc/diffusion/discrete/steady_state/primal_weak-formulation}.
  For any $p \in \{0, ..., D\}$ denote
  \begin{equation}
    n_p := \abs{K_p} = \dim(C_p K) = \dim(C^p K).
  \end{equation}
  The cochains $(N^0, ..., N^{n_0 - 1})$ form the standard basis of $C^0 K$.
  Define the matrix ${\bf A} \in M_{n_0 \times n_0}(\R)$ by
  \begin{equation}
    {\bf A}_{i, j} := \inner{\delta N^i}{\tilde{\kappa} \delta_0 N^i}_{K, 1}, \
    i, j = 0, ..., n_0 - 1,
  \end{equation}
  and the vectors ${\bf F}, {\bf G}, {\bf H} \in \R^{n_0}$ by
  \begin{subequations}
    \begin{alignat}{3}
      & {\bf F}_i
      && := (N^i \smile f)[K], \enspace
      && i = 0, ..., n_0 - 1, \\
      %
      & {\bf G}_i
      && := (\tr_{\Gamma_N, 0} N^i \smile g_N)[\Gamma_N], \enspace
      && i = 0, ..., n_0 - 1, \\
      %
      & {\bf H}
      && := {\bf F} - {\bf G}.
      &&
    \end{alignat}
  \end{subequations}
  Denote the unknown coefficients of $u$ as
  $\{{\bf U}_j\}_{j = 0}^{n_0 - 1}$,
  i.e.,
  \begin{equation}
    u = \sum_{j = 0}^{n_0 - 1} {\bf U}_j N^j
  \end{equation}
  Finally, let $J$ be the set of nodes on the Dirichlet boundary $\Gamma_D$,
  and $\overline{J} := \{0, ..., n_0 - 1\} \setminus J$.
  We get the system
  \begin{subequations}
    \begin{alignat}{3}
      & \sum_{j = 0}^{n_0 - 1} {\bf A}_{i, j} {\bf U}_j
      && = {\bf H}_i, \enspace
      && i \in \overline{J}, \\
      %
      & {\bf U}_i
      && = g_D(N_i), \enspace
      && i \in J.
    \end{alignat}
  \end{subequations}
  This leads to the system of equations
  \begin{equation}
    \sum_{j \in \overline{J}} {\bf A}_{i, j} x_j
    = {\bf H}_i - \sum_{j \in J} {\bf A}_{i, j} g_D(N_j),\ i \in \overline{J}.
  \end{equation}
  Denote by ${\bf A}$ the restriction of $A$ to the rows and columns in
  $\overline{J}$, by $\overline{\bf H}$ the right-hand side of the above
  equation (again only for the indices in $\overline{J}$),
  and by $\overline{\bf U}$ the restriction of ${\bf U}$ on the indices
  in $\overline{J}$.
  We arrive at the final linear system with positive-definite matrix
  $\overline{\bf A}$:
  \begin{equation}
    \overline{\bf A} \overline{\bf U} = \overline{\bf H}.
  \end{equation}
  After solving it, we get the final solution
  \begin{equation}
    \overline{\bf U}_i =
    \begin{cases}
      \overline{\bf U}_i, & i \in \overline{J} \\
      g_D(N_i), & i \in J
    \end{cases}.
  \end{equation}
\end{discussion}

\subsubsection{Transient}
\begin{formulation}
  \label{cmc/diffusion/discrete/steady_state/primal_weak-formulation}
  [Primal weak formulation for the steady-state discrete heat equation
    with discrete differential forms]
  The following formulation is a discrete version of
  \Cref{cmc/diffusion/continuous/steady_state/primal_weak-formulation}.
  Let:
  \begin{itemize}
    \item
      Let $D$ be a positive integer (space dimension);
    \item
      $K$ be an oriented quasi-cubical \hyperref[cmc:mesh:definition]{mesh} of
      dimension $D$ representing the material body;
    \item
      $[K]$ be the fundamental class of $K$;
    \item
      $\tilde{\kappa} [E L^{2 - D} T^{-1} \Theta^{-1}]
      \colon C^1 K \to C^1 K$
      be the thermal conductivity of the material, such that for any edge
      $c_1 \in K_1$, there exists some $\lambda > 0$ such that
      $\tilde{\kappa}(c^1) = \lambda c^1$;
    \item
      $f [E T^{-1}] \in C^D K$ be the internal production rate;
    \item
      $\partial K = \Gamma_D \cup \Gamma_N$ be the partition of the boundary of
      $K$ into Dirichlet ($\Gamma_D$) and Neumann ($\Gamma_N$) regions;
    \item
      $[\Gamma_N]$ be the fundamental class of $\Gamma_N$, where $\Gamma_N$
      has the boundary orientation induced from $K$;
    \item
      $g_D [\Theta] \in C^0 \Gamma_D$
      be the prescribed temperature on the Dirichlet boundary;
    \item
      $g_N [E T^{-1}] \in C^{D - 1} \Gamma_N$
      be the prescribed flow rate on the Neumann boundary.
  \end{itemize}
  Our unknown is temperature $u [\Theta] \in C^0 K$.
  We are solving the following problem for $u$:
  \begin{subequations}
    \begin{alignat}{4}
      & \forall v \in \Ker \tr_{\Gamma_D, 0}, \enspace
      && \inner{\delta_0 v}{\tilde{\kappa} \delta_0 u}_{K, 1}
      && = (v \smile f)[K]
         - (\tr_{\Gamma_N, 0} v \smile g_N)[\Gamma_N] \qquad
      && [E T^{-1} \Theta], \\
      %
      &
      && \tr_{\Gamma_D, 0} u
      && = g_D \qquad
      && [\Theta].
    \end{alignat}
  \end{subequations}
  The flow rate $q [E T^{-1}] \in C^{D - 1} K$
  is calculated in the post-processing phase by the formula
  \begin{equation}
    q(c_{D - 1}) :=
    \begin{cases}
      (- \star_1 \tilde{\kappa} \delta_0 u)(c_{D - 1}),
        & c_{D - 1} \in K_{D - 1} \setminus (\Gamma_N)_{D - 1} \\
      g_N(c_{D - 1}), & c_{D - 1} \in (\Gamma_N)_{D - 1}
    \end{cases}.
  \end{equation}
\end{formulation}

\begin{discussion}
  \label{cmc/diffusion/discrete/transient/primal_weak_solve_trapezoidal-discussion}
  We are going to derive a solution to
  \Cref{cmc/diffusion/discrete/transient/primal_weak-formulation}
  using the trapezoidal rule for time integration.
  We will assume that the heat capacity $\tilde{\pi}$ is time-independent which will
  allow us to rearrange the time derivative:
  \begin{equation}
    B(v, \frac{\partial u} {\partial t}) = \frac{d}{d t} B(v, u).
  \end{equation}
  For further simplicity we will also assume that all the rest input data
  internal production rate, thermal conductivity, boundary conditions)
  are also time-independent.
  Denote $H := F - G$.
  We can then integrate the equation
  \begin{equation}
    \frac{d}{d t} B(v, u) + A(v, u) = H(v)
  \end{equation}
  with respect to $t$ in the interval $[\alpha, \beta] \subset I$ to get
  \begin{equation}
    B(v, u(\beta)) - B(v, u(\alpha))
    + A(v, \int_\alpha^\beta u\, d t)
    = (\beta - \alpha) H(v).
  \end{equation}
  For an interval $[\alpha, \beta]$ the trapezoidal rule gives the approximation
  \begin{equation}
    A(v, \int_\alpha^\beta u\, d t)
    \approx A(v, \frac{\beta - \alpha}{2} (u(\alpha) + u(\beta))).
  \end{equation}
  Hence, if we partition $I$ into segements with size $\tau$ with moments of
  time $\{t_s := t_0 + \tau s\}_{s \geq 0}$, and if we denote
  $\{U^s := u(t_s)\}_{s \geq 0}$, we get
  \begin{equation}
    B(v, U^s) - B(v, U^{s - 1})
    + \frac{\tau}{2} (A(v, U^{s - 1}) + A(v, U^s))
    = \tau H(v).
  \end{equation}
  The above equation is restated as
  \begin{equation}
    (B - \frac{\tau}{2} A)(v, U^s)
    = (B + \frac{\tau}{2} A)(v, U^{s - 1}) + \tau H(v).
  \end{equation}
  Define the left-hand side and right-hand side operators
  \begin{subequations}
    \begin{alignat}{1}
      & L_\tau := B - \frac{\tau}{2} A, \\
      & R_\tau := B + \frac{\tau}{2} A.
    \end{alignat}
  \end{subequations}
  The initial condition corresponds to $U^0 = u_0(t_0, \cdot)$.
  At any moment $s > 0$ we get the following problem for $U^s \in C^0 K$:
  \begin{subequations}
    \begin{alignat}{3}
      & \forall v [\Theta] \in \Ker \tr_{\Gamma_D, 0}, \quad
      && L_\tau(v, U^s)
      && = R_\tau(v, U^{s - 1}) + \tau H(v), \\
      %
      &
      && \tr_{\Gamma_D, 0} U^s
      && = g_D.
    \end{alignat}
  \end{subequations}
  As in the steady-state case
  \Cref{cmc/diffusion/discrete/steady_state/primal_weak_solve-discussion},
  let $J$ be the set of nodes on the Dirichlet boundary $\Gamma_D$,
  and $\overline{J} := \{0, ..., n_0 - 1\} \setminus J$.
  Denote the unknown coefficients of $U^s$ as
  $\{{\bf U}^s_j\}_{j = 0}^{n_0 - 1}$,
  i.e.,
  \begin{equation}
    U^s = \sum_{j = 0}^{n_0 - 1} {\bf U}^s_j N^j.
  \end{equation}
  In an analogous derivation to the one in
  \Cref{cmc/diffusion/discrete/steady_state/primal_weak_solve-discussion},
  let $\overline{{\bf L}_\tau}$ be the matrix in the standard basis of the
  restriction of ${\bf L}_\tau$ to the rows and colums in $\overline{J}$,
  $\overline{\bf U}^s$ be the restriction of ${\bf U}^s$ on $\overline{J}$,
  and $\overline{{\bf H}_\tau} \in \R^{\abs{\overline{J}}}$
  be the vector defined by
  \begin{equation}
    \overline{{\bf H}_\tau}
    := \tau {\bf H}_i - \sum_{j \in J} ({\bf L}_\tau)_{i, j} g_D(N_j),\
    i \in \overline{J}.
  \end{equation}
  % where ${\bf S}_\tau$ realises the Cholesky decomposition of ${\bf R}_\tau$.
  This leads to the system
  \begin{equation}
    \overline{{\bf L}_\tau}\, \overline{{\bf U}}^s
    = \overline{{\bf R}_\tau {\bf U}^{s - 1}}
    + \overline{{\bf H}_\tau},
  \end{equation}
  where $\overline{{\bf R}_\tau {\bf U}^{s - 1}}$ is the restriction of
  ${\bf R}_\tau {\bf U}^{s - 1}$ to $\overline{J}$. 
  % This vector can also be written as
  % \begin{equation}
  %   \overline{{\bf H}_\tau}^s
  %   := \overline{(S_\tau {\bf U}^s)} + \overline{{\bf C}_\tau},
  % \end{equation}
  % where $\overline{{\bf S}_\tau {\bf U}^s}$ is the restriction of
  % ${\bf S}_\tau {\bf U}^s$ on $\overline{J}$ and
  % \begin{equation}
  %   \overline{{\bf C}_\tau}
  %   := \tau {\bf H}_i - \sum_{j \in J} ({\bf R}_\tau)_{i, j} g_D(N_j),\
  %   i \in \overline{J}
  % \end{equation}
  % Hence, at each time step we get the equation
  % \begin{equation}
  %   \overline{{\bf R}_\tau} \overline{\bf U}^s = \overline{{\bf H}_\tau}^s,
  % \end{equation}
  This leads to the the following iterative process.
\end{discussion}

\begin{algorithm}[Algorithm for solving the transient primal weak formulation
  for the discrete heat transfer phenomenon using trapezoidal rule for time
  integration, assuming time-independent input data]
  \label{cmc/diffusion/discrete/transient/primal_weak_solve_trapezoidal-algorithm}
  Let:
  \begin{itemize}
    \item
      Let $D$ be a positive integer (space dimension);
    \item
      $K$ be an oriented quasi-cubical \hyperref[cmc:mesh:definition]{mesh} of
      dimension $D$ representing the material body;
    \item
      $[K]$ be the fundamental class of $K$;
    \item
      $t_0 [T] \in \R$ be the initial time;
    \item
      $\tau [T] \in \R^+$ be the time step;
    \item
      $f [E T^{-1}] \in C^D K$ be the internal production rate;
    \item
      $u_0 [\Theta] \in C^0 K$ be the initial temperature;
    \item
      $\tilde{\pi} [E L^{-D} \Theta^{-1}] \colon I \times C^0 K \to C^0 K$
      be the heat capacity of the material;
    \item
      $\tilde{\kappa} [E L^{2 - D} T^{-1} \Theta^{-1}] \colon C^1 K \to C^1 K$
      be the thermal conductivity of the material;
    \item
      $\partial K = \Gamma_D \cup \Gamma_N$ be the partition of the boundary of
      $K$ into Dirichlet ($\Gamma_D$) and Neumann ($\Gamma_N$) regions;
    \item
      $[\Gamma_N]$ be the fundamental class of $\Gamma_N$, where $\Gamma_N$
      has the boundary orientation induced from $K$;
    \item
      $g_D [\Theta] \in C^0 \Gamma_D$
      be the prescribed temperature on the Dirichlet boundary;
    \item
      $g_N [E T^{-1}] \in C^{D - 1} \Gamma_N$
      be the prescribed flow rate on the Neumann boundary.
  \end{itemize}
  Our algorithm has $3$ phases.
  \begin{enumerate}
    \item
      \textbf{Time-independent phase.}
      Do the following calculations:
      \begin{itemize}
        \item
          $n_0 := \abs{K_0}$;
        \item
          the sparse matrix ${\bf A} \in M_{n_0 \times n_0}(\R)$,
          \begin{equation}
            {\bf A}_{i, j}
            := \inner{\delta_0 N^j}{\tilde{\kappa} \delta_0 N^i}_{K, 1},\
            i, j = 0, ..., n_0 - 1;
          \end{equation}
        \item
          the diagonal matrix ${\bf B} \in M_{n_0 \times n_0}(\R)$,
          \begin{equation}
            {\bf B}_{i, j} := \inner{N^j}{\tilde{\pi} N^i}_{K, 0},\
            i, j = 0, ..., n_0 - 1;
          \end{equation}
        \item
          the right-hand side vectors ${\bf F}, {\bf G}, {\bf H} \in \R^{n_0}$,
          \begin{subequations}
            \begin{alignat}{1}
              & {\bf F}_i := (N^i \smile f)[K], i = 0, ..., n_0 - 1, \\
              & {\bf G}_i := (\tr_{\Gamma_N, 0} N^i \smile g_N)[\Gamma_N],
                i = 0, ..., n_0 - 1, \\
              & {\bf H} := {\bf F} - {\bf G};
            \end{alignat}
          \end{subequations}
        \item
          the sparse matrices (having the same stencil as ${\bf A}$)
          ${\bf L}_\tau, {\bf R}_\tau \in M_{n_0 \times n_0}(\R)$,
          \begin{subequations}
            \begin{alignat}{1}
              & {\bf L}_\tau := {\bf B} - \frac{\tau}{2} {\bf A}, \\
              & {\bf R}_\tau := {\bf B} + \frac{\tau}{2} {\bf A};
            \end{alignat}
          \end{subequations}
        \item
          the sets $J := (\Gamma_D)_0$ and
          $\overline{J} := \{0, ..., n_0 - 1\} \setminus J$;
        \item
          the vector $\widehat{\bf U} \in \R^{\abs{J}}$
          of the prescribed temperature on the Dirichlet boundary:
          \begin{equation}
            \widehat{\bf U}_i = g_D(N_i),\ i \in J;
          \end{equation}
        \item
          the restricted matrix $\overline{{\bf L}_\tau}$, constructed out of
          ${\bf L}_\tau$ with rows and columns in $J$ removed,
          and the restricted vector
          $\overline{\bf H}_\tau \in \R^{\abs{\overline{J}}}$
          \begin{equation}
            ({\overline{\bf H}_\tau})_i
            := \tau {\bf H}_j
            - \sum_{j \in J} ({\bf L}_\tau)_{i, j} \widehat{\bf U}_j,\
            i \in \overline{J};
          \end{equation}
        \item
          the Cholesky decomposition
          \begin{equation}
            \overline{{\bf L}_\tau}
            = \overline{{\bf S}_\tau} \overline{{\bf S}_\tau}^T,
          \end{equation}
          where $\overline{{\bf S}_\tau}$ is a lower-triangular sparse matrix
          with positive diagonal;
        \item
          the time independent part of the restricted solution
          \begin{equation}
            \overline{{\bf Z}_\tau}
            := \overline{{\bf L}_\tau}^{-1} \overline{{\bf H}_\tau}
            = \overline{{\bf S}_\tau}^{-T} \overline{{\bf S}_\tau}^{-1}
            \overline{{\bf H}_\tau}
          \end{equation}
          (of course, we do not find the inverses of $\overline{{\bf S}_\tau}$
          and its transpose, but apply forward and back substitution);
        \item
          the initial coordinates ${\bf U}^0 \in \R^{n_0}$ of the temperature,
          \begin{equation}
            {\bf U}^0_i := u_0(N_i),\ i = 0, ..., n_0 - 1.
          \end{equation}
      \end{itemize}
    \item
      \textbf{Time-dependent (loop) phase.}
      For any $s > 0$ (until some predefined final moment is reached or some
      condition for small error is fulfilled) calculate:
      \begin{itemize}
        \item
          the time-dependent part
          $\overline{{\bf V}_\tau}^s$ of the right-hand side
          (allocated only once, updated on every step),
          \begin{equation}
            \overline{{\bf V}_\tau}^s
            := \overline{({\bf R}_\tau {\bf U}^{s - 1})};
          \end{equation}
        \item
          the time-dependent part $\overline{{\bf W}_\tau}^s$ of the solution
          (allocated only once, updated on every step),
          \begin{equation}
            \overline{{\bf W}_\tau}^s
            := \overline{{\bf S}_\tau}^{-T} \overline{{\bf S}_\tau}^{-1}
            \overline{v_\tau}
          \end{equation}
          (with forward and back substitution);
        \item
          the solution $\overline{\bf U}^s$ on the non-Dirichlet nodes
          (allocated only once, updated on every step),
          \begin{equation}
            \overline{\bf U}^s
            := \overline{{\bf W}_\tau}^s + \overline{{\bf Z}_\tau};
          \end{equation}
        \item
          the final solution
          \begin{equation}
            {\bf U}^s_i :=
            \begin{cases}
              \overline{\bf U}^s_i, & i \in \overline{J} \\
              \widehat{\bf U}_i,  & i \in J
            \end{cases}.
          \end{equation}
      \end{itemize}
    \item
      \textbf{Post-processing.}
      For each time moment $t_s$ the flow rate $q^s \in C^{D - 1} K$
      is as follows: for any $c \in K_{D - 1}$,
      \begin{equation}
        q^s(c_\bullet) :=
        \begin{cases}
          (- \star_1 \circ \tilde{\kappa} \circ \delta_0\, u^s)(c_\bullet),
          & c \in K_{D - 1} \setminus (\Gamma_N)_{D - 1} \\
          g_N(c_\bullet), & c \in (\Gamma_N)_{D - 1}
        \end{cases}.
      \end{equation}
  \end{enumerate}
\end{algorithm}

\subsection{Mixed weak formulation}
\subsubsection{Steady-state}
\begin{formulation}
  \label{cmc/diffusion/continuous/steady_state/mixed_weak-formulation}
  [Mixed weak formulation for the steady-state continuous heat
  equation with differential forms]
  Let:
  \begin{itemize}
    \item
      $D$ be a positive integer (space dimension);
    \item
      $X$ be a $D$-dimensional open region, representing a material body;
    \item
      $f [E T^{-1}] \in \Omega^D X$ be the internal production rate;
    \item
      $\kappa [E L^{2 - D} T^{-1} \Theta^{-1}]
      \colon \Omega^{D - 1} X \to \Omega^{D - 1} X$
      be the thermal conductivity of the material;
    \item
      $\partial X = \Gamma_D \cup \Gamma_N$ be the partition of the boundary of
      $X$ into Dirichlet ($\Gamma_D$) and Neumann ($\Gamma_N$) regions;
    \item
      $g_D [\Theta] \in \Omega^0 \Gamma_D$
      be the prescribed temperature on the Dirichlet boundary;
    \item
      $g_N [E T^{-1}] \in \Omega^{D - 1} \Gamma_N$
      be the prescribed flow rate on the Neumann boundary.
  \end{itemize}
  Define the following operators:
  \begin{subequations}
    \begin{alignat}{3}
      & A \colon \Omega^{D - 1} X \times \Omega^{D - 1} X \to \R, \quad
      && A(r, s)
        := \inner{r}{\kappa^{-1} s}_{X, D - 1} \qquad
      && [E^{-1} T \Theta], \\
      %
      & B \colon \Omega^D X \times \Omega^{D - 1} X \to \R, \quad
      && B(\tilde{w}, r)
        := \inner{d_{D - 1} r}{\tilde{w}}_{X, D} \qquad
      && [L^{-D}], \\
      %
      & G \colon \Omega^{D - 1} X \to \R, \quad
      && G(r)
        := \int_{\Gamma_D} (\tr_{\Gamma_D, D - 1} r \wedge g_D) \qquad
      && [\Theta], \\
      %
      & F \colon \Omega^D X \to \R, \quad
      && F(\tilde{w}) := \inner{f}{\tilde{w}}_{X, D} \qquad
      && [E T^{-1} L^{-D}].
    \end{alignat}
  \end{subequations}
  Our unknowns are:
  \begin{itemize}
    \item $q [E T^{-1}] \in \Omega^{D - 1} X$ (heat flow rate);
    \item $\tilde{u} [\Theta L^D] \in \Omega^D X$ (dual temperature).
  \end{itemize}
  We are solving the following problem for $q$ and $u$:
  \begin{subequations}
    \begin{alignat}{4}
      & \forall r [E T^{-1}] \in \Ker \tr_{\Gamma_N, D - 1}, \quad
      && A(r, q) - B^T(r, \tilde{u})
      && = - G(r) \qquad
      && [E T^{-1} \Theta], \\
      %
      & \forall \tilde{w} [\Theta L^D] \in \Omega^D X, \quad
      && - B(\tilde{w}, q)
      && = - F(\tilde{w}) \qquad
      && [E T^{-1} \Theta], \\
      %
      &
      && \tr_{\Gamma_N, D - 1} q
      && = g_N \qquad
      && [E T^{-1}].
    \end{alignat}
  \end{subequations}
  The temperature $u [\Theta] \in \Omega^0 X$ is calculated in the
  post-processing phase by the formula
  \begin{equation}
    u(x) :=
    \begin{cases}
      (\star_D \tilde{u})(x), & x \notin \Gamma_D \\
      g_D(x), & x \in \Gamma_D
    \end{cases}.
  \end{equation}
\end{formulation}

\begin{discussion}
  \label{cmc/diffusion/discrete/steady_state/mixed_weak_solve-discussion}
  We are going to derive a solution to
  \Cref{cmc/diffusion/discrete/steady_state/mixed_weak-formulation}.
  For any $p \in \{0, ..., d\}$ denote
  \begin{equation}
    n_p := \abs{K_p} = \dim(C_p K) = \dim(C^p K).
  \end{equation}
  The cochains $(c^{p, 0}, ..., c^{p, n_0 - 1})$ form the standard basis of
  $C^p K$.
  Define the diagonal matrix ${\bf A} \in M_{n_{D - 1} \times n_{D - 1}}(\R)$,
  the sparse matrix ${\bf B} \in M_{n_d \times n_{D - 1}}(\R)$, and
  the vectors ${\bf F} \in \R^{n_d}$, ${\bf G} \in \R^{n_{D - 1}}$ by
  \begin{subequations}
    \begin{alignat}{3}
      & {\bf A}_{i, j}
      && := \inner{c^{d - 1, j}}{\kappa^{-1} c^{d - 1, i}}, \enspace
      && i, j = 0, ..., n_{D - 1} - 1, \\
      %
      & {\bf B}_{k, i}
      && := \inner{\delta_{D - 1} c^{d - 1, i}}{c^{d, k}}, \enspace
      && k = 0, ..., n_d - 1,\ i = 0, ..., n_{D - 1} - 1, \\
      %
      & {\bf F}_k
      && := \inner{f}{c^{d, k}}, \enspace
      && k = 0, ..., n_d - 1, \\
      %
      & {\bf G}_i
      && := (\tr_{\Gamma_D, d - 1} c^{d - 1, i} \smile g_D)[\Gamma_D],
        \enspace
      && i = 0, ..., n_{D - 1} - 1.
    \end{alignat}
  \end{subequations}
  Denote the unknown coefficients of $q$ as
  $\{{\bf Q}_j\}_{j = 0}^{n_{D - 1} - 1}$, i.e.,
  \begin{equation}
    q = \sum_{j = 0}^{n_{s - 1} - 1} {\bf Q}_j c^{d - 1, j},
  \end{equation}
  and the unknown coefficients of $\tilde{u}$ as
  $\{\widetilde{\bf U}_k\}_{k = 0}^{n_d - 1}$, i.e.,
  \begin{equation}
    \tilde{u} = \sum_{k = 0}^{n_d - 1} \widetilde{\bf U}_k c^{d, k}.
  \end{equation}
  Finally, let $J$ be the set of $(d - 1)$-cells on the Neumann boundary
  $\Gamma_N$, and $\overline{J} := \{0, ..., n_{D - 1} - 1\} \setminus J$.
  We get the system
  \begin{subequations}
    \begin{alignat}{3}
      & \sum_{j = 0}^{n_{D - 1} - 1} {\bf A}_{i, j} {\bf Q}_j
        - \sum_{k = 0}^{n_d - 1} ({\bf B}^T)_{i, k} y_k
      && = - {\bf G}_i, \enspace
      && i \in \overline{J}, \\
      %
      & - \sum_{i = 0}^{n_{D - 1} - 1} {\bf B}_{k, i} {\bf Q}_i
      && = - {\bf F}_k, \enspace
      && k = 0, ..., n_d - 1, \\
      %
      & {\bf Q}_i
      && = g_N(c_{d - 1, i}), \enspace
      && i \in J.
    \end{alignat}
  \end{subequations}
  This leads to the system of equations
  \begin{subequations}
    \begin{alignat}{3}
      & \sum_{j \in \overline{J}} {\bf A}_{i, j} {\bf Q}_j
        - \sum_{k = 0}^{n_d - 1} ({\bf B}^T)_{i, k} y_k
      && = - {\bf G}_i
        - \sum_{j \in J} {\bf A}_{i, j} g_N(c_{d - 1, j}), \enspace
      && i \in \overline{J}, \\
      %
      & - \sum_{i \in J} {\bf B}_{k, i} {\bf Q}_i
      && = - {\bf F}_k
        + \sum_{i \in J} {\bf B}_{k, i} g_N(c_{d - 1, i}), \enspace
      && k = 0, ..., n_d - 1.
    \end{alignat}
  \end{subequations}
  (Note that since ${\bf A}$ is diagonal, ${\bf A}_{i, j} = 0$
  when $i \in \overline{J}$ and $j \in J$.)

  Define the matrices
  $\overline{\bf A} \in M_{\abs{\overline{J}} \times \abs{\overline{J}}}(\R)$,
  $\overline{\bf B} \in M_{n_d \times \abs{\overline{J}}}(\R)$,
  and vectors
  $\widehat{\bf Q} \in \R^{\abs{J}}$,
  $\widetilde{\bf F} \in \R^{n_d}$,
  $\overline{\bf G} \in \R^{\abs{\overline{J}}}$,
  $\overline{\bf Q} \in \R^{\abs{\overline{J}}}$
  by
  \begin{subequations}
    \begin{alignat}{4}
      & \overline{\bf A}_{i, j}
      && =\
      && {\bf A}_{i, j},\
      && i, j \in \overline{J}, \\
      %
      & \overline{\bf B}_{k, i}
      && =\
      && {\bf B}_{k, i},\
      && k = 0, ..., n_d - 1,\ i \in \overline{J}, \\
      %
      & \widehat{\bf Q}_i
      && =\
      && g_N(c_{d - 1, i}),\
      && i \in J, \\
      %
      & \widetilde{\bf F}_k
      && =\
      && {\bf F}_k - \sum_{i \in J} {\bf B}_{k, i} \widehat{\bf Q}_i,\
      && k = 0, ..., n_d - 1, \\
      %
      & \overline{\bf G}_{k, i}
      && =\
      && {\bf G}_i + \sum_{j \in J} {\bf A}_{i, j} \widehat{\bf Q}_i,\
      && i \in \overline{J}, \\
      %
      & \overline{\bf Q}_i
      && =\
      && {\bf Q}_i,\
      && i \in \overline{J}.
    \end{alignat}
  \end{subequations}
  Hence, we get the folloeing system of equations for
  $\overline{\bf Q}$ and $\widetilde{\bf U}$:
  \begin{subequations}
    \begin{alignat}{4}
      &
      && \overline{\bf A}\; \overline{\bf Q}
      && - \overline{\bf B}^T \widetilde{\bf U}
      && = - \overline{\bf G}, \\
      %
      & -
      && \overline{\bf B}\; \overline{\bf Q}
      &&
      && = - \widetilde{\bf F}.
    \end{alignat}
  \end{subequations}
  In general, when ${\bf A}$ is sparse but not diagonal, it is not beneficial to
  use the inverse of ${\bf A}$ in calculations since it will be a dense matrix.
  (This is the case in mixed finite element methods.)
  However, in our case ${\bf A}$ is diagonal, so the following calculation makes
  sense computationally.
  We can solve for $\overline{\bf Q}$ by
  \begin{equation}
    \label{eq:diffusion/discrete/steady_state/mixed_weak/flow_rate_from_potential}
    \overline{\bf Q}
    = \overline{\bf A}^{-1}
      (- \overline{\bf G} + \overline{\bf B}^T \widetilde{\bf U}).
  \end{equation}
  Hence,
  \begin{equation}
    \widetilde{\bf F}
    = \overline{\bf B}\; \overline{\bf Q}
    = \overline{\bf B}\; \overline{\bf A}^{-1}
        (- \overline{\bf G} + \overline{\bf B}^T \widetilde{\bf U}).
  \end{equation}
  This translates to
  \begin{equation}
    \overline{\bf B}\; \overline{\bf A}^{-1} \overline{\bf B}^T
      \widetilde{\bf U}
    = \overline{\bf B}\;
      \overline{\bf A}^{-1} \overline{\bf G} + \widetilde{\bf F}.
  \end{equation}
  Using the Cholesky decomposition, we can solve for $\widetilde{\bf U}$.
  We then calculate $\overline{\bf Q}$ by substituting $\widetilde{\bf U}$ in
  \Cref{eq:diffusion/discrete/steady_state/mixed_weak/flow_rate_from_potential}.
  Finally,
  \begin{equation}
    {\bf Q}_i =
    \begin{cases}
      \widehat{\bf Q}_i, & i \in J \\
      \overline{\bf Q}_i, & i \in \overline{J}
    \end{cases}.
  \end{equation}
\end{discussion}

\subsubsection{Transient}
\begin{formulation}
  \label{cmc/diffusion/continuous/steady_state/mixed_weak-formulation}
  [Mixed weak formulation for the steady-state continuous heat
  equation with differential forms]
  Let:
  \begin{itemize}
    \item
      $D$ be a positive integer (space dimension);
    \item
      $X$ be a $D$-dimensional open region, representing a material body;
    \item
      $f [E T^{-1}] \in \Omega^D X$ be the internal production rate;
    \item
      $\kappa [E L^{2 - D} T^{-1} \Theta^{-1}]
      \colon \Omega^{D - 1} X \to \Omega^{D - 1} X$
      be the thermal conductivity of the material;
    \item
      $\partial X = \Gamma_D \cup \Gamma_N$ be the partition of the boundary of
      $X$ into Dirichlet ($\Gamma_D$) and Neumann ($\Gamma_N$) regions;
    \item
      $g_D [\Theta] \in \Omega^0 \Gamma_D$
      be the prescribed temperature on the Dirichlet boundary;
    \item
      $g_N [E T^{-1}] \in \Omega^{D - 1} \Gamma_N$
      be the prescribed flow rate on the Neumann boundary.
  \end{itemize}
  Define the following operators:
  \begin{subequations}
    \begin{alignat}{3}
      & A \colon \Omega^{D - 1} X \times \Omega^{D - 1} X \to \R, \quad
      && A(r, s)
        := \inner{r}{\kappa^{-1} s}_{X, D - 1} \qquad
      && [E^{-1} T \Theta], \\
      %
      & B \colon \Omega^D X \times \Omega^{D - 1} X \to \R, \quad
      && B(\tilde{w}, r)
        := \inner{d_{D - 1} r}{\tilde{w}}_{X, D} \qquad
      && [L^{-D}], \\
      %
      & G \colon \Omega^{D - 1} X \to \R, \quad
      && G(r)
        := \int_{\Gamma_D} (\tr_{\Gamma_D, D - 1} r \wedge g_D) \qquad
      && [\Theta], \\
      %
      & F \colon \Omega^D X \to \R, \quad
      && F(\tilde{w}) := \inner{f}{\tilde{w}}_{X, D} \qquad
      && [E T^{-1} L^{-D}].
    \end{alignat}
  \end{subequations}
  Our unknowns are:
  \begin{itemize}
    \item $q [E T^{-1}] \in \Omega^{D - 1} X$ (heat flow rate);
    \item $\tilde{u} [\Theta L^D] \in \Omega^D X$ (dual temperature).
  \end{itemize}
  We are solving the following problem for $q$ and $u$:
  \begin{subequations}
    \begin{alignat}{4}
      & \forall r [E T^{-1}] \in \Ker \tr_{\Gamma_N, D - 1}, \quad
      && A(r, q) - B^T(r, \tilde{u})
      && = - G(r) \qquad
      && [E T^{-1} \Theta], \\
      %
      & \forall \tilde{w} [\Theta L^D] \in \Omega^D X, \quad
      && - B(\tilde{w}, q)
      && = - F(\tilde{w}) \qquad
      && [E T^{-1} \Theta], \\
      %
      &
      && \tr_{\Gamma_N, D - 1} q
      && = g_N \qquad
      && [E T^{-1}].
    \end{alignat}
  \end{subequations}
  The temperature $u [\Theta] \in \Omega^0 X$ is calculated in the
  post-processing phase by the formula
  \begin{equation}
    u(x) :=
    \begin{cases}
      (\star_D \tilde{u})(x), & x \notin \Gamma_D \\
      g_D(x), & x \in \Gamma_D
    \end{cases}.
  \end{equation}
\end{formulation}

\begin{discussion}
  \label{cmc/diffusion/discrete/transient/mixed_weak_solve_trapezoidal-discussion}
  We are going to derive a solution to
  \Cref{cmc/diffusion/discrete/transient/mixed_weak-formulation}
  using the trapezoidal rule for time integration.
  We will assume that the heat capacity $\tilde{\pi}$ is time-independent which will
  allow us to rearrange the time derivative:
  \begin{equation}
    C(\tilde{w}, \frac{\partial \tilde{u}} {\partial t})
    = \frac{d}{d t} C(\tilde{w}, \tilde{u}).
  \end{equation}
  For further simplicity we will also assume that all the rest input data (heat
  source, thermal conductivity, boundary conditions) are also time-independent.
  We can then integrate the equation (the conservation law)
  \begin{equation}
    - B(\tilde{w}, q) - \frac{d}{d t} C(\tilde{w}, \tilde{u}) = - F(\tilde{w})
  \end{equation}
  with respect to $t$ in the interval $[\alpha, \beta] \subset I$ to get
  \begin{equation}
    - B(\tilde{w}, \int_\alpha^\beta q\, d t)
    - (C(\tilde{w}, \tilde{u}(\beta)) - C(\tilde{w}, \tilde{u}(\alpha)))
    = - (\beta - \alpha) F(\tilde{w}).
  \end{equation}
  If we partition $I$ into segements with size $\tau$ with moments of
  time $\{t_s := t_0 + \tau s\}_{s \geq 0}$, and if we denote
  \begin{subequations}
    \begin{alignat}{3}
      & q^s
      && := q(t_s), \enspace
      && s \geq 0, \\
      %
      & \tilde{u}^s
      && := \tilde{u}(t_s), \enspace
      && s \geq 0,
    \end{alignat}
  \end{subequations}
  we get
  \begin{equation}
    - \frac{\tau}{2} (B(\tilde{w}, q^s) + B(\tilde{w}, q^{s + 1}))
    - (C(\tilde{w}, \tilde{u}^{s + 1}) - C(\tilde{w}, \tilde{u}^s))
    = - \tau F(\tilde{w}).
  \end{equation}
  By multiplying the above equation with $2 / \tau$ and rearranging we get:
  \begin{equation}
    - B(\tilde{w}, q^{s + 1}) - \frac{2}{\tau} C(\tilde{w}, \tilde{u}^{s + 1})
    = - 2 F(\tilde{w}) + B(\tilde{w}, q^s)
    - \frac{2}{\tau} C(\tilde{w}, \tilde{u}^s).
  \end{equation}
  At step $0$ we calculate initial data as
  \begin{subequations}
    \begin{alignat}{2}
      & q^0
      && = (- \kappa \star_1 \delta_0)(u_0), \\
      %
      & \tilde{u}^0
      && = \star_0 u_0.
    \end{alignat}
  \end{subequations}
  At any step $s > 0$ we get the following system for
  $(q^s, \tilde{u}^s) \in C^{D - 1} K \times C^D K$:
  \begin{subequations}
    \begin{alignat}{3}
      & \forall r [E T^{-1}] \in \Ker \tr_{\Gamma_N, D - 1}, \,
      && A(r, q^s) - B^T(r, \tilde{u}^s)
      && = - G(r), \\
      & \forall \tilde{w} [\Theta L^D] \in C^D K, \,
      && - B(\tilde{w}, q^s) - \frac{2}{\tau} C(\tilde{w}, \tilde{u}^s)
      && = - 2 F(\tilde{w}) + B(\tilde{w}, q^{s - 1})
        - \frac{2}{\tau} C(\tilde{w}, \tilde{u}^{s - 1}), \\
      %
      &
      && \tr_{\Gamma_N, d - 1} q^s
      && = g_N.
    \end{alignat}
  \end{subequations}
  Let
  \begin{subequations}
    \begin{alignat}{2}
      & J
      && := \set{i \in \{0, ..., n_{D - 1}\}}
        {c_{d - 1, i} \in (\Gamma_N)_{D - 1}}, \\
      %
      & \overline{J}
      && := \{0, ..., n_{D - 1}\} \setminus J.
    \end{alignat}
  \end{subequations}
  Initial conditions give us $Q^0$ and $U^0$.
  Denote the matrices ${\bf A}$, ${\bf B}$, ${\bf C}$ and vectors
  ${\bf Q}^s$, ${\bf U}^s$, ${\bf F}$, ${\bf G}$
  of the corresponding operators in standard bases.
  Let $s > 0$.
  We get the following system for ${\bf Q}^s$ and ${\bf U}^s$:
  \begin{subequations}
    \begin{alignat}{3}
      & \sum_{j = 0}^{n_{D - 1} - 1} {\bf A}_{i, j} {\bf Q}^s_j
        - \sum_{k = 0}^{n_d - 1} ({\bf B}^T)_{i, k} {\bf U}^s_k
      && = - {\bf G}_i, \:
      && i \in \overline{J}, \\
      %
      & - \sum_{i = 0}^{n_{D - 1} - 1} {\bf B}_{k, i} {\bf Q}^s
        - \frac{2}{\tau} ({\bf C} {\bf U}^s)_k
      && = - 2 {\bf F}_k
        + {\bf B} {\bf Q}^{s - 1}
        - \frac{2}{\tau} {\bf C} {\bf U}^{s - 1}_k, \:
      && k \in \{0, ..., n_d - 1\}, \\
      %
      & {\bf Q}^s_i
      && = g_N(c_{d - 1, i}), \:
      && i \in J.
    \end{alignat}
  \end{subequations}
  The system can be rewritten as:
  \begin{subequations}
    \begin{alignat}{3}
      & \sum_{j \in \overline{J}} {\bf A}_{i, j} {\bf Q}^s_j
        - \sum_{k = 0}^{n_d - 1} ({\bf B}^T)_{i, k} {\bf U}^s_k
      && = - {\bf G}_i -
        \sum_{j \in J} {\bf A}_{i, j} g_N(c_{d - 1, j}), \:
      && i \in \overline{J}, \\
      %
      & - \sum_{i \in J} {\bf B}_{k, i} {\bf Q}^s_i
        - \sum_{l = 0}^{n_d - 1}
          \frac{2}{\tau}{\bf C}_{k, l} {\bf U}^s_l
      && = - 2 {\bf F}_k
        + \sum_{i \in J} {\bf B}_{k, i} g_N(c_{d - 1, i})
        + {\bf B} {\bf Q}^{s - 1}
        - \frac{2}{\tau} {\bf C} {\bf U}^{s - 1}_k, \:
      && i \in \overline{J}.
    \end{alignat}
  \end{subequations}
  Let $\overline{\bf A}$ be the restriction of ${\bf A}$ to the rows and colums
  in $\overline{J}$,
  $\overline{\bf B}$ be the restristion of ${\bf B}$ to the colums in
  $\overline{J}$,
  $\overline{\bf Q}^s$ be the restriction of ${\bf Q}^{s}$ to the indices
  in $\overline{J}$,
  $\widetilde{\bf F} \in \R^{n_d}$ be defined as,
  \begin{equation}
    \widetilde{\bf F}_k
    := 2 {\bf F}_k - \sum_{i \in J} {\bf B}_{k, i} g_N(c_{d - 1, i}),\
    k = 0, ..., n_d - 1,
  \end{equation}
  $\overline{\bf G}$ be the restriction of ${\bf G}$ to the indices in
  $\overline{J}$ (since $A$ is diagonal, ${\bf G}$ is not modified before
  restriction).
  Hence, we get the restricted system
  \begin{subequations}
    \begin{alignat}{4}
      &
      && \overline{\bf A}\, \overline{\bf Q}^s
      && - \overline{\bf B}^T {\bf U}^s
      && = - \overline{\bf G}, \\
      %
      & - 
      && \overline{\bf B}\, \overline{\bf Q}^s
      && - \frac{2}{\tau} {\bf C} {\bf U}^s
      && = - \widetilde{\bf F}
        + {\bf B} {\bf Q}^{s - 1}
        - \frac{2}{\tau} {\bf C} {\bf U}^{s - 1}.
    \end{alignat}
  \end{subequations}
  We can solve for $\overline{\bf Q}^s$ as follows:
  \begin{equation}
    \overline{\bf Q}^s
    = \overline{\bf A}^{-1}
      (- \overline{\bf G} + \overline{\bf B}^T {\bf U}^s)
    = - \overline{\bf P} + \overline{\bf R} {\bf U}^s,
  \end{equation}
  where we have denoted
  \begin{align}
    \overline{\bf P} & := \overline{\bf A}^{-1} \overline{\bf G}, \\
    \overline{\bf R} & := \overline{\bf A}^{-1} \overline{\bf B}^T.
  \end{align}
  % Hence,
  % \begin{equation}
  %   {\bf B} {\bf Q}^s + \frac{2}{\tau} {\bf C} {\bf U}^s + \widetilde{\bf F}
  %   = - \overline{\bf B}\, \overline{\bf Q}^{s + 1}
  %     + \frac{2}{\tau} {\bf C} {\bf U}^{s + 1}
  %   = - \overline{\bf B}\, \overline{\bf A}^{-1}
  %     (- \overline{\bf G} + \overline{\bf B}^T {\bf U}^{s + 1})
  %     + \frac{2}{\tau} {\bf C} {\bf U}^{s + 1}.
  % \end{equation}
  This means that
  \begin{equation}
    - \overline{\bf B}\, \overline{\bf Q}^s - \frac{2}{\tau} {\bf C} {\bf U}^s
    = - \overline{\bf B}
      \overline{\bf A}^{-1} (- \overline{\bf G} + \overline{\bf B}^T {\bf U}^s)
    - \frac{2}{\tau} {\bf C} {\bf U}^s
    = - (\overline{\bf B} \overline{\bf A}^{-1} \overline{\bf B}^T
      + \frac{2}{\tau} {\bf C}) {\bf U}^s
      + \overline{\bf B} \overline{\bf A}^{-1} \overline{\bf G}.
  \end{equation}
  Define the left-hand side matrix ${\bf N}_\tau \in M_{n_d \times n_d}(\R)$,
  \begin{equation}
    {\bf N}_\tau
    := \overline{\bf B}\, \overline{\bf A}^{-1} \overline{\bf B}^T
      + \frac{2}{\tau} {\bf C}.
  \end{equation}
  Hence, the conversation law becomes
  \begin{equation}
    {\bf N}_\tau {\bf U}^s
    = \widetilde{\bf F}
    + \overline{\bf B} \overline{\bf A}^{-1} \overline{\bf G} 
    - {\bf B} {\bf Q}^{s - 1}
    + \frac{2}{\tau} {\bf C} {\bf U}^{s - 1}.
  \end{equation}
  % This translates to
  % \begin{equation}
  %   (\overline{\bf B}\, \overline{\bf A}^{-1} \overline{\bf B}^T
  %   - \frac{2}{\tau} {\bf C}) {\bf U}^{s + 1}
  %   = \overline{\bf B} \overline{\bf A}^{-1} \overline{\bf G}
  %   - \widetilde{\bf F} - {\bf B} {\bf Q}^s - \frac{2}{\tau} {\bf C} {\bf U}^s.
  % \end{equation}
  Define the constant right-hand side vector ${\bf Z} \in \R^{n_d}$,
  \begin{equation}
    {\bf Z}
    := \overline{\bf B} \overline{\bf A}^{-1} \overline{\bf G}
      + \widetilde{\bf F}
    = \overline{\bf B} \overline{\bf P} + \widetilde{\bf F}.
  \end{equation}
  This leads to the following linear $n_d \times n_d$ system:
  \begin{equation}
    {\bf N}_\tau {\bf U}^s
    = {\bf Z}
    - {\bf B} {\bf Q}^{s - 1}
    + \frac{2}{\tau} {\bf C} {\bf U}^{s - 1}.
  \end{equation}
  Define 
  \begin{align}
    {\bf V}_\tau & := {\bf N}_\tau^{-1} {\bf Z}, \\
    {\bf Y}_\tau^s
    & := - {\bf B} {\bf Q}^{s - 1} + \frac{2}{\tau} {\bf C} {\bf U}^{s - 1}, \\
    {\bf W}_\tau^s & := {\bf N}_\tau^{-1} {\bf Y}_\tau^s.
  \end{align}
  To find ${\bf V}_\tau$ and ${\bf W}_\tau^s$ we first find the Cholesky
  decomposition of ${\bf N}_\tau$:
  \begin{equation}
    {\bf N}_\tau = {\bf L}_\tau {\bf L}_\tau^T.
  \end{equation}
  Hence,
  \begin{equation}
    {\bf U}^s
    = {\bf N}_\tau^{-1}
      ({\bf Z} - {\bf B} {\bf Q}^s + \frac{2}{\tau} {\bf C} {\bf U}^s)
    = {\bf V}_\tau + {\bf W}_\tau^s.
  \end{equation}
  Summarasing, we get the following algorithmic procedure.
\end{discussion}

\input{diffusion/discrete/transient/mixed_weak_solve_trapezoidal-algorithm.tex}

\section{Examples of diffusion}
\label{section:examples_of_transport_phenomena}
\subsection{Steady-state}
\phantom{T}
\input{diffusion/continuous/steady_state/examples/2d_d00_p00-example.tex}
\input{diffusion/continuous/steady_state/examples/2d_d00_p00-figure.tex}
\input{diffusion/continuous/steady_state/examples/2d_d00_p01-example.tex}
\input{diffusion/continuous/steady_state/examples/2d_d00_p01-figure.tex}
\input{diffusion/continuous/steady_state/examples/2d_d00_p02-example.tex}
\input{diffusion/continuous/steady_state/examples/2d_d00_p02-figure.tex}
\input{diffusion/continuous/steady_state/examples/2d_d00_p03-example.tex}
\input{diffusion/continuous/steady_state/examples/2d_d00_p03-figure.tex}
\input{diffusion/continuous/steady_state/examples/2d_d00_p04-example.tex}
\input{diffusion/continuous/steady_state/examples/2d_d00_p04-figure.tex}
\input{diffusion/continuous/steady_state/examples/2d_d00_p05-example.tex}
\input{diffusion/continuous/steady_state/examples/2d_d00_p05-figure.tex}
\input{diffusion/continuous/steady_state/examples/2d_d01_p00-example.tex}
\input{diffusion/continuous/steady_state/examples/2d_d01_p00-figure.tex}
\input{diffusion/continuous/steady_state/examples/2d_d02_p00-example.tex}
\input{diffusion/continuous/steady_state/examples/2d_d02_p00-figure.tex}
\input{diffusion/continuous/steady_state/examples/2d_d02_p01-example.tex}
\input{diffusion/continuous/steady_state/examples/2d_d02_p01-figure.tex}
\input{diffusion/continuous/steady_state/examples/2d_d03_p00-example.tex}
\input{diffusion/continuous/steady_state/examples/2d_d03_p00-figure.tex}
\input{diffusion/continuous/steady_state/examples/2d_d03_p01-example.tex}
\input{diffusion/continuous/steady_state/examples/2d_d03_p01-figure.tex}
\input{diffusion/continuous/steady_state/examples/2d_d04_p00-example.tex}
\input{diffusion/continuous/steady_state/examples/2d_d04_p00-figure.tex}
\input{diffusion/continuous/steady_state/examples/2d_d04_p01-example.tex}
\input{diffusion/continuous/steady_state/examples/2d_d04_p01-figure.tex}
\input{diffusion/continuous/steady_state/examples/2d_d04_p02-example.tex}
\input{diffusion/continuous/steady_state/examples/2d_d04_p02-figure.tex}
\input{diffusion/continuous/steady_state/examples/2d_d04_p03-example.tex}
\input{diffusion/continuous/steady_state/examples/2d_d04_p03-figure.tex}
\input{diffusion/continuous/steady_state/examples/2d_parallelogram_20_15_degrees_45_p00-example.tex}
\input{diffusion/continuous/steady_state/examples/2d_parallelogram_20_15_degrees_45_p00-figure.tex}
\subsection{Transient}
\input{diffusion/continuous/transient/examples/2d_d00_p00-example.tex}
\input{diffusion/continuous/transient/examples/2d_d00_p00-figure.tex}
\input{diffusion/continuous/transient/examples/2d_d00_p01-example.tex}
\input{diffusion/continuous/transient/examples/2d_d00_p01-figure.tex}
\input{diffusion/continuous/transient/examples/2d_d00_p02-example.tex}
\input{diffusion/continuous/transient/examples/2d_d00_p02-figure.tex}
\input{diffusion/continuous/transient/examples/2d_d00_p03-example.tex}
\input{diffusion/continuous/transient/examples/2d_d00_p03-figure.tex}
\input{diffusion/continuous/transient/examples/2d_d00_p04-example.tex}
\input{diffusion/continuous/transient/examples/2d_d00_p04-figure.tex}
\input{diffusion/continuous/transient/examples/2d_d00_p05-example.tex}
\input{diffusion/continuous/transient/examples/2d_d00_p05-figure.tex}
\input{diffusion/continuous/transient/examples/2d_d01_p00-example.tex}
\input{diffusion/continuous/transient/examples/2d_d01_p00-figure.tex}
\input{diffusion/continuous/transient/examples/2d_d02_p00-example.tex}
\input{diffusion/continuous/transient/examples/2d_d02_p00-figure.tex}
\input{diffusion/continuous/transient/examples/2d_d02_p01-example.tex}
\input{diffusion/continuous/transient/examples/2d_d02_p01-figure.tex}
\input{diffusion/continuous/transient/examples/2d_d03_p00-example.tex}
\input{diffusion/continuous/transient/examples/2d_d03_p00-figure.tex}
\input{diffusion/continuous/transient/examples/2d_d03_p01-example.tex}
\input{diffusion/continuous/transient/examples/2d_d03_p01-figure.tex}

\newpage

\section{Lorentzian manifolds and relativity}
\label{section:lorentzian_manifolds_and_relativity}
\phantom{T}
\begin{definition}
  Let
    $R$ be a ring,
    $V$ be a finite-dimensional $R$-module,
    $\omega \in \Lambda^2 V^*$.
  We say that $\omega$ is \textbf{non-degenerate} or \textbf{symplectic}
  if the associated map
  \begin{equation}
    \tilde{\omega} \colon V \to V^*,\
    X \in V \mapsto \tilde{\omega}(X) := i_X \omega \in V^*,
  \end{equation}
  is an isomorphism.

  The pair $(V, \omega)$ is called a \textbf{symplectic module}
  (or \textbf{symplectic vector space} if $R$ is a field).
\end{definition}
\begin{proposition}
  Let
    $R$ be a ring without,
    $(V, \omega)$ be a finite-dimensional symplectic module over $R$.
  Assume that for any $x \in R,\ x + x = 0 \Rightarrow x = 0$.
  Then $\dim V$ is an even number.
\end{proposition}
\begin{proof}
  Let $n = \dim V$.
  In a basis of $V$ $\omega$ is represented by an antisymmetric matrix $A$.
  But then
  \begin{equation}
    \det A = \det(A^T) = \det(-A) = (-1)^n \det A.
  \end{equation}
  If $n$ is odd, then $\det A + \det A = 0$.
  By assumption this means that $\det A = 0$
  which contradicts the nondegeneracy of $\omega$.
  Hence, $n$ is even.
\end{proof}
\begin{definition}
  Let $M$ be a smooth manifold, $\omega \in \Omega^\bullet M$.
  We say that:
  \begin{enumerate}
    \item
      $\omega$ is \textbf{closed} if $d \omega = 0$
    \item
      $\omega$ is \textbf{exact} if there exists $\eta \in \Omega^\bullet M$
      such that $d \eta = \omega$.
  \end{enumerate}
\end{definition}
\begin{proposition}
  Let $M$ be a smooth manifold, $\omega \in \Omega^\bullet M$.
  If $\omega$ is exact, then it is closed.
\end{proposition}
\begin{proof}
  Let $\eta \in \Omega^\bullet M$ be such that $d \eta = \omega$.
  Then $d \omega = d (d \eta) = 0$, i.e., $\omega$ is closed.
\end{proof}
\begin{definition}
  Let $M$ be a smooth manifold, $\omega \in \Omega^2 M$.
  We say that $\omega$ is a \textbf{symplectic form}
  if it is non-degenerate
  (with base module $\mathfrak{X} M$ over $\mathcal{F} M$) and closed.

  The pair $(M, \omega)$ is called a \textbf{symplectic manifold}.
\end{definition}
\begin{proposition}
  Let $(M, \omega)$ be a symplectic manifold.
  Then $M$ is even-dimensional.
\end{proposition}
\begin{definition}
  Let $Q$ be a smooth manifold.
  Consider the cotangent bundle $T^* Q$ with bundle projection
  $\pi \colon T^* Q \to Q$
  with differential $d \pi \colon T(T^* Q) \to T Q$.
  Define the \textbf{tautological one-form}
  $\theta \colon T^* Q \to T^* (T^* Q)$ as follows:
  for any $(q, p) \in T^*Q$ (i.e,. $q \in Q$, $p \in \Hom(T_q Q, \R)$),
  \begin{equation}
    \restrict{\theta}{(q, p)}
    := p \circ \restrict{d \pi}{(q, p)} \in T^*_{(q, p)}(T^* Q).
  \end{equation}
  In other words, if we denote $M := T^* Q$, then $\theta$ is a section of its
  cotangent bundle $T^* M$, i.e., an $1$-form on $M$. 
\end{definition}
\begin{discussion}
  Let $Q$ be a smooth manifold, $\pi \colon T^* Q \to Q$ be the projection.
  Then a $1$-form on $Q$ is a section of $\colon T^* Q$, i.e., a smooth map
  $\mu \colon Q \to \colon T^* Q$ such that $\pi \circ \mu = \id_Q$.
  As such it has a pullback
  $\mu^* \colon \Omega^\bullet(T^* Q) \to \Omega^\bullet Q$.
\end{discussion}
\begin{proposition}
  Let
    $Q$ be a smooth manifold,
    $\theta$ be the tautological one-form on $T^* Q$,
    $\mu \in \Omega^1 Q$.
  Then
  \begin{equation}
    \mu^* \theta = \mu.
  \end{equation}
\end{proposition}
\begin{proof}
  Let $q \in Q$.
  Then
  \begin{equation}
    \restrict{\mu^* \theta}{q}
    = \restrict{\theta}{\mu q} \circ \restrict{d \mu}{q}
    = \restrict{\mu}{q} \circ \restrict{(d \pi)}{\mu q}
      \circ \restrict{d \mu}{q}
    = \restrict{\mu}{q} \circ \restrict{d(\pi \circ \mu)}{q}
    = \restrict{\mu}{q}.
  \end{equation}
  Since $q$ is arbitrary, $\mu^* \theta = \mu$.
\end{proof}
\begin{definition}
  Let
    $Q$ be a smooth manifold,
    $\theta \in \Omega^1(T^* Q)$ be the tautological one-form.
  Define $\omega := - d \theta$.
  The pair $(T^* Q, \omega)$ is called the \textbf{phase space} of $Q$.
  (In this setting $Q$ is usually called the \textbf{configuration space}.)
\end{definition}
\begin{remark}
  Let $Q$ be a smooth manifold.
  The elements of $T^* Q$ are of the form $(q, p)$ where $q \in Q$ and
  $p \in T^*_q Q = \Hom(T_q Q, \R)$.
  $q$ is called \textbf{generalised position}, while $p$ is called
  \textbf{generalised momentum}.
\end{remark}
\begin{proposition}
  Let
    $Q$ be a smooth manifold,
    $(T^* Q, \omega)$ be its phase space.
  Then $(T^* Q, \omega)$ is a symplectic manifold.
\end{proposition}
\begin{definition}
  Let $Q$ be a smooth manifold of dimension $n$.
  Consider a point $q_0 \in Q$ and let $(U, \hat{\varphi})$ be a chart around
  $q_0$, i.e., $U$ is a neighbourhood of $q_0$ and
  $\hat{\varphi} \colon U \to \R^n$ is a diffeomorphism.
  Let $\{\hat{q}^i \colon U \to \R\}_{i = 1}^n$ be the corresponding local
  coordinates, i.e., if $\{\pi^i \colon \R^n \to \R\}_{i = 1}^n$ are the
  projection maps, then $\{\hat{q}^i = \pi^i \circ \hat{\varphi}\}_{i = 1}^n$.
  Let $i \in \{1, ..., n\}$.
  Define \textbf{position coordinate} $q^i \colon T^* U \to \R$ by
  \begin{equation}
    q^i := \hat{q}^i \circ \restrict{\pi}{U}.
  \end{equation}
  Also, define \textbf{momentum coordinate} $p_i \colon T^* U \to \R$
  as follows: for any $(q, p) \in T^* U$,
  \begin{equation}
    p_i(q, p)
    := p\left(\restrict{\frac{\partial}{\partial \hat{q}^i}}{q}\right).
  \end{equation}
\end{definition}
\begin{proposition}
  Let
    $Q$ be a smooth manifold of dimension $n$,
    $q_0 \in Q$,
    $(U, \hat{\varphi})$ be a chart around $q_0$,
    $\{\hat{q}^i \colon U \to \R\}_{i = 1}^n$ be the corresponding local
      coordinates,
    $\{q^i \colon T^* U \to \R\}_{i = 1}^n$ be the corresponding position
      coordinates,
    $\{p_i \colon T^* U \to \R\}_{i = 1}^n$ be the corresponding momentum
      coordinates.
  Then the map $\varphi \colon T^* U \to \R^{2 n}$ defined by
  \begin{equation}
    \varphi(q, p) = (q^1(q, p), ..., q^n(q, p), p_1(q, p), ..., p_n(q, p))
  \end{equation}
  is a diffeomorphism, i.e., $(T^* U, \varphi)$ is a chart around $(q_0, 0)$.
  (The covector in $T^*_{q_0}$ does not matter, so we make the trivial choice by
  taking zero.)

  These local coordinates are called \textbf{generalised coordinates}.
\end{proposition}
\begin{remark}
  From now on, given a manifold $Q$ and a chart $(U, \hat{\varphi})$, unless
  stated otherwise, we will fix the notation and use the objects defined above:
  the projection map $\pi \colon T^* Q \to Q$, the tautological one-form
  $\theta$ and the canonical symplectic form $\omega = - d \theta$;
  for $i = 1, ..., n$ the coordinate maps $\hat{q}^i$, $q^i$, and $p_i$;
  the chart $(T^* U, \varphi)$.
\end{remark}
\begin{proposition}
  Let
    $Q$ be a smooth manifold,
    $\xi \in \Omega^1(T^* Q)$ has the following property:
    for any $1$-form $\mu$ on $Q$, $\mu^* \xi= 0$.
  Then $\xi = 0$.
\end{proposition}
\begin{proof}
  Let
    $n := \dim Q$, $(U, \hat{\varphi})$ be a chart on $Q$ and
    $\{f_i, g^i \in \mathcal{F}(T^* U)\}_{i = 1}^n$ be such that
  \begin{equation}
    \restrict{\xi}{U} = \sum_{i = 1}^n f_i\, d q^i + \sum_{i = 1}^n g^i\, d p_i.
  \end{equation}
  Take arbitrary $\{h_j \in \mathcal{F} U\}_{j = 1}^n$ so that
  \begin{equation}
    \restrict{\mu}{U} = \sum_{j = 1}^n h_j\, d \hat{q}^j.
  \end{equation}
  Note that
  $q^i \circ \restrict{\mu}{U} = \hat{q}^i$ and
  $p_i \circ \restrict{\mu}{U} = h_i$.
  Hence,
  \begin{equation}
    0 
    = \restrict{(\mu^* \xi)}{U}
    = \sum_{i = 1}^n (f_i \circ \restrict{\mu}{U})\,
      d(q^i \circ \restrict{\mu}{U})
    + \sum_{i = 1}^n (g^i \circ \restrict{\mu}{U})\,
      d(p_i \circ \restrict{\mu}{U})
    = \sum_{i = 1}^n (f_i \circ \restrict{\mu}{U})\, d \hat{q}^i
    + \sum_{i = 1}^n (g^i \circ \restrict{\mu}{U})\, d h_i.
  \end{equation}
  Fix $q_0 \in U$, $p_0 \in T^*_{q_0} Q$ so that $(p_0, q_0) \in T^* U$.
  Denote
  \begin{equation}
    c_i
    :=
    p_0\left(\restrict{\frac{\partial}{\partial \hat{q}^i}}{q_0}\right),
    i = 1, ..., n,
  \end{equation}
  so that
  \begin{equation}
    p_0 = \sum_{i = 1}^n c_i \restrict{d \hat{q}^i}{q_0}.
  \end{equation}
  \begin{enumerate}
    \item
      We will first prove that
      for any $i \in \{1, ..., n\}$, $f_i(q_0, p_0) = 0$.
      Define the constant functions
      \begin{equation}
        h_i(q) := c_i,\ i \in \{1, ..., n\},\ q \in U.
      \end{equation}
      Then for any $i \in \{1, ..., n\}$, $d h_i = 0$.
      Hence,
      \begin{equation}
        \begin{split}
          0
          & = \restrict{(\mu^* \xi)}{q_0} \\
          & = \sum_{i = 1}^n
              f_i(q_0, \sum_{j = 1}^n h_j(q_0) \restrict{d \hat{q}^j}{q_0})\,
              \restrict{d \hat{q}^i}{q_0} \\
          & = \sum_{i = 1}^n
              f_i(q_0, \sum_{j = 1}^n c_j \restrict{d \hat{q}^j}{q_0})\,
              \restrict{d \hat{q}^i}{q_0} \\
          & = \sum_{i = 1}^n f_i(q_0, p_0)\, \restrict{d \hat{q}^i}{q_0}.
        \end{split}
      \end{equation}
      Therefore, for any $i \in \{1, ..., n\}$, $f_i(q_0, p_0) = 0$.
    \item
      We will now prove that
      for any $i \in \{1, ..., n\}$, $g^i(q_0, p_0) = 0$.
      Define the linear functions
      \begin{equation}
        h_i(q) := c_i + \hat{q}^i(q) - \hat{q}^i(q_0).
      \end{equation}
      Then for any $i \in \{1, ..., n\}$,
      $d h_i = d \hat{q}^i$ and $h_i(q) = c_i$.
      Hence,
      \begin{equation}
        \begin{split}
          0
          & = \restrict{(\mu^* \xi)}{q_0} \\
          & = \sum_{i = 1}^n
              g^i(q_0, \sum_{j = 1}^n h_j(q_0) \restrict{d \hat{q}^j}{q_0})\,
              \restrict{d h_i}{q_0} \\
          & = \sum_{i = 1}^n
              g^i(q_0, \sum_{j = 1}^n c_j \restrict{d \hat{q}^j}{q_0})\,
              \restrict{d \hat{q}^i}{q_0} \\
          & = \sum_{i = 1}^n g^i(q_0, p_0)\, \restrict{d \hat{q}^i}{q_0}.
        \end{split}
      \end{equation}
      Therefore, for any $i \in \{1, ..., n\}$, $g^i(q_0, p_0) = 0$.
  \end{enumerate}
  Since $(q_0, p_0) \in T^* U$ was arbitrary, we conclude that
  for any $i \in \{1, ..., n\}$, $f_i = g^i = 0$.
  Hence, $\restrict{\xi}{U} = 0$.
  Taking an atlas $\{(U_\alpha, \hat{\varphi}_\alpha)\}_{\alpha \in A}$ of $Q$
  (for some index set $A$), we conclude that $\xi = 0$.
\end{proof}
\begin{corollary}
  Let
    $Q$ be a smooth manifold,
    $\theta$ be the tautological one-form on $T^* Q$,
    $\eta \in \Omega^1(T^* Q)$ has the following property:
    for any $1$-form $\mu$ on $Q$, $\mu^* \eta = \mu$.
  Then $\eta = \theta$.
\end{corollary}
\begin{proof}
  Write $\eta = \theta + \xi$, i.e., $\xi := \eta - \theta$.
  Then, for any $\mu \in \Omega^1 Q$,
  \begin{equation}
    \mu
    = \mu^* \eta
    = \mu^* \theta + \mu^* \xi
    = \mu + \mu^* \xi
    \Rightarrow \mu^* \xi = 0.
  \end{equation}
  But from the previous proposition it follows that $\xi = 0$,
  and hence $\eta = \theta$.
\end{proof}
\begin{proposition}
  Let
    $Q$ be a smooth manifold of dimension $n$,
    $(U, \hat{\varphi})$ be a chart,
    $(q, p) \in T^* U$,
    $ i \in \{1, ..., n\}$.
  Then
  \begin{equation}
    \restrict{d \pi}{(q, p)}
    \left(\restrict{\frac{\partial}{\partial q^i}}{(q, p)}\right)
    = \restrict{\frac{\partial}{\partial \hat{q}^i}}{q}
  \end{equation}
  and
  \begin{equation}
    \restrict{d \pi}{(q, p)}
    \left(\restrict{\frac{\partial}{\partial p^i}}{(q, p)}\right)
    = 0.
  \end{equation}
\end{proposition}
\begin{proof}
  Let $f \colon Q \to \R$ be smooth.
  Define the functions
  $\hat{g} := f \circ \hat{\varphi}^{-1} \colon \R^n \to \R$ and
  $g := f \circ \pi \circ \varphi^{-1} \colon \R^{2 n} \to \R$.
  Let $(X^1, ..., X^n, Y^1, ..., Y^n) := \varphi(p, q) \in \R^n$.
  This means that $(X^1, ..., X^n) = \hat{\varphi}(q)$.
  Then
  \begin{equation}
    g(X^1, ..., X^n, Y^1, ..., Y^n)
    = f(\pi(q, p))
    = f(q)
    = \hat{g}(X^1, ..., X^n).
  \end{equation}
  Hence,
  \begin{equation}
    \begin{split}
      \frac{\partial g}{\partial x^i}(X^1, ..., X^n, Y^1, ..., Y^n)
      & = \lim_{h \to 0}
        \frac
        {g(X^1, ..., X^i + h, ..., X^n, Y^1, ..., Y^n)
         - g(X^1, ..., X^n, Y^1, ..., Y^n)}
        {h} \\
      & = \lim_{h \to 0}
        \frac{\hat{g}(X^1, ..., X^i + h, ..., X^n) - \hat{g}(X^1, ..., X^n)}{h}
        \\
      & = \frac{\partial \hat{g}}{\partial \hat{x}^i}(X^1, ..., X^n).
    \end{split}
  \end{equation}
  Similarly, since $g$ is constant with respect to the last $n$ coordinates,
  \begin{equation}
    \frac{\partial g}{\partial x^{n + i}}(X^1, ..., X^n, Y^1, ..., Y^n) = 0
  \end{equation}
  Take the standard coordinate systems (given by projections)
  $\{\hat{x}^k\}_{k = 1}^n$ on $\R^n$ and
  $\{x^k\}_{k = 1}^{2 n}$ on $\R^{2 n}$.
  Then, by the definitions of differential and partial derivative on manifold,
  \begin{equation}
    (\restrict{d \pi}{(q, p)}
      \left(\restrict{\frac{\partial}{\partial q^i}}{(q, p)}\right)) f
    = \restrict{\frac{\partial}{\partial q^i}}{(q, p)}(f \circ \pi)
    = \frac{\partial(f \circ \pi \circ \varphi^{-1})}{x^i}(\varphi(q, p))
    = \frac{\partial(f \circ \hat{\varphi}^{-1})}{\hat{x}^i}(\hat{\varphi}(q))
    = \restrict{\frac{\partial}{\partial \hat{q}^i}}{q} f,
  \end{equation}
  from which it follows that the first equality holds.
  Similarly,
  \begin{equation}
    (\restrict{d \pi}{(q, p)}
      \left(\restrict{\frac{\partial}{\partial p^i}}{(q, p)}\right)) f
    = \restrict{\frac{\partial}{\partial p^i}}{(q, p)}(f \circ \pi)
    = \frac{\partial(f \circ \pi \circ \varphi^{-1})}{x^{i + n}}(\varphi(q, p))
    = 0,
  \end{equation}
  from which it follows that the second equality holds.
\end{proof}
\begin{proposition}[Tautological one-form in generalised coordinates]
  Let
    $Q$ be a smooth manifold of dimension $n$,
    $(U, \hat{\varphi})$ be a chart.
  Then
  \begin{equation}
    \restrict{\theta}{U} = \sum_{i = 1}^n p_i\, d q^i.
  \end{equation}
\end{proposition}
\begin{proof}
  Let $(q, p) \in T^* U$.
  Recall that $\restrict{\theta}{(q, p)} = p \circ \restrict{d \pi}{(q, p)}$.
  Hence,
  \begin{equation}
    \restrict{\theta}{(q, p)}
    \left(\restrict{\frac{\partial}{\partial q^i}}{(q, p)}\right)
    = p\left(\restrict{\frac{\partial}{\partial \hat{q}^i}}{q}\right)
    = p_i(q, p),
  \end{equation}
  and
  \begin{equation}
    \restrict{\theta}{(q, p)}
    \left(\restrict{\frac{\partial}{\partial p^i}}{(q, p)}\right)
    = p(0)
    = 0.
  \end{equation}
  Therefore,
  \begin{equation}
    \restrict{\theta}{(q, p)}
    = \sum_{i = 1}^n
      \restrict{\theta}{(q, p)}
      \left(\restrict{\frac{\partial}{\partial q^i}}{(q, p)}\right)\,
      \restrict{d q^i}{(q, p)}
    + \sum_{i = 1}^n
      \restrict{\theta}{(q, p)}
      \left(\restrict{\frac{\partial}{\partial p^i}}{(q, p)}\right)\,
      \restrict{d p^i}{(q, p)}
    = \sum_{i = 1}^n p_i(q, p)\, \restrict{d q^i}{(q, p)},
  \end{equation}
  from which the proposition follows.
\end{proof}
\begin{corollary}[Canonical symplectic in generalised coordinates]
  Let
    $Q$ be a smooth manifold of dimension $n$,
    $(U, \hat{\varphi})$ be a chart.
  Then
  \begin{equation}
    \restrict{\omega}{U} = \sum_{i = 1}^n d q^i \wedge d p_i.
  \end{equation}
\end{corollary}
\begin{proof}
  Let $i \in \{1, ..., n\}$.
  Then
  \begin{equation}
    - d(p_i\, d q^i) = - d p_i \wedge d q^i = d q^i \wedge d p_i.
  \end{equation}
  Summing up for all $i$, we get the desired result.
\end{proof}
\begin{definition}
  Let $(M, \omega)$ be a symplectic manifold, $f \in \mathcal{F} M$.
  We say that $X \in \mathfrak{X} M$ is a \textbf{Hamiltonian vector field} for
  $f$ if
  \begin{equation}
    i_X \omega + d_0 f = 0.
  \end{equation}
\end{definition}
\begin{proposition}
  Let $(M, \omega)$ be a symplectic manifold, $f \in \mathcal{F} M$.
  Then there exists a unique Hamiltonian vector field for $f$.
\end{proposition}
\begin{proof}
  The non-degeneracy of $\omega$ means that we can interpret the symplectic form
  as the isomorphism $\tilde{\omega} \colon \mathfrak{X} M \to \Omega^1 M$,
  given by
  \begin{equation}
    (\tilde{\omega} X) := i_X \omega,\ X \in \mathfrak{X} M.
  \end{equation}
  Hence, the problem at hand has a unique solution
  $X = \tilde{\omega}^{-1}(- d_0 f)$.
\end{proof}
\begin{definition}
  Let $(M, \omega)$ be a symplectic manifold.
  Define the map $\hamiltonian \colon \mathcal{F} M \to \mathfrak{X} M$ by
  \begin{equation}
    \hamiltonian = - \tilde{\omega}^{-1} \circ d_0.
  \end{equation}
  It maps a function to its corresponding Hamiltonian vector field.
  We will write $\hamiltonian_f$ instead of $\hamiltonian(f)$
  for $f \in \mathcal{F} M$.
\end{definition}
\begin{proposition}
  Let
    $Q$ be a smooth manifold of dimension $n$,
    $(U, \hat{\varphi})$ be a chart on $Q$,
    $f \in \mathcal{F}(T^* Q)$.
  Then
  \begin{equation}
    \hamiltonian_f
    = \sum_{i = 1}^n
    \left(
      - \frac{\partial f}{\partial p^i} \frac{\partial}{\partial q^i}
      + \frac{\partial f}{\partial q^i} \frac{\partial}{\partial p^i}
    \right).
  \end{equation}
\end{proposition}
\begin{proof}
  First, note that
  $i_{\frac{\partial}{\partial q^i}} \omega = d p^i$ and
  $i_{\frac{\partial}{\partial p^i}} \omega = - d q^i$.
  Denote
  \begin{equation}
    X
    := \sum_{i = 1}^n
    \left(
      - \frac{\partial f}{\partial p^i} \frac{\partial}{\partial q^i}
      + \frac{\partial f}{\partial q^i} \frac{\partial}{\partial p^i}
    \right).
  \end{equation}
  Then
  \begin{equation}
    i_X \omega
    = \sum_{i}^n
    \left(
      - \frac{\partial f}{\partial p^i} i_{\frac{\partial}{\partial q^i}} \omega
      + \frac{\partial f}{\partial q^i} i_{\frac{\partial}{\partial p^i}} \omega
    \right)
    = \sum_{i}^n
    \left(
      - \frac{\partial f}{\partial p^i} d p^i
      - \frac{\partial f}{\partial q^i} d q^i
    \right)
    = - d f.
  \end{equation}
  Hence, $\hamiltonian_f = X$.
\end{proof}
\begin{proposition}
  Let $(M, \omega)$ be a symplectic manifold, $f, g \in \mathcal{F} M$.
  Then
  \begin{equation}
    \hamiltonian_{f g} = f \hamiltonian_g + g \hamiltonian_f.
  \end{equation}
\end{proposition}
\begin{proof}
  Follows directly from the Leibniz rule for $d_0$.
\end{proof}
\begin{definition}
  Let $(M, \omega)$ be a symplectic manifold, $X \in \mathfrak{X} M$.
  We say that $X$ is a \textbf{symplectic vector field} if $L_X \omega = 0$.
\end{definition}
\begin{remark}
  Since $L_{\lie{X}{Y}} = \lie{L_X}{L_Y} = L_X \circ L_Y - L_Y \circ L_X$,
  the symplectic vector fields form a Lie subalgebra of the Lie algebra of
  vector fields.
\end{remark}
\begin{proposition}
  Let $(M, \omega)$ be a symplectic manifold, $f \in \mathcal{F} M$.
  Then $\hamiltonian_f$ is a symplectic vector field.
\end{proposition}
\begin{proof}
  $
    L_{\hamiltonian_f} \omega
    = i_{\hamiltonian_f}(d \omega) + d(i_{\hamiltonian_f} \omega)
    = i_{\hamiltonian_f} 0 - d(d f)
    = 0.
  $
\end{proof}
\begin{proposition}
  Let
    $(M, \omega)$ be a symplectic manifold,
    $X \in \mathfrak{X} M$ be a symplectic vector fields.
  Then
  \begin{equation}
    d(i_X \omega) = 0.
  \end{equation}
\end{proposition}
\begin{proof}
  $
    d(i_X \omega)
    = L_X \omega - i_X(d \omega)
    = 0 - 0
    = 0.
  $
\end{proof}
\begin{proposition}
  Let $M$ be a smooth manifold, $X, Y \in \mathfrak{X} M$.
  Then
  \begin{equation}
    L_X \circ i_Y = i_{\lie{X}{Y}} + i_Y \circ L_X.
  \end{equation}
\end{proposition}
\begin{proposition}
  Let
    $(M, \omega)$ be a symplectic manifold,
    $X, Y \in \mathfrak{X} M$ be symplectic vector fields.
  Then
  \begin{equation}
    \lie{X}{Y} = \hamiltonian_{i_Y(i_X \omega)}.
  \end{equation}
\end{proposition}
\begin{proof}
  \begin{equation}
    i_{\lie{X}{Y}} \omega
    = (L_X \circ i_Y - i_Y \circ L_X) \omega
    = (L_X \circ i_Y) \omega
    = ((d \circ i_X + i_X \circ d) \circ i_Y) \omega
    = d(i_X(i_Y \omega))
    = - d(i_Y(i_X \omega)).
  \end{equation}
  We get the desired result from the definition of $\hamiltonian$.
\end{proof}
\begin{definition}
  Let $(M, \omega)$ be a symplectic manifold.
  Define the \textbf{Poisson bracket}
  $\poisson{\cdot}{\cdot} \colon \mathcal{F} M \to \mathcal{F} M$ by
  \begin{equation}
    \poisson{f}{g}
    := i_{\hamiltonian_g}(i_{\hamiltonian_f} \omega),\
    f, g \in \mathcal{F} M.
  \end{equation}
\end{definition}
\begin{corollary}
  Let $(M, \omega)$ be a symplectic manifold, $f, g \in \mathcal{F} M$.
  Then
  \begin{equation}
    \lie{\hamiltonian_f}{\hamiltonian_g}
    = \hamiltonian_{i_{\hamiltonian_g}(i_{\hamiltonian_f} \omega)}
    = \poisson{f}{g}.
  \end{equation}
\end{corollary}
\begin{proposition}[Leibniz rule holds for the Poisson bracket]
  Let $(M, \omega)$ be a symplectic manifold, $f, g, h \in \mathcal{F} M$.
  Then
  \begin{equation}
    \poisson{f}{g h} = \poisson{f}{g} h + g \poisson{f}{h}.
  \end{equation}
\end{proposition}
\begin{proof}
  $
    \poisson{f}{g h}
    = i_{\hamiltonian_{g h}}(i_{\hamiltonian_f} \omega)
    = i_{h \hamiltonian_g + g \hamiltonian_{h}}(i_{\hamiltonian_f} \omega)
    = (i_{\hamiltonian_g}(i_{\hamiltonian_f} \omega))\, h
      + g\, (i_{\hamiltonian_h}(i_{\hamiltonian_f} \omega)) 
    = \poisson{f}{g} h + g \poisson{f}{h}.
  $
\end{proof}
\begin{proposition}
  Let $(M, \omega)$ be a symplectic manifold, $f, g, h \in \mathcal{F} M$.
  Then
  \begin{equation}
    \poisson{f}{g} = L_{\hamiltonian_f} g.
  \end{equation}
\end{proposition}
\begin{proof}
  $
    \poisson{f}{g}
    = i_{\hamiltonian_g}(i_{\hamiltonian_f} \omega)
    = - i_{\hamiltonian_f} \circ i_{\hamiltonian_g} \omega
    = i_{\hamiltonian_f}(d g)
    = L_{\hamiltonian_f} g.
  $
\end{proof}
\begin{definition}
  Let $(M, \omega)$ be a symplectic manifold.
  Define
  ${\rm ad} \colon \mathcal{F} M \to (\mathcal{F} M \to \mathcal{F} M)$ by
  \begin{equation}
    {\rm ad}_f g := \poisson{f}{g},\ f, g \in \mathcal{F} M,
  \end{equation}
\end{definition}
\begin{proposition}
  Let $(M, \omega)$ be a symplectic manifold.
  Then
  \begin{equation}
    \lie{{\rm ad}_f}{{\rm ad}_g} = {\rm ad}_{\poisson{f}{g}}.
  \end{equation}
  (Here the bracket $\lie{\cdot}{\cdot}$ is the commutator of operators.)
\end{proposition}
\begin{proof}
  From the previous proposition it follows that
  \begin{equation}
    {\rm ad}_f = L_{X_f},\ f \in \mathcal{F} M.
  \end{equation}
  Hence,
  \begin{equation}
    \lie{{\rm ad}_f}{{\rm ad}_g}
    = \lie{L_{\hamiltonian_f}}{L_{\hamiltonian_g}}
    = L_{\lie{\hamiltonian_f}{\hamiltonian_g}}
    = L_{\hamiltonian_{\poisson{f}{g}}}
    = {\rm ad}_{\poisson{f}{g}}.
  \end{equation}
\end{proof}
\begin{corollary}
  Let $(M, \omega)$ be a symplectic manifold.
  Then $(\mathcal{F} M, \poisson{\cdot}{\cdot})$ is a Lie algebra over $\R$.
\end{corollary}
\begin{proof}
  Bilinearity and antisymmetry are trivial to check.
  The Jacobi identity is equivalent to the adjoint map being a Lie algebra
  homomorphism, which was the previous proposition.
\end{proof}
\begin{definition}
  Let
    $R$ be a commutative ring with unity ring,
    $(A, +, \cdot)$ be an $R$-module
    with additional structures of
    an associative algebra $(A, *)$ and
    a Lie algebra $(A, \poisson{\cdot}{\cdot})$.
  We say that $A$ is a Poisson algebra if the Lie bracket acts as a derivation,
  i.e., for all $f, g, h \in A$,
  \begin{equation}
    \poisson{f}{g * h} = \poisson{f}{g} * h + g * \poisson{f}{h}.
  \end{equation}
\end{definition}
\begin{corollary}
  Let $(M, \omega)$ be a symplectic manifold.
  Then $\mathcal{F} M$ is a Poisson algebra over $\R$.
  Here, addition, scalar multiplication, and multiplication are given by the
  corresponding pointwise operations, while the Lie bracket is given by the
  Poisson bracket.
\end{corollary}


\section{Continuous electromagnetism}
\label{section:continuous_electromagnetism}
\phantom{T}
\input{electromagnetism/continuous/summary-discussion.tex}
\input{electromagnetism/continuous/quantities-table.tex}
\input{electromagnetism/continuous/laws-table.tex}
\input{electromagnetism/continuous/constitutive_relations-table.tex}
\input{electromagnetism/continuous/spacetime-discussion.tex}
\input{electromagnetism/continuous/spacetime_quantities-table.tex}
\input{electromagnetism/continuous/spacetime_laws-table.tex}

\section{Continuum mechanics}
\label{section:continuum_mechanics}
\phantom{T}
\begin{definition}
  Let
    $R$ be a ring,
    $V$ be a finite-dimensional $R$-module,
    $\omega \in \Lambda^2 V^*$.
  We say that $\omega$ is \textbf{non-degenerate} or \textbf{symplectic}
  if the associated map
  \begin{equation}
    \tilde{\omega} \colon V \to V^*,\
    X \in V \mapsto \tilde{\omega}(X) := i_X \omega \in V^*,
  \end{equation}
  is an isomorphism.

  The pair $(V, \omega)$ is called a \textbf{symplectic module}
  (or \textbf{symplectic vector space} if $R$ is a field).
\end{definition}
\begin{proposition}
  Let
    $R$ be a ring without,
    $(V, \omega)$ be a finite-dimensional symplectic module over $R$.
  Assume that for any $x \in R,\ x + x = 0 \Rightarrow x = 0$.
  Then $\dim V$ is an even number.
\end{proposition}
\begin{proof}
  Let $n = \dim V$.
  In a basis of $V$ $\omega$ is represented by an antisymmetric matrix $A$.
  But then
  \begin{equation}
    \det A = \det(A^T) = \det(-A) = (-1)^n \det A.
  \end{equation}
  If $n$ is odd, then $\det A + \det A = 0$.
  By assumption this means that $\det A = 0$
  which contradicts the nondegeneracy of $\omega$.
  Hence, $n$ is even.
\end{proof}
\begin{definition}
  Let $M$ be a smooth manifold, $\omega \in \Omega^\bullet M$.
  We say that:
  \begin{enumerate}
    \item
      $\omega$ is \textbf{closed} if $d \omega = 0$
    \item
      $\omega$ is \textbf{exact} if there exists $\eta \in \Omega^\bullet M$
      such that $d \eta = \omega$.
  \end{enumerate}
\end{definition}
\begin{proposition}
  Let $M$ be a smooth manifold, $\omega \in \Omega^\bullet M$.
  If $\omega$ is exact, then it is closed.
\end{proposition}
\begin{proof}
  Let $\eta \in \Omega^\bullet M$ be such that $d \eta = \omega$.
  Then $d \omega = d (d \eta) = 0$, i.e., $\omega$ is closed.
\end{proof}
\begin{definition}
  Let $M$ be a smooth manifold, $\omega \in \Omega^2 M$.
  We say that $\omega$ is a \textbf{symplectic form}
  if it is non-degenerate
  (with base module $\mathfrak{X} M$ over $\mathcal{F} M$) and closed.

  The pair $(M, \omega)$ is called a \textbf{symplectic manifold}.
\end{definition}
\begin{proposition}
  Let $(M, \omega)$ be a symplectic manifold.
  Then $M$ is even-dimensional.
\end{proposition}
\begin{definition}
  Let $Q$ be a smooth manifold.
  Consider the cotangent bundle $T^* Q$ with bundle projection
  $\pi \colon T^* Q \to Q$
  with differential $d \pi \colon T(T^* Q) \to T Q$.
  Define the \textbf{tautological one-form}
  $\theta \colon T^* Q \to T^* (T^* Q)$ as follows:
  for any $(q, p) \in T^*Q$ (i.e,. $q \in Q$, $p \in \Hom(T_q Q, \R)$),
  \begin{equation}
    \restrict{\theta}{(q, p)}
    := p \circ \restrict{d \pi}{(q, p)} \in T^*_{(q, p)}(T^* Q).
  \end{equation}
  In other words, if we denote $M := T^* Q$, then $\theta$ is a section of its
  cotangent bundle $T^* M$, i.e., an $1$-form on $M$. 
\end{definition}
\begin{discussion}
  Let $Q$ be a smooth manifold, $\pi \colon T^* Q \to Q$ be the projection.
  Then a $1$-form on $Q$ is a section of $\colon T^* Q$, i.e., a smooth map
  $\mu \colon Q \to \colon T^* Q$ such that $\pi \circ \mu = \id_Q$.
  As such it has a pullback
  $\mu^* \colon \Omega^\bullet(T^* Q) \to \Omega^\bullet Q$.
\end{discussion}
\begin{proposition}
  Let
    $Q$ be a smooth manifold,
    $\theta$ be the tautological one-form on $T^* Q$,
    $\mu \in \Omega^1 Q$.
  Then
  \begin{equation}
    \mu^* \theta = \mu.
  \end{equation}
\end{proposition}
\begin{proof}
  Let $q \in Q$.
  Then
  \begin{equation}
    \restrict{\mu^* \theta}{q}
    = \restrict{\theta}{\mu q} \circ \restrict{d \mu}{q}
    = \restrict{\mu}{q} \circ \restrict{(d \pi)}{\mu q}
      \circ \restrict{d \mu}{q}
    = \restrict{\mu}{q} \circ \restrict{d(\pi \circ \mu)}{q}
    = \restrict{\mu}{q}.
  \end{equation}
  Since $q$ is arbitrary, $\mu^* \theta = \mu$.
\end{proof}
\begin{definition}
  Let
    $Q$ be a smooth manifold,
    $\theta \in \Omega^1(T^* Q)$ be the tautological one-form.
  Define $\omega := - d \theta$.
  The pair $(T^* Q, \omega)$ is called the \textbf{phase space} of $Q$.
  (In this setting $Q$ is usually called the \textbf{configuration space}.)
\end{definition}
\begin{remark}
  Let $Q$ be a smooth manifold.
  The elements of $T^* Q$ are of the form $(q, p)$ where $q \in Q$ and
  $p \in T^*_q Q = \Hom(T_q Q, \R)$.
  $q$ is called \textbf{generalised position}, while $p$ is called
  \textbf{generalised momentum}.
\end{remark}
\begin{proposition}
  Let
    $Q$ be a smooth manifold,
    $(T^* Q, \omega)$ be its phase space.
  Then $(T^* Q, \omega)$ is a symplectic manifold.
\end{proposition}
\begin{definition}
  Let $Q$ be a smooth manifold of dimension $n$.
  Consider a point $q_0 \in Q$ and let $(U, \hat{\varphi})$ be a chart around
  $q_0$, i.e., $U$ is a neighbourhood of $q_0$ and
  $\hat{\varphi} \colon U \to \R^n$ is a diffeomorphism.
  Let $\{\hat{q}^i \colon U \to \R\}_{i = 1}^n$ be the corresponding local
  coordinates, i.e., if $\{\pi^i \colon \R^n \to \R\}_{i = 1}^n$ are the
  projection maps, then $\{\hat{q}^i = \pi^i \circ \hat{\varphi}\}_{i = 1}^n$.
  Let $i \in \{1, ..., n\}$.
  Define \textbf{position coordinate} $q^i \colon T^* U \to \R$ by
  \begin{equation}
    q^i := \hat{q}^i \circ \restrict{\pi}{U}.
  \end{equation}
  Also, define \textbf{momentum coordinate} $p_i \colon T^* U \to \R$
  as follows: for any $(q, p) \in T^* U$,
  \begin{equation}
    p_i(q, p)
    := p\left(\restrict{\frac{\partial}{\partial \hat{q}^i}}{q}\right).
  \end{equation}
\end{definition}
\begin{proposition}
  Let
    $Q$ be a smooth manifold of dimension $n$,
    $q_0 \in Q$,
    $(U, \hat{\varphi})$ be a chart around $q_0$,
    $\{\hat{q}^i \colon U \to \R\}_{i = 1}^n$ be the corresponding local
      coordinates,
    $\{q^i \colon T^* U \to \R\}_{i = 1}^n$ be the corresponding position
      coordinates,
    $\{p_i \colon T^* U \to \R\}_{i = 1}^n$ be the corresponding momentum
      coordinates.
  Then the map $\varphi \colon T^* U \to \R^{2 n}$ defined by
  \begin{equation}
    \varphi(q, p) = (q^1(q, p), ..., q^n(q, p), p_1(q, p), ..., p_n(q, p))
  \end{equation}
  is a diffeomorphism, i.e., $(T^* U, \varphi)$ is a chart around $(q_0, 0)$.
  (The covector in $T^*_{q_0}$ does not matter, so we make the trivial choice by
  taking zero.)

  These local coordinates are called \textbf{generalised coordinates}.
\end{proposition}
\begin{remark}
  From now on, given a manifold $Q$ and a chart $(U, \hat{\varphi})$, unless
  stated otherwise, we will fix the notation and use the objects defined above:
  the projection map $\pi \colon T^* Q \to Q$, the tautological one-form
  $\theta$ and the canonical symplectic form $\omega = - d \theta$;
  for $i = 1, ..., n$ the coordinate maps $\hat{q}^i$, $q^i$, and $p_i$;
  the chart $(T^* U, \varphi)$.
\end{remark}
\begin{proposition}
  Let
    $Q$ be a smooth manifold,
    $\xi \in \Omega^1(T^* Q)$ has the following property:
    for any $1$-form $\mu$ on $Q$, $\mu^* \xi= 0$.
  Then $\xi = 0$.
\end{proposition}
\begin{proof}
  Let
    $n := \dim Q$, $(U, \hat{\varphi})$ be a chart on $Q$ and
    $\{f_i, g^i \in \mathcal{F}(T^* U)\}_{i = 1}^n$ be such that
  \begin{equation}
    \restrict{\xi}{U} = \sum_{i = 1}^n f_i\, d q^i + \sum_{i = 1}^n g^i\, d p_i.
  \end{equation}
  Take arbitrary $\{h_j \in \mathcal{F} U\}_{j = 1}^n$ so that
  \begin{equation}
    \restrict{\mu}{U} = \sum_{j = 1}^n h_j\, d \hat{q}^j.
  \end{equation}
  Note that
  $q^i \circ \restrict{\mu}{U} = \hat{q}^i$ and
  $p_i \circ \restrict{\mu}{U} = h_i$.
  Hence,
  \begin{equation}
    0 
    = \restrict{(\mu^* \xi)}{U}
    = \sum_{i = 1}^n (f_i \circ \restrict{\mu}{U})\,
      d(q^i \circ \restrict{\mu}{U})
    + \sum_{i = 1}^n (g^i \circ \restrict{\mu}{U})\,
      d(p_i \circ \restrict{\mu}{U})
    = \sum_{i = 1}^n (f_i \circ \restrict{\mu}{U})\, d \hat{q}^i
    + \sum_{i = 1}^n (g^i \circ \restrict{\mu}{U})\, d h_i.
  \end{equation}
  Fix $q_0 \in U$, $p_0 \in T^*_{q_0} Q$ so that $(p_0, q_0) \in T^* U$.
  Denote
  \begin{equation}
    c_i
    :=
    p_0\left(\restrict{\frac{\partial}{\partial \hat{q}^i}}{q_0}\right),
    i = 1, ..., n,
  \end{equation}
  so that
  \begin{equation}
    p_0 = \sum_{i = 1}^n c_i \restrict{d \hat{q}^i}{q_0}.
  \end{equation}
  \begin{enumerate}
    \item
      We will first prove that
      for any $i \in \{1, ..., n\}$, $f_i(q_0, p_0) = 0$.
      Define the constant functions
      \begin{equation}
        h_i(q) := c_i,\ i \in \{1, ..., n\},\ q \in U.
      \end{equation}
      Then for any $i \in \{1, ..., n\}$, $d h_i = 0$.
      Hence,
      \begin{equation}
        \begin{split}
          0
          & = \restrict{(\mu^* \xi)}{q_0} \\
          & = \sum_{i = 1}^n
              f_i(q_0, \sum_{j = 1}^n h_j(q_0) \restrict{d \hat{q}^j}{q_0})\,
              \restrict{d \hat{q}^i}{q_0} \\
          & = \sum_{i = 1}^n
              f_i(q_0, \sum_{j = 1}^n c_j \restrict{d \hat{q}^j}{q_0})\,
              \restrict{d \hat{q}^i}{q_0} \\
          & = \sum_{i = 1}^n f_i(q_0, p_0)\, \restrict{d \hat{q}^i}{q_0}.
        \end{split}
      \end{equation}
      Therefore, for any $i \in \{1, ..., n\}$, $f_i(q_0, p_0) = 0$.
    \item
      We will now prove that
      for any $i \in \{1, ..., n\}$, $g^i(q_0, p_0) = 0$.
      Define the linear functions
      \begin{equation}
        h_i(q) := c_i + \hat{q}^i(q) - \hat{q}^i(q_0).
      \end{equation}
      Then for any $i \in \{1, ..., n\}$,
      $d h_i = d \hat{q}^i$ and $h_i(q) = c_i$.
      Hence,
      \begin{equation}
        \begin{split}
          0
          & = \restrict{(\mu^* \xi)}{q_0} \\
          & = \sum_{i = 1}^n
              g^i(q_0, \sum_{j = 1}^n h_j(q_0) \restrict{d \hat{q}^j}{q_0})\,
              \restrict{d h_i}{q_0} \\
          & = \sum_{i = 1}^n
              g^i(q_0, \sum_{j = 1}^n c_j \restrict{d \hat{q}^j}{q_0})\,
              \restrict{d \hat{q}^i}{q_0} \\
          & = \sum_{i = 1}^n g^i(q_0, p_0)\, \restrict{d \hat{q}^i}{q_0}.
        \end{split}
      \end{equation}
      Therefore, for any $i \in \{1, ..., n\}$, $g^i(q_0, p_0) = 0$.
  \end{enumerate}
  Since $(q_0, p_0) \in T^* U$ was arbitrary, we conclude that
  for any $i \in \{1, ..., n\}$, $f_i = g^i = 0$.
  Hence, $\restrict{\xi}{U} = 0$.
  Taking an atlas $\{(U_\alpha, \hat{\varphi}_\alpha)\}_{\alpha \in A}$ of $Q$
  (for some index set $A$), we conclude that $\xi = 0$.
\end{proof}
\begin{corollary}
  Let
    $Q$ be a smooth manifold,
    $\theta$ be the tautological one-form on $T^* Q$,
    $\eta \in \Omega^1(T^* Q)$ has the following property:
    for any $1$-form $\mu$ on $Q$, $\mu^* \eta = \mu$.
  Then $\eta = \theta$.
\end{corollary}
\begin{proof}
  Write $\eta = \theta + \xi$, i.e., $\xi := \eta - \theta$.
  Then, for any $\mu \in \Omega^1 Q$,
  \begin{equation}
    \mu
    = \mu^* \eta
    = \mu^* \theta + \mu^* \xi
    = \mu + \mu^* \xi
    \Rightarrow \mu^* \xi = 0.
  \end{equation}
  But from the previous proposition it follows that $\xi = 0$,
  and hence $\eta = \theta$.
\end{proof}
\begin{proposition}
  Let
    $Q$ be a smooth manifold of dimension $n$,
    $(U, \hat{\varphi})$ be a chart,
    $(q, p) \in T^* U$,
    $ i \in \{1, ..., n\}$.
  Then
  \begin{equation}
    \restrict{d \pi}{(q, p)}
    \left(\restrict{\frac{\partial}{\partial q^i}}{(q, p)}\right)
    = \restrict{\frac{\partial}{\partial \hat{q}^i}}{q}
  \end{equation}
  and
  \begin{equation}
    \restrict{d \pi}{(q, p)}
    \left(\restrict{\frac{\partial}{\partial p^i}}{(q, p)}\right)
    = 0.
  \end{equation}
\end{proposition}
\begin{proof}
  Let $f \colon Q \to \R$ be smooth.
  Define the functions
  $\hat{g} := f \circ \hat{\varphi}^{-1} \colon \R^n \to \R$ and
  $g := f \circ \pi \circ \varphi^{-1} \colon \R^{2 n} \to \R$.
  Let $(X^1, ..., X^n, Y^1, ..., Y^n) := \varphi(p, q) \in \R^n$.
  This means that $(X^1, ..., X^n) = \hat{\varphi}(q)$.
  Then
  \begin{equation}
    g(X^1, ..., X^n, Y^1, ..., Y^n)
    = f(\pi(q, p))
    = f(q)
    = \hat{g}(X^1, ..., X^n).
  \end{equation}
  Hence,
  \begin{equation}
    \begin{split}
      \frac{\partial g}{\partial x^i}(X^1, ..., X^n, Y^1, ..., Y^n)
      & = \lim_{h \to 0}
        \frac
        {g(X^1, ..., X^i + h, ..., X^n, Y^1, ..., Y^n)
         - g(X^1, ..., X^n, Y^1, ..., Y^n)}
        {h} \\
      & = \lim_{h \to 0}
        \frac{\hat{g}(X^1, ..., X^i + h, ..., X^n) - \hat{g}(X^1, ..., X^n)}{h}
        \\
      & = \frac{\partial \hat{g}}{\partial \hat{x}^i}(X^1, ..., X^n).
    \end{split}
  \end{equation}
  Similarly, since $g$ is constant with respect to the last $n$ coordinates,
  \begin{equation}
    \frac{\partial g}{\partial x^{n + i}}(X^1, ..., X^n, Y^1, ..., Y^n) = 0
  \end{equation}
  Take the standard coordinate systems (given by projections)
  $\{\hat{x}^k\}_{k = 1}^n$ on $\R^n$ and
  $\{x^k\}_{k = 1}^{2 n}$ on $\R^{2 n}$.
  Then, by the definitions of differential and partial derivative on manifold,
  \begin{equation}
    (\restrict{d \pi}{(q, p)}
      \left(\restrict{\frac{\partial}{\partial q^i}}{(q, p)}\right)) f
    = \restrict{\frac{\partial}{\partial q^i}}{(q, p)}(f \circ \pi)
    = \frac{\partial(f \circ \pi \circ \varphi^{-1})}{x^i}(\varphi(q, p))
    = \frac{\partial(f \circ \hat{\varphi}^{-1})}{\hat{x}^i}(\hat{\varphi}(q))
    = \restrict{\frac{\partial}{\partial \hat{q}^i}}{q} f,
  \end{equation}
  from which it follows that the first equality holds.
  Similarly,
  \begin{equation}
    (\restrict{d \pi}{(q, p)}
      \left(\restrict{\frac{\partial}{\partial p^i}}{(q, p)}\right)) f
    = \restrict{\frac{\partial}{\partial p^i}}{(q, p)}(f \circ \pi)
    = \frac{\partial(f \circ \pi \circ \varphi^{-1})}{x^{i + n}}(\varphi(q, p))
    = 0,
  \end{equation}
  from which it follows that the second equality holds.
\end{proof}
\begin{proposition}[Tautological one-form in generalised coordinates]
  Let
    $Q$ be a smooth manifold of dimension $n$,
    $(U, \hat{\varphi})$ be a chart.
  Then
  \begin{equation}
    \restrict{\theta}{U} = \sum_{i = 1}^n p_i\, d q^i.
  \end{equation}
\end{proposition}
\begin{proof}
  Let $(q, p) \in T^* U$.
  Recall that $\restrict{\theta}{(q, p)} = p \circ \restrict{d \pi}{(q, p)}$.
  Hence,
  \begin{equation}
    \restrict{\theta}{(q, p)}
    \left(\restrict{\frac{\partial}{\partial q^i}}{(q, p)}\right)
    = p\left(\restrict{\frac{\partial}{\partial \hat{q}^i}}{q}\right)
    = p_i(q, p),
  \end{equation}
  and
  \begin{equation}
    \restrict{\theta}{(q, p)}
    \left(\restrict{\frac{\partial}{\partial p^i}}{(q, p)}\right)
    = p(0)
    = 0.
  \end{equation}
  Therefore,
  \begin{equation}
    \restrict{\theta}{(q, p)}
    = \sum_{i = 1}^n
      \restrict{\theta}{(q, p)}
      \left(\restrict{\frac{\partial}{\partial q^i}}{(q, p)}\right)\,
      \restrict{d q^i}{(q, p)}
    + \sum_{i = 1}^n
      \restrict{\theta}{(q, p)}
      \left(\restrict{\frac{\partial}{\partial p^i}}{(q, p)}\right)\,
      \restrict{d p^i}{(q, p)}
    = \sum_{i = 1}^n p_i(q, p)\, \restrict{d q^i}{(q, p)},
  \end{equation}
  from which the proposition follows.
\end{proof}
\begin{corollary}[Canonical symplectic in generalised coordinates]
  Let
    $Q$ be a smooth manifold of dimension $n$,
    $(U, \hat{\varphi})$ be a chart.
  Then
  \begin{equation}
    \restrict{\omega}{U} = \sum_{i = 1}^n d q^i \wedge d p_i.
  \end{equation}
\end{corollary}
\begin{proof}
  Let $i \in \{1, ..., n\}$.
  Then
  \begin{equation}
    - d(p_i\, d q^i) = - d p_i \wedge d q^i = d q^i \wedge d p_i.
  \end{equation}
  Summing up for all $i$, we get the desired result.
\end{proof}
\begin{definition}
  Let $(M, \omega)$ be a symplectic manifold, $f \in \mathcal{F} M$.
  We say that $X \in \mathfrak{X} M$ is a \textbf{Hamiltonian vector field} for
  $f$ if
  \begin{equation}
    i_X \omega + d_0 f = 0.
  \end{equation}
\end{definition}
\begin{proposition}
  Let $(M, \omega)$ be a symplectic manifold, $f \in \mathcal{F} M$.
  Then there exists a unique Hamiltonian vector field for $f$.
\end{proposition}
\begin{proof}
  The non-degeneracy of $\omega$ means that we can interpret the symplectic form
  as the isomorphism $\tilde{\omega} \colon \mathfrak{X} M \to \Omega^1 M$,
  given by
  \begin{equation}
    (\tilde{\omega} X) := i_X \omega,\ X \in \mathfrak{X} M.
  \end{equation}
  Hence, the problem at hand has a unique solution
  $X = \tilde{\omega}^{-1}(- d_0 f)$.
\end{proof}
\begin{definition}
  Let $(M, \omega)$ be a symplectic manifold.
  Define the map $\hamiltonian \colon \mathcal{F} M \to \mathfrak{X} M$ by
  \begin{equation}
    \hamiltonian = - \tilde{\omega}^{-1} \circ d_0.
  \end{equation}
  It maps a function to its corresponding Hamiltonian vector field.
  We will write $\hamiltonian_f$ instead of $\hamiltonian(f)$
  for $f \in \mathcal{F} M$.
\end{definition}
\begin{proposition}
  Let
    $Q$ be a smooth manifold of dimension $n$,
    $(U, \hat{\varphi})$ be a chart on $Q$,
    $f \in \mathcal{F}(T^* Q)$.
  Then
  \begin{equation}
    \hamiltonian_f
    = \sum_{i = 1}^n
    \left(
      - \frac{\partial f}{\partial p^i} \frac{\partial}{\partial q^i}
      + \frac{\partial f}{\partial q^i} \frac{\partial}{\partial p^i}
    \right).
  \end{equation}
\end{proposition}
\begin{proof}
  First, note that
  $i_{\frac{\partial}{\partial q^i}} \omega = d p^i$ and
  $i_{\frac{\partial}{\partial p^i}} \omega = - d q^i$.
  Denote
  \begin{equation}
    X
    := \sum_{i = 1}^n
    \left(
      - \frac{\partial f}{\partial p^i} \frac{\partial}{\partial q^i}
      + \frac{\partial f}{\partial q^i} \frac{\partial}{\partial p^i}
    \right).
  \end{equation}
  Then
  \begin{equation}
    i_X \omega
    = \sum_{i}^n
    \left(
      - \frac{\partial f}{\partial p^i} i_{\frac{\partial}{\partial q^i}} \omega
      + \frac{\partial f}{\partial q^i} i_{\frac{\partial}{\partial p^i}} \omega
    \right)
    = \sum_{i}^n
    \left(
      - \frac{\partial f}{\partial p^i} d p^i
      - \frac{\partial f}{\partial q^i} d q^i
    \right)
    = - d f.
  \end{equation}
  Hence, $\hamiltonian_f = X$.
\end{proof}
\begin{proposition}
  Let $(M, \omega)$ be a symplectic manifold, $f, g \in \mathcal{F} M$.
  Then
  \begin{equation}
    \hamiltonian_{f g} = f \hamiltonian_g + g \hamiltonian_f.
  \end{equation}
\end{proposition}
\begin{proof}
  Follows directly from the Leibniz rule for $d_0$.
\end{proof}
\begin{definition}
  Let $(M, \omega)$ be a symplectic manifold, $X \in \mathfrak{X} M$.
  We say that $X$ is a \textbf{symplectic vector field} if $L_X \omega = 0$.
\end{definition}
\begin{remark}
  Since $L_{\lie{X}{Y}} = \lie{L_X}{L_Y} = L_X \circ L_Y - L_Y \circ L_X$,
  the symplectic vector fields form a Lie subalgebra of the Lie algebra of
  vector fields.
\end{remark}
\begin{proposition}
  Let $(M, \omega)$ be a symplectic manifold, $f \in \mathcal{F} M$.
  Then $\hamiltonian_f$ is a symplectic vector field.
\end{proposition}
\begin{proof}
  $
    L_{\hamiltonian_f} \omega
    = i_{\hamiltonian_f}(d \omega) + d(i_{\hamiltonian_f} \omega)
    = i_{\hamiltonian_f} 0 - d(d f)
    = 0.
  $
\end{proof}
\begin{proposition}
  Let
    $(M, \omega)$ be a symplectic manifold,
    $X \in \mathfrak{X} M$ be a symplectic vector fields.
  Then
  \begin{equation}
    d(i_X \omega) = 0.
  \end{equation}
\end{proposition}
\begin{proof}
  $
    d(i_X \omega)
    = L_X \omega - i_X(d \omega)
    = 0 - 0
    = 0.
  $
\end{proof}
\begin{proposition}
  Let $M$ be a smooth manifold, $X, Y \in \mathfrak{X} M$.
  Then
  \begin{equation}
    L_X \circ i_Y = i_{\lie{X}{Y}} + i_Y \circ L_X.
  \end{equation}
\end{proposition}
\begin{proposition}
  Let
    $(M, \omega)$ be a symplectic manifold,
    $X, Y \in \mathfrak{X} M$ be symplectic vector fields.
  Then
  \begin{equation}
    \lie{X}{Y} = \hamiltonian_{i_Y(i_X \omega)}.
  \end{equation}
\end{proposition}
\begin{proof}
  \begin{equation}
    i_{\lie{X}{Y}} \omega
    = (L_X \circ i_Y - i_Y \circ L_X) \omega
    = (L_X \circ i_Y) \omega
    = ((d \circ i_X + i_X \circ d) \circ i_Y) \omega
    = d(i_X(i_Y \omega))
    = - d(i_Y(i_X \omega)).
  \end{equation}
  We get the desired result from the definition of $\hamiltonian$.
\end{proof}
\begin{definition}
  Let $(M, \omega)$ be a symplectic manifold.
  Define the \textbf{Poisson bracket}
  $\poisson{\cdot}{\cdot} \colon \mathcal{F} M \to \mathcal{F} M$ by
  \begin{equation}
    \poisson{f}{g}
    := i_{\hamiltonian_g}(i_{\hamiltonian_f} \omega),\
    f, g \in \mathcal{F} M.
  \end{equation}
\end{definition}
\begin{corollary}
  Let $(M, \omega)$ be a symplectic manifold, $f, g \in \mathcal{F} M$.
  Then
  \begin{equation}
    \lie{\hamiltonian_f}{\hamiltonian_g}
    = \hamiltonian_{i_{\hamiltonian_g}(i_{\hamiltonian_f} \omega)}
    = \poisson{f}{g}.
  \end{equation}
\end{corollary}
\begin{proposition}[Leibniz rule holds for the Poisson bracket]
  Let $(M, \omega)$ be a symplectic manifold, $f, g, h \in \mathcal{F} M$.
  Then
  \begin{equation}
    \poisson{f}{g h} = \poisson{f}{g} h + g \poisson{f}{h}.
  \end{equation}
\end{proposition}
\begin{proof}
  $
    \poisson{f}{g h}
    = i_{\hamiltonian_{g h}}(i_{\hamiltonian_f} \omega)
    = i_{h \hamiltonian_g + g \hamiltonian_{h}}(i_{\hamiltonian_f} \omega)
    = (i_{\hamiltonian_g}(i_{\hamiltonian_f} \omega))\, h
      + g\, (i_{\hamiltonian_h}(i_{\hamiltonian_f} \omega)) 
    = \poisson{f}{g} h + g \poisson{f}{h}.
  $
\end{proof}
\begin{proposition}
  Let $(M, \omega)$ be a symplectic manifold, $f, g, h \in \mathcal{F} M$.
  Then
  \begin{equation}
    \poisson{f}{g} = L_{\hamiltonian_f} g.
  \end{equation}
\end{proposition}
\begin{proof}
  $
    \poisson{f}{g}
    = i_{\hamiltonian_g}(i_{\hamiltonian_f} \omega)
    = - i_{\hamiltonian_f} \circ i_{\hamiltonian_g} \omega
    = i_{\hamiltonian_f}(d g)
    = L_{\hamiltonian_f} g.
  $
\end{proof}
\begin{definition}
  Let $(M, \omega)$ be a symplectic manifold.
  Define
  ${\rm ad} \colon \mathcal{F} M \to (\mathcal{F} M \to \mathcal{F} M)$ by
  \begin{equation}
    {\rm ad}_f g := \poisson{f}{g},\ f, g \in \mathcal{F} M,
  \end{equation}
\end{definition}
\begin{proposition}
  Let $(M, \omega)$ be a symplectic manifold.
  Then
  \begin{equation}
    \lie{{\rm ad}_f}{{\rm ad}_g} = {\rm ad}_{\poisson{f}{g}}.
  \end{equation}
  (Here the bracket $\lie{\cdot}{\cdot}$ is the commutator of operators.)
\end{proposition}
\begin{proof}
  From the previous proposition it follows that
  \begin{equation}
    {\rm ad}_f = L_{X_f},\ f \in \mathcal{F} M.
  \end{equation}
  Hence,
  \begin{equation}
    \lie{{\rm ad}_f}{{\rm ad}_g}
    = \lie{L_{\hamiltonian_f}}{L_{\hamiltonian_g}}
    = L_{\lie{\hamiltonian_f}{\hamiltonian_g}}
    = L_{\hamiltonian_{\poisson{f}{g}}}
    = {\rm ad}_{\poisson{f}{g}}.
  \end{equation}
\end{proof}
\begin{corollary}
  Let $(M, \omega)$ be a symplectic manifold.
  Then $(\mathcal{F} M, \poisson{\cdot}{\cdot})$ is a Lie algebra over $\R$.
\end{corollary}
\begin{proof}
  Bilinearity and antisymmetry are trivial to check.
  The Jacobi identity is equivalent to the adjoint map being a Lie algebra
  homomorphism, which was the previous proposition.
\end{proof}
\begin{definition}
  Let
    $R$ be a commutative ring with unity ring,
    $(A, +, \cdot)$ be an $R$-module
    with additional structures of
    an associative algebra $(A, *)$ and
    a Lie algebra $(A, \poisson{\cdot}{\cdot})$.
  We say that $A$ is a Poisson algebra if the Lie bracket acts as a derivation,
  i.e., for all $f, g, h \in A$,
  \begin{equation}
    \poisson{f}{g * h} = \poisson{f}{g} * h + g * \poisson{f}{h}.
  \end{equation}
\end{definition}
\begin{corollary}
  Let $(M, \omega)$ be a symplectic manifold.
  Then $\mathcal{F} M$ is a Poisson algebra over $\R$.
  Here, addition, scalar multiplication, and multiplication are given by the
  corresponding pointwise operations, while the Lie bracket is given by the
  Poisson bracket.
\end{corollary}


\section{Discrete elasticity}
\label{section:discrete_elasticity}
\phantom{T}
\input{elasticity/discrete/model-discussion.tex}
\input{elasticity/discrete/model-example.tex}

\end{document}
