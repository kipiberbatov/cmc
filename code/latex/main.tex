\documentclass{article}
\usepackage[margin = 0.6in, paper = a4paper]{geometry}
\usepackage{xcolor}
\usepackage[fleqn]{amsmath}
\usepackage{amssymb}
\usepackage{amsthm}
\usepackage{makecell}
\usepackage{bookmark}
\addtocontents{toc}{\protect{\pdfbookmark[0]{\contentsname}{toc}}}
\addtocontents{lof}{\protect{\pdfbookmark[0]{\listfigurename}{lof}}}
\addtocontents{lot}{\protect{\pdfbookmark[0]{\listtablename}{lot}}}
\usepackage{hyperref}
\hypersetup{
  pdfauthor = {Kiprian Berbatov},
  pdftitle = {Combinatorial Mesh Calculus},
  pdfsubject = {Combinatorial Mesh Calculus},
  pdfkeywords = {Combinatorial, Mesh, Calculus, Geometry, Microstructure},
  pdfcreator = {pdflatex},
  pdfproducer = {Latex2e with hyperref},
  colorlinks = true,
  linkcolor = blue,
  citecolor = green,
  urlcolor = cyan,
  bookmarksnumbered = true,
  bookmarksopen = false
}
\usepackage[nameinlink]{cleveref}
\usepackage{graphicx}
\graphicspath{{build/release/pdf}}
\usepackage{subcaption}
\usepackage{cprotect}
\usepackage[most]{tcolorbox}
\newcommand\coloredcomponent[2]
{
  \tcolorboxenvironment{#1}
  {
    breakable,
    enhanced,
    colback = white,
    colframe = #2,
    boxrule = 1pt,
    left = 2pt,
    right = 2pt,
    top = 2pt,
    bottom = 2pt,
    sharp corners,
    before skip = \topsep,
    after skip = \topsep,
  }
}

\counterwithin{equation}{section}
\theoremstyle{definition}

\newtheorem{theorem}{Theorem}[section]
\coloredcomponent{theorem}{blue}

\newtheorem{proposition}[theorem]{Proposition}
\coloredcomponent{proposition}{blue}

\newtheorem{lemma}[theorem]{Lemma}
\coloredcomponent{lemma}{blue}

\newtheorem{corollary}[theorem]{Corollary}
\coloredcomponent{corollary}{blue}

\newtheorem{hypothesis}[theorem]{Hypothesis}
\coloredcomponent{hypothesis}{red}

\newtheorem{notation}[theorem]{Notation}
\coloredcomponent{notation}{green}

\newtheorem{definition}[theorem]{Definition}
\coloredcomponent{definition}{green}

\newtheorem{formulation}[theorem]{Formulation}
\coloredcomponent{formulation}{brown}

\newtheorem{discussion}[theorem]{Discussion}
\coloredcomponent{discussion}{yellow}

\newtheorem{example}[theorem]{Example}
\coloredcomponent{example}{purple}

\newtheorem{remark}[theorem]{Remark}
\coloredcomponent{remark}{orange}

\newtheorem{algorithm}[theorem]{Algorithm}
\coloredcomponent{algorithm}{cyan}

\newtheorem{solution}[theorem]{Solution}
\coloredcomponent{solution}{cyan}

\coloredcomponent{proof}{cyan}

\renewcommand{\thefootnote}{\arabic{footnote}}

\newcommand{\N}{\mathbb{N}}
\newcommand{\Z}{\mathbb{Z}}
\newcommand{\Q}{\mathbb{Q}}
\newcommand{\R}{\mathbb{R}}
\newcommand{\C}{\mathbb{C}}
\renewcommand{\S}{\mathbb{S}}

\newcommand{\set}[2]{\left\{ #1 \mid #2 \right\}}
\newcommand{\restrict}[2]{\left. #1 \right|_{#2}}

\newcommand{\norm}[1]{\left\lVert#1\right\rVert}
\newcommand{\abs}[1]{\left\lvert#1\right\rvert}
\newcommand{\inner}[2]{\langle#1,#2\rangle}
\newcommand{\lie}[2]{[#1,#2]}
\newcommand{\poisson}[2]{\{#1,#2\}}
\newcommand{\hamiltonian}{\nu}

\newcommand{\linearspan}{\mathop{\rm span}\nolimits}
\newcommand{\Ker}{\mathop{\rm Ker}\nolimits}
\renewcommand{\Im}{\mathop{\rm Im}\nolimits}
\newcommand{\Hom}{\mathop{\rm Hom}\nolimits}
\newcommand{\End}{\mathop{\rm End}\nolimits}
\newcommand{\Aut}{\mathop{\rm Aut}\nolimits}
\newcommand{\id}{\mathop{\rm id}\nolimits}
\newcommand{\tr}{\mathop{\rm tr}\nolimits}
\newcommand{\sym}{\mathop{\rm sym}\nolimits}

\newcommand{\grad}{\mathop{\rm grad}\nolimits}
\renewcommand{\div}{\mathop{\rm div}\nolimits}

\newcommand{\Aff}{\mathop{\rm Aff}\nolimits}
\newcommand{\Con}{\mathop{\rm Con}\nolimits}

\newcommand{\Cl}{\mathop{\rm Cl}\nolimits}
\newcommand{\Link}{\mathop{\rm Link}\nolimits}

\newcommand{\sgn}{\mathop{\rm sgn}\nolimits}
\newcommand{\OR}{\mathop{\rm or}\nolimits}
\newcommand{\vol}{\mathop{\rm vol}\nolimits}
\newcommand{\usmile}{\underline{\smile}}
\newcommand{\uwedge}{\underline{\wedge}}

\newcommand{\newterm}[1]{\textbf{#1}}

\newcommand{\amount}{\mathsf{X}}
\newcommand{\potential}{\mathsf{Y}}
\newcommand{\mass}{\mathsf{M}}
\newcommand{\length}{\mathsf{L}}
\renewcommand{\time}{\mathsf{T}}
\newcommand{\temperature}{\theta}
\newcommand{\charge}{\mathsf{C}}

\newcommand{\topStrut}{\rule{0pt}{2.6ex}}

\setcounter{tocdepth}{4}
\setcounter{secnumdepth}{4}

\title{Combinatorial Mesh Calculus (CMC) \\
       \url{https://github.com/kipiberbatov/cmc}}
\author{Kiprian Berbatov}
\date{23 September 2025}

\begin{document}

\pdfsuppresswarningpagegroup=1

\maketitle

\tableofcontents
\NewCommandCopy\oricontentsline\contentsline
\makeatletter
\RenewDocumentCommand\contentsline{mmmm}
{%
  \oricontentsline{#1}{#2}{#3}{#4}%
  {\let\numberline\@gobble
    \bookmark[
      rellevel=1,
      keeplevel,
      dest=#4,
    ]{#2}}%
}
\makeatother

\listoffigures

\listoftables

\bookmarksetup{startatroot}

\part{Algebra}

\section{Commutative rings with unity}
\label{section:commutative_rings_with_unity}
\input{commutative_ring_with_unity/concept-definition.tex}
\input{commutative_ring_with_unity/concept-example.tex}
\input{commutative_ring_with_unity/subring-definition.tex}
\input{commutative_ring_with_unity/closure_of_operations_leads_to_a_subring-proposition.tex}
\input{commutative_ring_with_unity/subring-example.tex}
\input{commutative_ring_with_unity/field-definition.tex}
\input{commutative_ring_with_unity/field-example.tex}
\input{commutative_ring_with_unity/why_using_them-remark.tex}

\section{Modules over commutative rings with unity}
\label{section:modules_over_commutative_rings_with_unity}
\input{module/concept-definition.tex}
\input{module/concept-remark.tex}
\input{module/concept-example.tex}
\input{module/linear_dependence_and_independence-definition.tex}
\input{module/span-definition.tex}
\input{module/basis-definition.tex}
\input{module/freeness-definition.tex}
\input{module/freeness-example.tex}
\begin{definition}
  Let
    $R$ be a commutative ring with unity,
    $V$ and $W$ be $R$-modules,
    $f \colon V \to W$.
  We say that $f$ is \textbf{linear} (or a \textbf{homomorphism}) if for any
  $\lambda, \mu \in R$, $u, v \in V$,
  \begin{equation}
    f(\lambda u + \mu v) = \lambda f(u) + \mu f(v).
  \end{equation}
  The space of all homomorphisms between $V$ and $W$ is denoted by $\Hom(V, W)$.
  An \textbf{endomorphism} is a homorphism from a module to itself.
  The space of all endomorphisms on $V$ is denoted by $\End V := \Hom(V, V)$.
\end{definition}

\begin{proposition}
  Let
    $R$ be a commutative ring with unity,
    $V$ and $W$ be $R$-modules,
  Then $\Hom(V, W)$ is an $R$-module under pointwise addition and scalar
  multiplication.
\end{proposition}

\begin{proof}
  Since $\Hom(V, W)$ is a subset of the space of functions from $V$ to $W$,
  which, as we already know, form a module, it is enough to show that
  $\Hom(V, W)$ is closed under addition and scalar multiplication.
  \begin{itemize}
    \item
      Let $f, g \in \Hom(V, W)$.
      We want to show that $f + g \in \Hom(V, W)$.
      Take arbitrary $u, v \in U$, $\lambda, \mu \in R$.
      Then
      \begin{equation}
        (f + g)(\lambda u + \mu v)
        = f(\lambda u + \mu v) + g(\lambda u + \mu v)
        = \lambda f(u) + \mu f(v) + \lambda g(u) + \mu g(v)
        = \lambda (f + g)(u) + \mu (f + g)(v).
      \end{equation}
      Hence, $f + g \in \Hom(V, W)$.
    \item
      Let $f \in \Hom(V, W)$, $r \in R$.
      We want to show that $r f \in \Hom(V, W)$.
      Take arbitrary $u, v \in U$, $\lambda, \mu \in R$.
      Then
      \begin{equation}
        (r f)(\lambda u + \mu v)
        = r f(\lambda u + \mu v)
        = r \lambda f(u) + r \mu f(v)
        = \lambda (r f)(u) + \mu (r f)(v).
      \end{equation}
      Hence, $r f\in \Hom(V, W)$.
  \end{itemize}
  Hence $\Hom(V, W)$ is a subspace of $V^W$.
\end{proof}


\section{Algebras over rings}
\label{section:algebras_over_rings}
\input{algebra_over_ring/concept-definition.tex}
\input{algebra_over_ring/different_types-definition.tex}
\input{algebra_over_ring/alternating_anticommutative_equivalence-proposition.tex}
\input{algebra_over_ring/alternating_anticommutative_equivalence-proof.tex}

\section{Lie algebras}
\label{section:lie_algebras}
\input{lie_algebra/concept-definition.tex}
\input{lie_algebra/commutator_on_associative_algebra_induces_lie_algebra-proposition.tex}
\input{lie_algebra/commutator_on_associative_algebra_induces_lie_algebra-proof.tex}
\begin{definition}
  Let
    $R$ be a commutative ring with unity,
    $V$ and $W$ be $R$-modules,
    $f \colon V \to W$.
  We say that $f$ is \textbf{linear} (or a \textbf{homomorphism}) if for any
  $\lambda, \mu \in R$, $u, v \in V$,
  \begin{equation}
    f(\lambda u + \mu v) = \lambda f(u) + \mu f(v).
  \end{equation}
  The space of all homomorphisms between $V$ and $W$ is denoted by $\Hom(V, W)$.
  An \textbf{endomorphism} is a homorphism from a module to itself.
  The space of all endomorphisms on $V$ is denoted by $\End V := \Hom(V, V)$.
\end{definition}

\input{lie_algebra/adjoint_map-definition.tex}
\input{lie_algebra/adjoint_map_is_a_homomorphism-proposition.tex}
\input{lie_algebra/adjoint_map_is_a_homomorphism-proof.tex}

\section{Derivations on algebras}
\label{section:derivations_on_algebras}
\input{derivation/concept-definition.tex}
\input{derivation/calculus_derivative-example.tex}
\input{derivation/adjoint_map_induces_derivation-example.tex}
\input{derivation/on_unital_commutative_associative_algebra_form_a_module-proposition.tex}
\input{derivation/on_unital_commutative_associative_algebra_form_a_module-proof.tex}
\input{derivation/on_unital_commutative_associative_algebra_commutator_is_a_derivation-proposition.tex}
\begin{proof}
  Linearity of $[X, Y]$ follows from module theory.
  Hence, we only need to show it satisfies the Leibniz rule.
  Let $f, g \in A$.
  Then
  \begin{equation}
    X(Y(f g))
    = X((Y f) * g + f * (Y g))
    = (X(Y f)) * g + (Y f) * (X g) + (X f) * (Y g) + f * (X (Y g)).
  \end{equation}
  Analogously,
  \begin{equation}
    Y(X(f g)) = (Y(X f)) * g + (X f) * (Y g) + (Y f) * (X g) + f * (Y (X g)).
  \end{equation}
  Hence,
  \begin{equation}
    \begin{split}
      [X, Y](f g)
      = {\color{red} (X(Y f)) * g}\,
        {\color{blue} + f * (X (Y g))}\,
        {\color{red} - (Y(X f)) * g}\,
        {\color{blue}- f * (Y (X g))}
      = {\color{red}([X, Y] f) * g} + {\color{blue} f * ([X, Y] g)}.
      \qedhere
    \end{split}
  \end{equation}
\end{proof}

\input{derivation/on_unital_commutative_associative_algebra_induces_a_lie_algebra-corollary.tex}
\input{derivation/there_are_no_nontrivial_derivations_on_ring_powers-proposition.tex}
\input{derivation/there_are_no_nontrivial_derivations_on_ring_powers-proof.tex}
\begin{definition}
  Let
  {
    $R$ be a commutative ring with unity,
    $A$ and $B$ be $R$-algebras,
    $D_A$ and $D_B$ be derivations on $A$ and $B$ respectively.
  }
  Define the map $D \colon A \otimes B \to A \otimes B$ as follows:
  it is the unique linear map such that for simple tensors
  $a \otimes b \in A \otimes B$,
  \begin{equation}
    D(a \otimes b) := D_A a \otimes b + a \otimes D_B b.
  \end{equation}
  We call the map $D$ the \textbf{tensor product of derivations}
  and denote it by $D_A \otimes D_B$.
\end{definition}

\begin{proposition}
  Let
  {
    $R$ be a commutative ring with unity,
    $A$ and $B$ be $R$-algebras,
    $D_A$ and $D_B$ be derivations on $A$ and $B$ respectively.
  }
  Then the tensor product of derivations $D := D_A \otimes D_B$
  is indeed a derivation on $A \otimes B$.
\end{proposition}

\begin{proof}
  Consider the simple tensors
  $c_1 := a_1 \otimes b_1,\ c_2 := a_2 \otimes b_2 \in A \otimes B$.
  Then
  \begin{equation}
    \begin{split}
      D(c_1 \cdot c_2)
      & = D((a_1 \otimes b_1) \cdot (a_2 \otimes b_2)) \\
      & = D((a_1 \cdot a_2) \otimes (b_1 \cdot b_2)) \\
      & = D_A(a_1 \cdot a_2) \otimes (b_1 \cdot b_2)
        + (a_1 \cdot a_2) \otimes D_B(b_1 \cdot b_2) \\
      & = (D_A a_1 \cdot a_2 + a_1 \cdot D_A a_2) \otimes (b_1 \cdot b_2)
        + (a_1 \cdot a_2) \otimes (D_B b_1 \cdot b_2 + b_1 \cdot D_B b_2) \\
      & = (D_A a_1 \otimes b_1 + a_1 \otimes D_B b_1) \cdot (a_2 \otimes b_2)
        + (a_1 \otimes b_1) \cdot (D_A a_2 \otimes b_2 + a_2 \otimes D_B b_2) \\
      & = D(a_1 \otimes b_1) \cdot (a_2 \otimes b_2)
        + (a_1 \otimes b_1) \cdot D(a_2 \otimes b_2) \\
      & = D c_1 \cdot c_2 + c_1 \cdot D c_2.
    \end{split}
  \end{equation}
  By linearity, the Leibniz rule is satisfied for all tensors.
\end{proof}


\section{Exterior algebra}
\label{section:exterior_algebra}
\input{exterior_algebra/concept-definition.tex}
\input{exterior_algebra/wedge_on_vectors_is_antisymmetric-proposition.tex}
\input{exterior_algebra/wedge_on_vectors_is_antisymmetric-proof.tex}
\input{exterior_algebra/ordered_lists_form_a_basis-proposition.tex}
\input{exterior_algebra/ordered_lists_form_a_basis-remark.tex}
\input{exterior_algebra/top_dimensional_exterior_power_has_dimension_one-proposition.tex}
\input{exterior_algebra/wedge_product_in_different_basis_by_determinant-proposition.tex}
\input{vector_space/dual-notation.tex}
\input{exterior_algebra/p_th_power_commutes_with_dual_up_to_isomorphism-proposition.tex}
\input{exterior_algebra/commutes_with_dual_up_to_isomorphism-corollary.tex}
\input{exterior_algebra/division_of_top_dimensional_forms-notation.tex}
\input{vector_space/orientation-definition.tex}
\input{combinatorics/set_of_combinations-definition.tex}
\input{exterior_algebra/multivector_with_multiindex-notation.tex}
\input{exterior_algebra/multivector_expressed_by_determinant_minors-proposition.tex}

\section{Inner products and Hodge star}
\label{section:inner_products_and_hodge_star}
\input{inner_product/concept-definition.tex}
\input{inner_product/induced_isomorphism_with_dual-proposition.tex}
\input{inner_product/orthogonal_and_orthonormal_basis-definition.tex}
\input{exterior_algebra/inner_product-definition.tex}
\input{exterior_algebra/volume_d_vector-definition.tex}
\input{inner_product/there_are_two_volume_vectors_with_norm_one-remark.tex}
\input{hodge_star/concept-definition.tex}
\input{hodge_star/standard_basis_in_2d-example.tex}
\input{hodge_star/alternative_formula-proposition.tex}
\input{hodge_star/duality_up_to_sign-proposition.tex}

\section{Chain complexes}
\label{section:chain_complexes}
\input{chain_complex/concept-definition.tex}
\input{chain_complex/tensor_product-definition.tex}
\input{chain_complex/tensor_product_is_chain_complex-proposition.tex}

\section{Differential graded algebras}
\label{section:differential_graded_algebras}
\input{differential_graded_algebra/concept-definition.tex}
\input{differential_graded_algebra/tensor_product-definition.tex}
\input{differential_graded_algebra/tensor_product_is_differential_graded_algebra-proposition.tex}

\part{Smooth manifolds}

\section{Vector fields on manifolds}
\label{section:vector_fields_on_manifolds}
\input{smooth_manifold/vector_field/concept-definition.tex}
\input{smooth_manifold/vector_field/commutator-closedness-proposition.tex}
\input{smooth_manifold/vector_field/commutator-closedness-proof.tex}
\input{smooth_manifold/vector_field/formula_in_coordinates-discussion.tex}
\begin{definition}
  Let $M$ be a smooth manifold.
  Define the \textbf{Lie derivative on functions},
  $L \colon \mathfrak{X} M \to {\rm Der}_{\mathcal{F} M}(\mathcal{F} M)$,
  as the evaluation map, i.e.,
  for any $X \in \mathfrak{X} M$, $f \in \mathcal{F} M$,
  \begin{equation}
    L_X f := X f.
  \end{equation}
\end{definition}
\begin{definition}
  Let $M$ be a smooth manifold.
  Define the \textbf{Lie derivative on vector fields},
  $L \colon \mathfrak{X} M \to {\rm Der}_{\mathcal{F} M}(\mathfrak{X} M)$,
  as the adjoint map, i.e., for any $X, Y \in \mathfrak{X} M$,
  \begin{equation}
    L_X Y := {\rm adj}_X Y = [X, Y].
  \end{equation}
\end{definition}
\begin{definition}
  Let $M$ be a smooth manifold, $E$ and $F$ be bundles so that there are
  Lie derivatives on their spaces of sections, i.e., for $Z \in \{E, F\}$,
  \begin{equation}
    L^{(Z)} \colon \mathfrak{X} M \to {\rm Der}_{\mathcal{F} M}(\Gamma(Z)).
  \end{equation}
  Define the \textbf{tensor product Lie derivative}
  \begin{equation}
    L^{(E \otimes F)}
    \colon \mathfrak{X} M \to {\rm Der}_{\mathcal{F} M}(\Gamma(E \otimes F)),
  \end{equation}
  such that for any $X \in \mathfrak{X} M$, $V \in \Gamma E$, $W \in \Gamma F$,
  \begin{equation}
    L^{(E \otimes F)}_X(V \otimes W)
    := L^{(E)}_X V \otimes W + V \otimes L^{(F)}_X W.
  \end{equation}
\end{definition}
\begin{definition}
  Let $M$ be a smooth manifold, $E$ and $F$ be bundles so that there are
  Lie derivatives $L^{(Z)}$ on their spaces of sections, for $Z \in \{E, F\}$.
  Define the \textbf{Hom Lie derivative}
  \begin{equation}
    L^{\Hom(E, F)}
    \colon \mathfrak{X} M
    \to {\rm Der}_{\mathcal{F} M}(\Hom(\Gamma E, \Gamma F)),
  \end{equation}
  such that for any
    $X \in \mathfrak{X} M$,
    $\varphi \in \Hom(\Gamma E, \Gamma F)$,
    $V \in \Gamma E$,
  \begin{equation}
    L^{(F)}_X(\varphi V) = (L^{\Hom(E, F)}_X \varphi) V + \varphi(L^{(E)}_X V),
  \end{equation}
  which gives the explicit formula
  \begin{equation}
    (L^{\Hom(E, F)}_X \varphi) V = L^{(F)}_X(\varphi V) - \varphi(L^{(E)}_X V).
  \end{equation}
\end{definition}
\begin{definition}
  Let $M$ be a smooth manifold, $E$ be a bundle so that there is a
  Lie derivative $L^{(E)}$ on $\Gamma E$.
  Then, specialising the definition of the Lie derivative on Hom spaces for the
  trivial bundle as the codomain, we get the \textbf{dual Lie derivative}
  \begin{equation}
    L^{(E^*)} \colon \mathfrak{X} M \to {\rm Der}_{\mathcal{F} M}(\Gamma E^*),
  \end{equation}
  defined so that for any
    $X \in \mathfrak{X} M$,
    $\varphi \in \Gamma E^*$,
    $V \in \Gamma E$,
  \begin{equation}
    (L^{(E^*)}_X \varphi) V = X(\varphi V) - \varphi(L^{(E)}_X V).
  \end{equation}
\end{definition}
\begin{definition}
  Let $M$ be a smooth manifold.
  Consider the Lie derivative on the whole tensor algebra
  (tensor products of vector and covector fields).
  For any $X, Y \in \mathfrak{X} M$ denote the Lie bracket
  \begin{equation}
    [L_X , L_Y] = L_X \circ L_Y - L_Y \circ L_X.
  \end{equation}
\end{definition}
\begin{proposition}
  Let $M$ be a smooth manifold, $X, Y \in \mathfrak{X} M$.
  Then
  \begin{equation}
    [L_X , L_Y] = L_{[X, Y]}.
  \end{equation}
\end{proposition}
\begin{proof}
  We will consider $4$ cases.
  \begin{enumerate}
    \item
      On functions the claim is obvious since $L = \id$.
    \item
      On vector fields the claim is restatement of the homomorphism nature of
      the adjoint map.
    \item
      (Tensor products.)
      Assume that
      $[L_X , L_Y] V = L_{[X, Y]} V$ and
      $[L_X , L_Y] W = L_{[X, Y]} W$.
      Then
      \begin{equation}
        \begin{split}
          [L_X , L_Y](V \otimes W)
          & = (L_X \circ L_Y - L_Y \circ L_X)(V \otimes W) \\
          & = L_X(L_Y V \otimes W + V \otimes L_Y W)
            - L_Y(L_X V \otimes W + V \otimes L_X W) \\
          & \begin{split}
              & = L_X(L_Y V) \otimes W + L_Y V \otimes L_X W
                + L_X V \otimes L_Y W + V \otimes L_X(L_Y W) \\
              & \quad - L_Y(L_X V) \otimes W + L_X V \otimes L_Y W
                - L_Y V \otimes L_X W + V \otimes L_Y(L_X W)
            \end{split} \\
          & = (L_X \circ L_Y - L_Y \circ L_X) V \otimes W
            + V \otimes (L_X \circ L_Y - L_Y \circ L_X) W \\
          & = [L_X, L_Y] V \otimes W + V \otimes [L_X, L_Y] W \\
          & = L_{[X, Y]} V \otimes W + V \otimes L_{[X, Y]} V \\
          & = L_{[X, Y]}(V \otimes W).
        \end{split}
      \end{equation}
    \item
      (Homomorphisms.)
      Assume that
      $[L_X , L_Y] \varphi = L_{[X, Y]} \varphi$ and
      $[L_X , L_Y] V = L_{[X, Y]} V$.
      Then
      \begin{equation}
        \begin{split}
          ([L_X , L_Y] \phi) V
          & = ((L_X \circ L_Y - L_Y \circ L_X) \phi) V \\
          & = L_X((L_Y \phi) V) - (L_Y \phi) (L_X V)
            - (L_Y((L_X \phi) V) - (L_X \phi) (L_Y V)) \\
          & \begin{split}
              & = L_X(L_Y(\phi V) - \phi(L_Y V))
                - (L_Y(\phi (L_X V)) - \phi(L_Y(L_X V)) \\
              & \quad - (L_Y(L_X(\phi V) - \phi(L_X V))
                         - (L_X (\phi (L_Y V)) - \phi(L_X(L_Y V)))
            \end{split} \\
          & \begin{split}
              & = (L_X \circ L_Y)(\phi V) - L_X(\phi(L_Y V))
                - L_Y(\phi (L_X V)) + \phi((L_Y \circ L_X) V) \\
              & \quad - (L_Y \circ L_X) (\phi V) + L_Y (\phi(L_X V))
                         - L_X(\phi (L_Y V)) + \phi((L_X \circ L_Y) V)
            \end{split} \\
          & = [L_X, L_Y](\phi V) - \phi([L_X, L_Y] V) \\
          & = L_{[X, Y]}(\phi V) - \phi(L_{[X, Y]} V) \\
          & = (L_{[X, Y]} \phi) V.
        \end{split}
      \end{equation}
  \end{enumerate}
  By structural induction on the space of tensors, we get the desired result on
  the whole tensor algebra of $\mathfrak{X} M$.
\end{proof}
\begin{definition}
  Let
    $(M, g)$ be a (pseudo-)Riemannian manifold.
    $X \in \mathfrak{X} M$.
  We say that $X$ is a (global) \textbf{Killing vector field}
  (named after German mathematician Wilhem Killing) if
  \begin{equation}
    L_X g = 0.
  \end{equation}
  Denote by ${\rm Killing}(M, g)$ the space of all Killing vector fields on $M$.
\end{definition}
\begin{proposition}
  Let $(M, g)$ be a pseudo-Riemannian manifold.
  Then ${\rm Killing}(M, g)$ is a subalgebra of the Lie algebra of vector fields
  on $M$.
\end{proposition}
\begin{proof}
  Let $X, Y \in {\rm Killing}(M, g)$, i.e., $L_X g = L_Y g = 0$.
  Then
  \begin{equation}
    \begin{split}
      L_{[X, Y]} g
      = [L_X, L_Y] g = L_X(L_Y g) - L_Y(L_X g) = L_X(0) - L_Y(0) = 0.
      \qedhere
    \end{split}
  \end{equation}
\end{proof}
\begin{example}
  We are going to calculate the Lie algebra of the Killing vector fields on a
  pseudo-Euclidean space $(V, g)$ of dimension $D$.
  By Sylvester's law of inertia we can choose a basis in which the metric has a
  diagonal form with values
  \begin{equation}
    g_{i, j} = s_i \delta_{i, j},\ s_i \in \{-1, 1\}\ (i, j = 1, ..., D).
  \end{equation}
  In other words, in the chosen coordinate system
  \begin{equation}
    g = \sum_{i = 1}^D s_i\, d x^i \otimes d x^i.
  \end{equation}
  Let $X = \sum_{i = 1}^D f^i \partial_{x_i}$ be a Killing vector field.
  First, for any $i \in 1, ..., D$ we calculate
  \begin{equation}
    L_X(d x^i)
    = d(L_X x^i)
    = d f^i
    = \sum_{j = 1}^D (\partial_{x_j} f^i)\, d x^j.
  \end{equation}
  Then
  \begin{equation}
    0
    = L_X g
    = \sum_{i = 1}^D s_i L_X(d x^i) \otimes d x^i + d x^i \otimes L_X(d x^i)
    = \sum_{i, j = 1}^D
      (s_i \partial_{x_j} f^i + s_j \partial_{x_i} f^j)\, d x^i \otimes d x^j.
  \end{equation}
  Hence, for $i, j = 1, ..., D$,
  \begin{equation}
    s_i \partial_{x_j} f^i + s_j \partial_{x_i} f^j = 0.
  \end{equation}
  We can show that the above system of PDE leads to the following
  $D (D + 1) / 2$ linearly independent solutions:
  \begin{enumerate}
    \item
      the $D$ basis vectors fields
      \begin{equation}
        C_i := \partial_{x_i},\ i = 1, ..., D;
      \end{equation}
    \item
      the $D (D - 1) / 2$ linear vector fields (boosts)
      \begin{equation}
        B_{i, j}
        := s_j x_j \partial_{x_i} - s_i x_i \partial_{x_j},\
        1 \leq i < j \leq D.
      \end{equation}
  \end{enumerate}
\end{example}

\begin{definition}
  Let $M$ be a smooth manifold, $I$ be a real interval.
  A \textbf{smooth curve} on $M$ with domain $I$ is a smooth function
  $\gamma \colon I \to M$.
\end{definition}
\begin{definition}
  Let
    $M$ be a smooth manifold,
    $I$ be a real interval,
    $\gamma \colon I \to M$ be a smooth curve.
  We define the \textbf{derivative} of $\gamma$,
  $\dot{\gamma} \colon I \to T M$, by
  \begin{equation}
    \restrict{\dot{\gamma}}{t} \in T_{\gamma(t) M},\ t \in I,
  \end{equation}
  such that for any $f \in \mathcal{F} M$,
  \begin{equation}
    \restrict{\dot{\gamma}}{t} f := (f \circ \gamma)'(t).
  \end{equation}
  Here, $'$ denotes the standard derivative of the single-variable function
  $f \circ \gamma \colon I \to R$.
\end{definition}
\begin{definition}
  Let $M$ be a smooth manifold, $X \in \mathfrak{X} M$, $I$ be a real interval.
  We say that a smooth curve $\gamma \colon I \to M$ is an
  \textbf{integral curve} for $X$ if
  \begin{equation}
    \dot{\gamma} = X \circ \gamma.
  \end{equation}
  In other words, for any $t \in I$,
  \begin{equation}
    \restrict{\dot{\gamma}}{t} = \restrict{X}{\gamma(t)}.
  \end{equation}
\end{definition}
\begin{proposition}
  Let $M$ be a smooth manifold, $X \in \mathfrak{X} M$, $x_0 \in M$.
  Then
  \begin{enumerate}
    \item
      \textbf{Existence.}
      There exists an open interval $I$ containing $0$ and an integral curve
      $\gamma \colon I \to \R$ of $X$ with $\gamma(0) = x_0$.
    \item
      \textbf{Uniqueness.}
      In any open interval $I$ containing $0$ there exists at most one integral
      curve $\gamma \colon I \to \R$ of $X$ with $\gamma(0) = x_0$.
  \end{enumerate}
\end{proposition}
\begin{definition}
  Let $M$ be a smooth manifold, $X \in \mathfrak{X} M$, $x_0 \in M$.
  The \textbf{flow} of $X$ is the unique function
  (may not be defined everywhere)
  $\varphi \colon \R \to (M \to M)$
  satisfying the following: for any $x \in M$, 
  $\varphi_{\cdot}(x) \colon \R \to M$ is the integral curve of $X$ with
  $\varphi_0(x) = x$.
\end{definition}
\begin{example}
  Let $M := \R^2$ and
  $X := - y \frac{\partial}{\partial x} + x \frac{\partial}{\partial y}$.
  We will first find the integral curves of $X$.
  Let $\gamma \colon \R \to \R^2$ be an integral curve,
  $\gamma(t) = (u(t), v(t))$.
  Then $\dot{\gamma}(t) = (\dot{u}(t), \dot{v}(t))$
  and $(X \circ \gamma)(t) = (- v(t), u(t))$.
  Hence, we get the system
  \begin{equation}
    \dot{u} = - v,\ \dot{v} = u.
  \end{equation}
  Its soultions are of the form
  \begin{equation}
    (u(t), v(t)) = (A \cos t - B \sin t, A \sin t + B \cos t),\ A, B \in \R.
  \end{equation}
  Geometrically, they are circles centred at $0$ with radii $\sqrt{A^2 + B^2}$.

  If $\varphi \colon \R \to (\R^2 \to \R^2)$ is the flow of $X$, and has the
  above form, then for any $(x, y) \in \R^2$,
  \begin{equation}
    (x, y) = \varphi_0(x, y) = (A, B).
  \end{equation}
  Hence, the flow of $X$ is given by
  \begin{equation}
    \varphi_t(x, y)
    = (x \cos t - y \sin t, x \sin t + y \cos t)
    =
    \begin{pmatrix}
      \cos t & - \sin t \\
      \sin t & \cos t
    \end{pmatrix}
    \begin{pmatrix}
      x \\
      y
    \end{pmatrix}.
  \end{equation}
\end{example}
\begin{proposition}
  Let
    $M$ be a smooth manifold,
    $X \in \mathfrak{X} M$,
    $\varphi$ be the flow of $X$,
    $s \in \R$.
  Then $\varphi_t \colon M \to M$ is a diffeomorphism (an automorphism).
\end{proposition}
\begin{proposition}
  Let
    $M$ be a smooth manifold,
    $X \in \mathfrak{X} M$,
    $\varphi$ be the flow of $X$,
    $s, t \in \R$.
  Then
  \begin{equation}
    \varphi_{s + t} = \varphi_s \circ \varphi_t.
  \end{equation}
\end{proposition}
\begin{corollary}
  Let
    $M$ be a smooth manifold,
    $X \in \mathfrak{X} M$.
    $\varphi$ be the flow of $X$.
  If $\varphi$ is defined everywhere, then it is a group action of $(\R, +)$ on
  $\Aut M$.
\end{corollary}
\begin{remark}
  For this reason, the flow of a vector field on a manifold is also called a
  \textbf{one-parameter group of diffeomorphisms}.
\end{remark}


\section{Symplectic manifolds}
\label{section:symplectic_manifolds}
\begin{definition}
  Let
    $\mathcal{K}$ be a quasi-cubical mesh,
    $D = \dim \mathcal{K}$,
    $E$ be a vector bundle on $\mathcal{K}$,
    $p \in \{0, ..., D\}$.
  The space of \textbf{$E$-valued differential $p$-forms} is the
  $(C^0 \mathcal{K})$-module
  \begin{equation}
    \Omega^{p}(\mathcal{K}, E)
    := \Gamma(E \otimes L^p \mathcal{K})
    \simeq \Gamma E \otimes \Omega^p \mathcal{K}.
  \end{equation}
  The \textbf{algebra of $E$-valued differential forms} is the
  $(C^0 \mathcal{K})$-algebra
  \begin{equation}
    \Omega^\bullet(\mathcal{K}, E)
    := \bigoplus_{p = 0}^D \Omega^p(\mathcal{K}, E).
  \end{equation}
\end{definition}
\begin{remark}
  Clearly, $\Omega^{0}(\mathcal{K}, E) \simeq \Gamma E$.
\end{remark}
\begin{definition}
  Let
    $\mathcal{K}$ be a quasi-cubical mesh,
    $D = \dim \mathcal{K}$,
    $E$ be a vector bundle on $\mathcal{K}$,
    $p, q \in \N$.
  The \textbf{cup product of an $E$-valued differential forms with a form}
  is the map
  \begin{equation}
    \usmile
    \colon \Omega^p(\mathcal{K}, E) \times \Omega^q \mathcal{K}
    \to \Omega^{p + q}(\mathcal{K}, E),
  \end{equation}
  defined as follows: it is the unique bilinear map such that for any
  $\sigma \in \Gamma E$,
  $\omega \in \Omega^p \mathcal{K}$,
  $\eta \in \Omega^q \mathcal{K}$,
  \begin{equation}
    (\sigma \otimes \omega) \usmile \eta :=
    \sigma \otimes (\omega \smile \eta).
  \end{equation}
\end{definition}
\begin{remark}
  The extension of the cup product define above is a straightforward
  generalisation of the cup product when one of the objects has two ``legs'':
  a bundle leg and a form leg.
  (Only the form legs multiply; the bundle leg stays the same.)
  We can further generalise if we have an operation on the bundle sections
  (e.g., a Lie bracket),
  or even an external operation between the sections of different bundles
  (e.g., a tensor product). 
\end{remark}
\begin{definition}
  Let
    $\mathcal{K}$ be a quasi-cubical mesh,
    $D = \dim \mathcal{K}$,
    $E, F, G$ be vector bundles on $\mathcal{K}$,
    $\mu \colon \Gamma E \times \Gamma F \to \Gamma G$ be a bilinear map,
    $p, q \in \N$.
  The \textbf{$\mu$-cup product of bundle-valued differential forms} is the map
  \begin{equation}
    \smile_\mu
    \colon \Omega^p(\mathcal{K}, E) \times \Omega^q(\mathcal{K}, F)
    \to \Omega^{p + q}(\mathcal{K}, G),
  \end{equation}
  defined as follows: it is the unique bilinear map such that for any
  $\sigma \in \Gamma E$,
  $\tau \in \Gamma F$,
  $\omega \in \Omega^p \mathcal{K}$,
  $\eta \in \Omega^q \mathcal{K}$,
  \begin{equation}
    (\sigma \otimes \omega) \smile_\mu (\tau \otimes \eta) :=
    \mu(\sigma, \tau) \otimes (\omega \smile \eta).
  \end{equation}
\end{definition}
\begin{example}
  Let
    $\mathcal{K}$ be a quasi-cubical mesh,
    $D = \dim \mathcal{K}$,
    $p, q \in \N$.
  The \textbf{$[\cdot, \cdot]$-cup product of vector-valued differential forms}
  is the map
  \begin{equation}
    \smile_{[\cdot, \cdot]}
    \colon \Omega^p(\mathcal{K}, \mathfrak{X} \mathcal{K})
    \times \Omega^q(\mathcal{K}, \mathfrak{X} \mathcal{K})
    \to \Omega^{p + q}(\mathcal{K}, \mathfrak{X} \mathcal{K}),
  \end{equation}
  defined as follows: it is the unique bilinear map such that for any
  $X, Y \in \mathfrak{X} \mathcal{K}$,
  $\omega \in \Omega^p \mathcal{K}$,
  $\eta \in \Omega^q \mathcal{K}$,
  \begin{equation}
    (X \otimes \omega) \smile_{[\cdot, \cdot]} (Y \otimes \eta) :=
    [X, Y] \otimes (\omega \smile \eta).
  \end{equation}
\end{example}


\part{Meshes}

\section{Meshes}
\label{section:meshes}
\input{mesh/concept-definition.tex}

\section{Relative orientation on meshes}
\label{section:relative_orientation_on_meshes}
\input{mesh/diamond_property-theorem.tex}
\input{mesh/relative_orientation/concept-definition.tex}
\input{mesh/relative_orientation/concept-remark.tex}
\input{mesh/relative_orientation/existence-theorem.tex}

\section{Chains and boundary operator on meshes}
\label{section:chains_and_boundary_operator_on_meshes}
\input{mesh/chain/p-definition.tex}
\input{mesh/chain/concept-definition.tex}
\input{mesh/boundary_operator/concept-definition.tex}
\input{mesh/boundary_operator/is_chain_complex-proposition.tex}
\input{mesh/boundary_operator/is_chain_complex-proof.tex}
\input{mesh/boundary_operator/is_unique_up_to_isomorphism-proposition.tex}
\input{mesh/boundary_operator/is_unique_up_to_isomorphism-remark.tex}

\section{Cochains and coboundary operator on meshes}
\label{section:cochains_and_coboundary_operator_on_meshes}
\input{mesh/cochain/concept-definition.tex}
\input{mesh/coboundary_operator/concept-definition.tex}
\input{mesh/coboundary_operator/is_cochain_complex-proposition.tex}
\input{mesh/coboundary_operator/is_cochain_complex-proof.tex}

\section{Combinatorial differential forms and Forman subdivision}
\label{section:combinatorial_differential_forms_and_forman_subdivision}
\input{mesh/combinatorial_differential_form/concept-definition.tex}
\input{mesh/combinatorial_differential_form/discrete_exterior_derivative-definition.tex}
\input{mesh/combinatorial_differential_form/is_cochain_complex-proposition.tex}
\input{mesh/combinatorial_differential_form/is_cochain_complex-proof.tex}
\input{mesh/forman_subdivision/concept-definition.tex}
\input{mesh/forman_subdivision/concept-example.tex}
\input{mesh/forman_subdivision/concept-figure.tex}
\input{mesh/forman_subdivision/relative_orientation-definition.tex}
\input{mesh/forman_subdivision/cochains_isomorphic_to_original_forms-theorem.tex}
\input{quasi_cube/concept-definition.tex}
\input{quasi_cube/concept-example.tex}
\input{mesh/quasi_cubical/concept-definition.tex}
\input{mesh/quasi_cubical/topological_parallelism-definition.tex}
\input{partially_ordered_set/closed_interval-definition.tex}
\input{mesh/interval_simplicial-definition.tex}
\input{mesh/forman_subdivision/is_quasi_cubical_iff_original_is_interval_simplicial-proposition.tex}
\input{mesh/forman_subdivision/is_quasi_cubical_in_up_to_2d-proposition.tex}
\input{polytope/simple-definition.tex}
\input{mesh/forman_subdivision/is_quasi_cubical_in_3d_iff_original_is_simple-proposition.tex}
\input{mesh/quasi_cubical/partial_circ_perp_equals_perp_circ_delta-proposition.tex}

\section{Metric-dependent calculus on quasi-cubical meshes}
\label{section:metric_dependent_calculus_on_quasi_cubical_meshes}
\input{mesh/interval_simplicial_are_not_uncommon-discussion.tex}
\input{mesh/quasi_cubical/topological_orthogonality-definition.tex}
\input{euclidean_measure-notation.tex}
\input{mesh/quasi_cubical/inner_product/concept-definition.tex}
\input{mesh/quasi_cubical/inner_product/regular_cube-example.tex}
\input{mesh/quasi_cubical/adjoint_coboundary/concept-definition.tex}
\begin{proposition}
  Let
    $D \in \N$,
    $K$ be a compatibly oriented quasi-cubical
    \hyperref[cmc:mesh:definition]{mesh} of dimension $D$,
    $[K] := \sum_{c_D \in K_D} c^D$ be the fundamental class of $K$,
    $\inner{\cdot}{\cdot}$ be an orthogonal inner product on $K$,
    $p \in \{0, ..., D\}$.
  The \hyperref[cmc/mesh/quasi_cubical/hodge_star/concept-definition]
               {Hodge star operator}
  $\star_p \colon C^p K \to C^{D - p} K$ has the following closed form:
  for any $\pi^p \in C^p K$ and any $c^{D - p} \in C^{D - p} K$,
  \begin{equation}
    (\star_p \pi^p)(b_{D - p})
    = \sum_{a_p \perp b_{D - p}}
      \frac{(a^p \smile b^{D - p})[K]}{\inner{b^{D - p}}{b^{D - p}}} \pi^p(a_p).
  \end{equation}
\end{proposition}

\input{mesh/quasi_cubical/adjoint_coboundary/regular_cube_closed_form-corollary.tex}
\input{mesh/quasi_cubical/hodge_star/concept-definition.tex}
\begin{proposition}
  Let
    $D \in \N$,
    $K$ be a compatibly oriented quasi-cubical
    \hyperref[cmc:mesh:definition]{mesh} of dimension $D$,
    $[K] := \sum_{c_D \in K_D} c^D$ be the fundamental class of $K$,
    $\inner{\cdot}{\cdot}$ be an orthogonal inner product on $K$,
    $p \in \{0, ..., D\}$.
  The \hyperref[cmc/mesh/quasi_cubical/hodge_star/concept-definition]
               {Hodge star operator}
  $\star_p \colon C^p K \to C^{D - p} K$ has the following closed form:
  for any $\pi^p \in C^p K$ and any $c^{D - p} \in C^{D - p} K$,
  \begin{equation}
    (\star_p \pi^p)(b_{D - p})
    = \sum_{a_p \perp b_{D - p}}
      \frac{(a^p \smile b^{D - p})[K]}{\inner{b^{D - p}}{b^{D - p}}} \pi^p(a_p).
  \end{equation}
\end{proposition}


\section{Product meshes}
\label{section:product_meshes}
\input{mesh/product-all.tex}

\section{Approximating vector fields with 1-cochains}
\label{section:approximating_vector_fields_with_1_cochains}
\input{moore_penrose_inverse/concept-definition.tex}
\input{moore_penrose_inverse/physical_dimension-remark.tex}
\input{moore_penrose_inverse/existence_and_uniqueness-theorem.tex}
\input{moore_penrose_inverse/full_rank_closed_form-remark.tex}
\input{mesh/flat/concept-definition.tex}
\input{mesh/flat/node_matrix-definition.tex}
\input{mesh/relative_orientation/neighbor_representation_of_1_cochain-definition.tex}
\input{mesh/flat/embedding_of_1_cochain-definition.tex}
\input{mesh/flat/embedding_of_1_cochain-example.tex}
\input{mesh/flat/approximation_of_vector_field-definition.tex}
\input{mesh/flat/approximation_of_vector_field-example.tex}
\input{mesh/flat/summary_of_operators-discussion.tex}

\section{Vector fields on combinatorial meshes}
\label{section:vector_fields_on_combinatorial_meshes}
\input{mesh/quasi_cubical/vector_field/concept-definition.tex}
\input{mesh/quasi_cubical/vector_field/interior_product_on_1_cochains-definition.tex}
\input{mesh/quasi_cubical/vector_field/lie_derivative_on_0_cochains-definition.tex}
\input{mesh/quasi_cubical/vector_field/commutator-definition.tex}
\input{mesh/quasi_cubical/vector_field/discrete_lie_bracket_approximates_continuum_lie_bracket_on_a_regular_grid-discussion.tex}

\section{Discrete vector bundles}
\label{section:discrete_vector_bundles}
\begin{definition}
  Let
    $\mathcal{K}$ be a quasi-cubical mesh,
    $D = \dim \mathcal{K}$,
    $E$ be a vector bundle on $\mathcal{K}$,
    $p \in \{0, ..., D\}$.
  The space of \textbf{$E$-valued differential $p$-forms} is the
  $(C^0 \mathcal{K})$-module
  \begin{equation}
    \Omega^{p}(\mathcal{K}, E)
    := \Gamma(E \otimes L^p \mathcal{K})
    \simeq \Gamma E \otimes \Omega^p \mathcal{K}.
  \end{equation}
  The \textbf{algebra of $E$-valued differential forms} is the
  $(C^0 \mathcal{K})$-algebra
  \begin{equation}
    \Omega^\bullet(\mathcal{K}, E)
    := \bigoplus_{p = 0}^D \Omega^p(\mathcal{K}, E).
  \end{equation}
\end{definition}
\begin{remark}
  Clearly, $\Omega^{0}(\mathcal{K}, E) \simeq \Gamma E$.
\end{remark}
\begin{definition}
  Let
    $\mathcal{K}$ be a quasi-cubical mesh,
    $D = \dim \mathcal{K}$,
    $E$ be a vector bundle on $\mathcal{K}$,
    $p, q \in \N$.
  The \textbf{cup product of an $E$-valued differential forms with a form}
  is the map
  \begin{equation}
    \usmile
    \colon \Omega^p(\mathcal{K}, E) \times \Omega^q \mathcal{K}
    \to \Omega^{p + q}(\mathcal{K}, E),
  \end{equation}
  defined as follows: it is the unique bilinear map such that for any
  $\sigma \in \Gamma E$,
  $\omega \in \Omega^p \mathcal{K}$,
  $\eta \in \Omega^q \mathcal{K}$,
  \begin{equation}
    (\sigma \otimes \omega) \usmile \eta :=
    \sigma \otimes (\omega \smile \eta).
  \end{equation}
\end{definition}
\begin{remark}
  The extension of the cup product define above is a straightforward
  generalisation of the cup product when one of the objects has two ``legs'':
  a bundle leg and a form leg.
  (Only the form legs multiply; the bundle leg stays the same.)
  We can further generalise if we have an operation on the bundle sections
  (e.g., a Lie bracket),
  or even an external operation between the sections of different bundles
  (e.g., a tensor product). 
\end{remark}
\begin{definition}
  Let
    $\mathcal{K}$ be a quasi-cubical mesh,
    $D = \dim \mathcal{K}$,
    $E, F, G$ be vector bundles on $\mathcal{K}$,
    $\mu \colon \Gamma E \times \Gamma F \to \Gamma G$ be a bilinear map,
    $p, q \in \N$.
  The \textbf{$\mu$-cup product of bundle-valued differential forms} is the map
  \begin{equation}
    \smile_\mu
    \colon \Omega^p(\mathcal{K}, E) \times \Omega^q(\mathcal{K}, F)
    \to \Omega^{p + q}(\mathcal{K}, G),
  \end{equation}
  defined as follows: it is the unique bilinear map such that for any
  $\sigma \in \Gamma E$,
  $\tau \in \Gamma F$,
  $\omega \in \Omega^p \mathcal{K}$,
  $\eta \in \Omega^q \mathcal{K}$,
  \begin{equation}
    (\sigma \otimes \omega) \smile_\mu (\tau \otimes \eta) :=
    \mu(\sigma, \tau) \otimes (\omega \smile \eta).
  \end{equation}
\end{definition}
\begin{example}
  Let
    $\mathcal{K}$ be a quasi-cubical mesh,
    $D = \dim \mathcal{K}$,
    $p, q \in \N$.
  The \textbf{$[\cdot, \cdot]$-cup product of vector-valued differential forms}
  is the map
  \begin{equation}
    \smile_{[\cdot, \cdot]}
    \colon \Omega^p(\mathcal{K}, \mathfrak{X} \mathcal{K})
    \times \Omega^q(\mathcal{K}, \mathfrak{X} \mathcal{K})
    \to \Omega^{p + q}(\mathcal{K}, \mathfrak{X} \mathcal{K}),
  \end{equation}
  defined as follows: it is the unique bilinear map such that for any
  $X, Y \in \mathfrak{X} \mathcal{K}$,
  $\omega \in \Omega^p \mathcal{K}$,
  $\eta \in \Omega^q \mathcal{K}$,
  \begin{equation}
    (X \otimes \omega) \smile_{[\cdot, \cdot]} (Y \otimes \eta) :=
    [X, Y] \otimes (\omega \smile \eta).
  \end{equation}
\end{example}


\section{Sections on discrete vector bundles}
\label{section:sections_on_discrete_vector bundles}
\begin{definition}
  Let
    $\mathcal{K}$ be a quasi-cubical mesh,
    $D = \dim \mathcal{K}$,
    $E$ be a vector bundle on $\mathcal{K}$,
    $p \in \{0, ..., D\}$.
  The space of \textbf{$E$-valued differential $p$-forms} is the
  $(C^0 \mathcal{K})$-module
  \begin{equation}
    \Omega^{p}(\mathcal{K}, E)
    := \Gamma(E \otimes L^p \mathcal{K})
    \simeq \Gamma E \otimes \Omega^p \mathcal{K}.
  \end{equation}
  The \textbf{algebra of $E$-valued differential forms} is the
  $(C^0 \mathcal{K})$-algebra
  \begin{equation}
    \Omega^\bullet(\mathcal{K}, E)
    := \bigoplus_{p = 0}^D \Omega^p(\mathcal{K}, E).
  \end{equation}
\end{definition}
\begin{remark}
  Clearly, $\Omega^{0}(\mathcal{K}, E) \simeq \Gamma E$.
\end{remark}
\begin{definition}
  Let
    $\mathcal{K}$ be a quasi-cubical mesh,
    $D = \dim \mathcal{K}$,
    $E$ be a vector bundle on $\mathcal{K}$,
    $p, q \in \N$.
  The \textbf{cup product of an $E$-valued differential forms with a form}
  is the map
  \begin{equation}
    \usmile
    \colon \Omega^p(\mathcal{K}, E) \times \Omega^q \mathcal{K}
    \to \Omega^{p + q}(\mathcal{K}, E),
  \end{equation}
  defined as follows: it is the unique bilinear map such that for any
  $\sigma \in \Gamma E$,
  $\omega \in \Omega^p \mathcal{K}$,
  $\eta \in \Omega^q \mathcal{K}$,
  \begin{equation}
    (\sigma \otimes \omega) \usmile \eta :=
    \sigma \otimes (\omega \smile \eta).
  \end{equation}
\end{definition}
\begin{remark}
  The extension of the cup product define above is a straightforward
  generalisation of the cup product when one of the objects has two ``legs'':
  a bundle leg and a form leg.
  (Only the form legs multiply; the bundle leg stays the same.)
  We can further generalise if we have an operation on the bundle sections
  (e.g., a Lie bracket),
  or even an external operation between the sections of different bundles
  (e.g., a tensor product). 
\end{remark}
\begin{definition}
  Let
    $\mathcal{K}$ be a quasi-cubical mesh,
    $D = \dim \mathcal{K}$,
    $E, F, G$ be vector bundles on $\mathcal{K}$,
    $\mu \colon \Gamma E \times \Gamma F \to \Gamma G$ be a bilinear map,
    $p, q \in \N$.
  The \textbf{$\mu$-cup product of bundle-valued differential forms} is the map
  \begin{equation}
    \smile_\mu
    \colon \Omega^p(\mathcal{K}, E) \times \Omega^q(\mathcal{K}, F)
    \to \Omega^{p + q}(\mathcal{K}, G),
  \end{equation}
  defined as follows: it is the unique bilinear map such that for any
  $\sigma \in \Gamma E$,
  $\tau \in \Gamma F$,
  $\omega \in \Omega^p \mathcal{K}$,
  $\eta \in \Omega^q \mathcal{K}$,
  \begin{equation}
    (\sigma \otimes \omega) \smile_\mu (\tau \otimes \eta) :=
    \mu(\sigma, \tau) \otimes (\omega \smile \eta).
  \end{equation}
\end{definition}
\begin{example}
  Let
    $\mathcal{K}$ be a quasi-cubical mesh,
    $D = \dim \mathcal{K}$,
    $p, q \in \N$.
  The \textbf{$[\cdot, \cdot]$-cup product of vector-valued differential forms}
  is the map
  \begin{equation}
    \smile_{[\cdot, \cdot]}
    \colon \Omega^p(\mathcal{K}, \mathfrak{X} \mathcal{K})
    \times \Omega^q(\mathcal{K}, \mathfrak{X} \mathcal{K})
    \to \Omega^{p + q}(\mathcal{K}, \mathfrak{X} \mathcal{K}),
  \end{equation}
  defined as follows: it is the unique bilinear map such that for any
  $X, Y \in \mathfrak{X} \mathcal{K}$,
  $\omega \in \Omega^p \mathcal{K}$,
  $\eta \in \Omega^q \mathcal{K}$,
  \begin{equation}
    (X \otimes \omega) \smile_{[\cdot, \cdot]} (Y \otimes \eta) :=
    [X, Y] \otimes (\omega \smile \eta).
  \end{equation}
\end{example}


\section{Chain and cochain spaces on quasi-cubical meshes}
\label{section:chain_and_cochain_spaces_on_quasi-cubical_meshes}
\begin{definition}
  Let
    $\mathcal{K}$ be a quasi-cubical mesh,
    $D = \dim \mathcal{K}$,
    $E$ be a vector bundle on $\mathcal{K}$,
    $p \in \{0, ..., D\}$.
  The space of \textbf{$E$-valued differential $p$-forms} is the
  $(C^0 \mathcal{K})$-module
  \begin{equation}
    \Omega^{p}(\mathcal{K}, E)
    := \Gamma(E \otimes L^p \mathcal{K})
    \simeq \Gamma E \otimes \Omega^p \mathcal{K}.
  \end{equation}
  The \textbf{algebra of $E$-valued differential forms} is the
  $(C^0 \mathcal{K})$-algebra
  \begin{equation}
    \Omega^\bullet(\mathcal{K}, E)
    := \bigoplus_{p = 0}^D \Omega^p(\mathcal{K}, E).
  \end{equation}
\end{definition}
\begin{remark}
  Clearly, $\Omega^{0}(\mathcal{K}, E) \simeq \Gamma E$.
\end{remark}
\begin{definition}
  Let
    $\mathcal{K}$ be a quasi-cubical mesh,
    $D = \dim \mathcal{K}$,
    $E$ be a vector bundle on $\mathcal{K}$,
    $p, q \in \N$.
  The \textbf{cup product of an $E$-valued differential forms with a form}
  is the map
  \begin{equation}
    \usmile
    \colon \Omega^p(\mathcal{K}, E) \times \Omega^q \mathcal{K}
    \to \Omega^{p + q}(\mathcal{K}, E),
  \end{equation}
  defined as follows: it is the unique bilinear map such that for any
  $\sigma \in \Gamma E$,
  $\omega \in \Omega^p \mathcal{K}$,
  $\eta \in \Omega^q \mathcal{K}$,
  \begin{equation}
    (\sigma \otimes \omega) \usmile \eta :=
    \sigma \otimes (\omega \smile \eta).
  \end{equation}
\end{definition}
\begin{remark}
  The extension of the cup product define above is a straightforward
  generalisation of the cup product when one of the objects has two ``legs'':
  a bundle leg and a form leg.
  (Only the form legs multiply; the bundle leg stays the same.)
  We can further generalise if we have an operation on the bundle sections
  (e.g., a Lie bracket),
  or even an external operation between the sections of different bundles
  (e.g., a tensor product). 
\end{remark}
\begin{definition}
  Let
    $\mathcal{K}$ be a quasi-cubical mesh,
    $D = \dim \mathcal{K}$,
    $E, F, G$ be vector bundles on $\mathcal{K}$,
    $\mu \colon \Gamma E \times \Gamma F \to \Gamma G$ be a bilinear map,
    $p, q \in \N$.
  The \textbf{$\mu$-cup product of bundle-valued differential forms} is the map
  \begin{equation}
    \smile_\mu
    \colon \Omega^p(\mathcal{K}, E) \times \Omega^q(\mathcal{K}, F)
    \to \Omega^{p + q}(\mathcal{K}, G),
  \end{equation}
  defined as follows: it is the unique bilinear map such that for any
  $\sigma \in \Gamma E$,
  $\tau \in \Gamma F$,
  $\omega \in \Omega^p \mathcal{K}$,
  $\eta \in \Omega^q \mathcal{K}$,
  \begin{equation}
    (\sigma \otimes \omega) \smile_\mu (\tau \otimes \eta) :=
    \mu(\sigma, \tau) \otimes (\omega \smile \eta).
  \end{equation}
\end{definition}
\begin{example}
  Let
    $\mathcal{K}$ be a quasi-cubical mesh,
    $D = \dim \mathcal{K}$,
    $p, q \in \N$.
  The \textbf{$[\cdot, \cdot]$-cup product of vector-valued differential forms}
  is the map
  \begin{equation}
    \smile_{[\cdot, \cdot]}
    \colon \Omega^p(\mathcal{K}, \mathfrak{X} \mathcal{K})
    \times \Omega^q(\mathcal{K}, \mathfrak{X} \mathcal{K})
    \to \Omega^{p + q}(\mathcal{K}, \mathfrak{X} \mathcal{K}),
  \end{equation}
  defined as follows: it is the unique bilinear map such that for any
  $X, Y \in \mathfrak{X} \mathcal{K}$,
  $\omega \in \Omega^p \mathcal{K}$,
  $\eta \in \Omega^q \mathcal{K}$,
  \begin{equation}
    (X \otimes \omega) \smile_{[\cdot, \cdot]} (Y \otimes \eta) :=
    [X, Y] \otimes (\omega \smile \eta).
  \end{equation}
\end{example}


\section{Discrete bundle-valued differential forms}
\label{section:discrete_bundle-valued_differential_forms}
\begin{definition}
  Let
    $\mathcal{K}$ be a quasi-cubical mesh,
    $D = \dim \mathcal{K}$,
    $E$ be a vector bundle on $\mathcal{K}$,
    $p \in \{0, ..., D\}$.
  The space of \textbf{$E$-valued differential $p$-forms} is the
  $(C^0 \mathcal{K})$-module
  \begin{equation}
    \Omega^{p}(\mathcal{K}, E)
    := \Gamma(E \otimes L^p \mathcal{K})
    \simeq \Gamma E \otimes \Omega^p \mathcal{K}.
  \end{equation}
  The \textbf{algebra of $E$-valued differential forms} is the
  $(C^0 \mathcal{K})$-algebra
  \begin{equation}
    \Omega^\bullet(\mathcal{K}, E)
    := \bigoplus_{p = 0}^D \Omega^p(\mathcal{K}, E).
  \end{equation}
\end{definition}
\begin{remark}
  Clearly, $\Omega^{0}(\mathcal{K}, E) \simeq \Gamma E$.
\end{remark}
\begin{definition}
  Let
    $\mathcal{K}$ be a quasi-cubical mesh,
    $D = \dim \mathcal{K}$,
    $E$ be a vector bundle on $\mathcal{K}$,
    $p, q \in \N$.
  The \textbf{cup product of an $E$-valued differential forms with a form}
  is the map
  \begin{equation}
    \usmile
    \colon \Omega^p(\mathcal{K}, E) \times \Omega^q \mathcal{K}
    \to \Omega^{p + q}(\mathcal{K}, E),
  \end{equation}
  defined as follows: it is the unique bilinear map such that for any
  $\sigma \in \Gamma E$,
  $\omega \in \Omega^p \mathcal{K}$,
  $\eta \in \Omega^q \mathcal{K}$,
  \begin{equation}
    (\sigma \otimes \omega) \usmile \eta :=
    \sigma \otimes (\omega \smile \eta).
  \end{equation}
\end{definition}
\begin{remark}
  The extension of the cup product define above is a straightforward
  generalisation of the cup product when one of the objects has two ``legs'':
  a bundle leg and a form leg.
  (Only the form legs multiply; the bundle leg stays the same.)
  We can further generalise if we have an operation on the bundle sections
  (e.g., a Lie bracket),
  or even an external operation between the sections of different bundles
  (e.g., a tensor product). 
\end{remark}
\begin{definition}
  Let
    $\mathcal{K}$ be a quasi-cubical mesh,
    $D = \dim \mathcal{K}$,
    $E, F, G$ be vector bundles on $\mathcal{K}$,
    $\mu \colon \Gamma E \times \Gamma F \to \Gamma G$ be a bilinear map,
    $p, q \in \N$.
  The \textbf{$\mu$-cup product of bundle-valued differential forms} is the map
  \begin{equation}
    \smile_\mu
    \colon \Omega^p(\mathcal{K}, E) \times \Omega^q(\mathcal{K}, F)
    \to \Omega^{p + q}(\mathcal{K}, G),
  \end{equation}
  defined as follows: it is the unique bilinear map such that for any
  $\sigma \in \Gamma E$,
  $\tau \in \Gamma F$,
  $\omega \in \Omega^p \mathcal{K}$,
  $\eta \in \Omega^q \mathcal{K}$,
  \begin{equation}
    (\sigma \otimes \omega) \smile_\mu (\tau \otimes \eta) :=
    \mu(\sigma, \tau) \otimes (\omega \smile \eta).
  \end{equation}
\end{definition}
\begin{example}
  Let
    $\mathcal{K}$ be a quasi-cubical mesh,
    $D = \dim \mathcal{K}$,
    $p, q \in \N$.
  The \textbf{$[\cdot, \cdot]$-cup product of vector-valued differential forms}
  is the map
  \begin{equation}
    \smile_{[\cdot, \cdot]}
    \colon \Omega^p(\mathcal{K}, \mathfrak{X} \mathcal{K})
    \times \Omega^q(\mathcal{K}, \mathfrak{X} \mathcal{K})
    \to \Omega^{p + q}(\mathcal{K}, \mathfrak{X} \mathcal{K}),
  \end{equation}
  defined as follows: it is the unique bilinear map such that for any
  $X, Y \in \mathfrak{X} \mathcal{K}$,
  $\omega \in \Omega^p \mathcal{K}$,
  $\eta \in \Omega^q \mathcal{K}$,
  \begin{equation}
    (X \otimes \omega) \smile_{[\cdot, \cdot]} (Y \otimes \eta) :=
    [X, Y] \otimes (\omega \smile \eta).
  \end{equation}
\end{example}


\section{Connections on discrete vector bundles}
\label{section:connections_on_discrete_vector_bundles}
\begin{definition}
  Let
    $\mathcal{K}$ be a quasi-cubical mesh,
    $D = \dim \mathcal{K}$,
    $E$ be a vector bundle on $\mathcal{K}$,
    $p \in \{0, ..., D\}$.
  The space of \textbf{$E$-valued differential $p$-forms} is the
  $(C^0 \mathcal{K})$-module
  \begin{equation}
    \Omega^{p}(\mathcal{K}, E)
    := \Gamma(E \otimes L^p \mathcal{K})
    \simeq \Gamma E \otimes \Omega^p \mathcal{K}.
  \end{equation}
  The \textbf{algebra of $E$-valued differential forms} is the
  $(C^0 \mathcal{K})$-algebra
  \begin{equation}
    \Omega^\bullet(\mathcal{K}, E)
    := \bigoplus_{p = 0}^D \Omega^p(\mathcal{K}, E).
  \end{equation}
\end{definition}
\begin{remark}
  Clearly, $\Omega^{0}(\mathcal{K}, E) \simeq \Gamma E$.
\end{remark}
\begin{definition}
  Let
    $\mathcal{K}$ be a quasi-cubical mesh,
    $D = \dim \mathcal{K}$,
    $E$ be a vector bundle on $\mathcal{K}$,
    $p, q \in \N$.
  The \textbf{cup product of an $E$-valued differential forms with a form}
  is the map
  \begin{equation}
    \usmile
    \colon \Omega^p(\mathcal{K}, E) \times \Omega^q \mathcal{K}
    \to \Omega^{p + q}(\mathcal{K}, E),
  \end{equation}
  defined as follows: it is the unique bilinear map such that for any
  $\sigma \in \Gamma E$,
  $\omega \in \Omega^p \mathcal{K}$,
  $\eta \in \Omega^q \mathcal{K}$,
  \begin{equation}
    (\sigma \otimes \omega) \usmile \eta :=
    \sigma \otimes (\omega \smile \eta).
  \end{equation}
\end{definition}
\begin{remark}
  The extension of the cup product define above is a straightforward
  generalisation of the cup product when one of the objects has two ``legs'':
  a bundle leg and a form leg.
  (Only the form legs multiply; the bundle leg stays the same.)
  We can further generalise if we have an operation on the bundle sections
  (e.g., a Lie bracket),
  or even an external operation between the sections of different bundles
  (e.g., a tensor product). 
\end{remark}
\begin{definition}
  Let
    $\mathcal{K}$ be a quasi-cubical mesh,
    $D = \dim \mathcal{K}$,
    $E, F, G$ be vector bundles on $\mathcal{K}$,
    $\mu \colon \Gamma E \times \Gamma F \to \Gamma G$ be a bilinear map,
    $p, q \in \N$.
  The \textbf{$\mu$-cup product of bundle-valued differential forms} is the map
  \begin{equation}
    \smile_\mu
    \colon \Omega^p(\mathcal{K}, E) \times \Omega^q(\mathcal{K}, F)
    \to \Omega^{p + q}(\mathcal{K}, G),
  \end{equation}
  defined as follows: it is the unique bilinear map such that for any
  $\sigma \in \Gamma E$,
  $\tau \in \Gamma F$,
  $\omega \in \Omega^p \mathcal{K}$,
  $\eta \in \Omega^q \mathcal{K}$,
  \begin{equation}
    (\sigma \otimes \omega) \smile_\mu (\tau \otimes \eta) :=
    \mu(\sigma, \tau) \otimes (\omega \smile \eta).
  \end{equation}
\end{definition}
\begin{example}
  Let
    $\mathcal{K}$ be a quasi-cubical mesh,
    $D = \dim \mathcal{K}$,
    $p, q \in \N$.
  The \textbf{$[\cdot, \cdot]$-cup product of vector-valued differential forms}
  is the map
  \begin{equation}
    \smile_{[\cdot, \cdot]}
    \colon \Omega^p(\mathcal{K}, \mathfrak{X} \mathcal{K})
    \times \Omega^q(\mathcal{K}, \mathfrak{X} \mathcal{K})
    \to \Omega^{p + q}(\mathcal{K}, \mathfrak{X} \mathcal{K}),
  \end{equation}
  defined as follows: it is the unique bilinear map such that for any
  $X, Y \in \mathfrak{X} \mathcal{K}$,
  $\omega \in \Omega^p \mathcal{K}$,
  $\eta \in \Omega^q \mathcal{K}$,
  \begin{equation}
    (X \otimes \omega) \smile_{[\cdot, \cdot]} (Y \otimes \eta) :=
    [X, Y] \otimes (\omega \smile \eta).
  \end{equation}
\end{example}


\section{Discerete covariant exterior derivative}
\label{section:discrete_covariant_exterior_derivative}
\begin{definition}
  Let
    $\mathcal{K}$ be a quasi-cubical mesh,
    $D = \dim \mathcal{K}$,
    $E$ be a vector bundle on $\mathcal{K}$,
    $p \in \{0, ..., D\}$.
  The space of \textbf{$E$-valued differential $p$-forms} is the
  $(C^0 \mathcal{K})$-module
  \begin{equation}
    \Omega^{p}(\mathcal{K}, E)
    := \Gamma(E \otimes L^p \mathcal{K})
    \simeq \Gamma E \otimes \Omega^p \mathcal{K}.
  \end{equation}
  The \textbf{algebra of $E$-valued differential forms} is the
  $(C^0 \mathcal{K})$-algebra
  \begin{equation}
    \Omega^\bullet(\mathcal{K}, E)
    := \bigoplus_{p = 0}^D \Omega^p(\mathcal{K}, E).
  \end{equation}
\end{definition}
\begin{remark}
  Clearly, $\Omega^{0}(\mathcal{K}, E) \simeq \Gamma E$.
\end{remark}
\begin{definition}
  Let
    $\mathcal{K}$ be a quasi-cubical mesh,
    $D = \dim \mathcal{K}$,
    $E$ be a vector bundle on $\mathcal{K}$,
    $p, q \in \N$.
  The \textbf{cup product of an $E$-valued differential forms with a form}
  is the map
  \begin{equation}
    \usmile
    \colon \Omega^p(\mathcal{K}, E) \times \Omega^q \mathcal{K}
    \to \Omega^{p + q}(\mathcal{K}, E),
  \end{equation}
  defined as follows: it is the unique bilinear map such that for any
  $\sigma \in \Gamma E$,
  $\omega \in \Omega^p \mathcal{K}$,
  $\eta \in \Omega^q \mathcal{K}$,
  \begin{equation}
    (\sigma \otimes \omega) \usmile \eta :=
    \sigma \otimes (\omega \smile \eta).
  \end{equation}
\end{definition}
\begin{remark}
  The extension of the cup product define above is a straightforward
  generalisation of the cup product when one of the objects has two ``legs'':
  a bundle leg and a form leg.
  (Only the form legs multiply; the bundle leg stays the same.)
  We can further generalise if we have an operation on the bundle sections
  (e.g., a Lie bracket),
  or even an external operation between the sections of different bundles
  (e.g., a tensor product). 
\end{remark}
\begin{definition}
  Let
    $\mathcal{K}$ be a quasi-cubical mesh,
    $D = \dim \mathcal{K}$,
    $E, F, G$ be vector bundles on $\mathcal{K}$,
    $\mu \colon \Gamma E \times \Gamma F \to \Gamma G$ be a bilinear map,
    $p, q \in \N$.
  The \textbf{$\mu$-cup product of bundle-valued differential forms} is the map
  \begin{equation}
    \smile_\mu
    \colon \Omega^p(\mathcal{K}, E) \times \Omega^q(\mathcal{K}, F)
    \to \Omega^{p + q}(\mathcal{K}, G),
  \end{equation}
  defined as follows: it is the unique bilinear map such that for any
  $\sigma \in \Gamma E$,
  $\tau \in \Gamma F$,
  $\omega \in \Omega^p \mathcal{K}$,
  $\eta \in \Omega^q \mathcal{K}$,
  \begin{equation}
    (\sigma \otimes \omega) \smile_\mu (\tau \otimes \eta) :=
    \mu(\sigma, \tau) \otimes (\omega \smile \eta).
  \end{equation}
\end{definition}
\begin{example}
  Let
    $\mathcal{K}$ be a quasi-cubical mesh,
    $D = \dim \mathcal{K}$,
    $p, q \in \N$.
  The \textbf{$[\cdot, \cdot]$-cup product of vector-valued differential forms}
  is the map
  \begin{equation}
    \smile_{[\cdot, \cdot]}
    \colon \Omega^p(\mathcal{K}, \mathfrak{X} \mathcal{K})
    \times \Omega^q(\mathcal{K}, \mathfrak{X} \mathcal{K})
    \to \Omega^{p + q}(\mathcal{K}, \mathfrak{X} \mathcal{K}),
  \end{equation}
  defined as follows: it is the unique bilinear map such that for any
  $X, Y \in \mathfrak{X} \mathcal{K}$,
  $\omega \in \Omega^p \mathcal{K}$,
  $\eta \in \Omega^q \mathcal{K}$,
  \begin{equation}
    (X \otimes \omega) \smile_{[\cdot, \cdot]} (Y \otimes \eta) :=
    [X, Y] \otimes (\omega \smile \eta).
  \end{equation}
\end{example}


\part{Physics}

\section{Continuous heat transport}
\label{section:continuous_diffusion}
\input{diffusion/continuous/introduction-discussion.tex}
\input{diffusion/continuous/model_derivation-discussion.tex}
\subsection{Primal strong formulation}
\subsubsection{Transient}
\phantom{T}
\input{diffusion/continuous/transient/primal_strong-formulation.tex}
\subsubsection{Steady-state}
\input{diffusion/continuous/steady_state/primal_strong-formulation.tex}
\subsection{Primal weak formulation}
\subsubsection{Transient}
\input{diffusion/continuous/transient/primal_weak-discussion.tex}
\input{diffusion/continuous/transient/primal_weak-formulation.tex}
\subsubsection{Steady-state}
\input{diffusion/continuous/steady_state/primal_weak-formulation.tex}
\subsection{Mixed weak formulation}
\subsubsection{Transient}
\input{diffusion/continuous/transient/mixed_weak-discussion.tex}
\input{diffusion/continuous/transient/mixed_weak-formulation.tex}
\subsubsection{Steady-state}
\input{diffusion/continuous/steady_state/mixed_weak-formulation.tex}

\section{Discrete heat transport}
\label{section:discrete_diffusion}
\input{restriction_of_free_vector_space-notation.tex}
\input{generalized-external_normal-notation.tex}
\input{diffusion/discrete/model_derivation-discussion.tex}
\subsection{Primal strong formulation}
\subsubsection{Steady-state}
\input{diffusion/discrete/steady_state/primal_strong_with_normals-formulation.tex}
\subsubsection{Transient}
\input{diffusion/discrete/transient/primal_strong_with_normals-formulation.tex}
\input{diffusion/discrete/transient/primal_strong_with_normals_solve_trapezoidal-discussion.tex}
\subsection{Primal weak formulation}
\subsubsection{Steady-state}
\input{diffusion/discrete/steady_state/primal_weak-formulation.tex}
\input{diffusion/discrete/steady_state/primal_weak_solve-discussion.tex}
\subsubsection{Transient}
\input{diffusion/discrete/transient/primal_weak-formulation.tex}
\input{diffusion/discrete/transient/primal_weak_solve_trapezoidal-discussion.tex}
\input{diffusion/discrete/transient/primal_weak_solve_trapezoidal-algorithm.tex}
\subsection{Mixed weak formulation}
\subsubsection{Steady-state}
\input{diffusion/discrete/steady_state/mixed_weak-formulation.tex}
\input{diffusion/discrete/steady_state/mixed_weak_solve-discussion.tex}
\subsubsection{Transient}
\input{diffusion/discrete/transient/mixed_weak-formulation.tex}
\input{diffusion/discrete/transient/mixed_weak_solve_trapezoidal-discussion.tex}
\input{diffusion/discrete/transient/mixed_weak_solve_trapezoidal-algorithm.tex}

\section{Examples of diffusion}
\label{section:examples_of_transport_phenomena}
\subsection{Steady-state}
\phantom{T}
\input{diffusion/continuous/steady_state/examples/2d_d00_p00-example.tex}
\input{diffusion/continuous/steady_state/examples/2d_d00_p00-figure.tex}
\input{diffusion/continuous/steady_state/examples/2d_d00_p01-example.tex}
\input{diffusion/continuous/steady_state/examples/2d_d00_p01-figure.tex}
\input{diffusion/continuous/steady_state/examples/2d_d00_p02-example.tex}
\input{diffusion/continuous/steady_state/examples/2d_d00_p02-figure.tex}
\input{diffusion/continuous/steady_state/examples/2d_d00_p03-example.tex}
\input{diffusion/continuous/steady_state/examples/2d_d00_p03-figure.tex}
\input{diffusion/continuous/steady_state/examples/2d_d00_p04-example.tex}
\input{diffusion/continuous/steady_state/examples/2d_d00_p04-figure.tex}
\input{diffusion/continuous/steady_state/examples/2d_d00_p05-example.tex}
\input{diffusion/continuous/steady_state/examples/2d_d00_p05-figure.tex}
\input{diffusion/continuous/steady_state/examples/2d_d01_p00-example.tex}
\input{diffusion/continuous/steady_state/examples/2d_d01_p00-figure.tex}
\input{diffusion/continuous/steady_state/examples/2d_d02_p00-example.tex}
\input{diffusion/continuous/steady_state/examples/2d_d02_p00-figure.tex}
\input{diffusion/continuous/steady_state/examples/2d_d02_p01-example.tex}
\input{diffusion/continuous/steady_state/examples/2d_d02_p01-figure.tex}
\input{diffusion/continuous/steady_state/examples/2d_d03_p00-example.tex}
\input{diffusion/continuous/steady_state/examples/2d_d03_p00-figure.tex}
\input{diffusion/continuous/steady_state/examples/2d_d03_p01-example.tex}
\input{diffusion/continuous/steady_state/examples/2d_d03_p01-figure.tex}
\input{diffusion/continuous/steady_state/examples/2d_d04_p00-example.tex}
\input{diffusion/continuous/steady_state/examples/2d_d04_p00-figure.tex}
\input{diffusion/continuous/steady_state/examples/2d_d04_p01-example.tex}
\input{diffusion/continuous/steady_state/examples/2d_d04_p01-figure.tex}
\input{diffusion/continuous/steady_state/examples/2d_d04_p02-example.tex}
\input{diffusion/continuous/steady_state/examples/2d_d04_p02-figure.tex}
\input{diffusion/continuous/steady_state/examples/2d_d04_p03-example.tex}
\input{diffusion/continuous/steady_state/examples/2d_d04_p03-figure.tex}
\input{diffusion/continuous/steady_state/examples/2d_parallelogram_20_15_degrees_45_p00-example.tex}
\input{diffusion/continuous/steady_state/examples/2d_parallelogram_20_15_degrees_45_p00-figure.tex}
\subsection{Transient}
\input{diffusion/continuous/transient/examples/2d_d00_p00-example.tex}
\input{diffusion/continuous/transient/examples/2d_d00_p00-figure.tex}
\input{diffusion/continuous/transient/examples/2d_d00_p01-example.tex}
\input{diffusion/continuous/transient/examples/2d_d00_p01-figure.tex}
\input{diffusion/continuous/transient/examples/2d_d00_p02-example.tex}
\input{diffusion/continuous/transient/examples/2d_d00_p02-figure.tex}
\input{diffusion/continuous/transient/examples/2d_d00_p03-example.tex}
\input{diffusion/continuous/transient/examples/2d_d00_p03-figure.tex}
\input{diffusion/continuous/transient/examples/2d_d00_p04-example.tex}
\input{diffusion/continuous/transient/examples/2d_d00_p04-figure.tex}
\input{diffusion/continuous/transient/examples/2d_d00_p05-example.tex}
\input{diffusion/continuous/transient/examples/2d_d00_p05-figure.tex}
\input{diffusion/continuous/transient/examples/2d_d01_p00-example.tex}
\input{diffusion/continuous/transient/examples/2d_d01_p00-figure.tex}
\input{diffusion/continuous/transient/examples/2d_d02_p00-example.tex}
\input{diffusion/continuous/transient/examples/2d_d02_p00-figure.tex}
\input{diffusion/continuous/transient/examples/2d_d02_p01-example.tex}
\input{diffusion/continuous/transient/examples/2d_d02_p01-figure.tex}
\input{diffusion/continuous/transient/examples/2d_d03_p00-example.tex}
\input{diffusion/continuous/transient/examples/2d_d03_p00-figure.tex}
\input{diffusion/continuous/transient/examples/2d_d03_p01-example.tex}
\input{diffusion/continuous/transient/examples/2d_d03_p01-figure.tex}

\newpage

\section{Lorentzian manifolds and relativity}
\label{section:lorentzian_manifolds_and_relativity}
\phantom{T}
\begin{definition}
  Let
    $\mathcal{K}$ be a quasi-cubical mesh,
    $D = \dim \mathcal{K}$,
    $E$ be a vector bundle on $\mathcal{K}$,
    $p \in \{0, ..., D\}$.
  The space of \textbf{$E$-valued differential $p$-forms} is the
  $(C^0 \mathcal{K})$-module
  \begin{equation}
    \Omega^{p}(\mathcal{K}, E)
    := \Gamma(E \otimes L^p \mathcal{K})
    \simeq \Gamma E \otimes \Omega^p \mathcal{K}.
  \end{equation}
  The \textbf{algebra of $E$-valued differential forms} is the
  $(C^0 \mathcal{K})$-algebra
  \begin{equation}
    \Omega^\bullet(\mathcal{K}, E)
    := \bigoplus_{p = 0}^D \Omega^p(\mathcal{K}, E).
  \end{equation}
\end{definition}
\begin{remark}
  Clearly, $\Omega^{0}(\mathcal{K}, E) \simeq \Gamma E$.
\end{remark}
\begin{definition}
  Let
    $\mathcal{K}$ be a quasi-cubical mesh,
    $D = \dim \mathcal{K}$,
    $E$ be a vector bundle on $\mathcal{K}$,
    $p, q \in \N$.
  The \textbf{cup product of an $E$-valued differential forms with a form}
  is the map
  \begin{equation}
    \usmile
    \colon \Omega^p(\mathcal{K}, E) \times \Omega^q \mathcal{K}
    \to \Omega^{p + q}(\mathcal{K}, E),
  \end{equation}
  defined as follows: it is the unique bilinear map such that for any
  $\sigma \in \Gamma E$,
  $\omega \in \Omega^p \mathcal{K}$,
  $\eta \in \Omega^q \mathcal{K}$,
  \begin{equation}
    (\sigma \otimes \omega) \usmile \eta :=
    \sigma \otimes (\omega \smile \eta).
  \end{equation}
\end{definition}
\begin{remark}
  The extension of the cup product define above is a straightforward
  generalisation of the cup product when one of the objects has two ``legs'':
  a bundle leg and a form leg.
  (Only the form legs multiply; the bundle leg stays the same.)
  We can further generalise if we have an operation on the bundle sections
  (e.g., a Lie bracket),
  or even an external operation between the sections of different bundles
  (e.g., a tensor product). 
\end{remark}
\begin{definition}
  Let
    $\mathcal{K}$ be a quasi-cubical mesh,
    $D = \dim \mathcal{K}$,
    $E, F, G$ be vector bundles on $\mathcal{K}$,
    $\mu \colon \Gamma E \times \Gamma F \to \Gamma G$ be a bilinear map,
    $p, q \in \N$.
  The \textbf{$\mu$-cup product of bundle-valued differential forms} is the map
  \begin{equation}
    \smile_\mu
    \colon \Omega^p(\mathcal{K}, E) \times \Omega^q(\mathcal{K}, F)
    \to \Omega^{p + q}(\mathcal{K}, G),
  \end{equation}
  defined as follows: it is the unique bilinear map such that for any
  $\sigma \in \Gamma E$,
  $\tau \in \Gamma F$,
  $\omega \in \Omega^p \mathcal{K}$,
  $\eta \in \Omega^q \mathcal{K}$,
  \begin{equation}
    (\sigma \otimes \omega) \smile_\mu (\tau \otimes \eta) :=
    \mu(\sigma, \tau) \otimes (\omega \smile \eta).
  \end{equation}
\end{definition}
\begin{example}
  Let
    $\mathcal{K}$ be a quasi-cubical mesh,
    $D = \dim \mathcal{K}$,
    $p, q \in \N$.
  The \textbf{$[\cdot, \cdot]$-cup product of vector-valued differential forms}
  is the map
  \begin{equation}
    \smile_{[\cdot, \cdot]}
    \colon \Omega^p(\mathcal{K}, \mathfrak{X} \mathcal{K})
    \times \Omega^q(\mathcal{K}, \mathfrak{X} \mathcal{K})
    \to \Omega^{p + q}(\mathcal{K}, \mathfrak{X} \mathcal{K}),
  \end{equation}
  defined as follows: it is the unique bilinear map such that for any
  $X, Y \in \mathfrak{X} \mathcal{K}$,
  $\omega \in \Omega^p \mathcal{K}$,
  $\eta \in \Omega^q \mathcal{K}$,
  \begin{equation}
    (X \otimes \omega) \smile_{[\cdot, \cdot]} (Y \otimes \eta) :=
    [X, Y] \otimes (\omega \smile \eta).
  \end{equation}
\end{example}


\section{Continuous electromagnetism}
\label{section:continuous_electromagnetism}
\phantom{T}
\input{electromagnetism/continuous/summary-discussion.tex}
\begin{table}[!ht]
  \caption{Quantities in electromagnetism with forms}
  \label{table:electromagnetism/continuous/quantities}
  \centering
  \begin{tabular}{|l|l|l|l|l|}
    \hline
    Quantity
    & Variable
    & Spatial domain
    & Definition
    & Dimension \topStrut \\[2pt]
    \hline
    \hline
    Electric charge
    & $Q$
    & $\Omega^3 M$
    & given
    & $\charge$ \topStrut \\[2pt]
    \hline
    Electric current
    & $J$
    & $\Omega^2 M$
    & given (or via constitutive law)
    & $\time^{-1} \charge$ \topStrut \\[2pt]
    \hline
    Electric potential
    & $\varphi$
    & $\Omega^0 M$
    & unknown (gauge freedom)
    & $\mass \length^2 \time^{-2} \charge^{-1}$ \topStrut \\[2pt]
    \hline
    Magnetic potential
    & $A$
    & $\Omega^1 M$
    & unknown (gauge freedom)
    & $\mass \length^2 \time^{-1} \charge^{-1}$ \topStrut \\[2pt]
    \hline
    Electric field
    & $E$
    & $\Omega^1 M$
    & $- d \varphi - \frac{\partial A}{\partial t}$
    & $\mass \length^2 \time^{-2} \charge^{-1}$ \topStrut \\[2pt]
    \hline
    Magnetic field
    & $B$
    & $\Omega^2 M$
    & $d_1 A$
    & $\mass \length^2 \time^{-1} \charge^{-1}$ \topStrut \\[2pt]
    \hline
    Electric displacement
    & $D$
    & $\Omega^2 M$
    & via constitutive law
    & $\charge$ \topStrut \\[2pt]
    \hline
    Magnetisation
    & $H$
    & $\Omega^1 M$
    & via constitutive law
    & $\time^{-1} \charge$ \topStrut \\[2pt]
    \hline
    Poynting form
    & $S$
    & $\Omega^2 M$
    & $E \wedge H$
    & $\mass \length^2 \time^{-3}$ \topStrut \\[2pt]
    \hline
    Electric energy form
    & $u_\mathcal{E}$
    & $\Omega^3 M$
    & $(E \wedge D) / 2$
    & $\mass \length^2 \time^{-2}$ \topStrut \\[2pt]
    \hline
    Magnetic energy form
    & $u_\mathcal{M}$
    & $\Omega^3 M$
    & $(B \wedge H) / 2$
    & $\mass \length^2 \time^{-2}$ \topStrut \\[2pt]
    \hline
    Electromagnetic energy form
    & $u$
    & $\Omega^3 M$
    & $u_\mathcal{E} + u_\mathcal{M}$
    & $\mass \length^2 \time^{-2}$ \topStrut \\[2pt]
    \hline
    Lorentz force form
    & F
    & $\Omega^2 M$
    & $\star_1 (\star_3 Q \wedge E) + \star_2 J \wedge \star_2 B$
    & $\mass \time^{-2}$ \topStrut \\[2pt]
    \hline
    Permittivity
    & $\varepsilon$
    & $\Omega^2 M \to \Omega^2 M$
    & material parameter
    & $\mass^{-1} \length^{-3} \time^2 \charge^2$ \topStrut \\[2pt]
    \hline
    Permeability
    & $\mu$
    & $\Omega^2 M \to \Omega^2 M$
    & material parameter
    & $\mass \length \charge^{-2}$ \topStrut \\[2pt]
    \hline
    Conductivity
    & $\sigma$
    & $\Omega^2 M \to \Omega^2 M$
    & material parameter
    & $\mass^{-1} \length^{-3} \time \charge^2$ \topStrut \\[2pt]
    \hline
    Speed of light
    & $c$
    & $\R$
    & universal constant
    & $\length \time^{-1}$ \topStrut \\[2pt]
    \hline
  \end{tabular}
\end{table}

\input{electromagnetism/continuous/laws-table.tex}
\input{electromagnetism/continuous/constitutive_relations-table.tex}
\input{electromagnetism/continuous/spacetime-discussion.tex}
\input{electromagnetism/continuous/spacetime_quantities-table.tex}
\begin{table}[!ht]
  \caption{Laws of spacetime electromagnetism}
  \label{table:electromagnetism/continuous/spacetime_laws}
  \centering
  \begin{tabular}{|l|l|l|l|}
    \hline
    Name
    & Equation
    & Domain
    & Dimension \topStrut \\[2pt]
    \hline
    \hline
    Electromagnetic field is closed
    & $d_2 \mathcal{F} = 0$
    & $\Omega^3(I \times M)$
    & $\mass \length^2 \time^{-1} \charge^{-1}$ \topStrut \\[2pt]
    \hline
    Conservation of electric charge: strong
    & $d_2 \mathcal{D} = \mathcal{Q}$
    & $\Omega^3(I \times M)$
    & $\charge$ \topStrut \\[2pt]
    \hline
    Conservation of electric charge: weak
    & $d_3 \mathcal{Q} = 0$
    & $\Omega^4(I \times M)$
    & $\charge$ \topStrut \\[2pt]
    \hline
    Electromagnetic form is exact
    & $d_1 \mathcal{A} = \mathcal{F}$
    & $\Omega^2(I \times M)$
    & $\mass \length^2 \time^{-1} \charge^{-1}$ \topStrut \\[2pt]
    \hline
    Constitutive law
    & $\mathcal{F} = \mu c \star_2 \mathcal{D}$
    & $\Omega^2(I \times M)$
    & $\mass \length^2 \time^{-1} \charge^{-1}$ \topStrut \\[2pt]
    \hline
  \end{tabular}
\end{table}


\section{Continuum mechanics}
\label{section:continuum_mechanics}
\phantom{T}
\begin{definition}
  Let
    $\mathcal{K}$ be a quasi-cubical mesh,
    $D = \dim \mathcal{K}$,
    $E$ be a vector bundle on $\mathcal{K}$,
    $p \in \{0, ..., D\}$.
  The space of \textbf{$E$-valued differential $p$-forms} is the
  $(C^0 \mathcal{K})$-module
  \begin{equation}
    \Omega^{p}(\mathcal{K}, E)
    := \Gamma(E \otimes L^p \mathcal{K})
    \simeq \Gamma E \otimes \Omega^p \mathcal{K}.
  \end{equation}
  The \textbf{algebra of $E$-valued differential forms} is the
  $(C^0 \mathcal{K})$-algebra
  \begin{equation}
    \Omega^\bullet(\mathcal{K}, E)
    := \bigoplus_{p = 0}^D \Omega^p(\mathcal{K}, E).
  \end{equation}
\end{definition}
\begin{remark}
  Clearly, $\Omega^{0}(\mathcal{K}, E) \simeq \Gamma E$.
\end{remark}
\begin{definition}
  Let
    $\mathcal{K}$ be a quasi-cubical mesh,
    $D = \dim \mathcal{K}$,
    $E$ be a vector bundle on $\mathcal{K}$,
    $p, q \in \N$.
  The \textbf{cup product of an $E$-valued differential forms with a form}
  is the map
  \begin{equation}
    \usmile
    \colon \Omega^p(\mathcal{K}, E) \times \Omega^q \mathcal{K}
    \to \Omega^{p + q}(\mathcal{K}, E),
  \end{equation}
  defined as follows: it is the unique bilinear map such that for any
  $\sigma \in \Gamma E$,
  $\omega \in \Omega^p \mathcal{K}$,
  $\eta \in \Omega^q \mathcal{K}$,
  \begin{equation}
    (\sigma \otimes \omega) \usmile \eta :=
    \sigma \otimes (\omega \smile \eta).
  \end{equation}
\end{definition}
\begin{remark}
  The extension of the cup product define above is a straightforward
  generalisation of the cup product when one of the objects has two ``legs'':
  a bundle leg and a form leg.
  (Only the form legs multiply; the bundle leg stays the same.)
  We can further generalise if we have an operation on the bundle sections
  (e.g., a Lie bracket),
  or even an external operation between the sections of different bundles
  (e.g., a tensor product). 
\end{remark}
\begin{definition}
  Let
    $\mathcal{K}$ be a quasi-cubical mesh,
    $D = \dim \mathcal{K}$,
    $E, F, G$ be vector bundles on $\mathcal{K}$,
    $\mu \colon \Gamma E \times \Gamma F \to \Gamma G$ be a bilinear map,
    $p, q \in \N$.
  The \textbf{$\mu$-cup product of bundle-valued differential forms} is the map
  \begin{equation}
    \smile_\mu
    \colon \Omega^p(\mathcal{K}, E) \times \Omega^q(\mathcal{K}, F)
    \to \Omega^{p + q}(\mathcal{K}, G),
  \end{equation}
  defined as follows: it is the unique bilinear map such that for any
  $\sigma \in \Gamma E$,
  $\tau \in \Gamma F$,
  $\omega \in \Omega^p \mathcal{K}$,
  $\eta \in \Omega^q \mathcal{K}$,
  \begin{equation}
    (\sigma \otimes \omega) \smile_\mu (\tau \otimes \eta) :=
    \mu(\sigma, \tau) \otimes (\omega \smile \eta).
  \end{equation}
\end{definition}
\begin{example}
  Let
    $\mathcal{K}$ be a quasi-cubical mesh,
    $D = \dim \mathcal{K}$,
    $p, q \in \N$.
  The \textbf{$[\cdot, \cdot]$-cup product of vector-valued differential forms}
  is the map
  \begin{equation}
    \smile_{[\cdot, \cdot]}
    \colon \Omega^p(\mathcal{K}, \mathfrak{X} \mathcal{K})
    \times \Omega^q(\mathcal{K}, \mathfrak{X} \mathcal{K})
    \to \Omega^{p + q}(\mathcal{K}, \mathfrak{X} \mathcal{K}),
  \end{equation}
  defined as follows: it is the unique bilinear map such that for any
  $X, Y \in \mathfrak{X} \mathcal{K}$,
  $\omega \in \Omega^p \mathcal{K}$,
  $\eta \in \Omega^q \mathcal{K}$,
  \begin{equation}
    (X \otimes \omega) \smile_{[\cdot, \cdot]} (Y \otimes \eta) :=
    [X, Y] \otimes (\omega \smile \eta).
  \end{equation}
\end{example}


\section{Discrete elasticity}
\label{section:discrete_elasticity}
\phantom{T}
\input{elasticity/discrete/model-discussion.tex}
\input{elasticity/discrete/model-example.tex}

\end{document}
